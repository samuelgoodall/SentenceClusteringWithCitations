Evaluating the relative importance of nodes in a graph is a fundamental operation in network analysis, and the literature is full of well-studied approaches for quantifying node importance (typically called centrality measures)~\cite{estrada2012structure,newman2010networks}.
Functions of matrices are a natural candidate for such rankings~\cite{estrada2010network}; in particular the matrix resolvent~\cite{gleich2015pagerank,katz1953new,page1999pagerank} (Katz and PageRank centrality) and the matrix exponential~\cite{benzi2013total,benzi2014matrix,chung2007heat,estrada2005subgraph} (heat kernel and subgraph centrality) have been widely studied and used in practice.

Like many approaches to centrality, functions of matrices actually give rise to infinite families of specific centrality measures based on the value of some parameter in the definition.
For example, the ($\beta$-)\emph{subgraph centrality} of a node $i$ is given by the diagonal entry of the matrix exponential $\exp(\beta \mA)_{ii}$ for non-negative $\beta$~\cite{estrada2005subgraph}.
In general, there is no consensus on the ``best'' parameter value(s) for a centrality measure, and the effects of specific choices can be hard to characterize.
Recent work has considered how relative node rankings change as matrix-based centrality
parameters vary~\cite{benzi2015limiting,paton2017centrality}.
In this work, we study when subgraph centralities can assign identical scores to nodes that are structurally different in the underlying graph.

The notion of ``structural equivalence'' we consider is based on prior studies of the interplay between the uniformity of subgraph centrality (or \emph{walk entropy}~\cite{benzi2014note,estrada2014walk}) and the \emph{walk-regularity} of a graph.
To be more precise, consider a graph $G$ with adjacency matrix $\mA$.
We say $G$ is \emph{walk-regular} if for each $\ell \geq 0$, every node has the same number of closed walks of length $\ell$ (equivalently, $\mA^{\ell}$ has constant diagonal), and we say nodes $i$ and $j$ are in the same \emph{walk class} (structurally equivalent) if $\mA^{\ell}_{ii} = \mA^{\ell}_{jj}$.
Early studies suggested that walk-regularity might be completely characterized by attaining maximum walk entropy;
that is, it was conjectured that a graph is walk-regular if and only if there exists at least one $\beta_0$ such that $\exp(\beta_0 \mA)$ gives all nodes the same score (resulting in maximum walk entropy)~\cite{benzi2014note}.
Recent work~\cite{Kloster2018115} disrupted this line of research by presenting a single graph which is non--walk-regular yet has uniform $\beta$-subgraph centrality for a particular value of $\beta$.
We call such a non--walk-regular graph \emph{entropic}, and any value $\beta$ for which $G$ attains maximum walk entropy an \emph{entropic value} for $G$.

In this work, we resolve several outstanding questions regarding the interplay between
walk-regularity and centrality. We begin by exhibiting an infinite family of entropic graphs (Section~\ref{sec:cartesian-graph}).
Our construction proves that for each entropic value $\beta_0$ there are infinitely many graphs entropic with respect to $\beta_0$.
Interestingly, this result does not produce any \emph{new} entropic values $\beta_0$;
however, in Section~\ref{sec:beta-distribution}
we establish that the set of entropic values is not only infinite, but contains a limit point at 0.

We then consider the more general class of $f$-\emph{subgraph centralities} given by the
diagonal entries of $f(\beta\mA)$ for a parameter $\beta > 0$ and suitable function $f$ defined on the spectrum of $\mA$.
If a graph has uniform $f$-subgraph centrality for some parameter $\beta$,
we say $G$ is \emph{$f$-entropic}, the value $\beta$ is $f$-entropic with respect to $G$, and $f$ is an entropic function.
In Section~\ref{sec:infinite-tensor} we prove that there are infinitely many functions $f_i$ that are entropic with respect to at least one graph.

Finally, we consider the related question of when a subset of a graph's walk-classes
have the same $f$-subgraph centrality score for some parameter $\beta$;
when this occurs, we say the walk classes \emph{collide} at $f$,
or that $f$ \emph{induces a collision} at the walk-classes.
Note that a graph is $f$-entropic exactly when $f$ induces a collision at all of the graph's walk-classes.
We present a sufficient condition for determining that a set of walk-classes collide under some function $f$,
and a sufficient condition for concluding that a set of walk-classes do not collide under \emph{any} suitable function $f$.
These sufficient conditions are practical to compute on modest-sized graphs.
