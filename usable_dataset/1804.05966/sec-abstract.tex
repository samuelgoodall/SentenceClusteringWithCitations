Matrix-based centrality measures have enjoyed significant popularity in network analysis, in
no small part due to our ability to rigorously analyze their behavior as parameters vary.
Recent work has considered the relationship between subgraph centrality,
which is defined using the matrix exponential $f(x) = \exp(x)$, and the walk structure of a network.
In a walk-regular graph, the number of closed walks of each length must be the same for all nodes, implying uniform $f$-subgraph centralities for any $f$ (or maximum $f$-\emph{walk entropy}).
We consider when non--walk-regular graphs can achieve maximum entropy, calling such graphs \emph{entropic}.
For parameterized measures, we are also interested in which values of the parameter witness this
uniformity.
To date, only one entropic graph has been identified, with only two witnessing parameter values, raising the question of how many such graphs and parameters exist.
We resolve these questions by constructing infinite families of entropic graphs, as well as a family of witnessing parameters with a limit point at zero.\newline

%  05C50, % Graphs and linear algebra (matrices, eigenvalues, etc.)
%  05C75, % Structural characterization of families of graphs
%  15A16 % Matrix exponential and similar functions of matrices
\noindent\textit{MSC:} 05C50, 05C75, 15A16 \newline
\noindent\textit{Keywords:} centrality; graph entropy; walk-regularity; functions of matrices; network analysis
