%%%%%%%%%%%%%%%%%%%%%%%%%%%%%%%%%%%%%%%%%%%%%%%%%%%%%%%%%%%%%%%%%%%%%%%%%%%%%
\clearemptydoublepage
\chapter{Introduction} \label{chap:intro}
%%%%%%%%%%%%%%%%%%%%%%%%%%%%%%%%%%%%%%%%%%%%%%%%%%%%%%%%%%%%%%%%%%%%%%%%%%%%%
%% \mycitation
%% {Start your dissertation with your thesis}
%% {Olivier Danvy, BRICS retreat, January~19, 2007.}
%%%%%%%%%%%%%%%%%%%%%%%%%%%%%%%%%%%%%%%%%%%%%%%%%%%%%%%%%%%%%%%%%%%%%%%%%%%%%
%% It is natural to assume a bound on the quantum memory of adversarial
%% players in a cryptographic protocol.

In the quest for interesting \index{cryptographic model}cryptographic
models, bounding the quantum memory of adversarial players is a great
% (good, useful, fruitful, practical, nice, valuable) 
assumption.

\section{Cryptographic Models and Basic Primitives} \label{sec:cryptomodels}
It is a fascinating art to come up with
\emph{\index{protocol}protocols}\footnote{A protocol consists of
  clear-cut instructions for the participating players.} that achieve a
cryptographic task like encryption, authentication, identification,
voting, secure function evaluation to name just a famous few. To
define a notion of security for such protocols, one needs to specify a
\emph{\index{cryptographic model}cryptographic model}, i.e. an
environment in which the protocol is run. The model states for example
the number of honest and dishonest players, the allowed running time
and amount of memory available to honest and dishonest players, how
dishonest players are allowed to deviate from the protocol, the use of
external resources like (quantum) communication channels or other
already established cryptographic functionalities etc.

While coming up with more and more protocols for different models,
cryptographers realized that some basic \emph{primitives}
(i.e.~precisely defined cryptographic tasks) are useful as
``benchmarks'' of how powerful a particular cryptographic model is.
An example is the two-party primitive \emph{
\index{oblivious transfer}Oblivious Transfer} (\pOT).  It comes in different flavors,
but all of these variants are equivalent in the sense that anyone of
them can be implemented using (possibly several instances of) an
other.  The \emph{\index{oblivious transfer!one-out-of-two}one-out-of-two} variant \OT was originally
introduced by Wiesner around 1970 (but only published much later
in~\cite{Wiesner83}) in the very first paper about quantum
cryptography, and later rediscovered by Even, Goldreich, and Lempel~\cite{EGL82}. It lets a sender Alice transmit two bits to a
receiver Bob who can choose which of them to receive. A secure
implementation of \OT does not allow a dishonest sender to learn which
of the two bits was received and it does not allow a dishonest
receiver to learn any information about the second bit. It was a
surprising insight when Kilian showed that this simple primitive is \emph{complete} for
two-party cryptography \cite{Kilian88}. In other words, a model in
which \OT can be securely implemented allows to implement any
cryptographic functionality between two players\footnote{If the model
  can be reasonably extended to more players, this usually allows to
  implement secure multi-party protocols as well.}. Another variant we
are concerned with in this thesis was introduced by Rabin
\cite{Rabin81} and is hence called \index{oblivious transfer!Rabin}Rabin Oblivious Transfer
(\RabinOT). It is basically a ``secure erasure channel'': the sender
Alice sends a bit which with probability one half is absorbed and with
probability one half finds its way to the receiver Bob. The security
requirements are the following: whatever a dishonest Alice does, she cannot find
out whether the bit was received or not; and whatever a dishonest
receiver does, he does not get any information about the bit with
probability one half.

Yet another basic two-party primitive of interest is \index{bit
  commitment} Bit Commitment (\BC) which allows a player to commit
himself to a choice of a bit $b$ by communicating with a verifier. The
verifier should not learn $b$ (we say the commitment is
\emph{\index{bit commitment!hiding}hiding}), yet the committer can
later choose to reveal $b$ in a convincing way, i.e. only the value
fixed at commitment time will be accepted by the verifier (we say the
commitment is \emph{\index{bit commitment!binding}binding}).  Bit
Commitment is a fundamental building block of virtually every more
complicated cryptographic protocol. Implementing secure \BC with a
secure \OT at hand is not difficult\footnote{To commit to a bit $b$,
  the committer sends random bits of parity $b$ via (several instances
  of) \OT and the verifier picks randomly one of the bits. To open,
  the committer sends all the random bits he was using, the verifier
  checks whether these are consistent with what he received.}. On the
other hand, there are cryptographic models allowing to securely
implement \BC, but not \OT. Moran and Naor gave an example of such a
model by assuming the physical device of a tamper-proof seal~\cite{MN05}.

It is not hard to see that the two security requirements for \BC are
in a sense contradictory, so perfectly secure bit commitment cannot be
implemented ``from scratch'', that is if only error-free communication
is available and there is no limitation assumed on the computing power
and memory of the players. The informal reason for this is that the
hiding property implies that when 0 is committed to, exactly the same
information exchange could have happened when committing to 1.
Hence, even if 0 was actually committed to, the committer could always
compute a complete view of the protocol consistent with having
committed to 1, and pretend that this view was what he had in mind
originally. By the reduction of \BC to \OT follows that also \OT and
many other cryptographic functionalities cannot be perfectly secure
when built from scratch.

One might hope that allowing the protocol to make use of quantum
communication would make a difference. Here, information is stored in
qubits, i.e., in the state of two-level quantum mechanical systems,
such as the polarization state of a single photon. Quantum information
behaves in a way that is fundamentally different from classical
information, enabling, for instance, unconditionally secure key
exchange between two honest players (so-called
\emph{\index{quantum key distribution}Quantum Key Distribution}).
However, in the case of two mutually distrusting parties, we are not
so fortunate: even with quantum communication, unconditionally secure
\BC and \OT remain \index{impossibility!of quantum bit
  commitment}impossible. This is the infamous
impossibility result by Mayers and by Lo and Chau~\cite{Mayers97,LC97}.

For this reason, cryptographers have tried hard to exhibit more
restricted models where these impossibility results do not apply. The
high art in this process is to find assumptions that are as realistic
as possible -- thus only minimally restricting the model, but still
strong enough to allow for implementing interesting functionalities.
There are at least three kinds of possible assumptions, namely
\begin{itemize}
\item bounding the computing power of players,
\item using the noise in the communication channel,
\item exploiting some physical limitation of the adversary, e.g., if
  the size of the available memory is bounded.
\end{itemize}

The first scenario is the basis of many well known solutions based on
plausible but unproven complexity assumptions, such as hardness of
factoring or discrete logarithms. A term often used for such schemes
is ``\index{computational security}computational security'', meaning
that it is \emph{not impossible} for an adversary to behave
dishonestly, but it is \emph{computationally infeasible} for him to do
so. Security proofs are usually done by reduction in the sense that
breaking the security of the protocol would imply solving a hard
problem like factoring the product of two large prime numbers. The
second scenario has been used to construct both \BC and \pOT protocols
in various models for the noise by Cr\'epeau, Kilian, Damg{\aa}rd,
Salvail, Fehr, Morozov, Wolf, and Wullschleger
\cite{CK88,DKS99,DFMS04,CMW04,Wullschleger07}.

The third scenario is the focus of this thesis. In contrast to the
first scenario, we deal with ``\index{unconditional security}unconditional security'' where (depending on the task a
protocol aims to achieve) an adversary has no way whatsoever to gain
illegal information. Proofs are not done by reduction, but we can
prove in information-theoretic terms that except with negligible
probability, the adversary does not learn \emph{any information} that
is meant to remain secret.

\section{Classical Bounded-Storage Model} \label{sec:ClassicalBSMIntro}
In the \index{classical bounded-storage model}classical
bounded-storage model, we assume the players to use classical
error-free communication and to be computationally unbounded, but on the
other hand restrict the size of their memory. In the usual setting,
there is a large random source $R$ (often called the
\emph{\index{randomizer}randomizer}) which all players can access, but
which is too large (or transmitted too quickly) to store as a whole.
One can think of $R$ as a deep-space radio source or a satellite broadcasting
random bits at a very high rate.

When Maurer introduced the classical bounded-storage model
in~\cite{Maurer90}, the goal was \emph{secure message transmission}.
He showed that two honest parties Alice and Bob sharing an initial key
can \index{key expansion}expand that key unless the eavesdropper Eve
can store more than a large fraction of the randomizer.  The basic
idea of the technique allowing Alice and Bob to get an advantage
over Eve is that their initial secret key indexes some positions in
the randomizer about which Eve has some uncertainty if she cannot
store the whole randomizer. Therefore, the bits at these positions can
be combined to yield more secure key bits and so to expand the initial
key.

A line of subsequent work by Maurer, Cachin, Aumann, Ding, Rabin,
Dziembowski, Lu, and Vadhan \cite{Maurer92, CM97, ADR02, DM04, Lu04,
  Vadhan04} improved this original protocol in terms of efficiency and
security. Aumann, Ding and Rabin~\cite{ADR02} noticed that
protocols in this model enjoy the property of ``\index{everlasting
security}everlasting security'' in the sense that the newly
generated key remains secure even when the initial key is later
revealed and Eve is no longer memory-bounded, under the sole condition
that the original randomizer cannot be accessed any
more. Ding~\cite{Ding05} showed how to do 
\index{error correction!classical bounded-storage model}error correction in the
bounded-storage model and therefore how to cope with the situation
when the honest parties do not have exactly the same view on the
randomizer.

Cachin, Cr\'epeau and Marcil illustrated the power of the
bounded-storage model by exhibiting in~\cite{CCM98} a protocol for
\OT. Ding improved on this \cite{Ding01} and later showed a
constant-round protocol for oblivious transfer in joint work with
Harnik, Rosen and Shaltiel \cite{DHRS04}.

All these protocols are shown secure as long as the adversary's memory
size is at most quadratic in the memory size of the honest players.
Considering the ease and low cost of storing massive amounts of classical
data nowadays, it is questionable how practical such an assumption on the
memory size of the players is. It would be clearly more satisfactory
to have a larger than quadratic separation between the memory size of
honest players and that of the adversary. However, this was shown to
be impossible by Dziembowski and Maurer~\cite{DM04}.


\section{Contributions} \label{sec:contributions}
In this section, we give an overview of the contributions of this
thesis.
The results about classical oblivious transfer described in
Chapter~\ref{chap:ClassicalOT} and summarized in
Section~\ref{sec:ClassicalOTReductions} are joint
work with Damg{\aa}rd, Fehr and Salvail~\cite{DFSS06}. 
All other results are based on two papers co-authored with Damg{\aa}rd, Fehr,
Salvail and Renner: \cite{DFSS05} and \cite{DFRSS07}. A journal version of \cite{DFSS05}
is to appear in a special issue of the SIAM Journal of Computing
\cite{DFSS08journal}.


\subsection{Bounded-Quantum-Storage Model}
\index{bounded-quantum-storage model}In this thesis, we study for the
first time protocols where quantum communication is used and we place
a bound on the adversary's {\em quantum} memory size.  There are two
reasons why this may be a good idea: first, if we do not bound the
classical memory size, we avoid the impossibility result of
\cite{DM04}.  Second, the adversary's typical goal is to obtain a
certain piece of classical information that we want to keep hidden
from him. However, if he cannot store all the quantum information that
is sent, he must convert some of it to classical information by
measuring. This may irreversibly destroy information, and we may be
able to arrange it in such a way that the adversary cannot afford to lose
information this way, while honest players can.

It turns out that this can be achieved indeed: we present protocols for
both \BC and \pOT in which $n$ qubits are transmitted, where honest
players need {\em no quantum memory}, but where the adversary must
store at least a large fraction (typically $n/2$ or $n/4$) of the $n$
transmitted qubits to break the protocol. We emphasize that no bound
is assumed on the adversary's computing power, nor on his classical
memory. This is clearly much more satisfactory than the classical
case, not only from a theoretical point of view, but also in practice:
while sending qubits and measuring them immediately as they arrive is
well within reach of current technology, storing even a single qubit
for more than a fraction of a second is a formidable technological
challenge.

Furthermore, we show that our protocols also work in a non-ideal
setting where we allow the quantum source to be imperfect and the
quantum communication to be noisy. We emphasize that what makes \pOT and
\BC possible in our model is not so much the memory bound per se, but
rather the loss of information on the part of the adversary. Indeed,
our results also hold if the adversary's memory device holds an
arbitrary number of qubits, but is imperfect in certain ways.

% underlying my thesis (see first sentence in intro):
All these factors make the assumption of \index{bounded-quantum-storage
model}bounded quantum memory a very attractive cryptographic model.
On one hand, as for the \index{classical bounded-storage
model}classical bounded-storage model, it is simple to work with and
yields beautiful theoretical results. On the other hand, it is much
more reasonable to assume the difficulty of storing quantum
information compared to storing classical one and hence, we are very
close to the physical reality and get schemes that can actually be
implemented!

%This is discussed in more detail in Section~\ref{sec:noisymem}.

\subsection{Characterization of Security of Classical \OT} \label{sec:ClassicalOTReductions}
While the task of formally defining \index{unconditional security}unconditional security of classical protocols for \RabinOT
and \BC is well understood, capturing the security of \OT in
information-theoretic terms is considerably more delicate, as was
pointed out by Cr\'epeau, Savvides, Schaffner and
Wullschleger~\cite{CSSW06}. 
For \OT of bits, it is clear that the
security for a honest sender against a cheating receiver guarantees
that the receiver does not learn any information about the XOR of the
two bits. Somewhat surprisingly, the converse is true as well, not
having any information about the XOR of the two bits sent implies that
we can point at one bit which the dishonest receiver does not know
(given the other).

This idea can be generalized to \OT of strings where the ignorance of
the XOR becomes ignorance of the outcome of all \index{non-degenerate
linear function}Non-Degenerate Linear binary Functions (NDLFs)
applied to the two strings sent. Such a characterization of
\index{sender-security!characterization of}sender-security in terms of
NDLF composes well with \emph{strongly \index{two-universal
    hashing!strongly}two-universal hashing} and hereby yields a
powerful technique to improve the analyses of the standard \index{reduction}reductions
from \OT to weaker variants of \pOT.

As a historical side note, the original motivation for this classical
characterization was the hope that it translates to the quantum
setting and thereby yields a security proof of the \OT scheme in the
\index{bounded-quantum-storage model}bounded-quantum-storage model. We
will point out why this approach does \emph{not} work.


\subsection{Quantum Security Definitions and Protocols}
When the players are allowed to use quantum communication, the output
of a dishonest player is a quantum state even when the protocol
implements a classical primitive. Therefore, security definitions for
\RabinOT, \OT and \BC have to be phrased in quantum terms. As an
easy-to-use \index{composability}composability framework has not yet
been established for quantum protocols\footnote{Some rather
complicated frameworks are known. They have been put forward by Ben-Or and
Mayers \cite{BM04} and Unruh \cite{Unruh02}.}, various \emph{ad-hoc}
security requirements are commonly used. The definitions in this
thesis are the strongest so far proposed, and as they are based on the
(classical) considerations in \cite{CSSW06}, we believe that they are
best suited to provide \emph{sequential composability}.

Most of the presented protocols in the bounded-quantum-storage model
can be cast in a non-interactive form, i.e.~only one party sends
information when doing \pOT, commitment or opening. We show the following.

\medskip
\noindent
{\bf {\em \pOT in the Bounded-Quantum-Storage Model:}} {\em There
  exist non-interactive protocols for \RabinOT and 1-out-of-2
  Oblivious Transfer (\OT[2]) of $\ell$-bit messages, secure in the
  bounded-quantum-storage model against adversaries with
  quantum-memory size at most $n/2- \ell$ for \RabinOT and $n/4 -
  2\ell$ for \OT. Here, $n$ is the number of qubits transmitted in the
  protocol and $\ell$ can be a constant fraction of $n$. Honest
  players need no quantum memory at all.}  \medskip

For the case of bit commitment, the standard definition of the
\index{bit commitment!binding}binding property used in the quantum
setting was introduced by Dumais, Mayers and Salvail~\cite{DMS00}. For
$b \in \set{0,1}$, let $p_b$ denote the probability that a dishonest
committer successfully opens the commitment to value $b$. The binding
condition then requires that the sum of $p_0$ and $p_1$ does
essentially not exceed 1. More formally, $p_0 + p_1 \leq 1 + \negl{n}$
where $\negl{n}$ stands for a term which is negligible in $n$ such as
$2^{-cn}$ (for a constant $c>0$) which is exponentially small in $n$.
This is to capture that a quantum committer can always commit to the
values $0$ and $1$ in superposition. We call this notion \emph{weakly
  binding} in the following. A shortcoming of this notion is that
committing bit by bit is not guaranteed to yield a secure string
commitment---the argument that one is tempted to use requires
independence of the $p_{b}$'s between the different executions, which
in general does not hold.

% Note that this definition of the binding
% condition allows protocols in which with probability one half a
% dishonest committer cannot open the commitment to any of the two
% values and with probability one half, he can open it to the bit of his
% choice.  Clearly, such behavior is not what one intuitively expects
% from a commitment scheme.

Instead, we propose the following \emph{strong binding} condition:
After the commitment phase, there exists a binary random variable $D
\in \set{0,1}$ such that a dishonest committer cannot open the
commitment to value $D$ except with negligible probability. The point
is that the distribution of $D$ is not under control of the dishonest
committer. We will point out that using this definition, we can easily
derive the security of a string commitment from the security of the
individual bits.

\medskip
\noindent
{\bf {\em \BC in the Bounded-Quantum-Storage Model:}} {\em There
  exists a protocol for bit commitment which is non-interactive.
 It is perfectly hiding and weakly binding in the
  bounded-quantum-storage model against dishonest committers with
  quantum-memory size at most $n/2$. It is strongly binding against
  memory sizes of at most $n/4$. Here, $n$ is the number of qubits
  transmitted in the protocol. Honest players need no quantum memory
  at all.}  \medskip


Furthermore, the commitment protocol has the interesting property that
the only message is sent \emph{to} the committer, i.e., it is possible
to commit while only {\em receiving} information.  Such a scheme
clearly does not exist without a bound on the committer's memory, even
under computational assumptions and using quantum communication: a
corrupt committer could always store (possibly quantumly) all the
information sent, until opening time, and only then follow the honest
committer's algorithm to figure out what should be sent to
convincingly open a 0 or a~1.  

Note that in the \index{classical bounded-storage model}classical
bounded-storage model, it has been shown by Moran, Shaltiel and
Ta-Shma~\cite{MST04} how to do \index{time-stamping}time-stamping
that is non-interactive in our sense: a player can time-stamp a
document while only receiving information.  However, no reasonable
protocol for \BC or for time-stamping a single bit exists in this
model.  It is straightforward to see that any such protocol can be
broken by an adversary with classical memory of size twice that of an
honest player, while our protocol requires no quantum memory for the
honest players and remains secure against any adversary unable to
store more than half the size of the quantum transmission.

We also note that it has been shown earlier by Salvail
\cite{Salvail98} that \BC is possible using quantum communication,
assuming a different type of physical limitation, namely a bound on
the size of coherent measurement that can be implemented. This
limitation is incomparable to ours: it does not limit the total size
of the memory, instead it limits the number of bits that can be
simultaneously operated on to produce a classical result. Our
adversary has a limit on the total quantum memory size, but can
measure all of it coherently. The protocol from \cite{Salvail98} is
interactive, and requires a bound on the maximal measurement size that
is sub-linear in $n$.

% Finally, our techniques imply security of the
% practical BB84-based protocol for \ROT\ introduced in \cite{DFSS05} with
% a memory bound twice that obtained therein. %\cite{DFSS05}.


\subsection{Quantum Uncertainty Relations}
\index{uncertainty relation}A problem often encountered in
\index{quantum cryptography}quantum cryptography is the following:
through some interaction between the players, a quantum state is
generated and then measured by one of the players (we call her Alice
in the following). Assuming Alice is honest, we want to know how
unpredictable her measurement outcome is to the adversary.  Once a
lower bound on the adversary's uncertainty about Alice's measurement
outcome is established, it is usually easy to prove the desired
security property of the protocol. Many existing constructions in
quantum cryptography have been proven secure following this paradigm.

Typically, Alice does not make her measurement in a fixed basis, but
chooses at random from a set of different bases. These bases are
usually chosen to be pairwise {\em \index{mutually unbiased
bases}mutually unbiased}, meaning that if the quantum state is
such that the measurement outcome in one basis is fixed, then this
implies that the uncertainty about the outcome of the measurement in
the other basis is maximal. In this way, one hopes to keep the
adversary's uncertainty high, even if the state is (partially) under
the adversary's control.

An inequality that lower bounds the adversary's uncertainty in such a
scenario is called an {\em uncertainty relation}.  There exist
uncertainty relations for different measures of uncertainty but
cryptographic applications typically require the adversary's
\index{entropy!min-}min-entropy to be bounded from below. Such uncertainty relations are
the key ingredient in the security proofs of our protocols in the
bounded-quantum-storage model.

In this thesis, we introduce new general and tight high-order entropic
uncertainty relations. Since the relations are expressed in terms of
lower bounds on the min-entropy or upper-bounds on large probabilities
respectively, they are applicable to a large class of natural
protocols in quantum cryptography.

The first uncertainty relation is concerned with the situation where a
\smash{$n$-qubit} state $\rho$ is measured in one out of two mutually
unbiased bases, say either in the
\index{basis!computational}computational basis (the $+$-basis) or in
the \index{basis!diagonal}diagonal basis (the $\times$-basis).

\medskip
\noindent
{\bf {\em First Uncertainty Relation:}} {\em Let $\rho$ be an
  arbitrary state of $n$ qubits, and let $\Qp(\cdot)$ and $\Qt(\cdot)$
  be the respective probability distributions over $\nbit$ of the
  outcome when $\rho$ is measured in the $+$-basis respectively the
  $\times$-basis.  Then, for any two sets $L^+ \subset \set{0,1}^n$
  and $L^{\times} \subset \set{0,1}^n$ it holds that
\[ \Qp(L^+)+\Qt(L^{\times}) \leq 1 + 2^{-n/2} \sqrt{|L^+|
  |L^{\times}|}. \] } 

Another \index{uncertainty relation}uncertainty relation is derived
for situations where an $n$-qubit state $\rho$ has each of its qubits
measured in a random and independent basis sampled uniformly from a
fixed set ${\cal B}$ of bases.  ${\cal B}$ does not necessarily have
to be \index{mutually unbiased bases}mutually unbiased, but we assume
a lower bound $h$---the so-called {\em \index{average entropic
uncertainty bound}average entropic uncertainty bound}---on the
average \index{entropy!Shannon}Shannon entropy of the distribution $P_{\vartheta}$, obtained
by measuring an arbitrary one-qubit state in basis $\vartheta \in
{\cal B}$, meaning that $\frac{1}{|{\cal B}|}\sum_{\vartheta}
\H(P_{\vartheta}) \geq h$.

\medskip
\noindent
{\bf {\em Second Uncertainty Relation (informal):}} {\em Let $\cal B$ be a
  set of bases with an average entropic uncertainty bound $h$ as
  above.  Let $P_{\theta}$ denote the probability distribution defined
  by measuring an arbitrary $n$-qubit state $\rho$ in basis $\theta
  \in {\cal B}^n$. For a uniform choice $\Theta \in_R {\cal B}^n$,
  it holds except with negligible probability (over $\Theta$ and over
  $P_{\theta}$) that %for every $\theta$, we have
\begin{equation}\label{main}
\hmin(P_\theta \mid \Theta=\theta) \gtrsim n h.
\end{equation}
} \medskip 

Observe that (\ref{main}) cannot be improved significantly since the
min-entropy of a distribution is at most equal to the Shannon entropy.
Our uncertainty relation is therefore asymptotically tight when the
bound $h$ is tight.

Any lower bound on the Shannon entropy associated to a set of
measurements ${\cal B}$ can be used in (\ref{main}).  In the special
case where the set of bases is ${\cal B}=\{+,\times\}$ (i.e. the two
BB84 bases named after Bennett and Brassard who used them in the first
quantum-key-distribution protocol~\cite{BB84}), 
$h$ is known precisely using Maassen and Uffink's
entropic relation~\cite{MU88}, see~(\ref{eq:maassenuffink}).  We
get $h=\frac{1}{2}$ and (\ref{main}) results in $\hmin(P_{\theta} \mid
\Theta=\theta) \gtrsim \frac{n}{2}$.
%% For ${\cal B}=\{+,\times,\oslash\}$ where 
%% $\oslash$ denotes the circular
%% basis, $h=\frac{2}{3}$ has also been shown tight in \cite{Ruiz93}.
%% In this case, (\ref{main}) provides 
%% $\hmin(P_{\theta} \mid {\cal E}_{\theta}) \geq \frac{2(1-\delta)n}{3}$.
Uncertainty relations for the \index{BB84 coding scheme}BB84 coding
scheme are useful, since this coding is widely used in
quantum cryptography.  Its resilience to imperfect quantum channels,
sources, and detectors is an important advantage in practice.

A major difference between the first and second uncertainty relation
is that while both relations can be used to bound the min-entropy
conditioned on an event, this event happens in the latter case with
probability essentially 1 (on average) whereas the corresponding event
from the first relation (defined in Corollary~\ref{cor:hadamard}) only
happens with probability about $1/2$.


\subsection{\QKD against Quantum-Memory-Bounded Eavesdropper}
We illustrate the versatility of our second uncertainty relation by
applying it to Quantum-Key-Distribution (\QKD) settings.  \QKD is the
art of distributing a secret key between two distant parties, Alice
and Bob, using only a completely insecure quantum channel and
authentic classical communication. \QKD protocols typically provide
\index{unconditional security}unconditional security, i.e., even an
adversary with unlimited resources cannot get any information about
the key.  A major difficulty when implementing \QKD schemes is that
they require a low-noise quantum channel.  The tolerated noise level
depends on the actual protocol and on the desired security of the key.
Because the quality of the channel typically decreases with its
length, the maximum tolerated noise level is an important parameter
limiting the maximum distance between Alice and Bob.

We consider a model in which the adversary has a limited amount of
quantum memory to store the information she intercepts during the
protocol execution. In this model, we show that the maximum
tolerated noise level is larger than in the standard scenario where
the adversary has unlimited resources.  
For {\em one-way \QKD protocols} which are protocols where error-correction is
performed non-interactively (i.e., a single classical message is sent
from one party to the other), we show the following result:

\medskip
\noindent
{\bf {\em \QKD Against Quantum-Memory-Bounded Eavesdroppers:}} {\em Let
  $\cB$ be a set of orthonormal bases of the two-dimensional Hilbert space $\cH_2$ with average entropic
  uncertainty bound $h$. Then, a \emph{one-way \QKD-protocol} produces
  a secure key against eavesdroppers whose quantum-memory size is
  sublinear in the length of the raw key at a positive rate, as long as
  the bit-flip probability $p$ of the quantum channel fulfills $h(p)
  < h $ where $h(\cdot)$ denotes the binary Shannon-entropy
  function.  } \medskip

Although this result does not allow us to improve (compared to
unbounded adversaries) the maximum error-rate for the BB84 protocol
(the \index{protocol!4-state}4-state protocol), the
\index{protocol!6-state}6-state (using three mutually unbiased bases)
protocol can be shown secure against adversaries with memory bound
sublinear in the secret-key length as long as the bit-flip error-rate
is less than $17\%$. This improves over the maximal error-rate of
$13\%$ for this protocol against unbounded adversaries.  We also show
that the generalization of the 6-state protocol to more bases (not
necessarily mutually unbiased) can be shown secure for a maximal
error-rate up to $20\%$ provided the number of bases is large enough.
Note that the best known one-way protocol based on qubits is proven
secure against general attacks for an error-rate of only up to roughly
$14.1\%$, and the theoretical maximum is $16.3\%$~\cite{RGK05}.

The quantum-memory-bounded eavesdropper model studied here is not
comparable to other restrictions on adversaries considered in the
literature (e.g. \emph{individual attacks}, where the eavesdropper is
assumed to apply independent measurements to each qubit sent over the
quantum channel as considered by Fuchs, Gisin, Griffiths, Niu, Peres,
and L\"utkenhaus~\cite{FGGNP97,Lutkenhaus00}).  In fact, these
assumptions are generally artificial and their purpose is to simplify
security proofs rather than to relax the conditions on the quality of
the communication channel from which secure key can be generated.  We
believe that the quantum-memory-bounded eavesdropper model is more
realistic.

%% On the technical side, we derive a new type of uncertainty relation
%% involving the min-entropy of a quantum encoding
%% (Theorem~\ref{thm:hadamard}, and
%% Corollary~\ref{cor:hadamard}), which might be useful in other
%% contexts as well.  The new relation is then used in combination with a
%% proof technique by Shor and Preskill \cite{SP00}, where the actions of
%% honest players are purified, and with privacy amplification against
%% quantum adversaries as introduced by Renner and K\"onig~\cite{RK05, Renner05}.


\section{Outline of the Thesis}
In Chapter~\ref{chap:prelim}, we introduce notation and present some
basic concepts from probability and quantum information theory like
quantum states and various kinds of their entropies. We prepare the
stage by reproducing and slightly extending the results about privacy
amplification via two-universal hashing from Renner's PhD thesis \cite{Renner05}.

Chapter~\ref{chap:ClassicalOT} is the only (almost) exclusively
classical chapter. It introduces the different flavors of oblivious
transfer and gives a characterization of the security for the sender
of \OT in terms of non-degenerate linear functions. It is cast in a
stand-alone manner and the rest of the thesis can be understood
without reading this chapter.

In Chapter~\ref{chap:uncertrelations}, the basis for the security
proofs of the following chapters is laid by establishing the quantum
min-entropic uncertainty relations. The following
Chapters~\ref{chap:RabinOT} and \ref{chap:12OT} contain the quantum
definitions, protocols and security proofs for \RabinOT and \OT,
respectively. Chapter~\ref{chap:qbc} treats quantum bit commitment.
Two flavors of the ``binding property'' are defined and the techniques
from the two previous chapters are used to prove security in the
bounded-quantum-storage model.

Chapter~\ref{chap:qkd} is devoted to another application of the
(second) uncertainty relation, quantum key distribution against a
quantum-memory-bounded eavesdropper. The last
Chapter~\ref{chap:conclusions} addresses some practical issues in
greater detail and concludes.

A short summary of the notation, the bibliography and an index
can be found at the end of the thesis.

% Enjoy the ride!

\section{Related Work}
The classical bounded-storage model is described in
Section~\ref{sec:ClassicalBSMIntro}. Besides work pointed out
in the overview of the contributions in
Section~\ref{sec:contributions} above, it is worth mentioning that several protocols aiming at achieving quantum oblivious transfer have been proposed. After Wiesner's original conjugate-coding protocol~\cite{Wiesner83}, Bennett, Brassard, Cr\'epeau, and Skubiszewska proposed an interactive protocol for \OT~\cite{BBCS91}, whose security was subsequently analyzed by Cr\'epeau~\cite{Crepeau94}, Mayers, Salvail~\cite{MS94, Mayers95}, and Yao~\cite{Yao95}. The protocol from \cite{BBCS91} is interactive and can be easily broken by a dishonest receiver with unbounded quantum memory. To ensure that the receiver actually performs a measurement, it was suggested to use (quantum) bit-commitment schemes such as~\cite{BCJL93} which were believed to be secure against such adversaries at this point in time. After the impossibility proofs of quantum bit-commitment by Lo and Chau~\cite{LC97}, and Mayers~\cite{Mayers97}, and of oblivious transfer by Lo~\cite{Lo97}, it became clear that assumptions are necessary in order to securely realize these primitives. Compared to these previous attempts, the protocols in this thesis are simpler, non-interactive, and provably secure according to stronger security definitions.

Work related to classical OT-reductions is referred to in the introductory sections to
Chapter~\ref{chap:ClassicalOT} in Sections~\ref{sec:introtoNDLF}
and~~\ref{sec:reductions}.  Previous work about quantum uncertainty
relations is described in Section~\ref{sec:uncerthistory}.



%% \subsubsection{Classical Bounded-Storage Model}

%% \subsubsection{OT Reductions}
%% \subsubsection{Uncertainty Relations}
%% \subsubsection{QKD}


%% The history of uncertainty relations starts with Heisenberg who showed
%% that the outcomes of two non-commuting observables $A$ and $B$ applied
%% to any state $\rho$ are not easy to predict simultaneously.  However,
%% Heisenberg only speaks about the variance of the measurement results,
%% and his result was shown to have several shortcomings in
%% \cite{HU88,Deutsch83}.  More general forms of uncertainty relations
%% were proposed in \cite{BiMy75} and \cite{Deutsch83} to resolve these
%% problems.  The new relations were called {\em entropic uncertainty
%%   relations}, because they are expressed using Shannon entropy instead
%% of the statistical variance.
%% % Such relations are called {\em entropic uncertainty relations}. 
%% %In addition to evade the problems of (\ref{robertson}), 
%% Entropic uncertainty relations have the advantage of being 
%% pure information theoretic statements. The first 
%% entropic uncertainty relation was introduced by Deutsch\cite{Deutsch83}
%% and stated that
%% %\begin{equation}\label{deutsch}
%%  $\H(P)+\H(Q) \geq -2\log{\frac{1+c}{2}}$,
%% %\end{equation}
%% where $P,Q$ are random variables representing the measurement
%% results and  $c$ is the maximum inner product norm between any eigenvectors
%% of $A$ and $B$. 
%% %Notice that another advantage of (\ref{deutsch}) over
%% %Heisneberg's result is that its right-hand side does not
%% %depend on $\ket{\psi}$ which makes it truly general
%% %and non-trivial as long as $c<1$.
%% First conjectured
%% by Kraus\cite{Kraus87}, Maassen and Uffink\cite{MU88} improved
%% %(\ref{deutsch}) 
%% Deutsch's relation to the optimal
%% \begin{equation}\label{maassenuffink}
%% \H(P)+\H(Q) \geq -2\log{c}.
%% \end{equation}
%% %Relation (\ref{maassenuffink}) 
%% %has been used for locking/unlocking 
%% %classical correlations in quantum states\cite{DHLST03} and for the analysis
%% %of the resilience of quantum symmetric encryption schemes
%% %against known plaintext attacks\cite{DPS04}.

%% Although a bound on Shannon entropy can be helpful
%% in some cases, it is usually not good enough in cryptographic applications. 
%% The main tool
%% to reduce the adversary's information --
%%  privacy amplification\cite{BBR88,ILL89,BBCM95,RK05,Renner05} --
%% only works if a bound on the adversary's min-entropy (in fact
%% collision entropy) is known. 
%% Unfortunately, knowing the Shannon entropy of a distribution
%% does in general not allow
%% to bound its higher order R\'enyi entropies.   

%% An entropic uncertainty relation involving R\'enyi entropy of order
%% $2$ (i.e. {\em collision entropy}) was introduced by
%% Larsen\cite{Larsen90,Ruiz95}.  Larsen's relation quantifies precisely
%% the collision entropy for the set $\{A_i\}_{i=1}^{d+1}$ of \emph{all}
%% maximally non-commuting observables, where $d$ is the dimension of the
%% Hilbert space.  Its use is therefore restricted to quantum coding
%% schemes that take advantage of \emph{all} $d+1$ observables, i.e. to
%% schemes that are difficult to implement in practice.


%% %%%%%%%%%%%%%%%%%%%%%%%%%%%%%%%%%%%%%%%%%%%%%%%%%%%%%%%%%%%%%%%%%%%%%%%%%%%%%
%% \section{Introduction from NDLF paper}
%% %%%%%%%%%%%%%%%%%%%%%%%%%%%%%%%%%%%%%%%%%%%%%%%%%%%%%%%%%%%%%%%%%%%%%%%%%%%%%

%% 1-2 Oblivious Transfer, \OT\ for short, is a two-party primitive which
%% allows a sender to send two bits (or, more generally, strings) $B_0$
%% and $B_1$ to a receiver, who is allowed to learn one of the two
%% according his choice $C$. Informally, it is required that the receiver only
%% learns $B_C$ but not $B_{1-C}$ ({\em obliviousness}), while at the
%% same time the sender does not learn $C$ ({\em privacy}). \OT\ was
%% introduced in~\cite{Wiesner83} under the name of ``multiplexing'' in the
%% context of quantum cryptography, and, inspired by~\cite{Rabin81} where a
%% different flavor was introduced, later re-discovered in~\cite{EGL82}.

%% \OT\ turned out to be very powerful in that it was shown to be sufficient
%% for secure general two-party computation~\cite{Kilian88}. On the other
%% hand, it is quite easy to see that unconditionally secure \OT\ is not
%% possible without any assumption.  Even with the help of quantum
%% communication and computation, unconditionally secure \OT\ remains
%% impossible~\cite{LC97,Mayers97}.  As a consequence, much effort has been
%% put into constructing unconditionally secure protocols for \OT\ using
%% physical assumptions like various models for noisy
%% channels~\cite{CK88,DKS99,DFMS04,CMW04}, or a memory bounded
%% adversary~\cite{CCM98,Din01,DHRS04}.  Similarly, much effort has been
%% put into reducing \OT\ to (seemingly) weaker flavors of \pOT, like
%% \RabinOT, \XOT, etc.~\cite{Crepeau87,BC97,Cachin98,Wolf00,BCW03,CS06}.  

%% In this work, we focus on a slightly modified notion of \OT, which we
%% call {\em Randomized} \OT, \RandOT\ for short, where the bits (or
%% strings) $B_0$ and $B_1$ are not {\em in}put by the sender, but
%% generated uniformly at random during the \RandOT\ and then {\em
%%   out}put to the sender. It is still required that the receiver only
%% learns the bit (or string) of his choice, $B_C$, whereas the sender
%% does not learn any information on $C$. It is obvious that a \RandOT\ can easily be turned
%% into an ordinary \OT\, simply by using the generated $B_0$ and $B_1$
%% to mask the actual input bits (or strings). Furthermore, all known
%% constructions of unconditionally secure \OT\ protocols make
%% implicitly the detour via \RandOT. 

%% In a first step, we observe that the obliviousness condition of a
%% \RandOT\ of {\em bits} is equivalent to requiring the XOR $B_0 \oplus
%% B_1$ to be (close to) uniformly distributed from the receiver's point
%% of view. The proof is very simple, and it is kind of surprising
%% that---to the best of our knowledge---this has not been realized
%% before. We then ask and answer the question whether there is a natural
%% generalization of this result to \RandOT\ of {\em strings}. Note that
%% requiring the bitwise XOR of the two strings to be uniformly
%% distributed is obviously not sufficient. We show that the
%% obliviousness condition for \RandOT\ of strings can be characterized
%% in terms of {\em non-degenerate linear functions} (bivariate binary
%% linear functions which non-trivially depend on both arguments, as
%% defined in Definition~\ref{def:linear}): obliviousness holds if and
%% only if the result of applying any non-degenerate linear function to
%% the two strings is (close to) uniformly distributed from the
%% receiver's point of view.

%% We then show the usefulness of this new understanding of \OT. We
%% demonstrate this on the problem of reducing \OT\ to weaker primitives.
%% Concretely, we show that the reducibility of an ordinary \OT\ to
%% weaker flavors via a non-interactive reduction follows by a trivial
%% argument from our characterization of the obliviousness condition.
%% This is in sharp contrast to the current literature: The proofs given
%% in~\cite{BC97,Wolf00,BCW03} for reducing \OT\ to \XOT, \GOT\ and \BUOT\ 
%% (we refer to Section~\ref{sec:application} for a description of these
%% flavors of \pOT) are rather complicated and tailored to a particular
%% class of privacy-amplifying hash functions; whether the reductions
%% also work for a less restricted class is left as an open
%% problem~\cite[page~222]{BCW03}. And, the proof given in~\cite{Cachin98}
%% for reducing \OT\ to one execution of a general \pUOT\ is not only
%% complicated, but also incorrect, as we will point out.  Thus, our
%% characterization of the obliviousness condition allows to simplify
%% existing reducibility proofs and, along the way, to solve the open
%% problem posed in~\cite{BCW03}, as well as to improve the reduction
%% parameters in most cases, but it also allows for new, respectively
%% until now only incorrectly proven reductions.  Furthermore, our
%% techniques may be useful for the construction and analysis of \OT\ 
%% protocols in other settings, for instance in a quantum setting as
%% demonstrated in~\cite{DFRSS06}, or for computationally secure \OT\ 
%% with unconditional obliviousness.

%% Finally, we extend our result and show how our characterization of
%% \RandOT\ in terms of non-degenerate linear functions translates
%% to \onenOT\ and to \OT in a \emph{quantum setting}. 





%% %% First things to try: restricting players memory or time.




%% %% (but by any mean does not learn anything about the other bit)


%% %% two primitives Bit Commitment
%% %% (BC) and Oblivious Transfer (OT) are the basic building blocks of
%% %% almost all





%% %% This thesis is only concerned with information-theoretical security,
%% %% the strongest reasonable security notion which means that adversarial
%% %% players have 



%% %% In cryptography, we want to assure that it is
%% %% impossible to ``cheat''. We call a protocol secure when there is a
%% %% guarantee for all players that honestly follow the protocol that the
%% %% dishonest players 



%% %% One art of cryptography is to come up with cryptographic protocols



%% %% This thesis opens up a new line of research by introducing a new
%% %% cryptographic model called the bounded-quantum-storage model.





%% \section{Intro from Rabin \pOT paper}
%% It is well known that non-trivial two-party cryptographic primitives
%% cannot be securely implemented if only error-free communication is
%% available and there is no limitation assumed on the computing power
%% and memory of the players. Fundamental examples of such primitives are
%% bit commitment (\BC) and oblivious transfer (\pOT). In \BC, a committer $\C$
%% commits himself to a choice of a bit $b$ by exchanging information
%% with a verifier $\V$. We want that $\V$ does not learn $b$ (we say the
%% commitment is hiding), yet $\C$ can later choose to reveal $b$ in a
%% convincing way, i.e., only the value fixed at commitment time will be
%% accepted by $\V$ (we say the commitment is binding). In (Rabin) OT, 
%% a sender $\A$
%% sends a bit $b$ to a receiver $\B$ by executing some protocol in such a way that
%% $\B$ receives $b$ with probability $\frac12$ and nothing with probability
%% $\frac12$, yet $\A$ does not learn what was received.


%% Informally, \BC is not possible with unconditional security since
%% hiding means that when 0 is committed, exactly the same information
%% exchange could have happened when committing to a 1. Hence, even if 0
%% was actually committed to, $\C$ could always compute a complete view of
%% the protocol consistent with having committed to 1, and pretend that
%% this was what he had in mind originally. A similar type of argument
%% shows that OT is also impossible in this setting.

%% One might hope that allowing the protocol to make use of quantum
%% communication would make a difference. Here, information is stored in
%% qubits, i.e., in the state of two-level quantum mechanical systems,
%% such as the polarization state of a single photon. It is well known
%% that quantum information behaves in a way that is fundamentally
%% different from classical information, enabling, for instance,
%% unconditionally secure key exchange between two honest players.
%% However, in the case of two mutually distrusting parties, we are not
%% so fortunate: even with quantum communication, unconditionally secure
%% \BC and \pOT remain impossible~\cite{LC97,Mayers97}.







%%% Local Variables: 
%%% mode: latex
%%% TeX-master: "diss"
%%% End: 


