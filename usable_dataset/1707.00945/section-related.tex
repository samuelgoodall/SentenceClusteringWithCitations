
\section{Related Work}
\label{sec:related-work}

Only a small number of experience reports about SPARK~2014 have been
published before. A look back at (old) SPARK's history
and its success, as well as an initial picture of SPARK~2014 is given
by Chapman and Schanda in~\cite{Chapman2014}. We can report that the
mentioned difficulties with floating-point numbers are solved in SPARK~2014, 
and that the goal to make verification more accessible, has been reached.  A small case study with SPARK~2014
is presented in~\cite{Trojanek2014}, but at that point multi-threading
(Ravenscar) was not yet supported, and floating point numbers have
been skipped in the proof. We can add to the conclusion given there,
that both are easily verified in ``real-world'' code, although
float proofs require more (computational and mental) effort.  Larger
case studies are summarized by Dross et al. in \cite{Dross2014}, with
whom we share the opinion of minor usability issues, and that some
small amount of developer training is required.  Finally, SPARK~2014
with Ravenscar has recently been announced to be used in the Lunar
IceCube~\cite{Brandon2016} satellite, a successor of the successful CubeSat project that was implemented in SPARK 2005. It will be a message-centric software, conceptually similar to NASA's cFE/CFS, but 
fully verified and striving to become an open source platform for spacecraft software.
In contrast to all the above publications, this paper is not focused on the application or case studies, but  pointing out typical sources of errors in SPARK programs, which a developer has to
know in order to get correct verification results.

%%% Local Variables: ***
%%% mode:latex ***
%%% TeX-master: "paper.tex"  ***
%%% End: ***