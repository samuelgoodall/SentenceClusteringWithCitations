%%%%%%%%%%%%%%%%%%%%%%%%%%%%%%%%%%%%%%%%%%%%%%%%%%%%%%%%%%%%%%%%%%%%%%%%%%%%%
%% Front pages
%%%%%%%%%%%%%%%%%%%%%%%%%%%%%%%%%%%%%%%%%%%%%%%%%%%%%%%%%%%%%%%%%%%%%%%%%%%%%

\vspace*{\fill}
\begin{flushright}
  {\Huge\sf Cryptography in the Bounded-Quantum-Storage Model}\\[3ex]
  {\huge\sf Christian Schaffner} 
\end{flushright}
\noindent\rule{\linewidth}{1mm}\\[-.5ex]
\noindent\rule{\linewidth}{2.5mm}
\vfill
\begin{center}
  {\huge\sf PhD Dissertation}\\[\fill]
  \includegraphics{./figures/au-segl}\\[\fill]
  {\sf BRICS Research School\\DAIMI -- Department of Computer Science\\University of Aarhus\\Denmark}
\end{center}
\vspace*{\fill}
\cleardoublepage
\begin{center}
  \vspace*{\stretch{1}}
  {\huge Cryptography in the\\[3mm]
Bounded-Quantum-Storage Model}\\[\fill]
  A Dissertation\\
  Presented to the Faculty of Science\\
  of the University of Aarhus\\
  in Partial Fulfillment of the Requirements for the\\
  PhD Degree\\[\stretch{2}]
  by\\
  Christian Schaffner\\
  official version submitted: March 2, 2007\\[5mm]

  final version: \makeatletter\@date\makeatother\\
\end{center}
\vspace*{\stretch{1}}

%%%%%%%%%%%%%%%%%%%%%%%%%%%%%%%%%%%%%%%%%%%%%%%%%%%%%%%%%%%%%%%%%%%%%%%%%%%%%
%% Abstract
%%%%%%%%%%%%%%%%%%%%%%%%%%%%%%%%%%%%%%%%%%%%%%%%%%%%%%%%%%%%%%%%%%%%%%%%%%%%%

\clearemptydoublepage
\pagestyle{plain}
\phantomsection
\addcontentsline{toc}{chapter}{Abstract}
\chapter*{{\Huge Abstract}}

Cryptographic primitives such as oblivious transfer and bit commitment
are impossible to realize if unconditional security is required against
adversaries who are unbounded in running time and memory size.
Therefore, it is a great challenge to come up with restrictions on the
adversary's capabilities such that on one hand interesting
cryptographic primitives become possible, but on the other hand the
model is still realistic and as close to practice as possible.

The \emph{bounded-quantum-storage model} is a prime example of
such a cryptographic model. In this thesis, we initiate the study of
cryptographic primitives with unconditional security under the sole
assumption that the adversary's {\em quantum} memory is of bounded
size.

Oblivious transfer and bit commitment can be implemented in this model
using protocols where honest parties need no quantum memory, whereas
an adversarial player needs to store \emph{at least a large fraction} of the
total number of transmitted qubits in order to break the protocol.
This is in sharp contrast to the classical bounded-memory model, where
we can only tolerate adversaries with memory of size polynomially
larger than the honest players' memory size.

On the practical side, our protocols are efficient, non-interactive
and can be adapted to cope with various kinds of noise in the
transmission. In fact, they can be \emph{implemented using today's
technology}. 

On the theoretical side, new \emph{entropic uncertainty relations}
involving min-entropy are established and used to prove the security
of protocols in the bounded-quantum-storage model according to new
strong security definitions. The uncertainty relations lower bound
the min-entropy of the encoding used in most quantum-cryptographic
protocols and therefore contribute to the understanding of the quantum
effects which these protocols are based upon. The most direct way to
make use of these lower bounds is by assuming a quantum-memory bound on
the adversary. For instance, in the realistic setting of \emph{Quantum
  Key Distribution (\QKD)} against quantum-memory-bounded
eavesdroppers, the uncertainty relation allows to prove the security
of QKD protocols while tolerating considerably higher error rates
compared to the standard model with unbounded adversaries. %%  For
%% instance, for the six-state protocol with one-way communication, a
%% bit-flip error rate of up to 17\% can be tolerated (compared to 13\%
%% in the standard model).

\vspace{2mm} In addition, though not directly related to the
bounded-quantum-storage model, a classical result about
unconditionally secure 1-out-of-2 Oblivious Transfer (\OT) is
obtained. It is pointed out that the standard security requirement for
\OT\ of bits, namely that the receiver only learns one of the bits
sent, holds if and only if the receiver has no information on the XOR
of the two bits. This result generalizes to \OT\ of strings, in which
case the security can be characterized in terms of \emph{binary linear
  functions}.  More precisely, it is shown that the receiver learns
only one of the two strings sent, if and only if he has no information
on the result of applying any binary linear function which
non-trivially depends on both inputs to the two strings. This result
not only gives new insight into the nature of \OT, but it in
particular provides a \emph{powerful tool for analyzing \OT\ 
  protocols}. With this characterization at hand, the reducibility of
\OT\ of strings to a wide range of weaker primitives follows by a very
simple argument.



%%%%%%%%%%%%%%%%%%%%%%%%%%%%%%%%%%%%%%%%%%%%%%%%%%%%%%%%%%%%%%%%%%%%%%%%%%%%%
%% Acknowledgements
%%%%%%%%%%%%%%%%%%%%%%%%%%%%%%%%%%%%%%%%%%%%%%%%%%%%%%%%%%%%%%%%%%%%%%%%%%%%%

\clearemptydoublepage
\phantomsection
\addcontentsline{toc}{chapter}{Acknowledgements}
\chapter*{{\Huge Acknowledgements}}

I am grateful to everyone who helped and supported me during my PhD
studies here in {\AA}rhus. 

First of all, I want to cordially thank my supervisors and co-authors
Louis Salvail and Ivan Damg{\aa}rd and the whole cryptology group at
DAIMI for providing an excellent environment for cryptographic
research. Countless are the hours I have spent discussing scientific
as well as non-scientific issues with Louis, \emph{merci beaucoup}! I
thank my other co-authors Claude Cr\'epeau, Serge Fehr, Renato Renner,
George Savvides and J\"urg Wullschleger for many inspiring visits and
discussions.

I appreciated very much being a PhD student in a well-organized and
well-funded research group and to be able to work in a brand-new building
with plenty of space, great infrastructure and always helpful and
friendly staff and secretaries: Ellen, Hanne, Karen, Lene,
Michael, and Uffe.

Studying in {\AA}rhus has been a great experience mainly because of
all the friends from the constantly changing ``gang'' of foreign and
Danish fellows at DAIMI including Allan, Claudio, Claus, Doina, Gabi,
Henrik, Jan, Jesper, Jooyong, Johan, Kevin, Michael, Mikkel, Mirka,
Rune, Tord, Thomas M, Tomas, and Troels; but not to forget the ones
who have left Denmark and are now spread around the world:
Barnie, Christopher, Emanuela and Paolo, Fitzi, Gosia and Darek, Jens,
Jes\'us, Karl, Kirill, Marco, Nelly and Antonio, Philipp, Thomas P, and
Saurabh. I thank you all for the wonderful time, both at and off the
table-soccer table. Special thanks to
Gosia and Henrik for constructive comments on the introduction of this
thesis and to J\"urg and Serge for further comments.

I would also like to thank Claude Cr\'epeau for hosting me for a
fantastic summer half-year at McGill university in Montr\'eal where I
had the chance to meet many interesting people doing quantum research
and experience the exciting spot where the francophone part of North
America meets the anglophone rest of the continent.

I thank Prof.~Andreas Winter from the University of Bristol and
Prof.~Stefan Wolf from ETH Z\"urich as well as Prof.~Susanne
B{\o}dker from the University of Aarhus for agreeing to constitute the
evaluation committee for my PhD thesis.

Last but not least, I want to express my gratitude to my family for
their immense love and support from the distance. I am infinitely
grateful for the great childhood they gave me which was and still is an
invaluable source of self-confidence for me.

\vspace{8mm}
This research was partially supported by the EU Project SECOQC,
No:~FP6-2002-IST-1-506813.

\vspace{2ex}
\begin{flushright}
  \emph{Christian Schaffner,}\\
  \emph{{\AA}rhus, March 2, 2007.}
\end{flushright}

%%% Local Variables: 
%%% mode: latex
%%% TeX-master: "diss"
%%% End: 
