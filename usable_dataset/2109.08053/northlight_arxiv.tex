\documentclass[conference]{IEEEtran}
\IEEEoverridecommandlockouts
% The preceding line is only needed to identify funding in the first footnote. If that is unneeded, please comment it out.
\usepackage{cite}
\usepackage{amsmath,amssymb,amsfonts}
\usepackage{algorithmic}
\usepackage{graphicx}
\usepackage{textcomp}
\usepackage{xcolor}
\usepackage{url}
\usepackage{subfig}
%\usepackage{subcaption}
\usepackage{tikz}
\usetikzlibrary{patterns}
\usepackage{gensymb}
\usepackage{listings}
\usepackage{xcolor}
\usepackage{array}
\usepackage{bm}
\usepackage{colortbl}
\usepackage{wrapfig}


\newcommand{\system}{Northlight}

\newcommand{\smalltt}[1]{{\texttt{\small #1}}}
\newcommand{\scriptsizett}[1]{{\texttt{\scriptsize #1}}}
\newtheorem{definition}{Definition}

\def\BibTeX{{\rm B\kern-.05em{\sc i\kern-.025em b}\kern-.08em
    T\kern-.1667em\lower.7ex\hbox{E}\kern-.125em}}

\lstdefinestyle{sql}{
  frame=tb,
  language=SQL,
  morekeywords={extension},
  deletendkeywords={TIME},
  aboveskip=2mm,
  belowskip=2mm,
  captionpos=b,
  showstringspaces=false,
  columns=flexible,
  basicstyle={\footnotesize\ttfamily},
  numbers=none,
  numberstyle=\small\color{gray},
  keywordstyle=\color{blue},
  commentstyle=\color{magenta},
  frame=none,
  breaklines=true,
  breakatwhitespace=true,
  tabsize=3,
}

\lstdefinestyle{pseudo}{
  frame=tb,
  language={scala},
  deletekeywords={with},
  aboveskip=2mm,
  belowskip=2mm,
  captionpos=b,
  showstringspaces=false,
  columns=flexible,
  basicstyle={\scriptsize\ttfamily},
  numbers=left,
  numberstyle=\tiny \color{black},
  keywordstyle=\color{blue},
  commentstyle=\color{magenta},
  frame=none,
  breaklines=true,
  breakatwhitespace=true,
  tabsize=3,
}

% !TEX root = my-thesis.tex

%%% QUICK GUIDE
%%%   \ref{sec:intro:structure}
%%%   \label{sec:intro:structure}
%%%   \cite{Franklin:1993}

%%% LEERZEICHENSETZUNG
%%%   \, hinter Komma
%%%   \, nach unärem +, -, \pm
%%%   \, vor und nach Summen und Integralen
%%%   \quad zwischen Formel und Text

\newcommand*{\todo}[1]{\textcolor{red}{[#1]}}

\newcommand*{\wrdn}[1]{\textcolor{teal}{[#1]}}

\newcommand*{\cross}{\ensuremath{\times}}

\newcommand*{\mr}{MapReduce}
\newcommand*{\kv}{key-value}
\newcommand*{\io}{I/O}

\newcommand*{\code}[1]{\texttt{\detokenize{#1}}}
\renewcommand{\emph}[1]{\textit{#1}}

\newcommand*{\reals}{\ensuremath{\mathbb{R}}}
\newcommand*{\naturals}{\ensuremath{\mathbb{N}}}


\definecolor{jgu}{RGB}{193,0,42}
\definecolor{grayshade}{RGB}{180,180,180}
\definecolor{lightgrayshade}{RGB}{230,230,230}


\makeatletter
\newcommand{\linebreakand}{%
  \end{@IEEEauthorhalign}
  \hfill\mbox{}\par
  \mbox{}\hfill\begin{@IEEEauthorhalign}
}
\makeatother



\title{\system{}: Declarative and Optimized Analysis of Atmospheric Datasets in SparkSQL}

\author{\IEEEauthorblockN{Justus Henneberg}
\IEEEauthorblockA{\textit{Institute of Computer Science} \\
\textit{Johannes Gutenberg-University}\\
Mainz, Germany\\
henneberg@uni-mainz.de}
\and
\IEEEauthorblockN{Felix Schuhknecht}
\IEEEauthorblockA{\textit{Institute of Computer Science} \\
\textit{Johannes Gutenberg-University}\\
Mainz, Germany \\
schuhknecht@uni-mainz.de}
\and 
\IEEEauthorblockN{Philipp Reutter}
\IEEEauthorblockA{\textit{Institute of Atmospheric Physics} \\
\textit{Johannes Gutenberg-University}\\
Mainz, Germany \\
preutter@uni-mainz.de}
\linebreakand
\IEEEauthorblockN{Nils Brast}
\IEEEauthorblockA{\textit{Institute of Atmospheric Physics} \\
\textit{Johannes Gutenberg-University}\\
Mainz, Germany \\
nibrast@uni-mainz.de}
\and
\IEEEauthorblockN{Peter Spichtinger}
\IEEEauthorblockA{\textit{Institute of Atmospheric Physics} \\
\textit{Johannes Gutenberg-University}\\
Mainz, Germany \\
spichtin@uni-mainz.de}
}

\hyphenation{SparkSQL}

\begin{document}

\maketitle

\begin{abstract}

Performing data-intensive analytics is an essential part of modern Earth science. As such, research in atmospheric physics and meteorology frequently requires the processing of very large observational and/or modeled datasets. Typically, these datasets (a)~have high dimensionality, i.e. contain various measurements per spatiotemporal point, (b)~are extremely large, containing observations over a long time period. Additionally, (c)~the analytical tasks being performed on these datasets are structurally complex. 

Over the years, the binary format NetCDF has been established as a de-facto standard in distributing and exchanging such multi-dimensional datasets in the Earth science community -- along with tools and APIs to visualize, process, and generate them. Unfortunately, these access methods typically lack either (1)~an easy-to-use but rich query interface or (2)~an automatic optimization pipeline tailored towards the specialities of these datasets. 
%Additionally, (3)~the handling of heterogeneous datasets often requires challenging manual adjustments and conversions before the actual processing can start. 
As such, researchers from the field of Earth sciences (which are typically not computer scientists) unnecessarily struggle in efficiently working with these datasets on a daily basis. 

Consequently, in this work, we aim at resolving the aforementioned issues. Instead of proposing yet another specialized tool and interface to work with atmospheric datasets, we integrate sophisticated NetCDF processing capabilities into the established SparkSQL dataflow engine -- resulting in our system~\system{}. In contrast to comparable systems, \system{} introduces a set of fully automatic optimizations specifically tailored towards NetCDF processing. We experimentally show that \system{} scales gracefully with the selectivity of the analysis tasks and outperforms the comparable state-of-the-art pipeline by up to a factor of~6x.
%Further, we introduce mechanism to handle the omnipresent heterogeneity within the datasets in an on-the-fly fashion. 
\end{abstract}

\begin{IEEEkeywords}
NetCDF, SparkSQL, Query Optimization, Earth Science, Atmospheric Physics
\end{IEEEkeywords}

% Introduction
\section{Introduction}
\label{sec:intro}

% Volume
Modern research in the field of Earth science has to process large volumes of observational/modeled data. Also, the joint processing and comparison of data from different sources is a common task in Earth science research~\cite{lit:rhi, Gierens:1999tf}, further increasing the total amount of data to process. For instance, in the field of atmospheric physics and meteorology, researchers perform an analysis on the ERA5~\cite{lit:ecmwf-era5,lit:era5} dataset provided by the European Centre for Medium-Range Weather Forecasts, which contains around 5~PB of multi-dimensional climate data estimates since~$1979$. Processing such a large and potentially distributed dataset efficiently is a non-trivial task.
%in particular, as the carried out scientific analysis is also structurally complex.  
%Another challenge besides the sheer volume of the data is that the faced datasets are often \textit{heterogeneous}. 
Further, the datasets are typically split across a large number of files, where certain dimensions might span over multiple files, while others remain consistent for each file.  
%contain gaps, or use a different coordinate system. Classical pipelines require either the manual cleaning and conversion of these datasets to generate uniformity or a careful adjustment of the query to handle these inconsistencies. 
Handling the aforementioned challenges manually is especially undesirable for users who originate from a domain other than computer science. Researchers from natural sciences, who want to focus on their actual task, are forced to think about efficient data management and how to carefully optimize complex query plans in order to achieve acceptable runtimes. 

\subsection{\system{}}

As a consequence, in the following work we propose the system~\system{}, which combines an easy-to-use declarative query interface with high processing performance, specifically tailored for atmospheric datasets. 

For decades, the distribution of observational/modeled data has been a cornerstone in atmospheric physics, meteorology and weather prediction, using the \textit{NetCDF}~\cite{lit:netcdf} format as a de-facto standard for more than $30$~years. Thus, we focus on datasets materialized in the NetCDF format in the following.
At the core, it is a self-describing, multi-dimensional binary format with a focus on portability and scalability. Interestingly, while there are plenty of low-level APIs available for creating, modifying, and accessing NetCDF files, there is hardly any NetCDF-capable processing system with a rich query interface that combines convenience with performance. This is especially surprising considering the typical userbase of the format.

Instead of proposing yet another specialized tool for NetCDF processing that requires the users to adopt to, we integrate NetCDF support into the well known \textit{SparkSQL}\cite{lit:spark-sql,lit:spark-sql-diss} framework. SparkSQL provides exactly what researchers from the Earth science community require: On the one hand, they need an abstraction layer which allows to easily connect NetCDF datasets and formulate analysis tasks in a simple declarative language such as SQL. On the other hand, they can be sure that their query is automatically and transparently optimized into an efficient parallel and distributed execution plan. 

In summary, our work makes the following contributions:

\subsection{Contributions}

% Core purpose
\textbf{(1)~Atmospheric Datasets in SparkSQL}: We integrate rich support for processing distributed NetCDF datasets into SparkSQL. Users are able to formulate queries against a convenient row-wise representation of their NetCDF dataset using SQL. \system{} ensures that the formulated query is automatically and transparently optimized down to the data source. \system{} seamlessly extends an existing Spark\cite{lit:spark,lit:spark-diss} installation -- no modifications to the vanilla Spark code or the optimization pipeline is required. This allows an easy installation and adoption of our techniques for users of both Spark and SparkSQL. The full source code of the system will be available upon acceptance. 

% Optimizations
\noindent \textbf{(2)~Transparent Query Optimization}: We propose several fully automatic data management optimizations to let the SparkSQL pipeline efficiently handle the characteristics of NetCDF datasets. These optimizations include vertical and horizontal pruning, which happens directly at the data source. We do so without precomputing any auxiliary index structure. 

\noindent \textbf{(3)~Optimizing Non-Convex Predicates}: Besides supporting the optimization of simple convex predicates, like done in comparable systems, we also target the optimization of complex non-convex predicates. These consist, for instance, of several conjuncted areas that might contain holes or overlap with each other. Our optimizations ensure that such non-convex predicates, which occur frequently in atmospheric analysis tasks, still result in loading the minimum amount of required data from the data source.

\noindent \textbf{(4)~Optimizing Joins via Envelopes}: Further, we support a transparent optimization of certain join queries via the introduction of envelopes. Envelopes are automatically computed lower and upper bounds on the coordinate axes. These are injected into SparkSQL's optimizer without changing the optimization pipeline in any way and can drastically speed up joins on coordinates. 

% Concrete Applications and Evaluation
\noindent \textbf{(5)~Evaluation of Real-world Applications}: We showcase how a set of real-world computations from the domain of atmospheric physics can be expressed and evaluated on real-world observational/modeled datasets within our framework. We show that \system{} gracefully scales with the selectivity of the workload for various workloads and optimizes join processing via envelopes. We compare our optimization pipeline to the one of ClimateSpark~\cite{lit:climatespark} and show that we optimize deeper without requiring any auxiliary data structures. 





%In~\cite{Reutter}, this ERA5 dataset is analyzed in combination with the IAGOS dataset, which contains observations collected on more than $60{,}000$~flights, to assess the accuracy of the ERA5 dataset.
%Obviously, processing such volumes of potentially heterogeneous data in order to solve complex tasks is highly challenging. 

% Hard for non-CS people
%As the researchers, who want to perform such analysis, are typically from a different domain than computer science, the aforementioned complexity hurts even more. While tools such as Panoply~\cite{panoply} to inspect and to visualize NetCDF datasets exist, only 


% Low-level APIs (Unidata Java/C/Fortran)
% CDO Climate Data Operators (limitations?)
% Visualization (Panoply)


This paper is structured as follows: In Section~\ref{sec:background}, we discuss the required background, such as the structure of the NetCDF format, as well as the related work. The related work also motivates and steers our overall architectural design, which we will describe in Section~\ref{sec:architecture}. Based on that, in Section~\ref{ssec:convex_query_processing}, we discuss how query processing and optimization work when facing convex predicates, whereas in Section~\ref{ssec:non_convex_query_processing}, we discuss the significantly more complex non-convex case. In Section~\ref{sec:envelopes}, we target the optimization of specific joins using envelopes. In Section~\ref{sec:experiments}, we conclude with an extensive experimental evaluation. 

\section{Background}
\label{sec:background}

To motivate the approach we are taking, in the following we first briefly introduce the NetCDF file format, then present state-of-the-art tools/frameworks/systems and also discuss their limitations.

\subsection{The NetCDF File Format}
\label{sec:system:problem:netcdf}

As mentioned before, we target the efficient processing of \textit{NetCDF}~\cite{lit:netcdf} datasets. This self-describing columnar file format for multidimensional data has emerged as the de-facto standard format used by the Earth science research community to materialize and distribute atmospheric datasets, along with an extensive list of metadata conventions\cite{lit:convention,lit:convention-monotonic}.
A NetCDF file consists of multiple \textit{dimensions} and \textit{variables}. A variable stores an observation for each point on a fixed multi-dimensional grid. For example, our local surface-level subset of the ERA5 dataset consists of the three dimensions time, longitude, and latitude, and each of the $19$~variables maps instances of the tuple (\smalltt{time}, \smalltt{lon}, \smalltt{lat}) to the observation at these coordinates. Figure~\ref{fig:netcdf_viz} visualizes the variable \smalltt{sp} (surface pressure) at a single point in time along varying longitude and latitude. Additionally, each dimension can be accompanied by a so-called \textit{coordinate variable}. This coordinate variable describes the coordinate values of the dimension, e.g. that the \smalltt{lat} dimension runs from $-90\degree$ to $90\degree$ in quarter-degree steps.
Apart from that, NetCDF datasets are typically materialized as a set of individual files that must be processed conjunctively. 

\begin{figure}[h!]
\includegraphics[width=\columnwidth]{netcdf_sp_viz.png}
\caption{Visualization of the \smalltt{sp} variable (surface pressure) of the ERA5 surface data subset in Panoply~\cite{lit:panoply}.}
\label{fig:netcdf_viz}
\end{figure}

Processing these datasets manually using the available low-level APIs~\cite{lit:netcdf-java,lit:netcdf-python,lit:xarray} imposes various difficulties on the user, in particular, if the user originates from a domain other than computer science. As a consequence, the community asks for frameworks that allow formulating complex queries in a declarative manner while ensuring the best possible performance during evaluation -- \system{} is our answer.

\subsection{Related Work}

While most NetCDF tools focus on small-scale data visualization~\cite{lit:panoply,lit:met3d,lit:cistools,lit:ncview} or conversion~\cite{lit:cdo}, a few existing tools support declarative queries against NetCDF datasets. Consequently, we discuss these in the following, as they also motivate our design in various stages: 

First of all, \emph{SciSpark}\cite{lit:scispark} extends the Spark framework with an RDD\cite{lit:rdd} implementation for multidimensional datasets. However, while SciSpark enables the distributed and parallel processing of NetCDF files, it lacks two crucial requirements: First, albeit mentioning Shark\cite{lit:shark} integration, there is no indication that SciSpark provides a high-level query interface. Therefore, the user is required to formulate the query against Spark's RDD interface, which requires deeper programming knowledge. Second, it does not apply any automatic optimizations, such as query rewriting, to the processing pipeline. This results in poor performance~\cite{lit:climatespark}, especially for selective queries. 

\textit{ClimateSpark}~\cite{lit:climatespark} goes one step further and integrates the support for processing multidimensional files into SparkSQL. Consequently, it utilizes the provided SQL interface and, at first glance, allows the convenient formulation of declarative queries that can be executed efficiently. However, a closer look reveals that various limitations are still imposed by the system, which drastically limit its usability: 
First, the provided SQL interface is restricted in several ways: Queries must factor in the reconstruction of actual tuples from the dimensions and variables of the multidimensional dataset. Furthermore, ClimateSpark only supports simple predicates that select a single convex region from the dataset. Frequently required compound predicates such as \smalltt{((lon > 0.0 AND lon < 20.0) OR (lon > 60.0 AND lon < 80.0))} either result in data being loaded unnecessarily or require a decomposition of the query at RDD level. In addition, overlapping conjunctive predicates cannot be handled at all by the engine.
Second, to speed up processing, ClimateSpark relies on a spatiotemporal index, which is materialized in a separate database. This index must be created for each variable of interest in advance to allow the pruning of data during loading. 
Third, ClimateSpark imposes certain requirements on the dataset: For example, only datasets that materialize one point in time per file are supported -- other datasets first require a manual conversion. 

\section{\system{}: Architectural Overview}
\label{sec:architecture}

In summary, the state-of-the-art systems do not satisfy the requirements of the community: We want to be able to (a)~formulate SQL queries against a convenient row-wise representation of the multidimensional dataset. (b)~We want to be able to use arbitrarily complex and potentially non-convex predicates. Still, the system should ensure to load the minimal amount of data from the data source.
(c)~We want to enable querying and optimization without any preprocessing of data. 
%(d)~We want to support a large variety of structurally heterogeneous NetCDF datasets.

%\subsection{Overview}
%\todo{High-level figure of the architecture of \system{}. In particular, show which parts come from SparkSQL and which parts come from us.} 

\subsection{Communicating with the NetCDF API}
\label{ssec:netcdf_api}

We utilize the official UCAR/Unidata NetCDF-Java library~\cite{lit:netcdf-java} to access NetCDF files. This library is based on Unidata's \textit{Common Data Model}\cite{lit:cdm} and can therefore handle both NetCDF-3 and NetCDF-4 files. To enable support for files stored in HDFS\cite{lit:hadoop-hdfs}, we open the files through a wrapper class\cite{lit:scispark-github} from SciSpark.
The NetCDF-Java library allows us to perform so-called \textit{subarray lookups} in a $d$-dimensional NetCDF file: By providing a multidimensional starting point along the dimensions, which is called the \textit{origin}~$o$, and a multidimensional length, called the \textit{extent}~$e$, the API returns a (multidimensional) subarray for a variable of interest.

It is important to note that a subarray lookup \textit{cannot} be done by passing concrete dimension values to the API, such as $(\smalltt{time}=01.01.2020, \smalltt{lon}=45.0\degree, \smalltt{lat}=12.25\degree)$. Instead, the API must receive a \textit{positional tuple} as the origin. For example, the positional tuple~$(\smalltt{time}=0, \smalltt{lon}=180, \smalltt{lat}=49)$ would refer to position~$0$ along the \smalltt{time} dimension, position~$180$ along the \smalltt{lon} dimension, and position $49$ along the \smalltt{lat} dimension. Accordingly, the extent describes the positional length for each dimension.
Together, origin and extent form a $d$-dimensional \textit{block}.

\subsection{Connecting NetCDF Datasets to SparkSQL}
With sufficient knowledge about the underlying API, let us now see how to connect a new NetCDF dataset to \system{}.
Essentially, the user only has to provide two things: 

\noindent \textbf{1)}~A set of NetCDF filenames, potentially distributed, representing the dataset to connect.

\noindent \textbf{2)}~For each dimension of the dataset, the information whether the dimension spans over multiple files or not. This information will become relevant to efficiently query the dataset. For example, in the ERA5 dataset, the \smalltt{time} dimension spans over multiple files: Each file contains exactly one point in time, representing a single hour. In contrast to that, the \smalltt{lon} and \smalltt{lat} dimensions do not span over multiple files: Each file contains the full $180\degree$~latitude and $360\degree$~longitude information. 

The schema of the dataset is automatically inferred by \system{}: We extract schema information about all dimensions and variables from the first file of the dataset, assuming the schema is shared by all files. The user can also specify a schema manually. 
Apart from schema inference, we also take into account how files are distributed across nodes. As a result, \system{} will allocate one RDD for the entire dataset, which contains one partition for each file. During execution, the scheduler of Spark will utilize this information when distributing tasks among the nodes of the cluster to execute them as locally as possible.

\subsection{NetCDF $\rightarrow$ RDD $\rightarrow$ Relation}
\label{ssec:netcdf_integration}

To represent our dataset internally, we utilize Spark's \textit{RDD}\cite{lit:rdd} as well as the \textit{Relations} concept from SparkSQL.
We use the RDD to load the dataset from disk and to convert it into a row-wise representation. As we expose one tuple per data point, users can conveniently operate on the dataset through the \textit{Dataset API}\cite{lit:datasets}.

The created row-wise representation follows the schema visualized in Table~\ref{table:schema}, using the ERA5 surface subset as an example: First, we expose the name of the file from which the data of the tuple originates. Then, for each dimension spanning multiple files, we expose a column with the dimension values (\smalltt{time}). Note that \system{} automatically converts the internal representation of \smalltt{time} into a user-friendly representation. For each non-spanning dimension, we expose a column with the dimension values and a column with the corresponding positions (\smalltt{lon} and \smalltt{lat}). The positional information is provided to allow advanced users to directly formulate queries against it. Finally, we expose a column for each variable.

\begin{table}[h!]
\setlength\tabcolsep{2.7pt}
\begin{tabular}{| c | c | c | c | c | c | c | c | c |}
\hline
\cellcolor{gray!25}\textbf{file} & \cellcolor{red!25}\textbf{time} & \cellcolor{green!25}\textbf{lon} & \cellcolor{green!25}\textbf{lonPos} & \cellcolor{green!25}\textbf{lat} & \cellcolor{green!25}\textbf{latPos} & \cellcolor{blue!25}\textbf{asn} & \cellcolor{blue!25}\textbf{blh} & \cellcolor{blue!25}...\\\hline\hline
04.nc & 2017-01-01 04:00:00 & 0 & 720 & 90.0 & 0 & 0.9 & 264.5 & ...\\\hline 
04.nc & 2017-01-01 04:00:00 & 0 & 720 & 89.75 & 1 & 0.9 & 333.8 & ...\\ \hline
04.nc & 2017-01-01 04:00:00 & 0 & 720 & ... & ... & ... & ... & ...\\  
\hline    
\end{tabular}
\caption{Exposed row-wise representation of ERA5 surface.}
\label{table:schema}
\end{table}

The Relation is responsible for exposing the schema of the data source, i.e.\ the data types of the columns, to the SparkSQL layer. The schema stores the name of each column, its corresponding data type, and whether the column is nullable. To obtain it, we convert the data types inferred earlier to the corresponding SparkSQL types.
%Also, the Relation is responsible for communication our optimizations down to the data-source.

With an overview of how we connect and represent distributed NetCDF datasets in SparkSQL, we can now discuss how our query processing layer works. In particular, we will discuss which optimizations we apply in order to prune data as close as possible to the data source when dealing with convex predicates (Section~\ref{ssec:convex_query_processing}) and non-convex predicates (Section~\ref{ssec:non_convex_query_processing}). Then, we discuss how to perform additional pruning in the presence of joins (Section~\ref{sec:envelopes}).

\section{Convex Query Processing \& Optimization}
\label{ssec:convex_query_processing}

We start the discussion by following the processing steps of a simple example query~$Q_1$, shown in Listing~\ref{listing:q1}. $Q_1$ loads a convex region of the previously mentioned ERA5 surface subset. Convex means that the selected region can be translated into a \textit{single} subarray lookup per file, represented by a single $d$-dimensional block.

\noindent 
\begin{minipage}{\linewidth}
\begin{lstlisting}[style=sql, caption={Example query~$Q_1$ (convex predicate).}, label={listing:q1}]
SELECT time, lat, lon, sp
FROM era_b
WHERE time > '2017-01-01 00:15:00' AND 
        lat > 20.2 AND lat < 60.5 
\end{lstlisting}
\end{minipage}

\noindent The query operates on the ERA5 surface subset, which is split into a set of files where each file stores all observations across longitude and latitude at a single point in time. More precisely, each file contains the following dimensions: \smalltt{time}~($1$~timestamp, resolution: hourly) $\times$ \smalltt{lat}~($721$~entries, resolution: $1/4\degree$) $\times$ \smalltt{lon}~($1440$~entries, resolution: $1/4\degree$). 
It is also noteworthy that the dataset may contain gaps, i.e. an hour of a day (and hence the corresponding file) might be simply missing.

\subsection{Overview}

Efficiently processing very large datasets relies on pruning all data that is not relevant for the query early on. Thus, fully scanning all files of the dataset, as done by SciSpark~\cite{lit:scispark} for instance, is not an option. In \system{}, we support both vertical and horizontal pruning, which is automatically pushed down from the SQL representation to the data source.

To implement vertical pruning, we filter and reorder the internal schema representation based on which columns are required by the query, such that the RDD only loads columns of interest.

Realizing horizontal pruning is more challenging. This holds true especially for handling non-convex and potentially overlapping predicates, as we will see in Section~\ref{ssec:non_convex_query_processing}. For now, let us focus on handling convex predicates. \system{} supports horizontal pruning for arbitrary numeric comparisons as well as SQL's \smalltt{IN} and \smalltt{NOT IN} operators. Further, predicates can be connected arbitrarily via \smalltt{AND} and \smalltt{OR} and be negated via \smalltt{NOT}.
We utilize the subarray lookup provided by the NetCDF API to realize filtering at data-source level.
However, as we have discussed in Section~\ref{ssec:netcdf_api}, a subarray lookup cannot be performed by passing the actual dimension values (as formulated in the query), but requires passing a multidimensional origin and extent. 
Consequently, we have to \textit{translate} all query predicates into the corresponding positional tuples of origin and extent. 
Note that such a translation could be implemented in form of an auxiliary index structure, as conceptually done by ClimateSpark~\cite{lit:climatespark}. However, this would (a)~impose additional storage overhead and (b)~require an expensive preprocessing pass that we want to avoid. Instead, we apply an on-the-fly approach \textit{during} query processing, as we will see in the following.

\subsection{Global and Local Query Rewriting}
\label{sssec:convex_workflow}

When $Q_1$ is submitted, \system{} first detects that the query filters on the two dimensions \smalltt{time} and \smalltt{lat}. As \smalltt{time} is a dimension that spans multiple files, whereas \smalltt{lat} is not, the two dimensions must be treated differently. Also, the \smalltt{lon} dimension, along which $Q_1$ does not filter, must still be considered, as all dimensions are required to perform the subarray lookup.

\system{} expresses the translation of the query predicates to the corresponding origin/extent representation by \textit{rewriting} the query in multiple steps. This rewriting happens on-the-fly during query processing and can be separated into two phases: \textit{global} rewriting and \textit{local} rewriting. Let us see how it works:

% Global rewriting
The global rewriting of~$Q_1$ starts at the first file of the dataset. For each dimension that does \textit{not} span multiple files, we rewrite all corresponding query predicates in this phase. 
This is the case for the \smalltt{lat} dimension and the \smalltt{lon} dimension. 
% Latitude
Since $Q_1$ filters along the \smalltt{lat} dimension, \system{} reads the array representation of \smalltt{lat} from the file and identifies the positional range that is relevant for our query via binary search. 
As the \smalltt{lat} dimension contains values from $90\degree$ to $-90.0\degree$ in steps of $1/4\degree$, a binary search of the predicate \smalltt{lat~>~20.2} will result in a rewriting of the predicate into \smalltt{latPos~<=~279}. Note that the change in direction of the comparison operation as well as the correct boundaries will be ensured automatically in the process. Similarly, the predicate \smalltt{lat~<~60.5} will be rewritten into \smalltt{latPos~>=~119}. 
% Longitude
As $Q_1$ does not filter along the \smalltt{lon} dimension, \system{} will generate a non-selective predicate in the global rewriting phase. As the dimension has $1440$~entries, we create \smalltt{lonPos~>=~0 AND lonPos~<=~1439}.

Overall, this results in the globally rewritten query $Q_1^{g}$, as shown in Listing~\ref{listing:q1_rewritten1}. 
Note that the core property of the globally rewritten query is that it is valid with respect to the global coordinate system spanning across all files of the dataset.

\noindent 
\begin{minipage}{\linewidth}
\begin{lstlisting}[style=sql, caption={Globally rewritten query~$Q_1^{g}$. Predicates are valid with respect to the global coordinate system of the dataset.}, label={listing:q1_rewritten1}]
SELECT time, lat, lon, sp
FROM era_b
WHERE time > '2017-01-01 00:15:00' AND 
        latPos >= 119 AND latPos <= 279 AND
        lonPos >= 0 AND lonPos <= 1439
\end{lstlisting}
\end{minipage}

%\noindent From the query predicates, we can now construct a so called \textit{logical block}. The logical block~$L(Q_1^{g})$ would be 
%$$L(Q_1^{g}) = ((\text{'2017-01-01 00:15:00'}, \infty), [119, 279], [0, 1439])\text{.}$$ 
%This intermediate representation will become necessary when handling non-convex predicates in Section~\ref{ssec:non_convex_query_processing}. 
%

Note that the \smalltt{time} dimension is not globally rewritten in $Q_1^g$. As \smalltt{time} spans over multiple files of our dataset, a global rewriting of the query predicate \smalltt{time~>~'2017-01-01~00:15:00'} would not yield correct results.
Instead, \system{} performs a local rewriting of the \smalltt{time} predicate in $Q_1^g$, which must happen \textit{individually per file}. Precisely, when processing file~$f$, we read the \smalltt{time} dimension array of $f$ and and identify which entries satisfy our predicate \smalltt{time~>~'2017-01-01~00:15:00'}. In the case of ERA5, the \smalltt{time} dimension of each file only contains one entry (at position~$0$). Thus, if the entry satisfies the predicate, we rewrite the query $Q_1^g$ into $Q_1^{l_{\text{sat}}}$, as shown in Listing~\ref{listing:q1_rewritten2_qual}. If the entry does not satisfy the predicate, we rewrite the query into $Q_1^{l_{\text{unsat}}}$, as shown in Listing~\ref{listing:q1_rewritten2_noqual}. 

\noindent 
\begin{minipage}{\linewidth}
\begin{lstlisting}[style=sql, caption={Locally rewritten query~$Q_1^{l_{\text{sat}}}$, which will be executed on a file that \textbf{satisfies} the \smalltt{time}-predicate.}, label={listing:q1_rewritten2_qual}]
SELECT time, lat, lon, sp
FROM era_b_f -- operates on a specific file f
WHERE timePos >= 0 AND timePos <= 0 AND -- all
        latPos >= 119 AND latPos <= 279 AND
        lonPos >= 0 AND lonPos <= 1439
\end{lstlisting}
\end{minipage}

\noindent 
\begin{minipage}{\linewidth}
\begin{lstlisting}[style=sql, caption={Locally rewritten query~$Q_1^{l_{\text{unsat}}}$, which will be executed on a file that \textbf{does not satisfy} the \smalltt{time}-predicate.}, label={listing:q1_rewritten2_noqual}]
SELECT time, lat, lon, sp
FROM era_b_f -- operates on a specific file f
WHERE timePos > 0 AND timePos <= 0 AND -- empty set
        latPos >= 119 AND latPos <= 279 AND
        lonPos >= 0 AND lonPos <= 1439
\end{lstlisting}
\end{minipage}

% Execution
After a local query has benn generated for a file, \system{} starts the retrieval of the projected dimensions and variables. While \system{} retrieves the projected dimensions individually via a single-dimensional lookup, 
it performs a final translation from the predicates of the local query to the origin/extent representation to retrieve the variables. For example, to retrieve the project variable~\smalltt{sp}, $Q_1^{l{\text{sat}}}$ is translated to origin
$o=(\smalltt{time}=0, \smalltt{lon}=0, \smalltt{lat}=119)$
and extent
$e=(\smalltt{time}=0, \smalltt{lon}=1439, \smalltt{lat}=160)$, then the subarray lookup is performed. Note that at this stage, \system{} also detects that the \smalltt{time}-predicate of~$Q_1^{l_{\text{unsat}}}$ results in an empty set. Thus, it eliminates this query at this stage entirely.

As previously described, \system{} materializes the data in a convenient row-wise representation after loading. 
Note that the result of a single subarray lookup can potentially yield enough entries to exceed the available main memory capacity of the worker node. Thus, before performing the lookup, we split a single lookup into multiple smaller ones if necessary. When splitting, we avoid splitting the \textit{fastest-varying dimension}, which is stored continuously in memory, to preserve the sequential read performance.

\section{Non-Convex Query Processing \& Optimization}
\label{ssec:non_convex_query_processing}

So far, we have the discussed how to efficiently process convex predicates, such as shown in query~$Q_1$. Again, convex means that all predicates of the query result in the selection and consequently the loading of a \textit{single} block per file.
However, various applications in the Earth science community often require the formulation of non-convex predicates and can therefore require multiple blocks per file. For example, certain atmospheric analyses require the area of interest to be limited to regions covered by water, such as oceans. Also, certain areas are also often excluded from the analysis, creating "holes" in the region of interest. Of course, we could simply load a larger convex block enclosing all blocks described by the non-convex predicates and then filter the result afterwards. This is the only option in ClimateSpark, which cannot automatically push down such a non-convex predicate to the data source. However, this could result in loading significant amounts of unnecessary data. Instead, we want to ensure that only the data of interest is actually loaded from the data source. 

\subsection{Overview}

To do so, \system{} again applies a set of rewriting rules to the non-convex predicate in order to generate a set of convex predicates. Each generated convex block can then be loaded via a single subarray lookup, as done before. While doing so, two circumstances must be respected:
\begin{enumerate} 
\item \label{enum:overlap} The generated convex blocks potentially \textit{overlap arbitrarily} across one or multiple dimensions.
\item \label{enum:many} We might generate a \textit{large number of convex blocks} for a single query.
\end{enumerate}
Regarding~\ref{enum:overlap}), we have to ensure \textit{correctness} in case of overlapping blocks, i.e. we must not load data within overlapping regions multiple times. 
Regarding~\ref{enum:many}), we may have to handle a large number of blocks during overlap detection and subsequent lookups. Thus, we require efficient overlap detection strategies as well as reduce the number of generated blocks again before the actual subarray lookups are performed. 
In the following, we will describe how we deal with the aforementioned challenges.


%For the discussion, we first introduce the concept of \textit{global blocks} and \textit{local blocks}. 
%A global block is a block over all dimensions, which potentially spans over multiple files.
% In contrast to that, a local block is a block of all dimensions, that is local to a single file. They can be derived from the predicates during global and local query rewriting. 
%For example, in Section~\ref{ssec:convex_query_processing}, we generated the globally rewritten query~$Q_1^g$. The block selected therein is represented by the global block
%$$g=((\text{’2017-01-01 00:15:00’}, \infty) \times [119, 279] \times [0, 1439])\text{.}$$
%Correspondingly, the local block selected by the block from the locally rewritten query~$Q_1^{lsat}$ is
%$$l=([0, 0] \times [119, 279] \times [0, 1439])\text{.}$$
%
%Why do we introduce this distinction? Because in contrast to the convex case, where we have to reason about a single block, 
%we now have to reason over \textit{multiple} blocks. As they are compared against each other during overlap detection, they must be formulated with respect to a global coordinate system. 

%\begin{figure}[h!]
%\includegraphics[page=1, width=\columnwidth, trim={0 5.7cm 0 0}, clip]{figures/non_convex.pdf}
%\caption{Visual representation of the entries, that are selected by the WHERE clause of query~$Q_2$.}
%\label{fig:non_convex}
%\end{figure}


In short, we first rewrite the \smalltt{WHERE} clause, which consists of potentially many non-convex query predicates, into \textit{disjunctive normal form}, i.e. a disjunction of conjunctive clauses. This automatically generates a set of convex predicates, which individually represent blocks to load. 
Section~\ref{ssec:non_convex_to_convex} discusses the details of this step.
Note that this conversion could theoretically inflate the predicate exponentially, but we do not expect this to happen in practice.

Next, we globally rewrite the predicate according to the steps described in Section~\ref{ssec:convex_query_processing}. After rewriting the query globally, we ensure that no region of the dataset is loaded more than once to ensure correctness. In Section~\ref{ssec:overlap}, we propose multiple strategies to do so. The final transformation into local queries is performed afterwards.

\subsection{Non-Convex to Convex Predicate Rewriting}
\label{ssec:non_convex_to_convex}

Let us consider a concrete example. Query~$Q_2$, shown in Listing~\ref{listing:q2}, selects a certain surface region based on the \smalltt{lon} and \smalltt{lat} dimension. As we can see, the selection \textit{excludes} a certain subregion.
\begin{minipage}{\linewidth}
\begin{lstlisting}[style=sql, caption={Example query~$Q_2$ (non-convex predicates).}, label={listing:q2}]
SELECT time, lat, lon, sp
FROM era_b
WHERE lon >= 90.0 AND
 NOT (lat == 0.0 AND lon >= 163.0 AND lon <= 163.75)
\end{lstlisting}
\end{minipage}

\noindent In the first step, the \smalltt{WHERE} clause is automatically rewritten into disjunctive normal form (DNF). Listing~\ref{listing:q2_where_rewriting} shows the individual steps that are carried out. Initially, we eliminate any present negations; this changes the comparison and logical operators. Then, we express all equalities by the corresponding inequalities. Finally, we expand the clause into DNF. 

\begin{lstlisting}[style=sql, caption={Individual rewriting-steps of query~$Q_2$ into DNF.}, label={listing:q2_where_rewriting}]
-- original predicate
lon >= 90.0 AND 
NOT (lat == 0.0 AND lon >= 163.0 AND lon <= 163.75)
-- eliminate negation
lon >= 90.0 AND 
(lat != 0.0 OR lon < 163.0 OR lon > 163.75)
-- rewrite equalities into inequalities
lon >= 90.0 AND 
(lat < 0.0 OR lat > 0.0 OR 
  lon < 163.0 OR lon > 163.75)
-- disjunctive normal form
(lon >= 90.0 AND lat < 0.0) OR 
(lon >= 90.0 AND lat > 0.0) OR
(lon >= 90.0 AND lon < 163.0) OR 
(lon >= 90.0 AND lon > 163.75)
\end{lstlisting}

With the DNF at hand, we are ready to generate the globally rewritten query, shown in Listing~\ref{listing:q2_global}. The \smalltt{WHERE}-clause contains the predicates in DNF, where each predicate which operates on a dimension that does not span across files has been translated into the corresponding positional boundaries. Again, this changes the comparison operator for the \smalltt{lat} dimension. Further, in this step, we have added the non-selective predicate for the missing \smalltt{time} dimension. As \smalltt{time} spans across multiple files, we formulate this predicate with respect to the global coordinate system.

\noindent 
\begin{minipage}{\linewidth}
\begin{lstlisting}[style=sql, caption={\smalltt{WHERE}-clause of globally rewritten query~$Q_2^{g}$.}, label={listing:q2_global}]
WHERE (time >= minTime AND time <= maxTime AND
         lonPos >= 361 AND lonPos <= 1439 AND
         latPos >= 362 AND latPos <= 721)
         OR
         (time >= minTime AND time <= maxTime AND
         lonPos >= 361 AND lonPos <= 1439 AND
         latPos >= 0 AND latPos <= 360) 
         OR
         (time >= minTime AND time <= maxTime AND
         lonPos >= 361 AND lonPos <= 652 AND
         latPos >= 0 AND latPos <= 721) 
         OR
         (time >= minTime AND time <= maxTime AND
         lonPos>= 657 AND lonPos <= 1439 AND
         latPos >= 0 AND latPos <= 721) 
\end{lstlisting}
\end{minipage}

\noindent Note that in the last conjunctive clause, \system{} translated 
\mbox{\smalltt{(lon >= 90.0 AND lon > 163.75)}} into 
\mbox{\smalltt{lon > 163.75}} and subsequently into
\smalltt{(lonPos >= 657 AND lonPos <= 1439)}, as the predicate \smalltt{lon > 163.75} is more selective than \smalltt{lon >= 90.0}.
In summary, we get four conjunctive clauses, where each clause selects a block.

Note that the same predicate transformation has been performed by Wang et al.~\cite{lit:rewriting} to provide a query interface for HDF5~\cite{list:hdf5} files, which are structurally similar to NetCDF-4 files.
However, Wang et al. do not mention the case where multiple conjunctive clauses describe overlapping regions, which can lead to incorrect results as we will discuss in the next section.


%Figure~\ref{fig:non_convex_with_overlap} visualizes the selected blocks across the \smalltt{lat} and \smalltt{lon} dimensions, while omitting the non-selective \smalltt{time} dimension. 

%\begin{figure}[h!]
%\includegraphics[page=2, width=\columnwidth, trim={0 5.7cm 0 0}, clip]{figures/non_convex.pdf}
%\caption{Visual representation of the \smalltt{WHERE} clause of query~$Q_2^g$. We can see that the resulting blocks overlap.}
%\label{fig:non_convex_with_overlap}
%\end{figure}

\subsection{Overlap Detection and Elimination}
\label{ssec:overlap}

As we see from Listing~\ref{listing:q2_global}, the resulting blocks overlap across multiple dimensions. If we translated each block into a corresponding subarray lookup, we would generate incorrect results. Thus, we need to transform the generated blocks such that no overlap occurs anymore. To do so, we propose two different strategies, which we will discuss in the following. 

Let $n$ be the number of blocks and $d$ be the number of dimensions.
A single block corresponds to a lower and an upper bound along each dimension,
so there are at most $2n$ distinct bounds along each dimension, and at most $2n - 1$ intervals between adjacent bounds.

\subsubsection{Naive}
\label{sssec:strategy1}
In the naive strategy, we iterate over all blocks and split each block at each distinct interval boundary along all dimensions. This generates at most $(2n - 1)^d$~sub-blocks per block. The resulting blocks either overlap entirely or not at all. Thus, we remove all blocks that overlap next. Finally, we merge adjacent blocks into larger ones to keep the number of resulting blocks as small as possible. To improve performance when loading blocks from disk later on, we merge along the fastest-varying dimension of the dataset first, resulting in blocks having the largest possible extent along this dimension. 

Let us see how the naive strategy would eliminate the overlaps of the $n=4$ blocks of query~$Q_2^g$. We have $4$~distinct bounds along both the \smalltt{lat} and the \smalltt{lon} dimension, with $3$~intervals between adjacent bounds for each dimension. The \smalltt{time} dimension is non-selective and therefore contains only one interval. Now, we split the blocks along each dimension, as shown in Figure~\ref{fig:strategy1:split}. This results in $12$~independent sub-blocks. We then remove the 4 redundant blocks by inserting all blocks into a \smalltt{HashSet}, resulting in $8$~remaining sub-blocks in total. Finally, as shown in Figure~\ref{fig:strategy1:merge}, we can merge the $6$~adjacent sub-blocks into $2$~larger ones along the fastest-varying dimension. Overall, we get $4$~non-overlapping blocks.


\begin{figure}[h!]
\vspace*{-0.3cm}
\subfloat[Split of blocks into either fully overlapping or non-overlapping sub-blocks.]{
\includegraphics[page=1, width=\columnwidth, trim={0 11cm 0 0}, clip]{non_convex_small.pdf}
\label{fig:strategy1:split}
}\quad
\subfloat[After elemination of overlapping sub-blocks, merge of adjacent sub-blocks.]{
\includegraphics[page=2, width=\columnwidth, trim={0 11cm 0 0}, clip]{non_convex_small.pdf}
\label{fig:strategy1:merge}
}
\caption{Overlap elimination using the naive strategy.}
\vspace*{-0.4cm}
\label{fig:strategy1}
\end{figure}


\subsubsection{Optimized}
\label{sssec:strategy2}

The naive strategy has two limitations: We might generate a large number of sub-blocks in the process of splitting. For only $n=4$ and $d=3$, we generate up to $(2n - 1)^d = 343$~sub-blocks per block in the worst case. Thus, the number of generated sub-blocks might (a)~exceed the available memory capacity of the driver node and (b)~render the overlap detection and subsequent merge expensive. In the following, we thus propose an optimized version to address these issues. 
\begin{lstlisting}[style=pseudo, caption={Pseudo-code of the optimized strategy.}, label={listing:strategy2}, xleftmargin=4.0ex]
findCover(blocks, dim=0, lastDim):
   // find 1D interval cover if applicable
   if dim == lastDim:
      return findIntervalCover(blocks, lastDim)
   cover <- []
   boundaries <- [for b in blocks: b.start[dim]] 
                   + [for b in blocks: b.end[dim]]
   boundaries.sort()
   boundaries.removeDuplicates()
   // split the blocks along adjacent boundaries
   for i in range(boundaries.size - 1):
      start, end = boundaries[i], boundaries[i + 1]
      active <- blocks that overlap with [start, end]
      sliced <- copy(active)
      for b in sliced:
         b.start[dim], b.end[dim] = start, end
      // recursively find a cover for this subset
      subCover = findCover(sliced, dim + 1, lastDim)
      // merge blocks before the next iteration
      cover <- mergeAlignedBlocks(subCover, cover)
   return cover
\end{lstlisting}
In the optimized strategy, we first sort the distinct intervals along a specific dimension. Then, we process these intervals in sorted order one at a time: For each interval, we split only along the interval boundaries and try to merge the newly obtained sub-blocks with those obtained in the previous iteration. Listing~\ref{listing:strategy2} shows the workflow in pseudo-code. 
By applying this strategy recursively along each dimension, the intermediate result is kept as small as possible at all times, addressing problem~(a) of the naive strategy. Further, the set of sub-blocks on which we have to detect overlaps and perform the merging remains small, effectively addressing problem~(b) of the naive strategy.  
Further, we process the fastest-varying dimension in the innermost recursion step. Thus, we still generate blocks that allow fast sequential access. Overall, this strategy produces the same result as the naive strategy, albeit more efficiently, as we will evaluate in Section~\ref{sec:experiments}.

In Figure~\ref{fig:strategy2}, we show how the optimized strategy processes two intervals along the \smalltt{lat} dimension. In Figure~\ref{fig:strategy2:interval1}, which shows the handling of the first interval, we split $2$~blocks along the boundary between latitude~$-0.25$ and $0.0$. Afterwards, we detect that the generated sub-blocks overlap with the third block of this area, and consequently, eliminate both. Next, in Figure~\ref{fig:strategy2:interval2}, which shows the handling of the second interval, we again split $2$~blocks, this time at the boundary between latitude~$0.0$ and $0.25$. Again, we check for an overlap of the newly created sub-blocks, however, there are no overlaps this time. As a final step of the round, we check whether the newly created sub-blocks can be merged directly with the sub-blocks created when handling the previous interval. In this case, this is not possible and thus, this interval is fully processed. The procedure repeats until all intervals have been processed. 

\begin{figure}[h!]
\vspace*{-0.3cm}
\subfloat[Split by the first interval of \scriptsizett{lat} dimension to eliminate overlapping blocks.]{
\includegraphics[page=3, width=\columnwidth, trim={0 11cm 0 0}, clip]{non_convex_small.pdf}
\label{fig:strategy2:interval1}
}\quad
\subfloat[Split by the second interval of \scriptsizett{lat} dimension. This time, no overlap must be eliminated. Try to merge the sub-blocks of the current iteration with the sub-blocks of the previous iteration, which is not possible here.]{
\includegraphics[page=4, width=\columnwidth, trim={0 11cm 0 0}, clip]{non_convex_small.pdf}
\label{fig:strategy2:interval2}
}
\caption{Overlap elimination using the optimized strategy.}
\vspace*{-0.4cm}
\label{fig:strategy2}
\end{figure}



\section{Join Optimization via Envelopes}
\label{sec:envelopes}

In the context of Earth science, analytical tasks often require the joining of multiple independent datasets or subsets of those. For example, the ERA5 dataset is split into one subset consisting of surface data and into another model-level subset with an additional vertical dimension -- often, both are required together and must be joined consequently. As such, we want to enable automatic and early horizontal pruning of data in the presence such joins as well. 
Precisely, we target the common situation where dataset~$R$ is more selective than dataset~$S$ with respect to the dimension on which the join is performed. In this situation, we want to avoid that~$S$ is fully loaded since many entries of~$S$ will not find a join partner in $R$. Of course, we could require the user to add a corresponding filtering condition on $S$ to the query. However, this requires a manual adjustment according to the characteristics of the underlying datasets and thus stands in conflict with \system{}'s principle of applying \textit{automatic} optimizations wherever possible. 

Instead, \system{} support the generation and automatic usage of so-called \textit{envelopes}. Envelopes are a set of tight lower and upper bounds for each file of a dataset. The user can conveniently request to generate envelopes for a specific dataset and a set of dimensions when registering the dataset in SparkSQL. By calling our \smalltt{envelope()}~function on the DataFrame, the auxiliary envelope information will be generated and used during the processing of any query that operates on the enveloped dataset:
\noindent
\begin{minipage}{\linewidth}
\begin{lstlisting}[style=sql, label={listing:envelopes}]
envelope(ERA_Sur, "time", "longitude", "latitude")
   .createOrReplaceTempView("Era_Sur")
\end{lstlisting}
\end{minipage}

The challenge lies in communicating to SparkSQL that only such data of~$S$ should be loaded that lies within the envelopes of $R$. While SparkSQL implements this form of invariant propagation internally\cite{lit:invariants}, unfortunately, it does not expose access to this mechanism in its public API.
To overcome this limitation, we gather all envelopes into a single filter operation and apply this filter on the DataFrame of~$R$ during the call of \smalltt{envelope()}. Note that this filter does not actually remove data in $R$ because each record of $R$ is contained within the envelope. However, this filter information of $R$ will be picked up by the optimizer of SparkSQL and pushed down to the other side of the join, namely $S$. On $S$, it results in horizontal pruning at the data source, as described before. 

Of course, this filter operation also introduces execution overhead, as SparkSQL has to evaluate a complex auxiliary predicate for every record of both $R$~and~$S$, in addition to the cost of computing the envelope. Still, as we will see in Section~\ref{sec:experiments}, the overhead is compensated by having to load significantly less data from disk. 

\section{Experimental Evaluation}
\label{sec:experiments}

%In the following, we will experimentally evaluate \system{} with respect to different properties. %Before can start, let us discuss the precise setup we use.

\subsection{Setup and Datasets}

All tests were conducted on a SparkSQL~3.0.0 cluster with $8$~executors per node with $8$~cores each. If not stated otherwise, we use $5$~nodes in total. The cluster is set up on top of MOGON~\cite{lit:mogon}, the HPC infrastructure at Mainz University, where each node consists of two 16-core Xeon Gold 6130 CPUs paired with 177 GB of RAM and an OmniPath interconnect. The dataset is stored on a Lustre filesystem~\cite{lit:lustre} instead of HDFS, so \system{} has to ignore data locality for files.
All workers are configured to use a node-local SSD for temporary storage. Unless mentioned otherwise, each experiment was run exactly once to avoid caching artifacts.

We use the ERA5 dataset~\cite{lit:ecmwf-era5,lit:era5} for our evaluation, where we use the following three subsets:
\textbf{ERA5\_Sur}:~The 3D surface subset ($38$MB/file), which stores, among others, snow cover and surface pressure, as well as wind speeds and temperature just above the surface.
\textbf{ERA5\_Pre}:~The 3D precipitation subset, which contains rain and snowfall data ($26$MB/file).
\textbf{ERA5\_Mod}:~The 4D model-level subset, which includes an additional vertical dimension that ranges from the surface to the top of the atmosphere ($2.2$GB/file). Variables include specific humidity, but also wind speeds and temperature for each model level separately.
Each subset is stored in a separate set of files. 
All three subsets use the $1/4\degree$ by $1/4\degree$ surface grid mentioned before, with a temporal resolution of one hour.

The majority of our benchmark queries is inspired by real-world analysis tasks from atmospheric physics.
In some of the queries, we mention a quantity called \textit{relative humidity with respect to ice} (RHI)~\cite{lit:rhi}, which indicates the amount of water vapor with respect to the stable phase ice.
This quantity is calculated through a formula\cite{lit:rhi-formula} that depends on variables from both the surface and model-level subsets.

%\begin{minipage}{\linewidth}
%\begin{lstlisting}[style=pseudo, caption={UDF used to compute the RHI.}, label={listing:rhi_udf}]
%val rhi: UserDefinedFunction = udf { 
%  (p: Double, q: Double, t: Double) =>
%  val preal: Double = q * p / (0.622 + 0.378 * q)
%  val picesat: Double = Math.exp(9.550426 - 5723.265 / t 
%                     + 3.53068 * Math.log(t) - 0.00728332 * t)
%  100 * preal / picesat
%}.asNonNullable()
%\end{lstlisting}
%\end{minipage}

\subsection{Experiment: Effect of Pruning}

We start with two micro-benchmarks to evaluate the effect of vertical and horizontal pruning on the ERA5\_Mod subset.
In Figure~\ref{fig:opt:exp1}, we fire a query that filters a single model-level file along the \smalltt{lat} dimension and then calculates the mean of the variable~\smalltt{t} (temperature). We vary the selected range of the \smalltt{lat} dimension from $0^\degree$ to $180\degree$.
In Figure~\ref{fig:opt:exp2}, we fire a query that does not perform any filtering of the file, but computes the mean of a certain number of variables, which we vary from~$1$ to $6$ (\smalltt{t, cc, q, o3, u, v}). We evaluate \system{} both with activated and deactivated optimizations.

\begin{figure}[h!]
\vspace*{-0.7cm}
\subfloat[Effect of horizontal pruning.]{
\includegraphics[width=43mm]{opt-lat.pdf}
\label{fig:opt:exp1}
}
\subfloat[Effect of vertical pruning.]{
\includegraphics[width=43mm]{opt-cols.pdf}
\label{fig:opt:exp2}
}
\vspace*{-0.1cm}
\caption{Evaluation of Pruning Optimizations.}
\vspace*{-0.3cm}
\label{fig:opt}
\end{figure}

As we can see in Figure~\ref{fig:opt}, the execution time is significantly increased and does not reflect the query's selectivity when all optimizations are disabled.
This is because Spark has to fully read every single entry in the file and can only apply the projections and selections after a tuple has been loaded from the data source.
We can also see that the execution time increases linearly with the amount of data loaded if our optimizations are enabled.
In absolute terms, enabling pruning reduces the execution time of the query from well above five minutes to $60$~seconds, and in many cases far below that.
Note that since each file corresponds to a single partition, this experiment does \textit{not} include any form of parallelism, even through there are multiple nodes available.

\subsection{Experiment: Real-world Atmospheric Application}

As \system{} is meant to support atmospheric research, let us evaluate a real-world application from the domain, namely the computation of RHI histograms. 
\begin{wrapfigure}{r}{0.475\columnwidth}
\vspace*{-0.8cm}
\subfloat{
\hspace*{-0.2cm}\includegraphics[width=43mm]{data-load-lat.pdf}
\label{fig:data-load:lat}
}\\ \vspace*{-0.5cm}
\subfloat{
\hspace*{-0.2cm}\includegraphics[width=43mm]{data-load-lon.pdf}
\label{fig:data-load:lon}
}\\ \vspace*{-0.5cm}
\subfloat{
\hspace*{-0.2cm}\includegraphics[width=43mm]{data-load-time.pdf}
\label{fig:data-load:time}
}
\caption{RHI histogram.}
\vspace*{-0.2cm}
\label{fig:data-load}
\end{wrapfigure}
We provide one month of data from both ERA5\_Mod and ERA5\_Sur, amounting to a total of $1.6$TB stored across $1488$~files. 
In the experiment, we evaluate queries which join these two subsets along their \smalltt{time}, \smalltt{lat}, and \smalltt{lon} dimensions, and then compute a RHI histogram. We vary the following restrictions: 
In Figure~\ref{fig:data-load:lat}, we restrict the time dimension to a 12-hour window and vary the extent along the \smalltt{lat} dimension. In Figure~\ref{fig:data-load:lon}, we restrict the \smalltt{time} dimension to a 12-hour window and vary the extent along the \smalltt{lon} dimension. In Figure~\ref{fig:data-load:time}, we vary the temporal window between 0 and 12 hours.

As we can see in Figure~\ref{fig:data-load}, the performance of \system{} scales gracefully with the amount of data being selected by the computation under this real-world application. Note that the last bar in each plot corresponds to loading the entire 12-hour window, and results in a consistent time between experiments. 


%Note that for both the latitude and longitude case, there is a sudden drop-off at the end.
%\todo{erklärung (vllt optimierung in netcdf-library?)}

\vspace*{-0.1cm}
\subsection{Experiment: Effect of Envelopes on Join Performance}

Let us now evaluate the effect of generating envelopes to speed up join processing. 
We set up the experiment by cutting out a contiguous six-hour window covering the North Atlantic Ocean from ERA5\_Sur, retaining only the \smalltt{time}, \smalltt{lat}, \smalltt{lon}, and \smalltt{sp} columns, and export it to CSV format using Spark. We call this artificial dataset ERA5\_NAO. Note that ERA5\_NAO is more selective on the join dimensions than the remaining ERA5 subsets. We then join this dataset with $12$~files from ERA5\_Mod and measure the execution time.
The $12$~files are selected to contain all six hours from the artificial dataset, so that each row in ERA5\_NAO has a matching join partner. After performing the join, we count the number of records where the RHI exceeds 100\% and the temperature remains below 243 Kelvin.

\begin{figure}[h!]
\vspace*{-0.1cm}
\includegraphics[width=90mm]{envelopes.pdf}
\vspace*{-0.7cm}
\caption{Effect of Envelopes on Join Processing.}
\vspace*{-0.5cm}
\label{fig:envelopes}
\end{figure}

Figure~\ref{fig:envelopes} shows the runtime of the computation with and without the generation of envelopes on ERA5\_NAO. We can see that the generation of envelopes, which is included in the runtime, reduces the end-to-end processing time by a factor of almost $12$x.  This is due to the reduction in I/O operations for the ERA5\_Mod side of the join. 


\subsection{Experiment: \system{} vs ClimateSpark pipeline}

Next, we want to evaluate \system{} against a comparable baseline, namely ClimateSpark~\cite{lit:climatespark}. Unfortunately, the only publicly available version~\cite{lit:climatespark-github} of ClimateSpark depends on a legacy version of Spark and is not compatible with Spark 3.0.0. In addition, the implementation is specifically hard-coded against the MERRA and MERRA2 datasets and would require deep changes to support ERA5. 

\begin{figure}[h!]
\vspace*{-0.7cm}
\subfloat[Maximum wind speed.] {
\includegraphics[width=43mm]{vs-climate-spark-wind.pdf}
\label{fig:climatespark:wind}
}
\subfloat[Maximum convective precipitation.] {
\includegraphics[width=43mm]{vs-climate-spark-oceans.pdf}
\label{fig:climatespark:oceans}
}
\caption{\system{} vs ClimateSpark-Pipeline.}
\vspace*{-0.3cm}
\label{fig:climatespark}
\end{figure}

\begin{figure*}[h!]
\newcolumntype{C}{ >{\centering\arraybackslash} m{42mm} }
\begin{tabular}{>{\centering\arraybackslash} m{5mm} @{} C @{} C @{} C @{} C}
& \hspace*{2mm} WL\_aligned & \hspace*{2mm} WL\_misaligned & \hspace*{2mm} WL\_diagonal & \hspace*{2mm} WL\_centered \\
\rotatebox{90}{\hspace*{4mm}d = 1} & \includegraphics[width=40mm]{decomposition-align-overlap-1.pdf} & \includegraphics[width=40mm]{decomposition-misalign-overlap-1.pdf} & \includegraphics[width=40mm]{decomposition-diagonal-1.pdf} & \includegraphics[width=40mm]{decomposition-max-overlap-1.pdf} \\
\rotatebox{90}{\hspace*{4mm}d = 2} & \includegraphics[width=40mm]{decomposition-align-overlap-2.pdf} & \includegraphics[width=40mm]{decomposition-misalign-overlap-2.pdf} & \includegraphics[width=40mm]{decomposition-diagonal-2.pdf} & \includegraphics[width=40mm]{decomposition-max-overlap-2.pdf} \\
\rotatebox{90}{\hspace*{4mm}d = 4} & \includegraphics[width=40mm]{decomposition-align-overlap-4.pdf} & \includegraphics[width=40mm]{decomposition-misalign-overlap-4.pdf} & \includegraphics[width=40mm]{decomposition-diagonal-4.pdf} & \includegraphics[width=40mm]{decomposition-max-overlap-4.pdf} \\
\end{tabular}
\vspace*{-0.2cm}
\caption{Comparison of Naive Strategy and Optimized Strategy for Overlap Detection and Elimination.}
\vspace*{-0.6cm}
\label{fig:overlap}
\end{figure*}


Still, we are able to compare the workflow of both systems. To do so, we replicated the pipeline of ClimateSpark within \system{}. Precisely, we first hard-coded the list of files as well as the subarrays and variables to load for each file to simulate the effects of ClimateSpark's spatiotemporal index. Note that we do not measure the overhead that building a spatiotemporal index would normally imply. 
The resulting RDD maps tuples consisting of variable name, timestamp and spatial bounding box to entire subarrays, representing the values.
This RDD is then transformed into a row-wise representation following the example of \smalltt{queryPointTimeSeries} from the ClimateSpark repository~\cite{lit:climatespark-github}. The row-wise representation is converted to a SparkSQL \smalltt{DataFrame}, which includes five columns: The name of a variable, the three-dimensional coordinates, and the associated value.
If the query requires multiple variables, each value is stored in a separate row.

Again, we evaluate both pipelines at two applications from atmospheric physics.
In Figure~\ref{fig:climatespark:wind}, we query the maximum wind speed within a contiguous 24-hour window on ERA5\_Sur.
We vary the extent of the window along the latitude dimension.
Within the dataset, wind speeds are represented by two perpendicular components, so the total wind speed has to be computed from separate variables through a UDF.
However, since the ClimateSpark pipeline does not produce a DataFrame in which both components are in the same row, we need to perform a self-join first.
The execution time overhead associated with this self-join is apparent in the results.
Note that this self-join operation is not necessary in \system{} because all values belonging to the same coordinate are always grouped into a single row.


In Figure~\ref{fig:climatespark:oceans}, we compute the maximum convective precipitation over the North Atlantic and North Pacific regions on ERA5\_Pre.
We vary the number of files in our base dataset up to an entire month of data.
Since ClimateSpark only supports a single convex block to be loaded, we load the smallest possible block that fully contains our regions of interest, then filter the rows afterwards.
The difference in execution time shown in Figure~\ref{fig:climatespark} is likely due to this issue.

\vspace*{-0.2cm}
\subsection{Experiment: Overlap Detection and Elimination Strategies}
\vspace*{-0.1cm}
Lastly, we evaluate the two proposed strategies for overlap elimination, which is required to handle non-convex predicates.
Note that since the size of the blocks does not impact the execution time of both strategies, we only vary the number of blocks~$n$ from $2$ to $2^{16}$ and the number of dimensions~$d$ from $1$ to $4$ in each experiment.
We provide four workloads, where the first two workloads are rather common in practice. The last two workloads are artificially designed to maximize the number of distinct interval boundaries.
\textbf{WL\_aligned}:~$n$~blocks sharing a common origin and extent along every dimension except dimension 0, where blocks are arranged next to each other. Two adjacent blocks overlap with a probability of 50\%.
\textbf{WL\_misaligned}:~Similar to WL\_aligned, but every block has a 50\% chance to be shifted by half its extent along any dimension except 0.
\textbf{WL\_diagonal}:~\mbox{$n-1$}~blocks arranged diagonally inside a single large block.
\textbf{WL\_centered}:~$n$~overlapping blocks sharing a common center where block $i$ has greater extent along dimension 0 but smaller extent along all other dimensions when compared to block $i-1$.
%We expect the first two workloads to be rather common in practice since they would be the result of a query targeting one or multiple fixed spatial regions within different (possibly overlapping) time intervals.
%The last two workloads are artificially designed to maximize the number of distinct interval boundaries, with each instance achieving the theoretical maximum of $2n$ distinct boundaries for each of the $d$ dimensions.
%While workload 3 can be trivially resolved by keeping the single large block and discarding all other blocks, workload 4 requires at least $2n - 1$ unique blocks.

Figure~\ref{fig:overlap} shows the average execution time over five runs.
%Since block decomposition happens during query optimization, we deemed three seconds to be the largest acceptable runtime for all practical applications, which is why the vertical axis is limited to five seconds where necessary.
As we can see, both strategies scale linearly with the number of blocks for WL\_aligned. This behavior is expected since each additional block introduces exactly two new interval boundaries.
The naive strategy is slightly slower because every block has to be decomposed explicitly before being merged, whereas the optimized strategy has to perform merges only.
WL\_misaligned introduces additional interval boundaries due to some blocks being misaligned, thus further increasing the gap between the two strategies.
For WL\_diagonal, the naive strategy splits the large block into $(2n-1)^d$ smaller blocks. This exponential growth with increasing dimensionality drastically hurts its performance. The optimized strategy slices and reassembles the large block one interval at a time, so the intermediate results are much smaller.
For WL\_centered, the naive strategy generates lots of copies of sub-blocks close to the center, up to $n$ copies for the region where all blocks overlap. In contrast, the optimized strategy eliminates these redundant blocks immediately in the interval merge step.

Overall, we can see that while the naive strategy quickly becomes infeasible with increasing $n$, the optimized strategy remains usable for small high-overlap scenarios. Thus, by default, \system{} uses the optimized strategy. 


%\begin{tikzpicture}[scale=0.2]
%\draw[draw=black] (1,2) rectangle ++(5,3);
%\draw[draw=black] (1.5,1.5) rectangle ++(4,4);
%\draw[draw=black] (2,1) rectangle ++(3,5);
%\end{tikzpicture}
%
%\begin{tikzpicture}[scale=0.2]
%\draw[draw=black] (1,1) rectangle ++(5,5);
%\draw[draw=black] (1.5,1.5) rectangle ++(1,1);
%\draw[draw=black] (3,3) rectangle ++(1,1);
%\draw[draw=black] (4.5,4.5) rectangle ++(1,1);
%\end{tikzpicture}
%
%\begin{tikzpicture}[scale=0.3]
%\draw[draw=black,pattern=north west lines] (0,0) rectangle ++(0.8,1);
%\draw[draw=black,pattern=north east lines] (1,0) rectangle ++(0.8,1);
%\draw[draw=black,pattern=north west lines] (2,0) rectangle ++(1.2,1);
%\draw[draw=black,pattern=north east lines] (3,0) rectangle ++(1.2,1);
%\draw[draw=black,pattern=north west lines] (4,0) rectangle ++(0.8,1);
%\draw[draw=black,pattern=north east lines] (5,0) rectangle ++(0.8,1);
%\end{tikzpicture}\newline
%\vspace*{1cm}
%\begin{tikzpicture}[scale=0.3]
%\draw[draw=black,pattern=north west lines] (0,0) rectangle ++(0.8,1);
%\draw[draw=black,pattern=north east lines] (1,0.5) rectangle ++(0.8,1);
%\draw[draw=black,pattern=north west lines] (2,0.5) rectangle ++(1.2,1);
%\draw[draw=black,pattern=north east lines] (3,0) rectangle ++(1.2,1);
%\draw[draw=black,pattern=north west lines] (4,0.5) rectangle ++(0.8,1);
%\draw[draw=black,pattern=north east lines] (5,0) rectangle ++(0.8,1);
%\end{tikzpicture}



\vspace*{-0.3cm}
\section{Conclusion}
\vspace*{-0.2cm}
In this work we proposed~\system{}, a system which allows researchers from Earth sciences to formulate their analytical tasks in convenient SQL \textit{while} being able to rely on several automatic optimizations in the background. These optimizations are tailored towards observational datasets in NetCDF format and enable a maximal amount of pruning of data directly at the data source -- even for queries, that formulate complex non-convex predicates. We show that the performance of~\system{} scales gracefully with the horizontal and vertical selectivity of the queries and that \system{} outperforms the comparable state-of-the-art pipeline by up to a factor of 6x. 

\textbf{Acknowledgements}:
This work contributes to the project  “Big Data in Atmospheric Physics (BINARY)”, funded by the Carl Zeiss Foundation (grant P2018-02-003).
%Further, parts of this research were conducted using the supercomputer Mogon and/or advisory services offered by Johannes Gutenberg University Mainz~\cite{lit:mogon}, which is a member of the AHRP (Alliance for High Performance Computing in Rhineland Palatinate~\cite{lit:ahrp}) and the Gauss Alliance e.V.
%Further, the authors gratefully acknowledge the computing time granted on the supercomputer Mogon at Johannes Gutenberg University Mainz~\cite{lit:mogon}.






%\subsection{NetCDF Projection Pushdown}
%\label{sec:system:details:projection}
%
%When Spark SQL instructs a relation to instantiate the corresponding RDD, it passes a list of the required columns as an argument.
%To implement projection pushdown, we prune and reorder the internal schema according to this list before passing it to the RDD. \todo{How? More details here.}
%
%Because of projection pushdown, a Spark SQL Relation is free to expose columns redundantly without incurring a performance penalty.
%For example, the ERA5 dataset could expose both the compressed and the decompressed representation of a NetCDF variable \todo{mention compression before}.
%If the programmer only requires one of the two representations, projection pushdown guarantees that the other one will not be computed.
%
%\section{Dataset-specific API}
%
%\subsection{Datasets from Atmospheric Physics}
%\label{sec:system:problem:context}
%
%\todo{Shorten this drastically. Only the high-level idea of these datasets should get through. Only the details required to understand our work must be shown.}
%
%One ongoing field of research in atmospheric physics focuses on the formation of \emph{cirrus clouds} and their impact on the Earth's atmosphere.
%Cirrus clouds can form in so-called \emph{ice supersaturated regions} (ISSRs) which occur frequently in an atmospheric region called the \emph{tropopause}.
%Any research on these ISSRs thus requires atmospheric measurements with sufficient spatial and temporal resolution from inside the tropopause.
%The tropopause occurs at different altitudes depending on the distance to the poles, varying between 9 and 17 kilometers above sea level,
%where reliable data acquisition poses a considerable challenge.
%
%Suitable data can be obtained from so-called \emph{reanalysis datasets} provided by the European Centre for Medium-Range Weather Forecasts (ECMWF).
%A \emph{reanalysis}\cite{lit:ecmwf-datasets} applies contemporary climate models to data measured in the past,
%resulting in climate data estimates on a grid covering the entire surface of the Earth at various altitudes,
%including the tropopause region.
%Data from reanalysis datasets can still be inaccurate depending on the quality of the climate model used.
%
%Another way of collecting measurements in the tropopause region is through sensors installed in passenger aircraft.
%The European research infrastructure \emph{IAGOS}\cite{lit:iagos-website,lit:iagos} provides climate data measurements,
%such as temperature and pressure values, from more than 60,000 flights between 1994 and today.
%This dataset has a superb resolution of around four seconds between measurements, but only for a few hours at a time.
%
%Reutter et al.\cite{lit:ice-supersat} compare the \emph{ERA-Interim}\cite{lit:ecmwf-era-interim} reanalysis dataset with the IAGOS dataset to assess the accuracy of ERA-Interim.
%As a key result of their analysis, they find that IAGOS contains a greater amount of small ice supersaturated regions than ERA-Interim.
%This is most likely due to the limited spatial resolution of ERA-Interim.
%The ECMWF provides a newer version of ERA-Interim with better resolution called \emph{ERA5}\cite{lit:ecmwf-era5}, so running the same analysis on ERA5 could provide better results.
%However, ERA5 is considerably larger than ERA-Interim at about 19.5 terabytes per year of data, making the time span between 2010 and 2019 by itself occupy around 200 terabytes.
%The entire ERA5 dataset ranges from 1950 to today and receives daily updates.
%This dataset is too large to be stored and analyzed on a single machine, and therefore requires a distributed computing approach that we will discuss in this chapter.
%
%
%The IAGOS dataset consists of multiple NetCDF files with every file corresponding to a single flight.
%All measurements are provided in separate one-dimensional variables sharing a single common dimension, namely the time of measurement.
%The time variable stores the number of seconds elapsed since midnight on the day the flight starts, implying that the date offset can differ between files.
%This means we need to convert the time to a unified representation before we can compare flight data to any other dataset.
%Individual measurements in IAGOS are sorted by time, and two adjacent measurements are mostly four, but sometimes five seconds apart,
%most likely because the actual measurement interval is slightly longer than exactly four seconds.
%
%The ERA5 dataset is a collection of various atmospheric parameters originating from a climate reanalysis.
%Data is available at every full hour since January 1950 with time consistently being measured in hours since midnight of January 1, 1900.
%ERA5 arranges parameters on a \emph{regular gaussian grid}\cite{lit:gaussian-grid}, which is a rectangular latitude/longitude grid, using a resolution of a 0.25 degrees in both dimensions.
%Therefore, ERA5 provides 721 by 1440 surface reference points, where the distance between adjacent points on the same longitude decreases as they get closer to the poles.
%While these three dimensions suffice for surface parameters, non-surface parameters also depend on their corresponding altitude.
%
%ERA5 employs a \emph{hybrid sigma-pressure level system}\cite{lit:era5-levels,lit:era5-hybrid}
%for vertical coordinates, with a total number of 137 levels arranged from the top of the atmosphere to the surface of the Earth.
%The level system is designed to correspond to surface topography at low altitudes and gradually transition towards levels of constant pressure at higher altitudes.
%This implies that the atmospheric pressure at a fixed level depends on the surface pressure, which in turn depends on both the time and the horizontal coordinates.
%
%For a fixed time $t$ and fixed horizontal coordinates $lat$ and $lon$, we can compute the pressure at level $lvl$ using
%\begin{equation*}
%    p^{(t, lat, lon)}_{lvl} = a_{lvl} + b_{lvl} * sp^{(t, lat, lon)}
%\end{equation*}
%where $sp^{(t, lat, lon)}$ denotes the surface pressure.
%$a_{lvl}$ and $b_{lvl}$ are pre-defined constants called \emph{hybrid coefficients}.
%
%ERA5 parameters are stored in NetCDF variables and partitioned into multiple NetCDF files by their time dimension.
%ERA5 also separates surface parameters and atmospheric parameters into different files, therefore a NetCDF file never contains three-dimensional and four-dimensional variables at the same time.
%However, every file still contains coordinate variables that are strictly one-dimensional.
%
%Furthermore, ERA5 compresses parameter values into 16-bit integers.
%To decompress a value, we need to convert the value into a floating-point number, scale it by a constant and then add another constant to the result.
%These constants differ for every parameter and are stored as attributes inside each variable of the NetCDF file.
%Even with this form of compression, a single four-dimensional NetCDF file from the ERA5 dataset still has a size of around 2.5 gigabytes.
%
%Ultimately, given the large size of the datasets,
%the fact that both datasets are updated continuously,
%the inconsistent offsets in IAGOS files,
%and the non-standardized way of compression in ERA5 files,
%we argue that our input data effectively satisfies all of Laney's three Vs of Big Data (\emph{Volume}, \emph{Velocity}, and \emph{Variety}).



%\begin{figure}
%\hspace*{-0.3cm}
%\subfloat[Prior to decomposition, global blocks can overlap each other and span multiple files.]{
%        \begin{tikzpicture}[scale=0.55]
%
%            \fill[grayshade, rounded corners=3pt] (0.6,0.6) rectangle ++(1.8,5.8);
%            \node[below,gray] at (1.5, 0.6) {\small\textbf{FILE 1}};
%            \fill[grayshade, rounded corners=3pt] (2.6,0.6) rectangle ++(2.8,5.8);
%            \node[below,gray] at (4.0, 0.6) {\small\textbf{FILE 2}};
%            \fill[grayshade, rounded corners=3pt] (5.6,0.6) rectangle ++(1.8,5.8);
%            \node[below,gray] at (6.5, 0.6) {\small\textbf{FILE 3}};
%
%            \draw[line width=1.5pt, rounded corners=3pt] (1.6,1.6) rectangle ++(5.8,2.8);
%            \draw[line width=1.5pt, rounded corners=3pt] (3.6,0.6) rectangle ++(1.8,5.8);
%
%            \foreach \row in {1, ..., 6} {
%            \foreach \col in {1, ..., 7} {
%            \filldraw [black] (\col, \row) circle (2.5pt);
%            }
%            }
%        \end{tikzpicture}
%}\quad
%\subfloat[Global and local blocks after decomposition. Local blocks never span multiple files and cannot exceed the maximum size.]{
%        \begin{tikzpicture}[scale=0.55]
%
%            \fill[grayshade, rounded corners=3pt] (0.6,0.6) rectangle ++(1.8,5.8);
%            \node[below,gray] at (1.5, 0.6) {\small\textbf{FILE 1}};
%            \fill[grayshade, rounded corners=3pt] (2.6,0.6) rectangle ++(2.8,5.8);
%            \node[below,gray] at (4.0, 0.6) {\small\textbf{FILE 2}};
%            \fill[grayshade, rounded corners=3pt] (5.6,0.6) rectangle ++(1.8,5.8);
%            \node[below,gray] at (6.5, 0.6) {\small\textbf{FILE 3}};
%
%            \fill[lightgrayshade, rounded corners=3pt] (1.75,1.75) rectangle ++(0.5,2.5);
%            \fill[lightgrayshade, rounded corners=3pt] (2.75,1.75) rectangle ++(2.5,0.5);
%            \fill[lightgrayshade, rounded corners=3pt] (2.75,2.75) rectangle ++(2.5,1.5);
%            \fill[lightgrayshade, rounded corners=3pt] (5.75,1.75) rectangle ++(1.5,2.5);
%            \fill[lightgrayshade, rounded corners=3pt] (3.75,4.75) rectangle ++(1.5,1.5);
%            \fill[lightgrayshade, rounded corners=3pt] (3.75,0.75) rectangle ++(1.5,0.5);
%
%            \draw[line width=1.5pt, rounded corners=3pt] (1.6,1.6) rectangle ++(5.8,2.8);
%            \draw[line width=1.5pt, rounded corners=3pt] (3.6,4.6) rectangle ++(1.8,1.8);
%            \draw[line width=1.5pt, rounded corners=3pt] (3.6,0.6) rectangle ++(1.8,0.8);
%
%            \foreach \row in {1, ..., 6} {
%            \foreach \col in {1, ..., 7} {
%            \filldraw [black] (\col, \row) circle (2.5pt);
%            }
%            }
%
%        \end{tikzpicture}
%}
%\caption{Visualization of the relation between files (dark gray), local blocks (light gray), and global blocks (black outline).}
%\end{figure}


%\begin{figure}
%   \subfloat[Starting configuration of logical blocks]{
%        \begin{tikzpicture}[scale=0.3]
%            % rectangles
%            \draw [line width=1pt] (2, 0) rectangle ++(4, 3);
%            \draw [line width=1pt] (5, 1) rectangle ++(3, 6);
%            \draw [line width=1pt] (0, 2) rectangle ++(4, 2);
%            \draw [line width=1pt] (1, 5) rectangle ++(3, 3);
%
%            % dashes
%            \foreach \x in {0, 1, 2, 4, 5, 6, 8} {
%            \draw [line width=1pt] (\x, 9) -- (\x, 9.2);
%            }
%            \foreach \y in {0, 1, 2, 3, 4, 5, 7, 8} {
%            \draw [line width=1pt] (9, \y) -- (9.2, \y);
%            }
%        \end{tikzpicture}
%        \label{fig:blockmerge-start}
%    }
%    \subfloat[Visualization of the total area covered by the blocks in \ref{fig:blockmerge-start}]{
%        \begin{tikzpicture}[scale=0.3]
%            % outline
%            \draw [line width=1pt] (0, 3) -- ++(0, 1) -- ++(4, 0) -- ++(0, -1) -- ++(1, 0) -- ++(0, 4) -- ++(3, 0) -- ++(0, -6) -- ++(-2, 0) -- ++(0, -1) -- ++(-4, 0) -- ++(0, 2) -- ++(-2, 0) -- ++(0, 1);
%
%            \draw [line width=1pt] (1, 5) rectangle ++(3, 3);
%
%            % dashes
%            \foreach \x in {0, 1, 2, 4, 5, 6, 8} {
%            \draw [line width=1pt] (\x, 9) -- (\x, 9.2);
%            }
%            \foreach \y in {0, 1, 2, 3, 4, 5, 7, 8} {
%            \draw [line width=1pt] (9, \y) -- (9.2, \y);
%            }
%        \end{tikzpicture}
%        \label{fig:blockmerge-area}
%    }\quad
%
%    \subfloat[Sub-blocks after splitting every block at each distinct interval boundary. Hatched areas indicate overlapping sub-blocks.]{
%        \begin{tikzpicture}[scale=0.3]
%            % rectangles
%            \draw [line width=1pt] (2, 0) rectangle ++(4, 3);
%            \draw [line width=1pt] (5, 1) rectangle ++(3, 6);
%            \draw [line width=1pt] (0, 2) rectangle ++(4, 2);
%            \draw [line width=1pt] (1, 5) rectangle ++(3, 3);
%
%            % redundant rectangles
%            \draw [pattern=north west lines] (2, 2) rectangle ++(2, 1);
%            \draw [pattern=north west lines] (5, 1) rectangle ++(1, 2);
%
%            % horizontal lines
%            \draw [line width=1pt] (2, 1) -- (5, 1);
%            \draw [line width=1pt] (4, 2) -- (8, 2);
%            \draw [line width=1pt] (0, 3) -- (2, 3);
%            \draw [line width=1pt] (6, 3) -- (8, 3);
%            \draw [line width=1pt] (5, 4) -- (8, 4);
%            \draw [line width=1pt] (5, 5) -- (8, 5);
%            \draw [line width=1pt] (1, 7) -- (4, 7);
%
%            % vertical lines
%            \draw [line width=1pt] (1, 2) -- (1, 4);
%            \draw [line width=1pt] (2, 3) -- (2, 4);
%            \draw [line width=1pt] (2, 5) -- (2, 8);
%            \draw [line width=1pt] (4, 0) -- (4, 2);
%            \draw [line width=1pt] (5, 0) -- (5, 1);
%            \draw [line width=1pt] (6, 3) -- (6, 7);
%
%            % dashes
%            \foreach \x in {0, 1, 2, 4, 5, 6, 8} {
%            \draw [line width=1pt] (\x, 9) -- (\x, 9.2);
%            }
%            \foreach \y in {0, 1, 2, 3, 4, 5, 7, 8} {
%            \draw [line width=1pt] (9, \y) -- (9.2, \y);
%            }
%        \end{tikzpicture}
%        \label{fig:blockmerge-fullsplit}
%    }\quad
%    
%     \subfloat[\ref{fig:blockmerge-fullsplit} after merging sub-blocks horizontally, then vertically where possible.]{
%        \begin{tikzpicture}[scale=0.3]
%            % rectangles
%            \draw [line width=1pt] (2, 0) rectangle ++(4, 1);
%            \draw [line width=1pt] (2, 1) rectangle ++(6, 1);
%            \draw [line width=1pt] (0, 2) rectangle ++(8, 1);
%            \draw [line width=1pt] (0, 3) rectangle ++(4, 1);
%            \draw [line width=1pt] (5, 3) rectangle ++(3, 4);
%            \draw [line width=1pt] (1, 5) rectangle ++(3, 3);
%
%            % dashes
%            \foreach \x in {0, 1, 2, 4, 5, 6, 8} {
%            \draw [line width=1pt] (\x, 9) -- (\x, 9.2);
%            }
%            \foreach \y in {0, 1, 2, 3, 4, 5, 7, 8} {
%            \draw [line width=1pt] (9, \y) -- (9.2, \y);
%            }
%        \end{tikzpicture}
%        \label{fig:blockmerge-horizontal}
%    }\quad
%
%     \subfloat[\ref{fig:blockmerge-fullsplit} after merging sub-blocks vertically, then horizontally where possible]{
%        \centering
%        \begin{tikzpicture}[scale=0.3]
%            % rectangles
%            \draw [line width=1pt] (0, 2) rectangle ++(2, 2);
%            \draw [line width=1pt] (1, 5) rectangle ++(3, 3);
%            \draw [line width=1pt] (2, 0) rectangle ++(2, 4);
%            \draw [line width=1pt] (4, 0) rectangle ++(1, 3);
%            \draw [line width=1pt] (5, 0) rectangle ++(1, 7);
%            \draw [line width=1pt] (6, 1) rectangle ++(2, 6);
%
%            % dashes
%            \foreach \x in {0, 1, 2, 4, 5, 6, 8} {
%            \draw [line width=1pt] (\x, 9) -- (\x, 9.2);
%            }
%            \foreach \y in {0, 1, 2, 3, 4, 5, 7, 8} {
%            \draw [line width=1pt] (9, \y) -- (9.2, \y);
%            }
%        \end{tikzpicture}
%        \label{fig:blockmerge-vertical}
%    }
%    \quad
%     \subfloat[\ref{fig:blockmerge-start} after decomposing only the block in the bottom center.]{
%        \begin{tikzpicture}[scale=0.3]
%            % rectangles
%            \draw [line width=1pt] (5, 1) rectangle ++(3, 6);
%            \draw [line width=1pt] (0, 2) rectangle ++(4, 2);
%            \draw [line width=1pt] (1, 5) rectangle ++(3, 3);
%
%            % decomposed rectangle
%            \draw [line width=1pt] (2, 0) rectangle ++(4, 1);
%            \draw [line width=1pt] (2, 1) rectangle ++(3, 1);
%            \draw [line width=1pt] (4, 2) rectangle ++(1, 1);
%
%            % dashes
%            \foreach \x in {0, 1, 2, 4, 5, 6, 8} {
%            \draw [line width=1pt] (\x, 9) -- (\x, 9.2);
%            }
%            \foreach \y in {0, 1, 2, 3, 4, 5, 7, 8} {
%            \draw [line width=1pt] (9, \y) -- (9.2, \y);
%            }
%        \end{tikzpicture}
%        \label{fig:blockmerge-smart}
%    }
%    
%    \caption{Visualization of logical block decomposition. The black dashes at the top and on the right denote the distinct interval boundaries along each dimension.}
%    \label{fig:blockmerge}
%\end{figure}


%\textbf{Overlap removal -- strategy 3} \todo{works well with little overlap}: In practice, we only expect a small subset of the blocks to overlap each other.
%Among those overlapping blocks, most of them will likely overlap trivially, i.e.\ one block fully contains another block.
%Therefore, we propose a third strategy that only considers pairs of overlapping blocks.
%For each pair, the strategy splits only one of the two blocks along the boundaries of the other, removing the overlapping part in the process.
%If a block is fully contained within another, it is dropped.
%Again, we take care not to split a block along the fastest-varying dimension unless necessary.
%Afterwards, we attempt to merge the resulting blocks to reduce the total block count.
%Except for the initial overhead of determining which pairs of blocks overlap, the execution time of this strategy entirely depends on the amount of overlapping blocks,
%making it very efficient in most practical situations.
%However, compared to the previous strategies, this strategy does not try to maximize each block's extent along the fastest-varying dimension (see Figure~\ref{fig:blockmerge-smart}).







%
%\subsection{Envelopes}
%\label{sec:system:details:envelopes}
%
%Each file in the IAGOS dataset corresponds to a single flight and therefore only spans a few hours at a time.
%Note that this information does not apply to arbitrary datasets, but is part of our knowledge about the domain.
%We utilize this knowledge by computing \emph{envelopes}, i.e.\ a set of tight lower and upper temporal bounds  \todo{aka zone maps}, for each file.
%When the IAGOS dataset is loaded for the first time, we use a separate RDD implementation to extract the time of the first and last measurement of each flight and save this information to disk. 
%
%The primary use of envelopes is to enable certain join optimizations.
%Whenever we join the IAGOS dataset with another dataset by their time columns,
%we only need to load records from the other dataset whose time value overlaps with a IAGOS flight. \todo{Isn't this also selection pushdown?}
%However, this optimization requires a way to propagate the envelopes through the dataflow graph so that they can be used in join optimization.
%
%Spark SQL implements this form of invariant propagation internally\cite{lit:invariants}, but it does not expose access to this mechanism in its public API.
%To overcome this limitation, we gather all envelopes into a single filter operation that is automatically inserted into the query whenever IAGOS is used as a data source \todo{Show envelope explicitly, example}.
%This filter does not actually remove data because each record is contained in at least one envelope,
%it only serves to inject the filter predicate into Spark SQL's query optimizer.
%Unfortunately, this filter operation also introduces additional execution overhead because Spark now has to evaluate a complex predicate for every record. Still, as we will see, the overhead is compensated by the gain during the join. 
%Even so, depending on the complexity of the rest of the query, this execution overhead might be negligible.
%
%To further increase the effect of reducing load operations, envelopes can be extended to include spatial bounds in addition to temporal bounds.
%However, computing spatial bounds is more complicated because latitude and longitude measurements in the IAGOS dataset are not necessarily monotonic,
%requiring a full traversal of each file to find the smallest and largest value.
%In addition, longitude measurements can ``wrap around'' at a certain point, potentially resulting in unusable maximum and minimum values.
%
%As a final optimization step, we could divide each flight into multiple parts and compute spatiotemporal envelopes for each part separately to obtain a better approximation of the flight path \todo{Remove this?}.
%This is similar to \emph{bounding volume hierarchies}\cite{lit:bvh}, a technique used in computer graphics to speed up intersection tests between triangle sets in three-dimensional space.
%We still need to take special care not to use too many envelopes at once as to not exceed the amount of available memory on the driver node.
%
%\section{Experimental Evaluation}
%
%TODO
%
%%\section{System}
%%\label{sec:system}
%%
%%In this chapter, we discuss the design and implementation of our Spark SQL extension.
%%Following the approach of Spark SQL, we attempt to provide a set of reusable components that do not sacrifice performance unnecessarily by capturing domain-specific programming patterns.
%%Most importantly, we want to find out how we can introduce our domain-specific knowledge into the optimization process.
%%We also want to see if we have to compromise on performance by solving the problem using Spark instead of writing a program from scratch.
%
%
%%\section{Implementation Overview}
%%\label{sec:system:overview}
%%
%%Our implementation extends Spark SQL with domain-specific features and optimizations, allowing us to express the problem as a Spark SQL query.
%%To see why this is a viable approach, let us take a step back and consider what a ``bare-bones'' implementation would look like,
%%i.e.\ an implementation that does not use any form of distributed computing abstraction.
%%
%%For simplicity, we focus on interpolating temperature values, but this procedure can be applied to other values as well.
%%We start with a fixed record $r$ from the IAGOS dataset.
%%\begin{itemize}[nosep]
%%    \item[\textbf{S1}] We find the coordinates of the neighboring ERA5 grid points in all four dimensions, resulting in up to 16 distinct coordinates which might be stored in different files.
%%    \item[\textbf{S2}] For each neighbor of $r$, we obtain the name of the ERA5 file containing it, then locate the physical node that stores this file.
%%    \item[\textbf{S3}] We load the temperature values from the ERA5 files by performing a subarray lookup.
%%    \item[\textbf{S4}] We collect all temperature values from step~\textbf{S3} on a single node.
%%    \item[\textbf{S5}] We interpolate the temperature values to receive our final result.
%%\end{itemize}
%%The resulting temperature value can be compared to the temperature stored in $r$.
%%These steps can be executed in parallel for each record in the IAGOS dataset.
%%If two records share the same neighbors, we only need to load them once.
%%If files are stored redundantly, step~\textbf{S3} can yield multiple physical nodes per file.
%%In this case, we would choose the nodes in a way that balances the total number of lookups per node evenly.
%%
%%However, this assumes we can readily compute the neighbors of $r$, which is not the case.
%%As we discussed before, the level neighbors of $r$ depend on the surface pressure, which is not part of $r$, but can only be obtained from ERA5 surface parameters.
%%This requires us to perform steps \textbf{S1}~through~\textbf{S5} two times in total,
%%first in three dimensions to obtain the surface pressure for $r$,
%%then in four dimensions to obtain the final temperature value for $r$.
%%The neighbors computed by the first iteration can be reused in the second one, provided there is enough memory available to cache them.
%%
%%Implementing these operations requires us to deal with the challenges associated with distributed computing (see Section~\ref{sec:related:absmatter}).
%%Alternatively, using our insights from the last chapter, we could employ a distributed dataflow system to solve this problem.
%%Therefore, let us consider how the temperature interpolation can be expressed as a Spark program.
%%
%%A Spark program requires the datasets to be accessible as RDDs.
%%Spark cannot read NetCDF files natively, but exposes binary files stored on disk or in HDFS through the \code{binaryFiles} primitive.
%%As such, we still need to implement the extraction of records from the file ourselves.
%%Once the datasets are available as RDDs, we can implement the steps listed above in the following way:
%%\begin{itemize}[nosep]
%%    \item[\textbf{S1}] We use Spark's \code{flatMap} operation for mapping $r$ to the coordinates of its neighbors.
%%    \item[\textbf{S2} \& \textbf{S3}] Moving data to a specific physical node cannot be expressed in Spark.
%%        However, we can rephrase subarray lookup as a \code{join} operation by using the grid coordinates as the join key.
%%    \item[\textbf{S4}] We use Spark's \code{groupBy} operation to collect the temperature values for $r$.
%%    \item[\textbf{S5}] We implement interpolation by applying the interpolation function to each group via the \code{map} operation.
%%\end{itemize}
%%Depending on the type of interpolation, steps \textbf{S4}~and~\textbf{S5} can be collapsed into a \code{reduceByKey} operation.
%%Again, these steps need to be performed twice because we need to obtain the surface pressure values first.
%%
%%\begin{figure}
%%    \centering
%%    \includegraphics[width=\columnwidth]{../images/fullquery}
%%    \caption{Dataflow graph for a Spark program comparing IAGOS measurements and ERA5 reanalysis data as described in Section~\ref{sec:system:overview}.
%%             The \code{keyBy} operations that would usually precede each join have been omitted for readability.
%%             Note that the vertex responsible for computing the lat/lon/time neighbors has two successors.}
%%    \label{fig:fullquery}
%%\end{figure}
%%
%%By using Spark, we are able to implement the distributed neighborhood lookup and the subsequent interpolation in very few high-level operations.
%%In addition, this implementation can easily be modified to support data from other sources such as text files.
%%However, since subarray lookups cannot be properly expressed in Spark, we always have to load the entire ERA5 dataset, even though we only need a small part of the dataset.
%%Our implementation tries to counteract this issue by only loading chunks of the dataset that overlap with at least one IAGOS flight.
%%We will discuss this limitation in more detail later on.
%%
%%Each of the required operations can also be expressed in Spark SQL:
%%NetCDF files have a fixed schema and can therefore be exposed as a Spark SQL relation.
%%Spark SQL can also accomodate multidimensional files by representing the file as a one-dimensional sequence of rows where each row includes the original array index for later reference.
%%Steps \textbf{S1}~and~\textbf{S5} can be expressed using Spark SQL's generalized \code{select} operator, and the remaining steps rely on typical SQL constructs.
%%In addition, some of Spark SQL's built-in optimizations would benefit our query as well, especially projection pushdown, predicate pushdown, and join reordering.
%%
%%Therefore, instead of constructing an entirely new domain-specific language for this problem and re-implementing these optimizations,
%%our case study will explore if Spark SQL's existing DSL can be extended to include our domain-specific features.
%%In the process of doing so, we will expose some of Spark SQL's limitations and see how a DSL built from scratch could have worked around them.
%
%
%\section{Related Work}
%\label{sec:system:related}
%
%
%\emph{SciSpark}\cite{lit:scispark} extends Spark with an RDD implementation for multidimensional datasets.
%SciSpark divides every file into multidimensional chunks and assigns them to different RDD partitions along with chunk metadata.
%The metadata includes the extent and offset of the chunk along each dimension.
%Within a partition, SciSpark stores each variable in a separate contiguous array, enabling cache-efficient accesses and vectorized computations.
%Unfortunately, this representation can cause problems when filtering records because a filter is likely to introduce ``holes'' into each chunk.
%Depending on the selectivity of the filter predicate, this can result in sparsely populated partitions.
%These partitions would have to be reorganized into a more compact representation in order to take up less space.
%In contrast to that, our implementation uses a strictly row-wise record representation and therefore does not have this drawback.
%However, our representation less cache-efficient unless we use a one-dimensional vectorized representation (see Section~\ref{sec:related:optauto:vectorization}).
%It is also more redundant since we have to include the coordinates with every record instead of storing an offset once per block.
%Further, there is no indication that SciSpark performs any automatic optimizations.
%As such, the user has to manually specify which variables SciSpark needs to load.
%In contrast, our implementation is able to automatically infer this information from the query and will only load the variables required.
%
%\emph{ClimateSpark}\cite{lit:climatespark} is another Spark extension designed for multidimensional datasets.
%Similar to SciSpark, ClimateSpark introduces a new type of RDD where each partition contains a multidimensional chunk of the dataset, separated by variable.
%However, ClimateSpark applies this concept at storage level by storing chunks in separate files along with metadata describing their offset and extent.
%This results in more fine-grained parallelism and hence better workload distribution.
%In addition, ClimateSpark builds a spatiotemporal index for the chunks to avoid unnecessary read operations.
%Again, ClimateSpark does not include automatic optimizations, and the ranges to be loaded still have to be specified by the user.
%
%ClimateSpark also includes more complex filter operations such as finding all surface coordinates contained in a user-provided two-dimensional polygon.
%Even so, these filters are applied only after a dataset has been loaded from disk and cannot benefit from ClimateSpark's spatiotemporal indices.
%The authors also mention a high-level operation for resampling a dataset through interpolation.
%Unfortunately, we were unable to find any details on how they implemented this operation.
%
%In summary, while there are other implementations available that utilize Spark to process scientific data,
%they exclusively rely on manual optimizations to achieve good query performance.
%
%By comparison, our implementation performs domain-specific optimizations automatically,
%allowing the user to focus on the actual data analysis instead.
%
%\section{Conclusion}
%
%TODO


\bibliographystyle{IEEEtran}
\bibliography{bib-refs}

\end{document}
