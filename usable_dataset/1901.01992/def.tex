
\let\inf\undefined
\DeclareMathOperator*{\inf}{\vphantom{\sup}inf}
\DeclareMathOperator*{\ex}{\mathbb E}
\DeclareMathOperator*{\tr}{tr}
\DeclareMathOperator*{\diag}{diag}
\DeclareMathOperator*{\supp}{supp}
\DeclareMathOperator*{\rank}{rank}
\DeclareMathOperator*{\conv}{conv}
\DeclareMathOperator*{\pr}{\mathbb P}
\DeclareMathOperator*{\sgn}{sgn}


\DeclareBoldMathCommand{\b}{b}
\DeclareBoldMathCommand{\c}{c}
\DeclareBoldMathCommand{\d}{d}
\DeclareBoldMathCommand{\e}{e}
\DeclareBoldMathCommand{\f}{f}
\DeclareBoldMathCommand{\g}{g}
\DeclareBoldMathCommand{\n}{n}
\DeclareBoldMathCommand{\m}{m}
\DeclareBoldMathCommand{\p}{p}
\DeclareBoldMathCommand{\q}{q}
\DeclareBoldMathCommand{\r}{r}
\DeclareBoldMathCommand{\s}{s}

\DeclareBoldMathCommand{\v}{v}
\DeclareBoldMathCommand{\w}{w}
%\DeclareBoldMathCommand{\x}{x}
%\DeclareBoldMathCommand{\y}{y}
\DeclareBoldMathCommand{\z}{z}

\DeclareBoldMathCommand{\A}{A}
\DeclareBoldMathCommand{\B}{B}
\DeclareBoldMathCommand{\C}{C}
\DeclareBoldMathCommand{\D}{D}
\DeclareBoldMathCommand{\F}{F}
\DeclareBoldMathCommand{\G}{G}
\DeclareBoldMathCommand{\H}{H}
\DeclareBoldMathCommand{\I}{I}
\DeclareBoldMathCommand{\J}{J}
\DeclareBoldMathCommand{\K}{K}
\DeclareBoldMathCommand{\L}{L}
\DeclareBoldMathCommand{\M}{M}
\DeclareBoldMathCommand{\P}{P}
\DeclareBoldMathCommand{\Q}{Q}
\DeclareBoldMathCommand{\R}{R}
\DeclareBoldMathCommand{\S}{S}


\DeclareBoldMathCommand{\V}{V}
\DeclareBoldMathCommand{\U}{U}
\DeclareBoldMathCommand{\W}{W}
\DeclareBoldMathCommand{\X}{X}
\DeclareBoldMathCommand{\Y}{Y}
\DeclareBoldMathCommand{\Z}{Z}

\DeclareBoldMathCommand{\ssigma}{\sigma}
\DeclareBoldMathCommand{\SSigma}{\Sigma}
\DeclareBoldMathCommand{\OOmega}{\Omega}

\DeclareBoldMathCommand{\mmu}{\mu}
\DeclareBoldMathCommand{\ones}{1}
\DeclareBoldMathCommand{\zeros}{0}

\let\top\intercal
\newcommand{\Graph}{\mathcal G}
\newcommand{\indicator}[2][]{\mathbf 1\sbr[#1]{#2}}

\newcommand{\norm}[1]{\left\Vert#1\right\Vert}
\newcommand{\set}[2]{\left\{#1|#2\right\}}
\newcommand{\smallnorm}[1]{\|#1\|}
\newcommand{\abs}[1]{\left\vert#1\right\vert}
\newcommand{\supnorm}[1]{\norm{#1}_\infty}
\newcommand{\trace}{\mathop{\rm trace}}
\newcommand{\maxEig}{\lambda_{\mbox{max}}}
%\newcommand{\sgn}{\mathop{\rm sgn}}
\newcommand{\Op}[1]{O\left(#1\right)}



\newcommand{\nat}{\mathbb N}                   

\newcommand{\Prob}[1]{{\mathbb P}\left(#1\right)}    % Probabilities; example: \Prob{X>\eps}<1-\delta
\newcommand{\PP}{{\mathbb P}}                         % Probabilities when we want to control the parenthesis
\newcommand{\EE}{{\mathbb E}}                         % Expectations  when we want to control the parenthesis
\newcommand{\Var}[1]{{\mathrm{Var}}\left[#1\right]}  % Variances
\newcommand{\one}[1]{{\mathbb I}{\{#1\}}}           % Characteristic function
\newcommand{\CX}{\mathcal X}
\newcommand{\CA}{\mathcal A}

\newcommand{\cD}{{\cal D}}
\newcommand{\cZ}{{\cal Z}}
\newcommand{\cX}{{\cal X}}
\newcommand{\cW}{{\cal W}}
\newcommand{\cC}{{\cal C}}
\newcommand{\cA}{{\cal A}}
\newcommand{\cY}{{\cal Y}}
\newcommand{\cT}{{\cal T}}

\newcommand{\GG}{\mathcal{G}}

\newcommand{\eps}{\varepsilon}                       % Nice epsilon

\newcommand{\Reals}{\mathbb R}
\newcommand{\Value}{\mathcal V}
\newcommand{\Regret}{\mathcal R}
\newcommand{\RefSet}{\mathcal U}

\DeclareMathOperator*{\argmin}{arg\,min}
\DeclareMathOperator*{\argmax}{arg\,max}


\newcommand{\TODO}[1]{
\ifmmode
\text{\textcolor{red}{TODO: #1}}
\else
\textcolor{red}{TODO: #1}
\fi
}
