
%\documentclass[conference]{IEEEtran}
% Add the compsoc option for Computer Society conferences.
%
% If IEEEtran.cls has not been installed into the LaTeX system files,
% manually specify the path to it like:
\documentclass[conference]{IEEEtran}
\newcommand\hmmax{0}
\newcommand\bmmax{0}
%\usepackage[ruled,vlined]{algorithm2e}
\usepackage{epsfig,endnotes}
\usepackage{varwidth}
\usepackage{algorithm}
\usepackage{algpseudocode}
\usepackage{balance}
\usepackage{xcolor}
\usepackage{nicefrac}
\usepackage{amsmath}
\usepackage{braket}
\usepackage{bm}
\usepackage{mathtools}
\usepackage{multirow}
\usepackage{bigdelim}
\usepackage{mathtools}
\usepackage{amssymb}
\usepackage{amsfonts}
\usepackage{algorithmicx}
\usepackage{indentfirst}
\usepackage{booktabs}
\let\labelindent\relax
\usepackage{enumitem}
\usepackage{boxedminipage}
\usepackage{float}
\usepackage{graphicx}
\usepackage{caption}
\usepackage{subcaption}
\usepackage[normalem]{ulem}
\usepackage{xpatch}
%\usepackage{MnSymbol}
\usepackage{xspace}
\usepackage{listings}
\usepackage{mathpartir}
\usepackage{makecell}
\usepackage[hyperfootnotes=false]{hyperref}%\usepackage{hyperref}
\usepackage{cleveref}
\usepackage{textcomp}
\usepackage{framed,mdframed}
\usepackage{tablefootnote}

\usepackage{flushend}

\usepackage{verbatim}
\usepackage{fancyvrb}
\usepackage{amsthm}

%\usepackage[pdftex,bookmarks=true,pdfstartview=FitH,colorlinks,linkcolor=blue,filecolor=blue,citecolor=blue,urlcolor=blue,pagebackref=true]{hyperref}
   % \urlstyle{sf}

\usepackage{stmaryrd}

\usepackage{tablefootnote}

\newlength{\saveparindent}
\setlength{\saveparindent}{\parindent}
\newlength{\saveparskip}
\setlength{\saveparskip}{\parskip}
 
\newenvironment{tiretnospace}{%
\begin{list}{\hspace{2pt}\rule[0.5ex]{6pt}{1pt}\hfill}{\labelwidth=15pt%
\labelsep=5pt \leftmargin=20pt \topsep=3pt%
\setlength{\listparindent}{\saveparindent}%
\setlength{\parsep}{\saveparskip}%
\setlength{\itemsep}{0pt}}}{\end{list}}
 
\newenvironment{tiret}{%
\begin{list}{\hspace{2pt}\rule[0.5ex]{6pt}{1pt}\hfill}{\labelwidth=15pt%
\labelsep=5pt \leftmargin=20pt \topsep=3pt%
\setlength{\listparindent}{\saveparindent}%
\setlength{\parsep}{\saveparskip}%
\setlength{\itemsep}{0pt}}}{\end{list}}
 
\newenvironment{enum}{%
\begin{list}{{\rm (\arabic{ctr})}\hfill}{\usecounter{ctr}\labelwidth=17pt%
\labelsep=6pt \leftmargin=23pt \topsep=5pt%
\setlength{\listparindent}{\saveparindent}%
\setlength{\parsep}{\saveparskip}%
\setlength{\itemsep}{3pt} }}{\end{list}}
 
\newenvironment{newenum}{%
\begin{list}{{\rm \arabic{ctr}.}\hfill}{\usecounter{ctr}\labelwidth=17pt%
\labelsep=6pt \leftmargin=23pt \topsep=.5pt%
\setlength{\listparindent}{\saveparindent}%
\setlength{\parsep}{\saveparskip}%
\setlength{\itemsep}{5pt} }}{\end{list}}

\newcommand{\KwIn}[1]{\textbf{Input: }#1\\}
\newcommand{\KwOut}[1]{\textbf{Output: }#1\\}

\newcommand{\tensorflow}{{{TensorFlow}}\xspace}
\newcommand{\tool}{{\textsc{CrypTFlow}}\xspace}
\newcommand{\minion}{{\textsc{MiniONN}}\xspace}
\newcommand{\bonsai}{{\textsc{Bonsai}}\xspace}
\newcommand{\R}{{\mathbb{R}}\xspace}
\newcommand{\mpc}{{MPC}\xspace}

\newtheorem{problem}{Problem}
\newtheorem{theorem}{Theorem}
\newtheorem{conjecture}[theorem]{Conjecture}
\newtheorem{definition}[theorem]{Definition}
\newtheorem{lemma}[theorem]{Lemma}
\newtheorem{proposition}[theorem]{Proposition}
\newtheorem{corollary}[theorem]{Corollary}
\newtheorem{claim}[theorem]{Claim}
\newtheorem{fact}[theorem]{Fact}
\newtheorem{remk}[theorem]{Remark}
\newtheorem{apdxlemma}{Lemma}


\newcommand{\namedref}[2]{\hyperref[#2]{#1~\ref*{#2}}\xspace}
\newcommand{\lemmaref}[1]{\namedref{Lemma}{lem:#1}}
\newcommand{\propref}[1]{\namedref{Proposition}{prop:#1}}
\newcommand{\theoremref}[1]{\namedref{Theorem}{theorem:#1}}
\newcommand{\claimref}[1]{\namedref{Claim}{clm:#1}}
\newcommand{\corolref}[1]{\namedref{Corollary}{corol:#1}}
\newcommand{\figureref}[1]{\namedref{Figure}{fig:#1}}
\newcommand{\tableref}[1]{\namedref{Table}{tbl:#1}}
\newcommand{\equationref}[1]{\namedref{Equation}{eq:#1}}
\newcommand{\defref}[1]{\namedref{Definition}{def:#1}}
\newcommand{\observationref}[1]{\namedref{Observation}{obs:#1}}
\newcommand{\procedureref}[1]{\namedref{Procedure}{proc:#1}}
\newcommand{\importedtheoremref}[1]{\namedref{Imported Theorem}{impthm:#1}}
\newcommand{\informaltheoremref}[1]{\namedref{Informal Theorem}{infthm:#1}}

\newcommand{\sectionref}[1]{\namedref{Section}{sec:#1}}
\newcommand{\appendixref}[1]{\namedref{Appendix}{app:#1}}
\newcommand{\propertyref}[1]{\namedref{Property}{prop:#1}}

\newcommand{\algoref}[1]{\namedref{Algorithm}{algo:#1}}



\newcommand{\TODO}[1]{{{\color{red} TODO: #1}}}
\newcommand{\dg}[1]{{{\color{blue} dg: #1}}}
\newcommand{\old}[1]{{{\color{red} OLD: #1}}}
\newcommand{\cmmt}[1]{{{\color{red} check: #1}}}
\newcommand{\nc}[1]{{{\color{purple} nc: #1}}}
\newcommand{\rs}[1]{{{\color{magenta} rs: #1}}}

\definecolor{mypink}{rgb}{1,0.2,0.4}
\newcommand{\aseem}[1]{{\color{mypink}{[{\footnotesize {\bf Aseem:} { {#1}}}]}}}

\newcommand{\nishant}[1]{{\textcolor{blue}{[{\footnotesize {\bf Nishant:} { {#1}}}]}}}
%\newcommand{\dg}[1]{{\textcolor{blue}{[{\footnotesize {\bf Nishant:} { {#1}}}]}}}
\newcommand{\mayank}[1]{{\textcolor{orange}{[{\footnotesize {\bf Mayank:} { {#1}}}]}}}

\newcommand{\adv}{\mathcal{A}}
\newcommand{\env}{\mathcal{Z}}
\newcommand{\prot}{\Pi}
\newcommand{\real}{\mathsf{REAL}}
\newcommand{\ideal}{\ensuremath{\mathsf{IDEAL}}}
\newcommand{\simu}{\mathcal{S}}
\newcommand{\F}{\mathcal{F}}
\newcommand{\secparam}{\kappa}
\newcommand{\matmul}{\cdot}
\newcommand{\sign}{\mathsf{Sign}}

\newcommand{\mulop}{\times}
%%circuits syntax
\newcommand{\gate}{{g}}
\newcommand{\wire}{w}
\newcommand{\bval}{\widetilde{r}}
\newcommand{\val}{\widetilde{v}}
\newcommand{\crct}{\chi}

\newcommand{\aEnv}{\ensuremath{\kappa}}
\newcommand{\aP}[1]{%
	\ifstrempty{#1}{%
		\ensuremath{\texttt{P}}%
	}%
	{%
		\ensuremath{\texttt{P}_{#1}}%
	}%
}
%\renewcommand{\ttdefault}{cmtt}

\lstset{ % 
    language=C,
    backgroundcolor=\color{white},   
    basicstyle=\footnotesize\ttfamily,
    breakatwhitespace=false,
    breaklines=false,
    belowskip=-0.3cm,
    captionpos=b,                    
    commentstyle=\color{blue}\bfseries,
    deletekeywords={...}, 
    escapeinside={\%*}{*)}, 
    extendedchars=true, 
    %% frame=single,
    keepspaces=true,
    keywordstyle=\color{black},
    keywordstyle=[2]\color{black},
    otherkeywords={*,...,uint,input1,input2,output,in},
    keywords=[2]{private},
    numbers=left,
    numbersep=5pt, 
    numberstyle=\tiny\color{gray}\bfseries, 
    rulecolor=\color{black},
    showspaces=false,
    showstringspaces=false, 
    showtabs=false, 
    stepnumber=2, 
    stringstyle=\color{mymauve},
    tabsize=2, 
    title=\lstname
}

\newcommand{\ls}[1]{{{\ensuremath{{\mathtt{#1}}}}}}

\newcommand{\fmpc}{\F_{\textrm{\tiny{mpc}}}}
\newcommand{\fattest}{\F_{\textrm{\tiny{attest}}}}
\newcommand{\fauth}{\F_{\textrm{\tiny{auth}}}}
\newcommand{\sid}{\mathsf{sid}}
\newcommand{\ircommit}{\mathsf{IRcommit}}
\newcommand{\gcommit}{\mathsf{Commit}}
\newcommand{\reject}{\mathsf{Reject}}
\newcommand{\compute}{\mathsf{Compute}}
\newcommand{\ctr}{\mathsf{ctr}}
\newcommand{\msg}{\mathsf{msg}}
\newcommand{\st}{\mathsf{state}}
\newcommand{\transcript}{\mathsf{Transcript}}
\newcommand{\zo}{\{0,1\}}
\newcommand{\sendmsg}{\mathsf{Message}}
\newcommand{\tee}{{\mathrm{secure~hardware~token}}}
\newcommand{\pk}{\mathsf{pk}}
\newcommand{\sk}{\mathsf{sk}}
\newcommand{\vk}{\mathsf{vk}}
\newcommand{\sksign}{\mathsf{sk}}
\newcommand{\protshtom}{\mathsf{Prot}_{\tiny{\mathsf{malicious}}}(P_1,\cdots,P_n)}
\newcommand{\verify}{\mathsf{Verify}}
\newcommand{\token}{\mathcal{T}}

\newcommand{\share}[3]{\langle #1\rangle^{#2}_{#3}}
\newcommand{\shareval}[2]{\langle #1\rangle^{#2}}
\newcommand{\genshare}[2]{\mathsf{Share}^{#1}(#2)}
\newcommand{\reconst}[2]{\mathsf{Reconst}^{#1}(#2)}
\newcommand{\atrand}{\xleftarrow[\text{}]{\$}}
\newcommand{\bbZ}{\mathbb{Z}}
\newcommand{\bbF}{\mathbb{F}}
\newcommand{\relu}{\mathsf{ReLU}}
\newcommand{\shareconvert}{\mathsf{ShareConvert}}
\newcommand{\computemsb}{\mathsf{ComputeMSB}}
\newcommand{\prm}{p}
\newcommand{\drelu}{\mathsf{DReLU}}
\newcommand{\maxpool}{\mathsf{MaxPool}}
\newcommand{\cryptflow}{\textsc{CrypTFlow}}
\newcommand{\resnet}{{\textsc{ResNet50}}\xspace}
\newcommand{\squeezenet}{\textsc{SqueezeNet}}
\newcommand{\densenet}{{\textsc{DenseNet121}}\xspace}
\newcommand{\alexnet}{\textsc{AlexNet}}
\newcommand{\etal}{{\em et al}.}


% To control widows and orphans
% orphans
%\clubpenalty=9986
% widows
%\widowpenalty=9999
\clubpenalty=9996
\widowpenalty=9999
\brokenpenalty=4991
\predisplaypenalty=10000
\postdisplaypenalty=1549
\displaywidowpenalty=1602

\usepackage{url}
\def\UrlBreaks{\do\/\do-}

\newenvironment{tffbox}{
\begin{figure}[ht!]\small
}{\end{figure}}


%\usepackage{blindtext,titlefoot}
\usepackage{lipsum}

\newcommand\blfootnote[1]{%
  \begingroup
  \renewcommand\thefootnote{}\footnote{#1}%
  \addtocounter{footnote}{-1}%
  \endgroup
}
%\widowpenalty10000
%\clubpenalty10000
%\pagestyle{plain}


% Some very useful LaTeX packages include:
% (uncomment the ones you want to load)


% *** MISC UTILITY PACKAGES ***
%
%\usepackage{ifpdf}
% Heiko Oberdiek's ifpdf.sty is very useful if you need conditional
% compilation based on whether the output is pdf or dvi.
% usage:
% \ifpdf
%   % pdf code
% \else
%   % dvi code
% \fi
% The latest version of ifpdf.sty can be obtained from:
% http://www.ctan.org/tex-archive/macros/latex/contrib/oberdiek/
% Also, note that IEEEtran.cls V1.7 and later provides a builtin
% \ifCLASSINFOpdf conditional that works the same way.
% When switching from latex to pdflatex and vice-versa, the compiler may
% have to be run twice to clear warning/error messages.






% *** CITATION PACKAGES ***
%
%\usepackage{cite}
% cite.sty was written by Donald Arseneau
% V1.6 and later of IEEEtran pre-defines the format of the cite.sty package
% \cite{} output to follow that of IEEE. Loading the cite package will
% result in citation numbers being automatically sorted and properly
% "compressed/ranged". e.g., [1], [9], [2], [7], [5], [6] without using
% cite.sty will become [1], [2], [5]--[7], [9] using cite.sty. cite.sty's
% \cite will automatically add leading space, if needed. Use cite.sty's
% noadjust option (cite.sty V3.8 and later) if you want to turn this off.
% cite.sty is already installed on most LaTeX systems. Be sure and use
% version 4.0 (2003-05-27) and later if using hyperref.sty. cite.sty does
% not currently provide for hyperlinked citations.
% The latest version can be obtained at:
% http://www.ctan.org/tex-archive/macros/latex/contrib/cite/
% The documentation is contained in the cite.sty file itself.






% *** GRAPHICS RELATED PACKAGES ***
%
\ifCLASSINFOpdf
  % \usepackage[pdftex]{graphicx}
  % declare the path(s) where your graphic files are
  % \graphicspath{{../pdf/}{../jpeg/}}
  % and their extensions so you won't have to specify these with
  % every instance of \includegraphics
  % \DeclareGraphicsExtensions{.pdf,.jpeg,.png}
\else
  % or other class option (dvipsone, dvipdf, if not using dvips). graphicx
  % will default to the driver specified in the system graphics.cfg if no
  % driver is specified.
  % \usepackage[dvips]{graphicx}
  % declare the path(s) where your graphic files are
  % \graphicspath{{../eps/}}
  % and their extensions so you won't have to specify these with
  % every instance of \includegraphics
  % \DeclareGraphicsExtensions{.eps}
\fi
% graphicx was written by David Carlisle and Sebastian Rahtz. It is
% required if you want graphics, photos, etc. graphicx.sty is already
% installed on most LaTeX systems. The latest version and documentation can
% be obtained at: 
% http://www.ctan.org/tex-archive/macros/latex/required/graphics/
% Another good source of documentation is "Using Imported Graphics in
% LaTeX2e" by Keith Reckdahl which can be found as epslatex.ps or
% epslatex.pdf at: http://www.ctan.org/tex-archive/info/
%
% latex, and pdflatex in dvi mode, support graphics in encapsulated
% postscript (.eps) format. pdflatex in pdf mode supports graphics
% in .pdf, .jpeg, .png and .mps (metapost) formats. Users should ensure
% that all non-photo figures use a vector format (.eps, .pdf, .mps) and
% not a bitmapped formats (.jpeg, .png). IEEE frowns on bitmapped formats
% which can result in "jaggedy"/blurry rendering of lines and letters as
% well as large increases in file sizes.
%
% You can find documentation about the pdfTeX application at:
% http://www.tug.org/applications/pdftex





% *** MATH PACKAGES ***
%
%\usepackage[cmex10]{amsmath}
% A popular package from the American Mathematical Society that provides
% many useful and powerful commands for dealing with mathematics. If using
% it, be sure to load this package with the cmex10 option to ensure that
% only type 1 fonts will utilized at all point sizes. Without this option,
% it is possible that some math symbols, particularly those within
% footnotes, will be rendered in bitmap form which will result in a
% document that can not be IEEE Xplore compliant!
%
% Also, note that the amsmath package sets \interdisplaylinepenalty to 10000
% thus preventing page breaks from occurring within multiline equations. Use:
%\interdisplaylinepenalty=2500
% after loading amsmath to restore such page breaks as IEEEtran.cls normally
% does. amsmath.sty is already installed on most LaTeX systems. The latest
% version and documentation can be obtained at:
% http://www.ctan.org/tex-archive/macros/latex/required/amslatex/math/





% *** SPECIALIZED LIST PACKAGES ***
%
%\usepackage{algorithmic}
% algorithmic.sty was written by Peter Williams and Rogerio Brito.
% This package provides an algorithmic environment fo describing algorithms.
% You can use the algorithmic environment in-text or within a figure
% environment to provide for a floating algorithm. Do NOT use the algorithm
% floating environment provided by algorithm.sty (by the same authors) or
% algorithm2e.sty (by Christophe Fiorio) as IEEE does not use dedicated
% algorithm float types and packages that provide these will not provide
% correct IEEE style captions. The latest version and documentation of
% algorithmic.sty can be obtained at:
% http://www.ctan.org/tex-archive/macros/latex/contrib/algorithms/
% There is also a support site at:
% http://algorithms.berlios.de/index.html
% Also of interest may be the (relatively newer and more customizable)
% algorithmicx.sty package by Szasz Janos:
% http://www.ctan.org/tex-archive/macros/latex/contrib/algorithmicx/




% *** ALIGNMENT PACKAGES ***
%
%\usepackage{array}
% Frank Mittelbach's and David Carlisle's array.sty patches and improves
% the standard LaTeX2e array and tabular environments to provide better
% appearance and additional user controls. As the default LaTeX2e table
% generation code is lacking to the point of almost being broken with
% respect to the quality of the end results, all users are strongly
% advised to use an enhanced (at the very least that provided by array.sty)
% set of table tools. array.sty is already installed on most systems. The
% latest version and documentation can be obtained at:
% http://www.ctan.org/tex-archive/macros/latex/required/tools/


%\usepackage{mdwmath}
%\usepackage{mdwtab}
% Also highly recommended is Mark Wooding's extremely powerful MDW tools,
% especially mdwmath.sty and mdwtab.sty which are used to format equations
% and tables, respectively. The MDWtools set is already installed on most
% LaTeX systems. The lastest version and documentation is available at:
% http://www.ctan.org/tex-archive/macros/latex/contrib/mdwtools/


% IEEEtran contains the IEEEeqnarray family of commands that can be used to
% generate multiline equations as well as matrices, tables, etc., of high
% quality.


%\usepackage{eqparbox}
% Also of notable interest is Scott Pakin's eqparbox package for creating
% (automatically sized) equal width boxes - aka "natural width parboxes".
% Available at:
% http://www.ctan.org/tex-archive/macros/latex/contrib/eqparbox/





% *** SUBFIGURE PACKAGES ***
%\usepackage[tight,footnotesize]{subfigure}
% subfigure.sty was written by Steven Douglas Cochran. This package makes it
% easy to put subfigures in your figures. e.g., "Figure 1a and 1b". For IEEE
% work, it is a good idea to load it with the tight package option to reduce
% the amount of white space around the subfigures. subfigure.sty is already
% installed on most LaTeX systems. The latest version and documentation can
% be obtained at:
% http://www.ctan.org/tex-archive/obsolete/macros/latex/contrib/subfigure/
% subfigure.sty has been superceeded by subfig.sty.



%\usepackage[caption=false]{caption}
%\usepackage[font=footnotesize]{subfig}
% subfig.sty, also written by Steven Douglas Cochran, is the modern
% replacement for subfigure.sty. However, subfig.sty requires and
% automatically loads Axel Sommerfeldt's caption.sty which will override
% IEEEtran.cls handling of captions and this will result in nonIEEE style
% figure/table captions. To prevent this problem, be sure and preload
% caption.sty with its "caption=false" package option. This is will preserve
% IEEEtran.cls handing of captions. Version 1.3 (2005/06/28) and later 
% (recommended due to many improvements over 1.2) of subfig.sty supports
% the caption=false option directly:
%\usepackage[caption=false,font=footnotesize]{subfig}
%
% The latest version and documentation can be obtained at:
% http://www.ctan.org/tex-archive/macros/latex/contrib/subfig/
% The latest version and documentation of caption.sty can be obtained at:
% http://www.ctan.org/tex-archive/macros/latex/contrib/caption/




% *** FLOAT PACKAGES ***
%
%\usepackage{fixltx2e}
% fixltx2e, the successor to the earlier fix2col.sty, was written by
% Frank Mittelbach and David Carlisle. This package corrects a few problems
% in the LaTeX2e kernel, the most notable of which is that in current
% LaTeX2e releases, the ordering of single and double column floats is not
% guaranteed to be preserved. Thus, an unpatched LaTeX2e can allow a
% single column figure to be placed prior to an earlier double column
% figure. The latest version and documentation can be found at:
% http://www.ctan.org/tex-archive/macros/latex/base/



%\usepackage{stfloats}
% stfloats.sty was written by Sigitas Tolusis. This package gives LaTeX2e
% the ability to do double column floats at the bottom of the page as well
% as the top. (e.g., "\begin{figure*}[!b]" is not normally possible in
% LaTeX2e). It also provides a command:
%\fnbelowfloat
% to enable the placement of footnotes below bottom floats (the standard
% LaTeX2e kernel puts them above bottom floats). This is an invasive package
% which rewrites many portions of the LaTeX2e float routines. It may not work
% with other packages that modify the LaTeX2e float routines. The latest
% version and documentation can be obtained at:
% http://www.ctan.org/tex-archive/macros/latex/contrib/sttools/
% Documentation is contained in the stfloats.sty comments as well as in the
% presfull.pdf file. Do not use the stfloats baselinefloat ability as IEEE
% does not allow \baselineskip to stretch. Authors submitting work to the
% IEEE should note that IEEE rarely uses double column equations and
% that authors should try to avoid such use. Do not be tempted to use the
% cuted.sty or midfloat.sty packages (also by Sigitas Tolusis) as IEEE does
% not format its papers in such ways.





% *** PDF, URL AND HYPERLINK PACKAGES ***
%
%\usepackage{url}
% url.sty was written by Donald Arseneau. It provides better support for
% handling and breaking URLs. url.sty is already installed on most LaTeX
% systems. The latest version can be obtained at:
% http://www.ctan.org/tex-archive/macros/latex/contrib/misc/
% Read the url.sty source comments for usage information. Basically,
% \url{my_url_here}.





% *** Do not adjust lengths that control margins, column widths, etc. ***
% *** Do not use packages that alter fonts (such as pslatex).         ***
% There should be no need to do such things with IEEEtran.cls V1.6 and later.
% (Unless specifically asked to do so by the journal or conference you plan
% to submit to, of course. )


% correct bad hyphenation here
%\hyphenation{op-tical net-works semi-conduc-tor}


\begin{document}
%
% paper title
% can use linebreaks \\ within to get better formatting as desired
\title{\cryptflow: Secure \tensorflow Inference} %Computation for Machine Learning Users}


% author names and affiliations
% use a multiple column layout for up to three different
% affiliations
\author{\IEEEauthorblockN{Nishant Kumar\IEEEauthorrefmark{1}}
\IEEEauthorblockA{Microsoft Research\\
t-niskum@microsoft.com}\\
\IEEEauthorblockN{Divya Gupta}
\IEEEauthorblockA{Microsoft Research\\
digup@microsoft.com}
\and
\IEEEauthorblockN{Mayank Rathee\IEEEauthorrefmark{1}}
\IEEEauthorblockA{Microsoft Research\\
t-may@microsoft.com}\\
\IEEEauthorblockN{Aseem Rastogi}
\IEEEauthorblockA{Microsoft Research\\
aseemr@microsoft.com}
\and
\IEEEauthorblockN{Nishanth Chandran}
\IEEEauthorblockA{Microsoft Research\\
nichandr@microsoft.com}\\
\IEEEauthorblockN{Rahul Sharma}
\IEEEauthorblockA{Microsoft Research\\
rahsha@microsoft.com}
}
%\and
%\IEEEauthorblockN{James Kirk\\ and Montgomery Scott}
%\IEEEauthorblockA{Starfleet Academy\\
%someemail@somedomain.com}}

% conference papers do not typically use \thanks and this command
% is locked out in conference mode. If really needed, such as for
% the acknowledgment of grants, issue a \IEEEoverridecommandlockouts
% after \documentclass

% for over three affiliations, or if they all won't fit within the width
% of the page, use this alternative format:
% 
%\author{\IEEEauthorblockN{Michael Shell\IEEEauthorrefmark{1},
%Homer Simpson\IEEEauthorrefmark{2},
%James Kirk\IEEEauthorrefmark{3}, 
%Montgomery Scott\IEEEauthorrefmark{3} and
%Eldon Tyrell\IEEEauthorrefmark{4}}
%\IEEEauthorblockA{\IEEEauthorrefmark{1}School of Electrical and Computer Engineering\\
%Georgia Institute of Technology,
%Atlanta, Georgia 30332--0250\\ Email: see http://www.michaelshell.org/contact.html}
%\IEEEauthorblockA{\IEEEauthorrefmark{2}Twentieth Century Fox, Springfield, USA\\
%Email: homer@thesimpsons.com}
%\IEEEauthorblockA{\IEEEauthorrefmark{3}Starfleet Academy, San Francisco, California 96678-2391\\
%Telephone: (800) 555--1212, Fax: (888) 555--1212}
%\IEEEauthorblockA{\IEEEauthorrefmark{4}Tyrell Inc., 123 Replicant Street, Los Angeles, California 90210--4321}}




% use for special paper notices
%\IEEEspecialpapernotice{(Invited Paper)}



%\IEEEoverridecommandlockouts
%\makeatletter\def\@IEEEpubidpullup{6.5\baselineskip}\makeatother
%\IEEEpubid{\parbox{\columnwidth}{
%    Network and Distributed Systems Security (NDSS) Symposium 2020\\
 %   23-26 February 2020, San Diego, CA, USA\\
 %   ISBN 1-891562-61-4\\
 %   https://dx.doi.org/10.14722/ndss.2020.23xxx\\
  %  www.ndss-symposium.org
%}
%\hspace{\columnsep}\makebox[\columnwidth]{}}

%\makeatletter
%\def\@copyrightspace{\relax}
%\makeatother

% make the title area
\maketitle

\begin{abstract}\blfootnote{\IEEEauthorrefmark{1} Equal contribution}
We present \cryptflow, a first of its kind system that converts
\tensorflow inference code into Secure
Multi-party Computation (MPC) protocols at the push of a
button. To do this, we build three components. Our first component,
Athos, is an end-to-end compiler from \tensorflow
to a variety of semi-honest MPC protocols. The second
component, Porthos, is an improved semi-honest 3-party protocol
that provides significant speedups for \tensorflow like applications. Finally, to
provide malicious
secure MPC protocols, our third component, Aramis, is a novel
technique that uses hardware with integrity guarantees to convert any semi-honest
MPC protocol into an
MPC protocol that provides malicious security. 
The malicious security of the
protocols output by Aramis relies on integrity of the hardware and semi-honest security of \mpc.
Moreover, our system matches the inference accuracy of plaintext \tensorflow. 

We experimentally demonstrate
the power of our system by showing the secure inference of real-world
neural networks such as \resnet\ and \densenet\ over the
ImageNet dataset with running times of about 30 seconds for
semi-honest security and under two minutes for malicious
security. Prior work in the area of secure inference 
%(SecureML, MiniONN, ABY$^3$, SecureNN, CHET, Gazelle, and  Delphi)
 has been limited to
semi-honest security of small networks over tiny datasets such as MNIST or CIFAR. 
%In contrast, our largest network ({\textsc{ResNet200}}) has over 200 layers, 65 million parameters and over 1000 ImageNet classes.
Even on MNIST/CIFAR, \tool outperforms prior work.


\end{abstract}
%

% IEEEtran.cls defaults to using nonbold math in the Abstract.
% This preserves the distinction between vectors and scalars. However,
% if the conference you are submitting to favors bold math in the abstract,
% then you can use LaTeX's standard command \boldmath at the very start
% of the abstract to achieve this. Many IEEE journals/conferences frown on
% math in the abstract anyway.

% no keywords




% For peer review papers, you can put extra information on the cover
% page as needed:
% \ifCLASSOPTIONpeerreview
% \begin{center} \bfseries EDICS Category: 3-BBND \end{center}
% \fi
%
% For peerreview papers, this IEEEtran command inserts a page break and
% creates the second title. It will be ignored for other modes.
%%\IEEEpeerreviewmaketitle



%%%%%%%%%%%%%%%%%%%%%%%%%%%%%%%%%%%%%%%%%%%%%%%%%%%%%%%%%%%%%%%%%%%%%%%%%%%%%
\clearemptydoublepage
\chapter{Introduction} \label{chap:intro}
%%%%%%%%%%%%%%%%%%%%%%%%%%%%%%%%%%%%%%%%%%%%%%%%%%%%%%%%%%%%%%%%%%%%%%%%%%%%%
%% \mycitation
%% {Start your dissertation with your thesis}
%% {Olivier Danvy, BRICS retreat, January~19, 2007.}
%%%%%%%%%%%%%%%%%%%%%%%%%%%%%%%%%%%%%%%%%%%%%%%%%%%%%%%%%%%%%%%%%%%%%%%%%%%%%
%% It is natural to assume a bound on the quantum memory of adversarial
%% players in a cryptographic protocol.

In the quest for interesting \index{cryptographic model}cryptographic
models, bounding the quantum memory of adversarial players is a great
% (good, useful, fruitful, practical, nice, valuable) 
assumption.

\section{Cryptographic Models and Basic Primitives} \label{sec:cryptomodels}
It is a fascinating art to come up with
\emph{\index{protocol}protocols}\footnote{A protocol consists of
  clear-cut instructions for the participating players.} that achieve a
cryptographic task like encryption, authentication, identification,
voting, secure function evaluation to name just a famous few. To
define a notion of security for such protocols, one needs to specify a
\emph{\index{cryptographic model}cryptographic model}, i.e. an
environment in which the protocol is run. The model states for example
the number of honest and dishonest players, the allowed running time
and amount of memory available to honest and dishonest players, how
dishonest players are allowed to deviate from the protocol, the use of
external resources like (quantum) communication channels or other
already established cryptographic functionalities etc.

While coming up with more and more protocols for different models,
cryptographers realized that some basic \emph{primitives}
(i.e.~precisely defined cryptographic tasks) are useful as
``benchmarks'' of how powerful a particular cryptographic model is.
An example is the two-party primitive \emph{
\index{oblivious transfer}Oblivious Transfer} (\pOT).  It comes in different flavors,
but all of these variants are equivalent in the sense that anyone of
them can be implemented using (possibly several instances of) an
other.  The \emph{\index{oblivious transfer!one-out-of-two}one-out-of-two} variant \OT was originally
introduced by Wiesner around 1970 (but only published much later
in~\cite{Wiesner83}) in the very first paper about quantum
cryptography, and later rediscovered by Even, Goldreich, and Lempel~\cite{EGL82}. It lets a sender Alice transmit two bits to a
receiver Bob who can choose which of them to receive. A secure
implementation of \OT does not allow a dishonest sender to learn which
of the two bits was received and it does not allow a dishonest
receiver to learn any information about the second bit. It was a
surprising insight when Kilian showed that this simple primitive is \emph{complete} for
two-party cryptography \cite{Kilian88}. In other words, a model in
which \OT can be securely implemented allows to implement any
cryptographic functionality between two players\footnote{If the model
  can be reasonably extended to more players, this usually allows to
  implement secure multi-party protocols as well.}. Another variant we
are concerned with in this thesis was introduced by Rabin
\cite{Rabin81} and is hence called \index{oblivious transfer!Rabin}Rabin Oblivious Transfer
(\RabinOT). It is basically a ``secure erasure channel'': the sender
Alice sends a bit which with probability one half is absorbed and with
probability one half finds its way to the receiver Bob. The security
requirements are the following: whatever a dishonest Alice does, she cannot find
out whether the bit was received or not; and whatever a dishonest
receiver does, he does not get any information about the bit with
probability one half.

Yet another basic two-party primitive of interest is \index{bit
  commitment} Bit Commitment (\BC) which allows a player to commit
himself to a choice of a bit $b$ by communicating with a verifier. The
verifier should not learn $b$ (we say the commitment is
\emph{\index{bit commitment!hiding}hiding}), yet the committer can
later choose to reveal $b$ in a convincing way, i.e. only the value
fixed at commitment time will be accepted by the verifier (we say the
commitment is \emph{\index{bit commitment!binding}binding}).  Bit
Commitment is a fundamental building block of virtually every more
complicated cryptographic protocol. Implementing secure \BC with a
secure \OT at hand is not difficult\footnote{To commit to a bit $b$,
  the committer sends random bits of parity $b$ via (several instances
  of) \OT and the verifier picks randomly one of the bits. To open,
  the committer sends all the random bits he was using, the verifier
  checks whether these are consistent with what he received.}. On the
other hand, there are cryptographic models allowing to securely
implement \BC, but not \OT. Moran and Naor gave an example of such a
model by assuming the physical device of a tamper-proof seal~\cite{MN05}.

It is not hard to see that the two security requirements for \BC are
in a sense contradictory, so perfectly secure bit commitment cannot be
implemented ``from scratch'', that is if only error-free communication
is available and there is no limitation assumed on the computing power
and memory of the players. The informal reason for this is that the
hiding property implies that when 0 is committed to, exactly the same
information exchange could have happened when committing to 1.
Hence, even if 0 was actually committed to, the committer could always
compute a complete view of the protocol consistent with having
committed to 1, and pretend that this view was what he had in mind
originally. By the reduction of \BC to \OT follows that also \OT and
many other cryptographic functionalities cannot be perfectly secure
when built from scratch.

One might hope that allowing the protocol to make use of quantum
communication would make a difference. Here, information is stored in
qubits, i.e., in the state of two-level quantum mechanical systems,
such as the polarization state of a single photon. Quantum information
behaves in a way that is fundamentally different from classical
information, enabling, for instance, unconditionally secure key
exchange between two honest players (so-called
\emph{\index{quantum key distribution}Quantum Key Distribution}).
However, in the case of two mutually distrusting parties, we are not
so fortunate: even with quantum communication, unconditionally secure
\BC and \OT remain \index{impossibility!of quantum bit
  commitment}impossible. This is the infamous
impossibility result by Mayers and by Lo and Chau~\cite{Mayers97,LC97}.

For this reason, cryptographers have tried hard to exhibit more
restricted models where these impossibility results do not apply. The
high art in this process is to find assumptions that are as realistic
as possible -- thus only minimally restricting the model, but still
strong enough to allow for implementing interesting functionalities.
There are at least three kinds of possible assumptions, namely
\begin{itemize}
\item bounding the computing power of players,
\item using the noise in the communication channel,
\item exploiting some physical limitation of the adversary, e.g., if
  the size of the available memory is bounded.
\end{itemize}

The first scenario is the basis of many well known solutions based on
plausible but unproven complexity assumptions, such as hardness of
factoring or discrete logarithms. A term often used for such schemes
is ``\index{computational security}computational security'', meaning
that it is \emph{not impossible} for an adversary to behave
dishonestly, but it is \emph{computationally infeasible} for him to do
so. Security proofs are usually done by reduction in the sense that
breaking the security of the protocol would imply solving a hard
problem like factoring the product of two large prime numbers. The
second scenario has been used to construct both \BC and \pOT protocols
in various models for the noise by Cr\'epeau, Kilian, Damg{\aa}rd,
Salvail, Fehr, Morozov, Wolf, and Wullschleger
\cite{CK88,DKS99,DFMS04,CMW04,Wullschleger07}.

The third scenario is the focus of this thesis. In contrast to the
first scenario, we deal with ``\index{unconditional security}unconditional security'' where (depending on the task a
protocol aims to achieve) an adversary has no way whatsoever to gain
illegal information. Proofs are not done by reduction, but we can
prove in information-theoretic terms that except with negligible
probability, the adversary does not learn \emph{any information} that
is meant to remain secret.

\section{Classical Bounded-Storage Model} \label{sec:ClassicalBSMIntro}
In the \index{classical bounded-storage model}classical
bounded-storage model, we assume the players to use classical
error-free communication and to be computationally unbounded, but on the
other hand restrict the size of their memory. In the usual setting,
there is a large random source $R$ (often called the
\emph{\index{randomizer}randomizer}) which all players can access, but
which is too large (or transmitted too quickly) to store as a whole.
One can think of $R$ as a deep-space radio source or a satellite broadcasting
random bits at a very high rate.

When Maurer introduced the classical bounded-storage model
in~\cite{Maurer90}, the goal was \emph{secure message transmission}.
He showed that two honest parties Alice and Bob sharing an initial key
can \index{key expansion}expand that key unless the eavesdropper Eve
can store more than a large fraction of the randomizer.  The basic
idea of the technique allowing Alice and Bob to get an advantage
over Eve is that their initial secret key indexes some positions in
the randomizer about which Eve has some uncertainty if she cannot
store the whole randomizer. Therefore, the bits at these positions can
be combined to yield more secure key bits and so to expand the initial
key.

A line of subsequent work by Maurer, Cachin, Aumann, Ding, Rabin,
Dziembowski, Lu, and Vadhan \cite{Maurer92, CM97, ADR02, DM04, Lu04,
  Vadhan04} improved this original protocol in terms of efficiency and
security. Aumann, Ding and Rabin~\cite{ADR02} noticed that
protocols in this model enjoy the property of ``\index{everlasting
security}everlasting security'' in the sense that the newly
generated key remains secure even when the initial key is later
revealed and Eve is no longer memory-bounded, under the sole condition
that the original randomizer cannot be accessed any
more. Ding~\cite{Ding05} showed how to do 
\index{error correction!classical bounded-storage model}error correction in the
bounded-storage model and therefore how to cope with the situation
when the honest parties do not have exactly the same view on the
randomizer.

Cachin, Cr\'epeau and Marcil illustrated the power of the
bounded-storage model by exhibiting in~\cite{CCM98} a protocol for
\OT. Ding improved on this \cite{Ding01} and later showed a
constant-round protocol for oblivious transfer in joint work with
Harnik, Rosen and Shaltiel \cite{DHRS04}.

All these protocols are shown secure as long as the adversary's memory
size is at most quadratic in the memory size of the honest players.
Considering the ease and low cost of storing massive amounts of classical
data nowadays, it is questionable how practical such an assumption on the
memory size of the players is. It would be clearly more satisfactory
to have a larger than quadratic separation between the memory size of
honest players and that of the adversary. However, this was shown to
be impossible by Dziembowski and Maurer~\cite{DM04}.


\section{Contributions} \label{sec:contributions}
In this section, we give an overview of the contributions of this
thesis.
The results about classical oblivious transfer described in
Chapter~\ref{chap:ClassicalOT} and summarized in
Section~\ref{sec:ClassicalOTReductions} are joint
work with Damg{\aa}rd, Fehr and Salvail~\cite{DFSS06}. 
All other results are based on two papers co-authored with Damg{\aa}rd, Fehr,
Salvail and Renner: \cite{DFSS05} and \cite{DFRSS07}. A journal version of \cite{DFSS05}
is to appear in a special issue of the SIAM Journal of Computing
\cite{DFSS08journal}.


\subsection{Bounded-Quantum-Storage Model}
\index{bounded-quantum-storage model}In this thesis, we study for the
first time protocols where quantum communication is used and we place
a bound on the adversary's {\em quantum} memory size.  There are two
reasons why this may be a good idea: first, if we do not bound the
classical memory size, we avoid the impossibility result of
\cite{DM04}.  Second, the adversary's typical goal is to obtain a
certain piece of classical information that we want to keep hidden
from him. However, if he cannot store all the quantum information that
is sent, he must convert some of it to classical information by
measuring. This may irreversibly destroy information, and we may be
able to arrange it in such a way that the adversary cannot afford to lose
information this way, while honest players can.

It turns out that this can be achieved indeed: we present protocols for
both \BC and \pOT in which $n$ qubits are transmitted, where honest
players need {\em no quantum memory}, but where the adversary must
store at least a large fraction (typically $n/2$ or $n/4$) of the $n$
transmitted qubits to break the protocol. We emphasize that no bound
is assumed on the adversary's computing power, nor on his classical
memory. This is clearly much more satisfactory than the classical
case, not only from a theoretical point of view, but also in practice:
while sending qubits and measuring them immediately as they arrive is
well within reach of current technology, storing even a single qubit
for more than a fraction of a second is a formidable technological
challenge.

Furthermore, we show that our protocols also work in a non-ideal
setting where we allow the quantum source to be imperfect and the
quantum communication to be noisy. We emphasize that what makes \pOT and
\BC possible in our model is not so much the memory bound per se, but
rather the loss of information on the part of the adversary. Indeed,
our results also hold if the adversary's memory device holds an
arbitrary number of qubits, but is imperfect in certain ways.

% underlying my thesis (see first sentence in intro):
All these factors make the assumption of \index{bounded-quantum-storage
model}bounded quantum memory a very attractive cryptographic model.
On one hand, as for the \index{classical bounded-storage
model}classical bounded-storage model, it is simple to work with and
yields beautiful theoretical results. On the other hand, it is much
more reasonable to assume the difficulty of storing quantum
information compared to storing classical one and hence, we are very
close to the physical reality and get schemes that can actually be
implemented!

%This is discussed in more detail in Section~\ref{sec:noisymem}.

\subsection{Characterization of Security of Classical \OT} \label{sec:ClassicalOTReductions}
While the task of formally defining \index{unconditional security}unconditional security of classical protocols for \RabinOT
and \BC is well understood, capturing the security of \OT in
information-theoretic terms is considerably more delicate, as was
pointed out by Cr\'epeau, Savvides, Schaffner and
Wullschleger~\cite{CSSW06}. 
For \OT of bits, it is clear that the
security for a honest sender against a cheating receiver guarantees
that the receiver does not learn any information about the XOR of the
two bits. Somewhat surprisingly, the converse is true as well, not
having any information about the XOR of the two bits sent implies that
we can point at one bit which the dishonest receiver does not know
(given the other).

This idea can be generalized to \OT of strings where the ignorance of
the XOR becomes ignorance of the outcome of all \index{non-degenerate
linear function}Non-Degenerate Linear binary Functions (NDLFs)
applied to the two strings sent. Such a characterization of
\index{sender-security!characterization of}sender-security in terms of
NDLF composes well with \emph{strongly \index{two-universal
    hashing!strongly}two-universal hashing} and hereby yields a
powerful technique to improve the analyses of the standard \index{reduction}reductions
from \OT to weaker variants of \pOT.

As a historical side note, the original motivation for this classical
characterization was the hope that it translates to the quantum
setting and thereby yields a security proof of the \OT scheme in the
\index{bounded-quantum-storage model}bounded-quantum-storage model. We
will point out why this approach does \emph{not} work.


\subsection{Quantum Security Definitions and Protocols}
When the players are allowed to use quantum communication, the output
of a dishonest player is a quantum state even when the protocol
implements a classical primitive. Therefore, security definitions for
\RabinOT, \OT and \BC have to be phrased in quantum terms. As an
easy-to-use \index{composability}composability framework has not yet
been established for quantum protocols\footnote{Some rather
complicated frameworks are known. They have been put forward by Ben-Or and
Mayers \cite{BM04} and Unruh \cite{Unruh02}.}, various \emph{ad-hoc}
security requirements are commonly used. The definitions in this
thesis are the strongest so far proposed, and as they are based on the
(classical) considerations in \cite{CSSW06}, we believe that they are
best suited to provide \emph{sequential composability}.

Most of the presented protocols in the bounded-quantum-storage model
can be cast in a non-interactive form, i.e.~only one party sends
information when doing \pOT, commitment or opening. We show the following.

\medskip
\noindent
{\bf {\em \pOT in the Bounded-Quantum-Storage Model:}} {\em There
  exist non-interactive protocols for \RabinOT and 1-out-of-2
  Oblivious Transfer (\OT[2]) of $\ell$-bit messages, secure in the
  bounded-quantum-storage model against adversaries with
  quantum-memory size at most $n/2- \ell$ for \RabinOT and $n/4 -
  2\ell$ for \OT. Here, $n$ is the number of qubits transmitted in the
  protocol and $\ell$ can be a constant fraction of $n$. Honest
  players need no quantum memory at all.}  \medskip

For the case of bit commitment, the standard definition of the
\index{bit commitment!binding}binding property used in the quantum
setting was introduced by Dumais, Mayers and Salvail~\cite{DMS00}. For
$b \in \set{0,1}$, let $p_b$ denote the probability that a dishonest
committer successfully opens the commitment to value $b$. The binding
condition then requires that the sum of $p_0$ and $p_1$ does
essentially not exceed 1. More formally, $p_0 + p_1 \leq 1 + \negl{n}$
where $\negl{n}$ stands for a term which is negligible in $n$ such as
$2^{-cn}$ (for a constant $c>0$) which is exponentially small in $n$.
This is to capture that a quantum committer can always commit to the
values $0$ and $1$ in superposition. We call this notion \emph{weakly
  binding} in the following. A shortcoming of this notion is that
committing bit by bit is not guaranteed to yield a secure string
commitment---the argument that one is tempted to use requires
independence of the $p_{b}$'s between the different executions, which
in general does not hold.

% Note that this definition of the binding
% condition allows protocols in which with probability one half a
% dishonest committer cannot open the commitment to any of the two
% values and with probability one half, he can open it to the bit of his
% choice.  Clearly, such behavior is not what one intuitively expects
% from a commitment scheme.

Instead, we propose the following \emph{strong binding} condition:
After the commitment phase, there exists a binary random variable $D
\in \set{0,1}$ such that a dishonest committer cannot open the
commitment to value $D$ except with negligible probability. The point
is that the distribution of $D$ is not under control of the dishonest
committer. We will point out that using this definition, we can easily
derive the security of a string commitment from the security of the
individual bits.

\medskip
\noindent
{\bf {\em \BC in the Bounded-Quantum-Storage Model:}} {\em There
  exists a protocol for bit commitment which is non-interactive.
 It is perfectly hiding and weakly binding in the
  bounded-quantum-storage model against dishonest committers with
  quantum-memory size at most $n/2$. It is strongly binding against
  memory sizes of at most $n/4$. Here, $n$ is the number of qubits
  transmitted in the protocol. Honest players need no quantum memory
  at all.}  \medskip


Furthermore, the commitment protocol has the interesting property that
the only message is sent \emph{to} the committer, i.e., it is possible
to commit while only {\em receiving} information.  Such a scheme
clearly does not exist without a bound on the committer's memory, even
under computational assumptions and using quantum communication: a
corrupt committer could always store (possibly quantumly) all the
information sent, until opening time, and only then follow the honest
committer's algorithm to figure out what should be sent to
convincingly open a 0 or a~1.  

Note that in the \index{classical bounded-storage model}classical
bounded-storage model, it has been shown by Moran, Shaltiel and
Ta-Shma~\cite{MST04} how to do \index{time-stamping}time-stamping
that is non-interactive in our sense: a player can time-stamp a
document while only receiving information.  However, no reasonable
protocol for \BC or for time-stamping a single bit exists in this
model.  It is straightforward to see that any such protocol can be
broken by an adversary with classical memory of size twice that of an
honest player, while our protocol requires no quantum memory for the
honest players and remains secure against any adversary unable to
store more than half the size of the quantum transmission.

We also note that it has been shown earlier by Salvail
\cite{Salvail98} that \BC is possible using quantum communication,
assuming a different type of physical limitation, namely a bound on
the size of coherent measurement that can be implemented. This
limitation is incomparable to ours: it does not limit the total size
of the memory, instead it limits the number of bits that can be
simultaneously operated on to produce a classical result. Our
adversary has a limit on the total quantum memory size, but can
measure all of it coherently. The protocol from \cite{Salvail98} is
interactive, and requires a bound on the maximal measurement size that
is sub-linear in $n$.

% Finally, our techniques imply security of the
% practical BB84-based protocol for \ROT\ introduced in \cite{DFSS05} with
% a memory bound twice that obtained therein. %\cite{DFSS05}.


\subsection{Quantum Uncertainty Relations}
\index{uncertainty relation}A problem often encountered in
\index{quantum cryptography}quantum cryptography is the following:
through some interaction between the players, a quantum state is
generated and then measured by one of the players (we call her Alice
in the following). Assuming Alice is honest, we want to know how
unpredictable her measurement outcome is to the adversary.  Once a
lower bound on the adversary's uncertainty about Alice's measurement
outcome is established, it is usually easy to prove the desired
security property of the protocol. Many existing constructions in
quantum cryptography have been proven secure following this paradigm.

Typically, Alice does not make her measurement in a fixed basis, but
chooses at random from a set of different bases. These bases are
usually chosen to be pairwise {\em \index{mutually unbiased
bases}mutually unbiased}, meaning that if the quantum state is
such that the measurement outcome in one basis is fixed, then this
implies that the uncertainty about the outcome of the measurement in
the other basis is maximal. In this way, one hopes to keep the
adversary's uncertainty high, even if the state is (partially) under
the adversary's control.

An inequality that lower bounds the adversary's uncertainty in such a
scenario is called an {\em uncertainty relation}.  There exist
uncertainty relations for different measures of uncertainty but
cryptographic applications typically require the adversary's
\index{entropy!min-}min-entropy to be bounded from below. Such uncertainty relations are
the key ingredient in the security proofs of our protocols in the
bounded-quantum-storage model.

In this thesis, we introduce new general and tight high-order entropic
uncertainty relations. Since the relations are expressed in terms of
lower bounds on the min-entropy or upper-bounds on large probabilities
respectively, they are applicable to a large class of natural
protocols in quantum cryptography.

The first uncertainty relation is concerned with the situation where a
\smash{$n$-qubit} state $\rho$ is measured in one out of two mutually
unbiased bases, say either in the
\index{basis!computational}computational basis (the $+$-basis) or in
the \index{basis!diagonal}diagonal basis (the $\times$-basis).

\medskip
\noindent
{\bf {\em First Uncertainty Relation:}} {\em Let $\rho$ be an
  arbitrary state of $n$ qubits, and let $\Qp(\cdot)$ and $\Qt(\cdot)$
  be the respective probability distributions over $\nbit$ of the
  outcome when $\rho$ is measured in the $+$-basis respectively the
  $\times$-basis.  Then, for any two sets $L^+ \subset \set{0,1}^n$
  and $L^{\times} \subset \set{0,1}^n$ it holds that
\[ \Qp(L^+)+\Qt(L^{\times}) \leq 1 + 2^{-n/2} \sqrt{|L^+|
  |L^{\times}|}. \] } 

Another \index{uncertainty relation}uncertainty relation is derived
for situations where an $n$-qubit state $\rho$ has each of its qubits
measured in a random and independent basis sampled uniformly from a
fixed set ${\cal B}$ of bases.  ${\cal B}$ does not necessarily have
to be \index{mutually unbiased bases}mutually unbiased, but we assume
a lower bound $h$---the so-called {\em \index{average entropic
uncertainty bound}average entropic uncertainty bound}---on the
average \index{entropy!Shannon}Shannon entropy of the distribution $P_{\vartheta}$, obtained
by measuring an arbitrary one-qubit state in basis $\vartheta \in
{\cal B}$, meaning that $\frac{1}{|{\cal B}|}\sum_{\vartheta}
\H(P_{\vartheta}) \geq h$.

\medskip
\noindent
{\bf {\em Second Uncertainty Relation (informal):}} {\em Let $\cal B$ be a
  set of bases with an average entropic uncertainty bound $h$ as
  above.  Let $P_{\theta}$ denote the probability distribution defined
  by measuring an arbitrary $n$-qubit state $\rho$ in basis $\theta
  \in {\cal B}^n$. For a uniform choice $\Theta \in_R {\cal B}^n$,
  it holds except with negligible probability (over $\Theta$ and over
  $P_{\theta}$) that %for every $\theta$, we have
\begin{equation}\label{main}
\hmin(P_\theta \mid \Theta=\theta) \gtrsim n h.
\end{equation}
} \medskip 

Observe that (\ref{main}) cannot be improved significantly since the
min-entropy of a distribution is at most equal to the Shannon entropy.
Our uncertainty relation is therefore asymptotically tight when the
bound $h$ is tight.

Any lower bound on the Shannon entropy associated to a set of
measurements ${\cal B}$ can be used in (\ref{main}).  In the special
case where the set of bases is ${\cal B}=\{+,\times\}$ (i.e. the two
BB84 bases named after Bennett and Brassard who used them in the first
quantum-key-distribution protocol~\cite{BB84}), 
$h$ is known precisely using Maassen and Uffink's
entropic relation~\cite{MU88}, see~(\ref{eq:maassenuffink}).  We
get $h=\frac{1}{2}$ and (\ref{main}) results in $\hmin(P_{\theta} \mid
\Theta=\theta) \gtrsim \frac{n}{2}$.
%% For ${\cal B}=\{+,\times,\oslash\}$ where 
%% $\oslash$ denotes the circular
%% basis, $h=\frac{2}{3}$ has also been shown tight in \cite{Ruiz93}.
%% In this case, (\ref{main}) provides 
%% $\hmin(P_{\theta} \mid {\cal E}_{\theta}) \geq \frac{2(1-\delta)n}{3}$.
Uncertainty relations for the \index{BB84 coding scheme}BB84 coding
scheme are useful, since this coding is widely used in
quantum cryptography.  Its resilience to imperfect quantum channels,
sources, and detectors is an important advantage in practice.

A major difference between the first and second uncertainty relation
is that while both relations can be used to bound the min-entropy
conditioned on an event, this event happens in the latter case with
probability essentially 1 (on average) whereas the corresponding event
from the first relation (defined in Corollary~\ref{cor:hadamard}) only
happens with probability about $1/2$.


\subsection{\QKD against Quantum-Memory-Bounded Eavesdropper}
We illustrate the versatility of our second uncertainty relation by
applying it to Quantum-Key-Distribution (\QKD) settings.  \QKD is the
art of distributing a secret key between two distant parties, Alice
and Bob, using only a completely insecure quantum channel and
authentic classical communication. \QKD protocols typically provide
\index{unconditional security}unconditional security, i.e., even an
adversary with unlimited resources cannot get any information about
the key.  A major difficulty when implementing \QKD schemes is that
they require a low-noise quantum channel.  The tolerated noise level
depends on the actual protocol and on the desired security of the key.
Because the quality of the channel typically decreases with its
length, the maximum tolerated noise level is an important parameter
limiting the maximum distance between Alice and Bob.

We consider a model in which the adversary has a limited amount of
quantum memory to store the information she intercepts during the
protocol execution. In this model, we show that the maximum
tolerated noise level is larger than in the standard scenario where
the adversary has unlimited resources.  
For {\em one-way \QKD protocols} which are protocols where error-correction is
performed non-interactively (i.e., a single classical message is sent
from one party to the other), we show the following result:

\medskip
\noindent
{\bf {\em \QKD Against Quantum-Memory-Bounded Eavesdroppers:}} {\em Let
  $\cB$ be a set of orthonormal bases of the two-dimensional Hilbert space $\cH_2$ with average entropic
  uncertainty bound $h$. Then, a \emph{one-way \QKD-protocol} produces
  a secure key against eavesdroppers whose quantum-memory size is
  sublinear in the length of the raw key at a positive rate, as long as
  the bit-flip probability $p$ of the quantum channel fulfills $h(p)
  < h $ where $h(\cdot)$ denotes the binary Shannon-entropy
  function.  } \medskip

Although this result does not allow us to improve (compared to
unbounded adversaries) the maximum error-rate for the BB84 protocol
(the \index{protocol!4-state}4-state protocol), the
\index{protocol!6-state}6-state (using three mutually unbiased bases)
protocol can be shown secure against adversaries with memory bound
sublinear in the secret-key length as long as the bit-flip error-rate
is less than $17\%$. This improves over the maximal error-rate of
$13\%$ for this protocol against unbounded adversaries.  We also show
that the generalization of the 6-state protocol to more bases (not
necessarily mutually unbiased) can be shown secure for a maximal
error-rate up to $20\%$ provided the number of bases is large enough.
Note that the best known one-way protocol based on qubits is proven
secure against general attacks for an error-rate of only up to roughly
$14.1\%$, and the theoretical maximum is $16.3\%$~\cite{RGK05}.

The quantum-memory-bounded eavesdropper model studied here is not
comparable to other restrictions on adversaries considered in the
literature (e.g. \emph{individual attacks}, where the eavesdropper is
assumed to apply independent measurements to each qubit sent over the
quantum channel as considered by Fuchs, Gisin, Griffiths, Niu, Peres,
and L\"utkenhaus~\cite{FGGNP97,Lutkenhaus00}).  In fact, these
assumptions are generally artificial and their purpose is to simplify
security proofs rather than to relax the conditions on the quality of
the communication channel from which secure key can be generated.  We
believe that the quantum-memory-bounded eavesdropper model is more
realistic.

%% On the technical side, we derive a new type of uncertainty relation
%% involving the min-entropy of a quantum encoding
%% (Theorem~\ref{thm:hadamard}, and
%% Corollary~\ref{cor:hadamard}), which might be useful in other
%% contexts as well.  The new relation is then used in combination with a
%% proof technique by Shor and Preskill \cite{SP00}, where the actions of
%% honest players are purified, and with privacy amplification against
%% quantum adversaries as introduced by Renner and K\"onig~\cite{RK05, Renner05}.


\section{Outline of the Thesis}
In Chapter~\ref{chap:prelim}, we introduce notation and present some
basic concepts from probability and quantum information theory like
quantum states and various kinds of their entropies. We prepare the
stage by reproducing and slightly extending the results about privacy
amplification via two-universal hashing from Renner's PhD thesis \cite{Renner05}.

Chapter~\ref{chap:ClassicalOT} is the only (almost) exclusively
classical chapter. It introduces the different flavors of oblivious
transfer and gives a characterization of the security for the sender
of \OT in terms of non-degenerate linear functions. It is cast in a
stand-alone manner and the rest of the thesis can be understood
without reading this chapter.

In Chapter~\ref{chap:uncertrelations}, the basis for the security
proofs of the following chapters is laid by establishing the quantum
min-entropic uncertainty relations. The following
Chapters~\ref{chap:RabinOT} and \ref{chap:12OT} contain the quantum
definitions, protocols and security proofs for \RabinOT and \OT,
respectively. Chapter~\ref{chap:qbc} treats quantum bit commitment.
Two flavors of the ``binding property'' are defined and the techniques
from the two previous chapters are used to prove security in the
bounded-quantum-storage model.

Chapter~\ref{chap:qkd} is devoted to another application of the
(second) uncertainty relation, quantum key distribution against a
quantum-memory-bounded eavesdropper. The last
Chapter~\ref{chap:conclusions} addresses some practical issues in
greater detail and concludes.

A short summary of the notation, the bibliography and an index
can be found at the end of the thesis.

% Enjoy the ride!

\section{Related Work}
The classical bounded-storage model is described in
Section~\ref{sec:ClassicalBSMIntro}. Besides work pointed out
in the overview of the contributions in
Section~\ref{sec:contributions} above, it is worth mentioning that several protocols aiming at achieving quantum oblivious transfer have been proposed. After Wiesner's original conjugate-coding protocol~\cite{Wiesner83}, Bennett, Brassard, Cr\'epeau, and Skubiszewska proposed an interactive protocol for \OT~\cite{BBCS91}, whose security was subsequently analyzed by Cr\'epeau~\cite{Crepeau94}, Mayers, Salvail~\cite{MS94, Mayers95}, and Yao~\cite{Yao95}. The protocol from \cite{BBCS91} is interactive and can be easily broken by a dishonest receiver with unbounded quantum memory. To ensure that the receiver actually performs a measurement, it was suggested to use (quantum) bit-commitment schemes such as~\cite{BCJL93} which were believed to be secure against such adversaries at this point in time. After the impossibility proofs of quantum bit-commitment by Lo and Chau~\cite{LC97}, and Mayers~\cite{Mayers97}, and of oblivious transfer by Lo~\cite{Lo97}, it became clear that assumptions are necessary in order to securely realize these primitives. Compared to these previous attempts, the protocols in this thesis are simpler, non-interactive, and provably secure according to stronger security definitions.

Work related to classical OT-reductions is referred to in the introductory sections to
Chapter~\ref{chap:ClassicalOT} in Sections~\ref{sec:introtoNDLF}
and~~\ref{sec:reductions}.  Previous work about quantum uncertainty
relations is described in Section~\ref{sec:uncerthistory}.



%% \subsubsection{Classical Bounded-Storage Model}

%% \subsubsection{OT Reductions}
%% \subsubsection{Uncertainty Relations}
%% \subsubsection{QKD}


%% The history of uncertainty relations starts with Heisenberg who showed
%% that the outcomes of two non-commuting observables $A$ and $B$ applied
%% to any state $\rho$ are not easy to predict simultaneously.  However,
%% Heisenberg only speaks about the variance of the measurement results,
%% and his result was shown to have several shortcomings in
%% \cite{HU88,Deutsch83}.  More general forms of uncertainty relations
%% were proposed in \cite{BiMy75} and \cite{Deutsch83} to resolve these
%% problems.  The new relations were called {\em entropic uncertainty
%%   relations}, because they are expressed using Shannon entropy instead
%% of the statistical variance.
%% % Such relations are called {\em entropic uncertainty relations}. 
%% %In addition to evade the problems of (\ref{robertson}), 
%% Entropic uncertainty relations have the advantage of being 
%% pure information theoretic statements. The first 
%% entropic uncertainty relation was introduced by Deutsch\cite{Deutsch83}
%% and stated that
%% %\begin{equation}\label{deutsch}
%%  $\H(P)+\H(Q) \geq -2\log{\frac{1+c}{2}}$,
%% %\end{equation}
%% where $P,Q$ are random variables representing the measurement
%% results and  $c$ is the maximum inner product norm between any eigenvectors
%% of $A$ and $B$. 
%% %Notice that another advantage of (\ref{deutsch}) over
%% %Heisneberg's result is that its right-hand side does not
%% %depend on $\ket{\psi}$ which makes it truly general
%% %and non-trivial as long as $c<1$.
%% First conjectured
%% by Kraus\cite{Kraus87}, Maassen and Uffink\cite{MU88} improved
%% %(\ref{deutsch}) 
%% Deutsch's relation to the optimal
%% \begin{equation}\label{maassenuffink}
%% \H(P)+\H(Q) \geq -2\log{c}.
%% \end{equation}
%% %Relation (\ref{maassenuffink}) 
%% %has been used for locking/unlocking 
%% %classical correlations in quantum states\cite{DHLST03} and for the analysis
%% %of the resilience of quantum symmetric encryption schemes
%% %against known plaintext attacks\cite{DPS04}.

%% Although a bound on Shannon entropy can be helpful
%% in some cases, it is usually not good enough in cryptographic applications. 
%% The main tool
%% to reduce the adversary's information --
%%  privacy amplification\cite{BBR88,ILL89,BBCM95,RK05,Renner05} --
%% only works if a bound on the adversary's min-entropy (in fact
%% collision entropy) is known. 
%% Unfortunately, knowing the Shannon entropy of a distribution
%% does in general not allow
%% to bound its higher order R\'enyi entropies.   

%% An entropic uncertainty relation involving R\'enyi entropy of order
%% $2$ (i.e. {\em collision entropy}) was introduced by
%% Larsen\cite{Larsen90,Ruiz95}.  Larsen's relation quantifies precisely
%% the collision entropy for the set $\{A_i\}_{i=1}^{d+1}$ of \emph{all}
%% maximally non-commuting observables, where $d$ is the dimension of the
%% Hilbert space.  Its use is therefore restricted to quantum coding
%% schemes that take advantage of \emph{all} $d+1$ observables, i.e. to
%% schemes that are difficult to implement in practice.


%% %%%%%%%%%%%%%%%%%%%%%%%%%%%%%%%%%%%%%%%%%%%%%%%%%%%%%%%%%%%%%%%%%%%%%%%%%%%%%
%% \section{Introduction from NDLF paper}
%% %%%%%%%%%%%%%%%%%%%%%%%%%%%%%%%%%%%%%%%%%%%%%%%%%%%%%%%%%%%%%%%%%%%%%%%%%%%%%

%% 1-2 Oblivious Transfer, \OT\ for short, is a two-party primitive which
%% allows a sender to send two bits (or, more generally, strings) $B_0$
%% and $B_1$ to a receiver, who is allowed to learn one of the two
%% according his choice $C$. Informally, it is required that the receiver only
%% learns $B_C$ but not $B_{1-C}$ ({\em obliviousness}), while at the
%% same time the sender does not learn $C$ ({\em privacy}). \OT\ was
%% introduced in~\cite{Wiesner83} under the name of ``multiplexing'' in the
%% context of quantum cryptography, and, inspired by~\cite{Rabin81} where a
%% different flavor was introduced, later re-discovered in~\cite{EGL82}.

%% \OT\ turned out to be very powerful in that it was shown to be sufficient
%% for secure general two-party computation~\cite{Kilian88}. On the other
%% hand, it is quite easy to see that unconditionally secure \OT\ is not
%% possible without any assumption.  Even with the help of quantum
%% communication and computation, unconditionally secure \OT\ remains
%% impossible~\cite{LC97,Mayers97}.  As a consequence, much effort has been
%% put into constructing unconditionally secure protocols for \OT\ using
%% physical assumptions like various models for noisy
%% channels~\cite{CK88,DKS99,DFMS04,CMW04}, or a memory bounded
%% adversary~\cite{CCM98,Din01,DHRS04}.  Similarly, much effort has been
%% put into reducing \OT\ to (seemingly) weaker flavors of \pOT, like
%% \RabinOT, \XOT, etc.~\cite{Crepeau87,BC97,Cachin98,Wolf00,BCW03,CS06}.  

%% In this work, we focus on a slightly modified notion of \OT, which we
%% call {\em Randomized} \OT, \RandOT\ for short, where the bits (or
%% strings) $B_0$ and $B_1$ are not {\em in}put by the sender, but
%% generated uniformly at random during the \RandOT\ and then {\em
%%   out}put to the sender. It is still required that the receiver only
%% learns the bit (or string) of his choice, $B_C$, whereas the sender
%% does not learn any information on $C$. It is obvious that a \RandOT\ can easily be turned
%% into an ordinary \OT\, simply by using the generated $B_0$ and $B_1$
%% to mask the actual input bits (or strings). Furthermore, all known
%% constructions of unconditionally secure \OT\ protocols make
%% implicitly the detour via \RandOT. 

%% In a first step, we observe that the obliviousness condition of a
%% \RandOT\ of {\em bits} is equivalent to requiring the XOR $B_0 \oplus
%% B_1$ to be (close to) uniformly distributed from the receiver's point
%% of view. The proof is very simple, and it is kind of surprising
%% that---to the best of our knowledge---this has not been realized
%% before. We then ask and answer the question whether there is a natural
%% generalization of this result to \RandOT\ of {\em strings}. Note that
%% requiring the bitwise XOR of the two strings to be uniformly
%% distributed is obviously not sufficient. We show that the
%% obliviousness condition for \RandOT\ of strings can be characterized
%% in terms of {\em non-degenerate linear functions} (bivariate binary
%% linear functions which non-trivially depend on both arguments, as
%% defined in Definition~\ref{def:linear}): obliviousness holds if and
%% only if the result of applying any non-degenerate linear function to
%% the two strings is (close to) uniformly distributed from the
%% receiver's point of view.

%% We then show the usefulness of this new understanding of \OT. We
%% demonstrate this on the problem of reducing \OT\ to weaker primitives.
%% Concretely, we show that the reducibility of an ordinary \OT\ to
%% weaker flavors via a non-interactive reduction follows by a trivial
%% argument from our characterization of the obliviousness condition.
%% This is in sharp contrast to the current literature: The proofs given
%% in~\cite{BC97,Wolf00,BCW03} for reducing \OT\ to \XOT, \GOT\ and \BUOT\ 
%% (we refer to Section~\ref{sec:application} for a description of these
%% flavors of \pOT) are rather complicated and tailored to a particular
%% class of privacy-amplifying hash functions; whether the reductions
%% also work for a less restricted class is left as an open
%% problem~\cite[page~222]{BCW03}. And, the proof given in~\cite{Cachin98}
%% for reducing \OT\ to one execution of a general \pUOT\ is not only
%% complicated, but also incorrect, as we will point out.  Thus, our
%% characterization of the obliviousness condition allows to simplify
%% existing reducibility proofs and, along the way, to solve the open
%% problem posed in~\cite{BCW03}, as well as to improve the reduction
%% parameters in most cases, but it also allows for new, respectively
%% until now only incorrectly proven reductions.  Furthermore, our
%% techniques may be useful for the construction and analysis of \OT\ 
%% protocols in other settings, for instance in a quantum setting as
%% demonstrated in~\cite{DFRSS06}, or for computationally secure \OT\ 
%% with unconditional obliviousness.

%% Finally, we extend our result and show how our characterization of
%% \RandOT\ in terms of non-degenerate linear functions translates
%% to \onenOT\ and to \OT in a \emph{quantum setting}. 





%% %% First things to try: restricting players memory or time.




%% %% (but by any mean does not learn anything about the other bit)


%% %% two primitives Bit Commitment
%% %% (BC) and Oblivious Transfer (OT) are the basic building blocks of
%% %% almost all





%% %% This thesis is only concerned with information-theoretical security,
%% %% the strongest reasonable security notion which means that adversarial
%% %% players have 



%% %% In cryptography, we want to assure that it is
%% %% impossible to ``cheat''. We call a protocol secure when there is a
%% %% guarantee for all players that honestly follow the protocol that the
%% %% dishonest players 



%% %% One art of cryptography is to come up with cryptographic protocols



%% %% This thesis opens up a new line of research by introducing a new
%% %% cryptographic model called the bounded-quantum-storage model.





%% \section{Intro from Rabin \pOT paper}
%% It is well known that non-trivial two-party cryptographic primitives
%% cannot be securely implemented if only error-free communication is
%% available and there is no limitation assumed on the computing power
%% and memory of the players. Fundamental examples of such primitives are
%% bit commitment (\BC) and oblivious transfer (\pOT). In \BC, a committer $\C$
%% commits himself to a choice of a bit $b$ by exchanging information
%% with a verifier $\V$. We want that $\V$ does not learn $b$ (we say the
%% commitment is hiding), yet $\C$ can later choose to reveal $b$ in a
%% convincing way, i.e., only the value fixed at commitment time will be
%% accepted by $\V$ (we say the commitment is binding). In (Rabin) OT, 
%% a sender $\A$
%% sends a bit $b$ to a receiver $\B$ by executing some protocol in such a way that
%% $\B$ receives $b$ with probability $\frac12$ and nothing with probability
%% $\frac12$, yet $\A$ does not learn what was received.


%% Informally, \BC is not possible with unconditional security since
%% hiding means that when 0 is committed, exactly the same information
%% exchange could have happened when committing to a 1. Hence, even if 0
%% was actually committed to, $\C$ could always compute a complete view of
%% the protocol consistent with having committed to 1, and pretend that
%% this was what he had in mind originally. A similar type of argument
%% shows that OT is also impossible in this setting.

%% One might hope that allowing the protocol to make use of quantum
%% communication would make a difference. Here, information is stored in
%% qubits, i.e., in the state of two-level quantum mechanical systems,
%% such as the polarization state of a single photon. It is well known
%% that quantum information behaves in a way that is fundamentally
%% different from classical information, enabling, for instance,
%% unconditionally secure key exchange between two honest players.
%% However, in the case of two mutually distrusting parties, we are not
%% so fortunate: even with quantum communication, unconditionally secure
%% \BC and \pOT remain impossible~\cite{LC97,Mayers97}.







%%% Local Variables: 
%%% mode: latex
%%% TeX-master: "diss"
%%% End: 



\section{Motivating Example}\label{sec:toolchain}

In this section, we describe the end-to-end working of \cryptflow\ through an example of logistic regression. The high-level toolchain is shown in Figure \ref{fig:cryptflowtoolchain}. We describe how code compilation happens from \tensorflow to \mpc protocols. 

\begin{figure}
  \includegraphics[width=\linewidth]{cryptflowtoolchain.pdf}
  \caption{\cryptflow: End-to-end toolchain}
  \label{fig:cryptflowtoolchain}
\end{figure}

%% The developer would write code in \tensorflow, which is the first step
%% of the toolchain.

The \cryptflow\ toolchain takes as input code written in vanilla
\tensorflow. For example, consider the code snippet for
logistic regression over MNIST
dataset in \tensorflow as shown in Figure \ref{fig:lrtf}. 
Our compiler
% compiles this code to \mpc protocols using the following
%sequence of steps. It
 first generates the
\tensorflow graph dump (as shown in Figure \ref{fig:tfGraphDef}) as
well as metadata to help compute the dimensions of all the tensors
(Figure \ref{fig:tfGraphMetadata}). \ref{subsec:athosfrontend}
provides more details on the frontend. Next, the \tensorflow graph
dump is compiled into a high-level intermediate language HLIL. The
code snippet for logistic regression in HLIL is shown in Figure
\ref{fig:lrseedot}. Next, Athos' float-to-fixed converter translates
the floating-point HLIL code to fixed-point code in a low-level
intermediate language LLIL. This step requires Athos to
compute the right precision to be used for maximum accuracy
(Section~\ref{subsec:athosquantizer}).
Figure \ref{fig:lrezpc} shows the
LLIL code snippet for logistic regression. The function calls in this
sequence can be implemented with a variety of secure computation
backends - e.g. ABY~\cite{aby} for the case of 2-party secure
computation, Porthos for the case of semi-honest 3-party secure
computation (Section \ref{sec:porthos}) and Aramis (Section
\ref{sec:aramis}) for the malicious secure variant. Different backends
provide different security guarantees and hence vary in their
performance. For this example, the three backends take
227ms, 6.5ms, and 10.2ms respectively.

\begin{figure}
\small
% \begin{minted}[mathescape,
%                linenos,
%                numbersep=5pt,
%                gobble=2,
%                frame=lines,
%                framesep=2mm]{python}
\begin{Verbatim}[frame=single]
# x is an MNIST image of shape (1,784).
# W and b are the model parameters.

print(tf.argmax(tf.matmul(x, W) + b, 1))
\end{Verbatim}
% \end{minted}
\caption{Logistic Regression: TensorFlow snippet}
\label{fig:lrtf}
\end{figure}


\begin{figure}
  \centering
  \resizebox{0.49\columnwidth}{!}{
    \begin{subfigure}{0.4\columnwidth}
      \centering
      \includegraphics[scale=0.8]{E2E_TF_Graph_Dump.pdf}
      \caption{}
      \label{fig:tfGraphDef}
    \end{subfigure}
  }
  \resizebox{0.38\columnwidth}{!}{
    \begin{subfigure}{0.4\columnwidth}
      \centering
      \def\arraystretch{1.2}
      \begin{tabular}{|p{1.2cm}|p{1.6cm}|}
        \hline
        Node & Outgoing \\ & dimensions \\ \hline \hline
        x & $1 \times 784$\\ \hline
        W & $784 \times 10$\\ \hline
        MatMul & $1 \times 10$ \\\hline
        b & $1 \times 10$ \\\hline
        MatAdd & $1 \times 10$ \\\hline
        ArgMax & $1 \times 1$ \\\hline
      \end{tabular}
    \caption{}
    \label{fig:tfGraphMetadata}
    \end{subfigure}
  }
  \caption{Logistic Regression: (a) \tensorflow graph definition (b) Metadata consisting of graph nodes and their outgoing dimensions}
  \label{fig:lrtfgraphdump}
\end{figure}

\begin{SaveVerbatim}{HLIL_LR_Verbatim}
xW = MatMul(x, W);
xWb = MatAdd(xW, b);
output(ArgMax(xWb));
\end{SaveVerbatim}

\begin{SaveVerbatim}[]{LLIL_LR_Verbatim}
//Assume Athos chooses
//15 bit precision

xW = MatMul(x, W);
ScaleDown(xW, 15);
xWb = MatAdd(xW, b);
output(ArgMax(xWb));
\end{SaveVerbatim}

\begin{figure}
  \centering
  \resizebox{0.47\columnwidth}{!}{
    \begin{subfigure}{0.49\columnwidth}
      \centering
      \setlength{\fboxsep}{1.7mm}
      \fbox{\BUseVerbatim[fontsize=\small]{HLIL_LR_Verbatim}}
      \caption{}
      \label{fig:lrseedot}
    \end{subfigure}
  }
  \resizebox{0.47\columnwidth}{!}{
    \begin{subfigure}{0.49\columnwidth}
      \centering
      \setlength{\fboxsep}{1.6mm}
      \fbox{\BUseVerbatim[fontsize=\small]{LLIL_LR_Verbatim}}
      \caption{}
      \label{fig:lrezpc}
    \end{subfigure}
  }
  \caption{Logistic Regression in (a) floating-point: HLIL syntax (b) fixed-point: LLIL syntax}
\end{figure}

%\section{Preliminaries}\label{subsec:preliminaries}

\subsection{Neural Networks} A Deep Neural Network (DNN) consists of multiple layers each performing a specific operation on the layer's input (typically a matrix) to prepare the output for the next layer. Some of these layers like Convolution and MaxPool employ stencil codes to sweep through the input using smaller stencils or filters. At a very high level, this is an iterative process where the filter moves along the input with the step value of this governed by a stride parameter. Inside an iteration, a well defined function, $\mathcal{G}$, is computed over the input cells. The result of this function yields a cell value of the output matrix. It is easy to see that such an operation would, at times, reduce the dimensions of the input. To sometimes prevent that from happening, input matrix is padded with 0s along the border. We define a few terms commonly used in the description of Convolutional Neural Network (CNN).
\begin{tiret}
\item $\relu(x)$: Defined as $\mathsf{max}(x,0)$ and its derivative, denoted by $\drelu(x)$ is $1$ if $x>0$ and $0$ otherwise.
\item \textsf{Fully Connected:} Denotes matrix multiplication between input and weights followed by the addition of a bias term.
\item \textsf{Convolution:} A stencil operation where $\mathcal{G}$ is a matrix product between stencil and the input cells beneath it. This is parameterized by the following: Input dimension of $(n\times n\times i)$ and stencil dimension being $(f\times f\times i\times o)$, where $n$ is the spatial dimension, the no. of input channels is denoted by $i$, $f$ is the filter dimension and $o$ denotes the no. of output channels.
\item \textsf{MaxPool:} A stencil operation where $\mathcal{G}$ is the max value among the input cells. Parameters are similar to that of a convolution.
\item \textsf{Batch Normalization:} A technique used in recent and popular CNNs where the output of some intermediate layers is normalized across a mini-batch of images.
\end{tiret}

\subsection{Threat Model and Security}

All security is modeled and proved using the simulation paradigm~\cite{gmw,canetti00}. At a very high level, all parties are modeled as non-uniform interactive Turing machines running in probabilistic polynomial time (PPT). The adversary $\adv$, that interacts with and acts as instructed by an environment $\env$, corrupts a subset of the parties (in our case, at most $1$ out of the parties). These corrupted parties are under the complete control of $\adv$ and $\adv$ has access to the {\em view} of the corrupted party (which includes all incoming/outgoing messages, inputs and random tape of the party). When considering {\em semi-honest security}, all parties are assumed to follow the protocol specification, while for {\em malicious security}, no such assumption is made. $\env$ receives the entire view of all adversarial parties and outputs a single bit. Security is modeled through two interactions - the {\em real} interaction where parties run a protocol $\prot$ in the presence of $\adv$ and  $\env$ and the {\em ideal} interaction where parties send their inputs to a {\em trusted functionality} $\F$ that executes the desired computation truthfully. $\simu$ (called the {\em simulator}) denotes the adversary in the ideal interaction. Let $\mathsf{REAL}_{\prot,\adv,\env}$ (resp. $\mathsf{IDEAL}_{\F,\simu,\env}$) denote the distribution describing $\env$'s output in the real (resp. ideal) interaction. A protocol {\em securely realizes} a functionality $\F$ if for all $\adv$ in the real interaction, there exists $\simu$ in the ideal interaction such that the distributions $\mathsf{REAL}_{\prot,\adv,\env}$ and  $\mathsf{IDEAL}_{\F,\simu,\env}$ are negligibly close (in a security parameter $\secparam$). Protocols can also invoke other sub-protocols and in this framework, the {\em $F-$hybrid model} is like a real interaction, except that some invocations of sub-protocols are replaced by invocations of an ideal functionality $\F$.

\subsection{Intel SGX}

A {\em Trusted Execution Environment (TEE)} is an area of the processor that promises to provide security guarantees like data privacy and code integrity to a program that is loaded inside of it, even in the presence of a malicious operating system and hypervisor. With the Skylake series of processors, Intel introduced a new set of instructions, called {\em Intel Software Guard Extensions} (SGX), which provides a way to realize TEE on Intel chipsets. This is enforced through hardware access control mechanisms for the pages belonging to an application loaded with SGX. Intel SGX aims to provide the guarantee that code and data of the secure application can neither be read nor modified by even privileged software on the machine. Naturally, providing these guarantees are very hard and reducing the assumption needed of the TEE is of paramount importance. 

Applications typically contain two parts: an {\em enclave} which contains all sensitive code and data, and the {\em standard} part, which handles all system calls related to input/output and sockets. A chunk of memory, called {\em Processor Reserved Memory} (around 128 MB) is set aside for holding the enclave data pages. There are two types of function calls in SGX - $\mathsf{ecall}$ and $\mathsf{ocall}$ with which the program control flow enters/exits an enclave - e.g., to make system calls. 

Enclave Attestation is a protocol by which an enclave proves its identity to another application or enclave. There are two types of enclave attestations: a) Local, when both enclaves reside on the same machine; and b) Remote, when the enclaves reside on different machines. In the latter case, the enclave proving its identity (say A) first proves its identity locally to a special enclave known as the Quoting Enclave (QE) which has keys provisioned by Intel. Upon successful verification, QE queries Intel's Attestation Service (IAS) for an attestation certificate (sometimes referred to as a report). This report can then be used by the second enclave (B) (or even a standard application) to verify the identity of A using IAS's public key. Once attestation is completed successfully, A can set up a private and authenticated channel with B using standard mechanisms.

\newcommand{\kw}[1]{{\ensuremath{\mathtt{#1}}}}
\newcommand{\ftext}[1]{\text{\small{#1}}}
\newcommand{\cond}[3]{\ensuremath{{{#1}\:?\:{#2}\::{#3}}}}
\newcommand{\forl}[4]{\ensuremath{\kw{for}\:{#1}\:\kw{in}\:[{#2}, {#3}]\:\kw{do}\:{#4}}}
\newcommand{\ite}[3]{\ensuremath{\kw{if}({#1}, {#2}, {#3})}}
\newcommand{\loops}[3]{\ensuremath{\kw{while}\:{#1} \leq {#2}\:\kw{do}\:{#3}}}

\section{Athos}
\label{sec:athos}
Athos compiles ML inference code written in \tensorflow to \mpc protocols. It has the following main components:

\begin{tiret}

\item \emph{Frontend.} Athos frontend compiles \tensorflow code to a
high-level intermediate language (HLIL). HLIL supports floating-point
tensors and sequence of function calls (corresponding to the
\tensorflow nodes) that manipulate tensors. The main challenge in the
frontend is to reconcile dynamic typing in \tensorflow to static
typing in HLIL. \tensorflow code, written in Python, does not have
tensor dimensions, whereas our HLIL has explicit tensor dimensions as
it enables the compiler to perform analyses and optimizations.

\item \emph{Float-to-fixed converter.} While ML models use
floating-point arithmetic, \mpc protocols operate on fixed-point
arithmetic. Rather than requiring the programmers to manually convert
(or re-train) their models to integers, Athos performs the conversion
automatically, without compromising on the inference accuracy.

\item \emph{Modular LLIL.} Athos compiles floating-point HLIL code to 
fixed-point code in a low-level
intermediate language (LLIL). LLIL is a C-like imperative language
that supports integer tensors, loops, conditionals, and functions. LLIL
also makes it easier for different cryptographic backends to be
plugged into Athos. It precisely specifies the interface that it
requires the cryptographic protocols to implement, while providing a
 library for other operations. The LLIL is compiled down to
the MPC protocol code.

\item \emph{Optimizations.} Athos implements 
 \mpc specific optimizations as well as
several standard dataflow
  analyses and compiler optimizations.
 The design of HLIL and LLIL, and the choice of them being
  statically typed, is partly motivated by the requirements of these
  analyses.

Below we explain each of these components in detail.
\end{tiret}
\subsection{Frontend and HLIL}
\label{subsec:athosfrontend}

Athos frontend compiles the input \tensorflow models to HLIL (described
next) with explicit tensor dimensions. To obtain these dimensions,
the frontend first runs \tensorflow code on one dummy input and
generates \tensorflow metadata that has all the required information.
The metadata is then translated to HLIL.

We discuss some details of the frontend.
A plain dump of  the \tensorflow metadata contains some nodes that 
are semantically irrelevant for actual inference,
e.g. {\tt identity}, {\tt assign}, etc. To avoid representing these
nodes in HLIL,
we first prune the \tensorflow graph to remove such nodes,
specifically we use the \tensorflow graph transform tool
\cite{tfGraphTransformTool} for this purpose. Next, Athos desugars the remaining (tens of) \tensorflow nodes  to HLIL, while keeping the number of functions in HLIL as small
as possible. \tensorflow also supports
``broadcasting'' \cite{tensorflowbroadcasting} that allows operations
on tensors of incompatible dimensions and sizes. For example, due to
broadcasting, addition of a four-dimensional tensor with a one-dimensional tensor is a valid operation. Athos frontend
passes the broadcasting information to HLIL,
which then accounts for it by compiling it to the appropriate LLIL
library function call.


% Finally, some \tensorflow graphs have ``control edges'' 
% \cite{abadi2017computational} that constrain the order of execution of
% the nodes -- Athos frontend also accounts for their mutability
% \aseem{Add detail how}.

\begin{figure}[htp]
  \footnotesize
  \[
  \begin{array}{rrcl}
    %\ftext{Function} & f &&\\
    \ftext{Constant} & n & ::= & 0 \mid 1 \mid 2 \mid \ldots\\
    \ftext{Float constant} & r & ::= & n.n\\
    \ftext{Type} & \hat{\tau} &::=& \kw{float} \mid \kw{int} \mid \hat{\tau}[n] \\
    \ftext{Matrix} & \hat{M} & ::= & \overline{r} \mid \overline{\hat{M}}\\
    \ftext{Expression} & \hat{e} &::=& n \mid x \mid \hat{M} \mid \hat{e_1} \oplus \hat{e_2} \mid x[\hat{e}] \\
    \ftext{Program} & \hat{p} & ::= &\kw{void}\ \kw{main}\;()\;\{\overline{\hat{\tau}\;x}\;;\overline{f(\overline{\hat{e}})}\}
  \end{array}
  \]
\caption{HLIL syntax}
\label{fig:hil}
\end{figure}

Figure~\ref{fig:hil} shows the HLIL (we use $\overline{r}$ to denote
sequences of floating-point constants, and similarly for other
syntactic categories). It is a simple language of floating-point
tensors ($\hat{M}$), with dimensions ($n$) and sizes as explicit type annotations ($\hat{\tau}[n]$), and
the $\kw{main}$ is a sequence of variable declarations and function calls.

We next discuss how Athos performs float-to-fixed conversion on HLIL
programs.

\subsection{Float-to-fixed}
\label{subsec:athosquantizer}
As observed earlier, most ML models are expressed using
floating-point, while \mpc protocols operate on
integers. For large models, we cannot expect the programmers to
manually translate or re-train floating-point ML models to integer
code (the common approach in literature on secure inference~\cite{secureml,minionn,gazelle,aby3,securenn,delphi,chameleon}).
Furthermore, it is well-known that floating-point operations are much
more inefficient than fixed-point when evaluated
securely~(\cite{secureml,aby3}) -- we re-confirm this by performing
two-party secure multiplication~\cite{ddkssz15} using both fixed-point
and floating-point arithmetic to showcase the difference. This is
illustrated in Table \ref{tab:floatvsfixed} which shows the huge
overheads associated with floating-point arithmetic.
In future, if efficient protocols for floating-point become
available then we can directly compile HLIL to them, but until then
Athos automatically performs the translation.

The translation is parametrized by a scale parameter
$s$ that determines the precision.
We discuss how this scale is set later in the section. Given a scale
$s\in\mathbb{Z}$, we define a map $\rho_s:\mathbb{R}\rightarrow \mathbb{Z}_{2^b}$
that maps Reals to $b$-bit integers: $\rho_s(r)=\lfloor r\cdot2^s\rfloor$. We abuse notation and also apply $\rho_s$ to
matrices $M$ of Reals where the result is a point-wise application of
$\rho_s$ to each matrix element.  
In the output fixed-point code, every Real number $r$ is represented by a $b$-bit integer.
The Real representation of an integer $n$ is given by $\frac{n}{2^s}$.
%the conversion algorithm
The float-to-fixed conversion (for select cases) is described in
the following algorithm (\kw{ScaleDown} is described in
Table~\ref{tab:smf}):

\[
\begin{array}{lcl}
%F(n) & = & n\\
%F(r) & = & \rho_s(r)\\
%F(\hat{M}) & = & \rho_s(\hat{M})\\
%F(\kw{float}\ x) & = & \kw{int}\ x\\
F(\kw{MatAdd}(A,B,C)) & = & \kw{MatAdd}(A,B,C)\\
F(\kw{MatMul}(A,B,C)) &= &\kw{MatMul}(A,B,C);\\
& &\kw{ScaleDown}(C,s)
\end{array}
\]

%Here, $\mathit{unop}$ denotes a generic unary operator (e.g., $\mathtt{argmax},\mathtt{maxpool},\mathtt{relu},\dots$).
%In the first step, we run the a procedure that replaces all floating-point numbers in the program with fixed-point integers. 
% The only operation which needs to change because of float-to-fixed is matrix multiplication. Since convolution calls matrix multiplication (Figure~\ref{convtomatmul}), it needs to be modified to call the new matrix multiplication (Figure~\ref{newmatmul}).

%example
\
As an example of the conversion process, consider the program $M_1*M_2$ that multiplies
the row vector $M_1=[400.1,200.1]$ with the column vector $M_2=[0.3,0.1]^T$.
Then in infinite precision Real arithmetic the result of the computation
$400.1*0.3+200.1*0.1$
is $140.04$. Single-precision floating-point 
arithmetic with 32 bits only has a 23-bit mantissa and computes the approximately correct result 140.040009.
%When using Athos, the computed result
%can be much more precise than the floating-point result. 
We use $0.1f$ to denote the floating-point number closest to the Real number $0.1$. 
Given $s=24$, $F(M_1*M_2)$ results 
into the following program over integers
\[
(\rho_{24}(400.1f)*\rho_{24}(0.3f)+\rho_{24}(200.1f)*\rho_{24}(0.1f)) >> {24}
\]
which results in the following computation with 64-bit integers
\[
(6712564224*3357121024+5033165*1677721)>>{24}
\]
The final result is 2349481329 that represents the real number
$\frac{2349481329}{2^{24}}=140.040000021457672119140625$ which is good approximation of the desired result $140.04$. Although it is feasible to constuct examples where fixed-point computations
can be imprecise, ML usually operates on normalized values and we have observed that Athos does not lose accuracy in practice (Table~\ref{tab:fixed-accuracy}).

Athos, assigns the same bit-width $b$ and the same scale $s$ to all
network parameters. While we could use different $b$ and $s$, our
experimental results show that same values for all parameters works
quite well in practice.
%Hence, no information about any individual parameter gets leaked. 
We keep the scale public for efficiency: division with $2^s$ when $s$ is secret is much more expensive than when $s$ is public.
Moreover, scaling down operations (division by $2^s$) cause loss of precision, as they lose significant bits, and hence need to be minimized.
 Therefore, Athos scales down only once per matrix multiplication and does not scale down matrix additions.
%These design choices make Athos ``crypto-aware" and effective for secure machine learning (Section~\ref{subsec:athosexperiments}).
%Prior float-to-fixed converters assign different bit-widths/scales to different parameters and  scale down after every operation~\cite{}.
%Because IEEE754 floating-point numbers have a 23-bit mantissa, if $s$ is set to 23 and all intermediate values occurring at runtime are below 128 then the fixed-point code has very similar accuracy to floating-point code. 
%However, in practice, the intermediate values do exceed 128, causing $s$ being set to smaller values, and some accuracy is lost.

%\subsubsection{Setting Scale}\label{subsec:athossettingscale}
%We try all possible $s$. If $s$ is large then overflows and garbage. If $s$ is small then imprecise %result. Autotuning gives a good $s$ that produces good results. 

While we use machine integer width (64) for $b$, finding a good value
of $s$ is difficult. We explain the various tradeoffs that
govern the choice of $s$ and then discuss our solution.

Suppose, in our example, $s$ is set too low: $s=2$.
Then $F([400.1f,200.1f]*[0.3f,0.1f])$ is $(1600*1+800*0)>>2$,
which represents the Real number $400/4=100$.
This result is far from 140.04. Here, low scale values have lead to
loss of significant bits. In particular, 0.1 has been rounded to zero
causing an imprecise result. Ideally we want to set the scale to a large value
so that the integers have many significant digits.

Next, suppose $s$ is set to a very high value, e.g., 60. Then, the
computation $\rho_{60}(400.1f)*\rho_{60}(0.3f)$ overflows 64-bit integers and
the result is garbage
(multiplication of these two large positive numbers would become a negative number).  
%Usually, at compile time the operations are of the form $x*y$ where the values of $x$
%and $y$ are known only at runtime. Hence, it is hard to decide whether a particular
%scale value would lead to overflows or not at compile time.

Thus, scale can neither be very low nor very high; we need to find a sweet spot.
To determine an appropriate value of $s$, we sweep over all its possible values $\{0,1,\ldots,b-1\}$
and choose the value that leads to the best accuracy. For the example $400.1f*0.3f+200.1f*0.1f$,
the most accurate result is obtained at $s=24$. In general, machine learning algorithms have a 
 validation dataset that is used for hyperparameter tuning. We consider scale as a hyperparameter
and select the scale that leads to a fixed-point classifier implementation that performs the best on the validation set.
The scale chosen by Athos is a {\em leakage} function that depends on the weights of the model.
Athos gives a methodical way of picking this scale that prior works did manually.
Hence, leakage by Athos is similar to all prior works on secure inference. 
%We point out that while the scale chosen by Athos is a ``leakage'' function that depends on the weights of the model, we stress that this is not unique to our work. Indeed, all prior works on secure evaluation of machine learning algorithms pick an appropriate scale with the hope that the accuracy of the fixed-point network will not be too much worse than the original floating-point version. Our scheme helps us select the scale that helps match the floating-point accuracy.

%Like all prior works on secure inference, we keep the scale public for efficiency reasons.
%Hence, the {\em{leakage}} by Athos is identical to prior work.

%The accuracy of the final classifier is reported on the test set. In particular, the test set
%is used only for evaluation and {\em not} to generate a classifier implementation. Athos only works with the validation
%set and does not have access to the test set. 



\begin{table}
  \centering
  \resizebox{\columnwidth}{!}{

      \begin{tabular}{|c|c|c|c|l|}
    \hline
    \# Sequential Multiplications & Fixed (ms) & Float (ms) & Overhead \\
    \hline
  $1$ & $2.57$ & $72.35$ & $28.11$x\\
  \hline
    $10$ & $4.88$ & $278.8$ & $57.1$x \\  \hline
    $100$ & $21.65$ & $2735$ & $126.34$x \\   \hline
    $1000$ & $199.6$ & $25281.42$ & $126.6$x \\   \hline
\end{tabular}
}
 \caption{Floating-point vs Fixed-point multiplication.}
\label{tab:floatvsfixed}
%\tableup
\end{table}

\subsection{Modular LLIL}
\label{sec:athosmodularity}

\begin{figure}[htp]
  \footnotesize
  \[
  \begin{array}{rrcl}
    \ftext{Constant} & n & ::= & 0 \mid 1 \mid 2 \mid \ldots\\
    \ftext{Type} & \tau &::=& \kw{int} \mid \tau[n] \\
    \ftext{Matrix} & M & ::= & \overline{n} \mid \overline{M}\\
    \ftext{Expression} & e &::=& n \mid x \mid M \mid e_{1} \oplus e_{2} \mid x[e]\\
    \ftext{Statement} & s &::=& \tau\:x \mid x = e \mid \kw{for}(x=e_{1};x<e_{2};x++)\{s\}\\
    & & \mid& x[e_{1}] = e_{2} \mid \kw{if}(e, s_{1}, s_2\} \mid
    s_{1}; s_{2} \mid \kw{return}\ e\\
    & & \mid & f\;(\overline{e}) \mid d\;(\overline{e})\\
    \ftext{Global} & g & ::= &\kw{extern}\;\tau\;d\;(\overline{\tau\;x}) \mid \tau\;f\;(\overline{\tau\;x})\{s\}\\
    \ftext{Program} & p & ::= & \overline{g};\;\kw{void}\ \kw{main}\;()\;\{s\}
  \end{array}
  \]
\caption{LLIL syntax}
\label{fig:lil}
\end{figure}

Athos compiles HLIL to LLIL, a crypto-aware, C-like intermediate
language that has only integer-valued tensors. Figure~\ref{fig:lil}
shows the syntax of LLIL. This language  has sufficient expressiveness
required to implement ML inference tasks. In particular it supports
arrays, basic arithmetic, loops, branching, functions, and \kw{extern}
declarations.
%
LLIL makes the Athos interface to the \mpc cryptographic protocols
explicit. We observe that the tensor operations in a typical
\tensorflow code fall into two categories: those that do not change
the values but just copy the data around (e.g. {\tt squeeze} to remove
dimensions of size 1 from a tensor, {\tt pad} to pad a tensor with
various kinds of paddings, {\tt transpose} to take the transpose of a
tensor, and {\tt concat} to concatenate two tensors into a single
tensor), and those that compute new values.
%
For functions that do not manipulate shares (denoted by $f$), LLIL
provides a library with their
implementations that is automatically added as a prelude to LLIL
programs. Changing the underlying crypto protocol does not require
changes to these library functions and this library can be used by all
crypto developers. These functions are implemented in LLIL and are
compiled to C++ code.

Share-manipulating functions ($\kw{extern}\;d$) are required to be
implemented in the cryptographic backend.
All a crypto developer needs to do is to implement these functions,
and then she would be able to directly evaluate
the protocols on ML models used in practice. We describe these
functions with their signatures and intended semantics in Table~\ref{tab:smf}.
Concretely, we provide three implementations of these functions: using the 2PC
protocols of ABY~\cite{aby}, 3PC protocols of SecureNN~\cite{securenn},
and Porthos (Section~\ref{sec:porthos}). 

\begin{table*}
\begin{tabular}{rl}
MatMul(int[L][M] A, int[M][N] B, int[L][N] C) & Multiply two tensors $A$ and $B$ and store results in $C$\\
MatAdd(int[L][M] A, int[L][M] B, int[L][M] C)& Add two tensors $A$ and $B$ into $C$\\
Conv(int[H][W][CI] A,  int[FH][FW][CI][CO] F) & Convolve a tensor $A$ with filter $F$\\
% Optional parameters include padding and strides. \\
Avg/Max Pool(a, b, int[H][W][C] A) & Apply a stencil that computes the average/max value in windows of size $a\times b$ of tensor $A$.\\
ArgMax(int[M] A) & Compute the index with maximum value in $A$ \\
FusedBatchNorm(int[K][L][M][N] A, int[N] B, int[N] C) & Returns $\forall k,l,m,n. B[n] \times A[k][l][m][n] + C[n]$\\
ReLU(int[M][N] A) & Returns $\forall i,j. {\sf Max}(A[i][j],0)$\\
ScaleDown(int[M][N] A, k) & Arithmetic right shift each entry of $A$ with $k$.\\
\end{tabular}
\caption{Share manipulating functions. These have been simplified for
  exposition by suppressing parameters such as padding and
  strides. For comprehensive signatures, see~\url{https://www.tensorflow.org/api_docs/python/tf/}.}
\label{tab:smf}
\end{table*} 


Finally, Athos compiles LLIL programs to C++ and links them with
the cryptographic \mpc protocol implementation.

\subsection{Optimizations}
\label{sec:athosopt}
Athos intermediate languages are designed to be amenable to static analysis. 
In particular, we have implemented several standard dataflow analyses and compiler optimizations~\cite{dragonbook}:
reaching definitions, liveness analysis, and so on. These analyses help with
optimizing memory utilization and we have observed
savings reaching up to 80\%. 
To demonstrate the ease of implementing analyses and optimizations, we provide an example each:
(a) a peephole optimization ReLU MaxPool Switching on HLIL to improve
efficiency of DNNs that use ReLU and MaxPool, and (b) an analysis
Counting Scale Down operations on LLIL to determine the number of scale
down operations done in order to prevent loss in
accuracy (a similar analysis was done manually in~\cite{secureml,securenn,aby3}).

\subsubsection{ReLU MaxPool Switching}
% The ReLU operation is one of the most time intensive task in secure inference of DNNs. For some DNNs, secure evaluation of ReLUs can consume up to 80\% of the total protocol execution %time. This is in contrast to evaluation in the clear where ReLUs consume only a fraction of the total time.
%Hence, it is plausible that ML developers can write \tensorflow code
%in a way that has no impact on cleartext evaluation  but can severely
%degrade the performance of secure evaluation. One such idiom involves
%applying ReLU to a matrix followed by MaxPool. Notice that ReLU and
%MaxPool are commutative operators: ReLU(MaxPool($\cdot$)) is
%functionally equivalent to MaxPool(ReLU($\cdot$)). Moreover, for
%cleartext performance, there is no discernible difference in the
%performance of these two alternatives. Hence,
Most \tensorflow
developers have adopted the convention of expressing DNN layers using the MaxPool(ReLU($\cdot$)) idiom.
For protocols like Porthos and SecureNN~\cite{securenn} that reduce ReLU and MaxPool to secure comparison protocols, ReLU(MaxPool($\cdot$)) can be much more efficient than  MaxPool(ReLU($\cdot$))  as this significantly reduces the number of comparisons. As opposed to SecureNN, where this was done manually, we have built a peephole optimization pass on HLIL
that replaces occurrences of $\kw{MaxPool}(a,b, \kw{ReLU}(A));$ with $\kw{ReLU}(\kw{MaxPool}(a,b,A));$. For example,
if the input matrix $A$ has dimensions $112\times112\times64$ and we compute a MaxPool with $2\times2$ windows.
Then, the output matrix has dimensions $56\times56\times 64$. Hence, the latter needs to compute only one fourth the number of ReLUs
compared to the former. In this case, the optimized code is over $3\times$ better in communication and over $2\times$ faster in our experimental setup (Section~\ref{sec:experiments}).
\subsubsection{Counting Scale Down operations}
We describe an analysis to count the number of scale down operations in an LLIL code. The analysis uses an environment $\rho$ that
maps tensors to the number of elements they contain. This
environment is populated using variable declarations in the code.
 The analysis makes a single pass over $\kw{main}$ and for each call
 $\kw{ScaleDown(A,s)}$ 
accumulates $\rho(A)$ into a counter. The final value of the counter
provides the number of scale down operations in the code.

Note that this analysis is easy to describe as the LLIL code contains
dimensions of all the tensors explicitly.
Hence, the compiler can statically populate $\rho$. This analysis is
impossible to perform on the \tensorflow Python code 
as the sizes of tensors are unknown at compile time. 



%% \aseem{Should we move the Discussion to the end of this section on Athos? It
%%   breaks the flow of describing the compiler.}



%% \section{Athos}\label{sec:athos}
%% %% The goal of Athos is to bridge the gap between the increasingly important task of secure inference and MPC protocols published in literature.
%% %% Achieving this goal requires addressing several challenges. %that we address in the following sections.

%% %\aseem{Should we change the mpc macro to be MPC rather than 2PC? Below
%% %  I use the macro everywhere. Also can we add mactos for Athos,
%% %  Porthos, and Aramis too? I added one for TensorFlow that we should
%% %  use consistently.}

%% Athos compiles ML inference code written in \tensorflow to \mpc
%% cryptographic protocols. Our ambition with Athos is to make it easier
%% for the ML programmers, who are comfortable writing \tensorflow code
%% and not so much cryptographic protocols, to use \mpc for secure
%% inference \emph{out of the box}. However, this requirement imposes
%% several challenges:

%% %% \aseem{May be we should summarize challenges together, and then
%% %%   say how we address them?}

%% \begin{tiret}
%% %\item Athos provides static typechecking of the input inference
%% %  code (e.g. checking the dimensions of the matrices in matrix
%% %  multiplications) so as to provide early feedback to the programmers
%% %  and to prevent undefined runtime behaviors in the crypto
%% %  protocols. Our first challenge then is to reconcile the dynamically
%% %  typed \tensorflow Python code with static typing in Athos.
%% \item The first challenge for Athos is to handle the floating-point
%%   arithmetic in the input \tensorflow code. Most cryptographic
%%   protocols use integer or fixed-point arithmetic, and given the size of the models
%%   we are targeting, it would be unreasonable to expect programmers to
%%   manually translate floating-point ML models to fixed-point code.
%%   Given a floating-point model in \tensorflow, we require Athos to produce fixed-point models with good accuracies without   any access to training data
%%   or the high computational overheads associated with re-training.
%% %  Furthermore, Athos operates directly on trained models, does not require access to training data,
%% %  and hence is compatible with all training procedures. 
%% \item We would also like to make it convenient to plug-in various \mpc
%%   protocols as backends in Athos. Designing a modular, crypto- and
%%   ML-aware compiler, that presents well-defined interface to the
%%   secure inference \mpc protocols for them to be plugged-in as
%%   backends, is our next challenge.
%% \item Finally, as with any other compiler, Athos implements several
%%   optimizations on the input code. We need to design Athos so that
%%   implementing these, and other analyses in future, is easy. To this end,
%%   Athos uses intermediate languages that are statically typed so that
%%   the analyses have more (e.g. tensor dimensions) to work with.
%%  Our final challenge then is to reconcile the dynamically
%%  typed \tensorflow Python code with static typing in Athos.
%% \end{tiret}

%% Below we describe the design of Athos and its various intermediate
%% languages that help us address these challenges.

%% %\aseem{Should we describe the compiler top-down or bottom-up? I am
%% %  thinking top-down, we can also add a figure with boxes etc. that
%% %  shows the compiler pipeline?}

%% \subsection{Modularity}
%% \label{sec:athosmodularity}
%% The first design challenge that we address is modularity: It should be
%% easy for an \mpc cryptographic protocol developer to plug her system
%% in Athos and evaluate the protocols on large DNNs. We observe that
%% large DNNs perform two kinds of operations: those that manipulate the
%% values of secret shares (e.g., addition, multiplication, etc.) and those which just copy them
%% around (e.g., reshaping a tensor, transposing a matrix, etc.). A cryptographic protocol should only need
%% to provide implementations of the former and should not be bothered
%% with the operations that fall in the latter class. 

%% %\aseem{Some minor comments: (a) No else in if? (b) Definitions are missing
%% %  statement body? (c) What is s + x ... in the main body? (d)
%% %  Definitions d have at least one argument?}

%% %\aseem{More than minor comments: (a) No function calls? (b) Perhaps we
%% %  should add a syntactic category f for functions that are expected
%% %  from the backend ... and then function calls can use either those
%% %  functions or definitions as head? A different syntactic category
%% %  would bring about the point from above.}

%% %\aseem{Depending on space, we could leave this in the main text?}

%% \begin{figure}[htp]
%%   \footnotesize
%%   \[
%%   \begin{array}{rrcl}
%%     \ftext{Constant} & n & ::= & 0 \mid 1 \mid 2 \mid \ldots\\
%%     \ftext{Type} & \tau &::=& \kw{int} \mid \tau[n] \\
%%     \ftext{Matrix} & M & ::= & \{n(, n)^*\} \mid \{M (, M)^*\}\\
%%     \ftext{Expression} & e &::=& n \mid x \mid M \mid e_{1} \oplus e_{2} \mid x[e]\\
%%     \ftext{Statement} & s &::=& \tau\:x \mid x = e \mid \kw{for}(x=e_{1};x<e_{2};x++)\{s\}\\
%%     & & \mid& x[e_{1}] = e_{2} \mid \kw{if}(e)\{s_{1}\} \mid  s_{1}; s_{2} \mid \kw{return}\ e\\
%%     \ftext{Declaration} & d & ::= &\kw{extern}\ \tau\ x(\tau\ x (, \tau x)^*)\\
%%     \ftext{Definition} & f & ::= & \tau\ x(\tau\ x (, \tau x)^*)\{s\}\\
%%     \ftext{Program} & p & ::= & (d+f)(;(d+f))^*;\kw{void}\ \kw{main}\{(s+ x(e(, e)^*))^+\}
%%   \end{array}
%%   \]
%% \caption{LLIL syntax \rs{move to appendix?}}
%% \label{fig:llil}
%% \end{figure}


%% To this end, the output of Athos is a program in a
%% low-level intermediate language (LLIL), described in
%% Figure~\ref{fig:llil}.
%% This language is a subset of C and has sufficient expressiveness
%% required to implement ML inference tasks. In particular it supports
%% arrays, basic arithmetic, loops, branching, and function calls.
%% For functions that do not manipulate shares (denoted by
%%   \textit{definitions} $f$), we have implemented a library with their
%%  implementations that is automatically added as a prelude to LLIL programs.
%% Changing the underlying crypto protocol does not require changes to
%% these library functions and this library can be used by all crypto
%% developers. 
%% Examples of these functions include {\tt squeeze} (remove dimensions
%% of size 1 from a tensor), {\tt pad} (pad a tensor with various kinds
%% of paddings), 
%% {\tt transpose} (the transpose of a tensor), {\tt concat} (concatenate
%% two tensors into a single tensor), etc. These functions are
%% implemented in LLIL and are compiled to C++ code.


%% For share-manipulating functions (\textit{declarations} $d$ using the $\kw{extern}$ keyword),
%% Athos provides a unified interface that the cryptographers can
%% target.
%% All a crypto developer needs to do is to implement these functions,
%% and then she would be able to directly evaluate
%% the protocols on ML models used in practice. We describe these
%% functions with their signatures and intended semantics in Table~\ref{tab:smf}.
%% We provide three implementations of these functions: using the 2PC
%% protocols of ABY~\cite{aby}, 3PC protocols of SecureNN~\cite{securenn},
%% and Porthos (Section~\ref{sec:porthos}). 

%% \begin{table*}
%% \begin{tabular}{rl}
%% MatMul(int[L][M] A, int[M][N] B, int[L][N] C) & Multiply two tensors $A$ and $B$ and store results in $C$\\
%% MatAdd(int[L][M] A, int[L][M] B, int[L][M] C)& Add two tensors $A$ and $B$ into $C$\\
%% Conv(int[H][W][CI] A,  int[FH][FW][CI][CO] F) & Convolve a tensor $A$ with filter $F$\\
%% % Optional parameters include padding and strides. \\
%% Avg/Max Pool(a, b, int[H][W][C] A) & Apply a stencil that computes the average/max value in windows of size $a\times b$ of tensor $A$.\\
%% ArgMax(int[M] A) & Compute the index with maximum value in $A$ \\
%% FusedBatchNorm(int[K][L][M][N] A, int[N] B, int[N] C) & Returns $\forall k,l,m,n. B[n] \times A[k][l][m][n] + C[n]$\\
%% ReLU(int[M][N] A) & Returns $\forall i,j. {\sf Max}(A[i][j],0)$\\
%% ScaleDown(int[M][N] A, k) & Divide each entry of $A$ with $2^k$.\\
%% \end{tabular}
%% \caption{Share manipulating functions. These have been simplified for
%%   exposition by suppressing parameters such as padding and
%%   strides. For comprehensive signatures, see~\url{https://www.tensorflow.org/api_docs/python/tf/}.}
%% \label{tab:smf}
%% \end{table*} 




%% \subsection{Floating-point}
%% \label{subsec:athosquantizer}
%% The programs written in LLIL manipulate integers. However, most ML models are expressed using floating-point.
%% For large models, we cannot expect the user to manually translate
%% floating-point ML models to integer code (the common approach in
%% literature on secure inference of toy
%% models~\cite{secureml,minionn,gazelle,aby3,securenn}).
%% Therefore Athos automatically translates floating-point ML models to
%% LLIL via a high-level intermediate language (HLIL), described in
%% Figure~\ref{fig:hlil}. HLIL is very similar to LLIL except that it permits
%% floating-point values ($r$) and arrays of floating-point values. In
%% this section, we describe HLIL and translation from HLIL to LLIL.
%% A frontend that translates \tensorflow code to HLIL is described in
%% Section~\ref{subsec:athosfrontend}.

%% \aseem{No statements in HLIL? No if, for ?}

%% \begin{figure}[htp]
%%   \footnotesize
%%   \[
%%   \begin{array}{rrcl}
%%     \ftext{Float constant} & r &&\\
%%     \ftext{Type} & \hat{\tau} &::=& \kw{float} \mid \kw{int} \mid \hat{\tau}[n] \\
%%     \ftext{Matrix} & \hat{M} & ::= & M \mid \{r(, r)^*\} \mid \{\hat{M} (, \hat{M})*\}\\
%%     \ftext{Expression} & \hat{e} &::=& e \mid r\mid \hat{M} \mid \hat{e}_{1} \oplus \hat{e}_{2} \\
%%     \ftext{Program} & \hat{p} & ::= &\kw{void}\ \kw{main}\{((\hat{\tau}\ x;)+ x(\hat{e}(, \hat{e})^*))^+\}
%%   \end{array}
%%   \]
%% \caption{HLIL syntax \rs{move to appendix?}}
%% \label{fig:hlil}
%% \end{figure}

%% The float-to-fixed translation is parametrized by a scale parameter $s$ that determines the precision.
%% We show how this scale is set in Section~\ref{subsec:athossettingscale}.
%% Given a scale $s$, we define a map $\rho_s:\mathbb{R}\rightarrow \mathbb{Z}_{2^b}$ that
%% maps Reals to $b$-bit integers: $\rho_s(r)=\lfloor r\cdot2^s\rfloor$. 
%% We abuse notation and
%% also apply $\rho_s$ to matrices $M$ of Reals where the result is a pointwise application of $\rho_s$ to each matrix element. 
%% In the output fixed-point code, every Real number $r$ is represented by a $b$-bit integer.
%% The Real representation of an integer $n$ is given by $\frac{n}{2^s}$.
%% %the conversion algorithm
%% The float-to-fixed conversion (for interesting cases) is described in
%% the following algorithm (\kw{ScaleDown} is described in
%% Table~\ref{tab:smf}):

%% \aseem{What is the type of F? hat\{e\} to e? But then (float x) or
%%   (x = e1) are not in HLIL?}

%% \[
%% \begin{array}{lcl}
%% F(n) & = & n\\
%% F(r) & = & \rho_s(r)\\
%% F(M) & = & \rho_s(M)\\
%% F(\kw{float}\ x) & = & \kw{int}\ x\\
%% F( x = e_1) & = & x = F(e_1)\\
%% F(e_1+e_2) & = & F(e_1) + F(e_2)\\
%% F(\kw{MatMul}(A,B,C)) &= &\kw{MatMul}(A,B,C);\\
%% & &\kw{ScaleDown}(C,s)
%% \end{array}
%% \]
%% The converter $F$ takes an HLIL program over floating-point as input and outputs an LLIL program over integers.
%% %Here, $\mathit{unop}$ denotes a generic unary operator (e.g., $\mathtt{argmax},\mathtt{maxpool},\mathtt{relu},\dots$).
%% %In the first step, we run the a procedure that replaces all floating-point numbers in the program with fixed-point integers. 
%% % The only operation which needs to change because of float-to-fixed is matrix multiplication. Since convolution calls matrix multiplication (Figure~\ref{convtomatmul}), it needs to be modified to call the new matrix multiplication (Figure~\ref{newmatmul}).

%% %example
%% \
%% As an example of the conversion process, consider the program $M_1*M_2$ that multiplies
%% the row vector $M_1=[400.1,200.1]$ with the column vector $M_2=[0.3,0.1]^T$.
%% Then in infinite precision Real arithmetic the result of the computation
%% $400.1*0.3+200.1*0.1$
%% is 140.04. Floating-point
%% arithmetic only has a 23-bit mantissa and computes the approximately correct result 140.040009.
%% When using Athos, the computed result
%% can be much more precise than the floating-point result. 
%% We use $0.1f$ to denote the floating-point number closest to the Real number $0.1$. 
%% Given $s=24$, $F(M_1*M_2)$ results 
%% into the following program over integers
%% \[
%% (\rho_{24}(400.1f)*\rho_{24}(0.3f)+\rho_{24}(200.1f)*\rho_{24}(0.1f))/{2^{24}}
%% \]
%% which results in the following computation with 64-bit integers
%% \[
%% (6712564224*3357121024+5033165*1677721)/2^{24}
%% \]
%% The final result is 2349481329 that represents the real number
%% $\frac{2349481329}{2^{24}}=140.040000021457672119140625$ which is a better approximation than
%% the floating-point result.


%% \aseem{It's not clear to me how can different b and s leak information
%%   about parameters? Aren't the parameters hidden by the MPC protocol
%%   anyway?}

%% Athos, assigns the same bitwidth $b$ and the same scale $s$ to all network parameters. 
%% %Hence, no information about any individual parameter gets leaked. 
%% We keep the scale public for efficiency: division with $2^s$ when $s$ is secret is much more expensive than when $s$ is public.
%% Moreover, scaling down operations (division by $2^s$) cause loss of precision, as they lose significant bits, and hence need to be minimized.
%%  Therefore, Athos scales down only once per matrix multiplication and does not scale down matrix additions.
%% %These design choices make Athos ``crypto-aware" and effective for secure machine learning (Section~\ref{subsec:athosexperiments}).
%% %Prior float-to-fixed converters assign different bitwidths/scales to different parameters and  scale down after every operation~\cite{}.
%% %Because IEEE754 floating-point numbers have a 23-bit mantissa, if $s$ is set to 23 and all intermediate values occuring at runtime are below 128 then the fixed-point code has very similar accuracy to floating-point code. 
%% %However, in practice, the intermediate values do exceed 128, causing $s$ being set to smaller values, and some accuracy is lost.

%% \subsubsection{Setting Scale}\label{subsec:athossettingscale}
%% %We try all possible $s$. If $s$ is large then overflows and grabage. If $s$ is small then imprecise %result. Autotuning gives a good $s$ that produces good results. 

%% Finding a good value of $s$ is difficult. We explain the various tradeoffs that
%% govern the choice of $s$ and then discuss our solution.

%% Suppose, in our example, $s$ is set too low: $s=2$.
%% Then $F([400.1f,200.1f]*[0.3f,0.1f])$ is
%% \[
%% (1600*1+800*0)/4
%% \]
%% which represents the Real number $400/4=100$.
%% This, result is far from 140.04. Here, low scale values have lead to
%% loss of significant bits. In particular, 0.1 has been rounded to zero
%% causing an imprecise result. Hence, ideally we want to set the scale to a large value
%% so that the integers have many significant digits.

%% Next, suppose $s$ is set to a very high value, e.g., 60. Then, the
%% computation $\rho_{60}(400.1f)*\rho_{60}(0.3f)$ overflows 64-bit integers and
%% the result is garbage
%% (multiplication of these two large positive numbers would become a negative number).  
%% %Usually, at compile time the operations are of the form $x*y$ where the values of $x$
%% %and $y$ are known only at runtime. Hence, it is hard to decide whether a particular
%% %scale value would lead to overflows or not at compile time. 

%% \aseem{Should we say how we choose b? Is it just the machine integer width?}

%% Thus, scale can neither be very low nor very high; we need to find a sweet spot.
%% To determine an appropriate value of $s$, we sweep over all its possible values $\{0,1,\ldots,b-1\}$
%% and choose the value that leads to the best accuracy. For the example $400.1f*0.3f+200.1f*0.1f$,
%% the most accurate result is obtained at $s=24$. In general, machine learning algorithms have a 
%%  validation dataset which is used for hyperparameter tuning. We consider scale as a hyperparameter
%% and select the scale that leads to a fixed-point classifier implementation which performs the best on the validation set.
%% This scheme helps us select values of scales that result in minimal accuracy loss.
%% %The accuracy of the final classifier is reported on the test set. In particular, the test set
%% %is used only for evaluation and {\em not} to generate a classifier implementation. Athos only works with the validation
%% %set and does not have access to the test set. 



%% \begin{table}
%%   \centering
%%   \resizebox{\columnwidth}{!}{

%%       \begin{tabular}{|c|c|c|c|l|}
%%     \hline
%%     \# Sequential Multiplications & Fixed (ms) & Float (ms) & Overhead \\
%%     \hline
%%   $1$ & $2.57$ & $72.35$ & $28.11$x\\
%%   \hline
%%     $10$ & $4.88$ & $278.8$ & $57.1$x \\  \hline
%%     $100$ & $21.65$ & $2735$ & $126.34$x \\   \hline
%%     $1000$ & $199.6$ & $25281.42$ & $126.6$x \\   \hline
%% \end{tabular}
%% }
%%  \caption{Floating-point vs Fixed-point multiplication.}
%% \label{tab:floatvsfixed}
%% %\tableup
%% \end{table}


%% \aseem{Should we move the Discussion to the end of this section on Athos? It
%%   breaks the flow of describing the compiler.}

%% \subsubsection{Discussion}
%% We discuss why an automatic float-to-fixed converter is necessary for \tool and how Athos is different from prior work.
%% %A question is that why target cryptographic protocols over integers and not compile HLIL to protocols over floating-point.
%% It is well-known that floating-point operations are much more inefficient than fixed-point when evaluated securely~(\cite{secureml,aby3}) -- we re-confirm this by performing two-party secure multiplication~\cite{ddkssz15} using both fixed-point and floating-point arithmetic to showcase the difference. This is illustrated in Table \ref{tab:floatvsfixed} which shows the huge overheads associated with floating-point arithmetic.
%% Since machine learning algorithms are typically expressed using floating-point, we use Athos to first convert floating-point programs to fixed-point.
%% In the future, if efficient protocols for floating-point become
%% available then we can directly compile HLIL to them (without going
%% through LLIL). \aseem{Not sure, does HLIL have the distinction of f vs
%%   d? If at all, convert TF to LLIL directly?}





%% \aseem{Should we move optimizations before Frontend, that way static
%%   typing would have already been motivated.}

%% \subsection{Frontend}
%% \label{subsec:athosfrontend}
%% Although the developers can directly implement ML models in HLIL, it raises their barrier to entry.
%% Usually developers train their models in \tensorflow, expecting them
%% to manually write HLIL code everytime they make
%% minor modifications to their \tensorflow code is untenable. We have
%% developed a frontend that translates \tensorflow code to HLIL automatically.

%% The main challenge here is that the \tensorflow programs are written in Python which has dynamic types.
%% However, HLIL has static types. For example, a Python developer does not need to explicitly annotate the dimensions of matrices.
%% However, HLIL requires the sizes of all matrices to be known at compile time. This information makes the code amenable to static analysis (Section~\ref{sec:athosopt}). To obtain this crucial information, we first run \tensorflow code on one dummy input
%% to generate \tensorflow metadata which has all the requisite information and then translate the metadata to HLIL.

%% We discuss some interesting low-level details of compilation from \tensorflow.
%% First, we need to desugar hundreds of \tensorflow nodes to the small number of functions in the library and the interface (Section~\ref{sec:athosmodularity}).
%% Second,  \tensorflow supports  ``broadcasting'' \cite{tensorflowbroadcasting} that  allows operations on 
%% tensors of seemingly incompatible dimensions and sizes. For example,
%% due to broadcasting, addition of a four dimensional tensor with a one
%% dimensional tensor is a valid operation. The compilation should
%% account for this and  pass the broadcasting information to the
%% statically-typed HLIL programs.
%% Third, some \tensorflow graphs have ``control edges'' 
%% \cite{abadi2017computational} that constrain the order of execution of the nodes and  the 
%% compilation needs to account for their mutability. \aseem{Elaborate
%%   more on mutability?}

%% \subsection{Optimizations}
%% \label{sec:athosopt}
%% The intermediate languages are designed to be amenable to static analysis. 
%% In paricular, we have implemented several standard dataflow analyses and compiler optimizations~\cite{dragonbook}:
%% reaching definitions, liveness analysis, etc. These analyses help with
%% optimizing memory utilization and we have observed
%% savings reaching upto 80\%. 
%% To demonstrate the ease of implementing analyses and optimizations, we provide an example each:
%% (a) a peephole optimization ReLU Maxpool Switching on HLIL to improve efficiency of DNNs that use ReLU and maxpool, and (b) an analysis Counting Scale Down operations on LLIL to determine the number of scale down operations that must be done in order to prevent any loss in accuracy (a similar analysis was done manually in~\cite{secureml}).

%% \subsubsection{ReLU Maxpool Switching} The ReLU operation is one of the most time intensive task in secure inference of DNNs. For some DNNs, secure evaluation of ReLUs can consume upto 80\% of the total protocol execution time. This is in contrast to evaluation in the clear where ReLUs consume only a fraction of the total time.
%% Hence, it is plausible that ML developers can write \tensorflow code
%% in a way that has no impact on cleartext evaluation  but can severely
%% degrade the performance of secure evaluation. One such idiom involves
%% applying ReLU to a matrix followed by MaxPool. Notice that ReLU and
%% MaxPool are commutative operators: ReLU(MaxPool($\cdot$)) is
%% functionally equivalent to MaxPool(ReLU($\cdot$)). Moreover, for
%% cleartext performance, there is no discernible difference in the
%% performance of these two alternatives. Hence, most \tensorflow
%% developers have adopted the convention of MaxPool(ReLU($\cdot$)).

%% For secure evaluation, MaxPool(ReLU($\cdot$)) can be much more inefficient than ReLU(MaxPool($\cdot$)) as this significantly reduces the number of ReLU operations that need to be performed in \mpc. Hence, we have built an optimization pass on HLIL
%% that replaces occurences of $\kw{MaPool}(a,b, \kw{ReLU}(A));$ with $\kw{ReLU}(\kw{MaxPool}(a,b,A));$. For example,
%% if the input matrix $A$ has dimensions $112\times112\times64$ and we compute a maxpool with $2\times2$ windows.
%% Then, the output matrix has dimensions $56\times56\times 64$. Hence, the latter needs to compute only one fourth the number of ReLUs
%% compared to the former. In this case, the optimized code is over $3\times$ better in communication and over $2\times$ faster in our experimental setup (Section~\ref{sec:experiments}).
%% \subsubsection{Counting Scale Down operations}
%% We describe an analysis to count the number of scale down operations in an LLIL code. The analysis uses an environment $\rho$ that
%% maps tensors to the number of elements they contain \aseem{rho maps to
%%   number of elements or to the number of scale down operations?}. This
%% environment is populated using variable declarations in the code.
%%  The analysis makes a single pass over $\kw{main}$ and for each call
%%  $\kw{ScaleDown(A,s)}$ 
%% accumulates $\rho(A)$ into a counter \aseem{Didn't get this
%%   ``accumulates into a counter''?}. The final value of the counter
%% provides the number of scale down operations in the code.

%% \aseem{We don't say why is this analysis useful.}

%% Note that this analysis is easy to describe as the LLIL code contains dimensions of all the tensors explicitly.
%% Hence, the compiler can statically populate $\rho$. This analysis is impossible to perform on the Python code 
%% as the sizes of tensors are unknown at compile time. 


%% \begin{comment}
%% %Previous schemes don't work

%% We first describe the frontend of Athos that compiles \tensorflow to the SeeDot intermediate language followed by the float-to-fixed conversion of a SeeDot program to secure fixed-point code.




%% \subsection{Frontend}\label{subsec:athosfrontend}
%% In this section, we describe the compilation from Tensorflow to SeeDot.
%% The syntax of the core of SeeDot language is shown in Figure~\ref{fig:syntax}.
%% (Details about the semantics of SeeDot can be found in the original reference~\cite{seedot}.)
%% %The Athos language is strongly typed with a type system shown in Figure~\ref{fig:type}.
%% Here, $r$ is a Real number, $n$ is a $b$-bit integer, and $M$ is a matrix. Unless specified otherwise, we use $b=64$.
%% The language supports basic linear algebra operations as well as commonly used ML operators.

%% \begin{figure}[t]
%% \centering
%% \begin{math}
%% \begin{array}{rcl}
%%   e & ::= & n\ |\ r \ |\ M\ |\ x\ \vert\  \mathtt{let}\ x = e_1\ \mathtt{in}\  e_2\\
%%   & &\vert\  e_1 + e_2\  \vert\ e_1 * e_2\ \vert\ \mathtt{argmax}(e)\\
%%     & &\vert\ \mathtt{maxpool}(e,n,n)\ \vert\ \mathtt{relu}(e)
%% \end{array}
%% \end{math}
%% \caption{Syntax of the core language of SeeDot~\cite{seedot}}
%% \label{fig:syntax}
%% \end{figure}

%% As shown in the toolchain in \figureref{cryptflowtoolchain}, as a first step, the compiler takes in a Tensorflow (TF) model 
%% code as input and produces TF metadata. This metadata consists of two things. 1) First, it contains the TF model graph definition (as shown in Figure \ref{fig:lrtfgraphdump}), which is a 
%% textual representation of a topological sort of the Directed Acyclic Graph representing the TF computation. 
%% For each node in the computation graph, this graph definition contains the identifier for the node, the operation it performs and a list of children of the node.
%% 2) Second, the metadata contains the sizes of all the tensors involved in the computation. To understand why this metadata is required, consider that on the input end of the toolchain, we have dynamically-typed python-based TensorFlow, 
%% in which sizes of tensors are not known statically at compile time, while on the output end, we desire something like C++, which requires us to declare sizes of all matrices statically.
%%  The metadata on the sizes of the tensors helps us in bridging this gap between the two.
%% % We remark that since we are compiling a 
%% % dynamically-typed Python like language to statically typed C++, extra information about the sizes of the tensors involved
%% % is a must and cannot be avoided.
%% In order to do this, we first execute the TF model code, as is, on a sample input, which helps us in profiling the execution and capturing the metadata on the sizes
%% of all tensors.
%% % We choose to execute the TF model code on garbage input of right size and dimension,
%% % which allows us to learn the sizes of all the other tensors involved in the computation.

%% The second part of the compiler takes this dumped TF metadata as input and compiles it to the SeeDot intermediate language.
%% The high-level idea here is to traverse the TF computation graph sorted topologically and 
%% output recursive let-bindings for each node that is found. At the end of the traversal, this compiled let-binding 
%% SeeDot code is output, which is fed to the float-to-fixed converter. 

%% % The second part of the TF metadata - the size information about the tensors, is also used 
%% % in this compilation step to output Athos matrices of fixed sizes.

%% % We mention a few design choices we made here. Since Athos is a float-to-fixed point compiler, any fixed-point computation on inputs which do not
%% % involve multiplication \nishant{Rahul: i have mentioned multiplication - thats ok, right ?} (e.g. tensor restructuring operations and $\relu(x)$ (defined as $\mathsf{max}(x,0)$), 
%% % as opposed to operations like convolution and matrix multiplication)
%% % can be passed obliviously through Athos. Keeping this in mind, such computations are compiled to uninterpreted function calls which are passed unaltered through Athos 
%% % (see for example the uninterpreted calls in \figureref{lrseedot}). 
%% % On the other hand, computations like convolution are compiled to interpreted calls in Athos, allowing the Athos compiler
%% % to do the necessary quantization.

%% % \nishant{There is more to be talked about here - about TF quirks while doing this compilation (like broadcasting, Batch norm) etc. Plus, mention 
%% % Athos is strongly-typed language and so the size info in this compilation step is a must.}

%% We remark here some of the lower-level details of TensorFlow which make this compilation step non-trivial to perform. 
%% Firstly, along with the need to deal with a wide variety of node operations, TF supports what is referred to as ``broadcasting'' \cite{tensorflowbroadcasting}, which essentially allows operations on 
%% two tensors of different sizes and dimensions. For example, this makes addition between a 4D tensor and a 1D tensor a valid operation in TF. The non-triviality arises
%% since the compilation should account for this and somehow pass this information onto the strongly-typed SeeDot program.
%% Secondly, TF graph definitions are not always as simplified as shown in Figure \ref{fig:lrtfgraphdump}. Some of the edges of the graph are what are known as ``control edges'' 
%% \cite{abadi2017computational}, which constrain the order of execution of the nodes of the graph. In essence, these edges have to dealt with differently during the 
%% compilation to account for their mutability aspect.


%% \iffalse

%% Discuss complicated things like batch normalization, matrix dimensions
%% from program execution, instrumenting Tensorflow code, graph dumps,
%% etc.

%% \aseem{04/10: discussions with Nishant, see tf-to-sdot.jpg, main
%% points:

%% TF graph is a DAG capturing the input-output relationship between the
%% operators in the graph. The roots (or leaves, depending on how you
%% look at it) of the DAG are the inputs and the leaf (or root) is the
%% final output. The nodes metadata contain properties like
%% node kind etc. The graph does not contain the dimensions and sizes of
%% the tensors. To get this information, we run the TF graph on some
%% input image (it doesn't matter what the image is, since we are only
%% interested in the sizes and dimensions). Depending on the batch size
%% $n$ we are interested in for inference, we can also run the graph on
%% $n$ images and get the sizes and dimensions accordingly.

%% So now we have the TF graph dump and separately the sizes etc. We have
%% written a compiler in Python that takes these as inputs and outputs
%% the SeeDot AST. Operators that change the values (can we make it more
%% precise?) such as multiplication, convolution etc. are faithfully (for
%% the lack of better term for now) compiled to SeeDot, whereas operators
%% that just change the tensor structures are compiled to uninterpreted
%% functions in SeeDot -- these are given hand-written implementations in
%% EzPC.

%% Other than this, the compilation is straightforward -- it traverses
%% the TF graph in the topological sort order and outputs straightline
%% SeeDot code with a let-binding for each node and threading the inputs
%% outputs properly.

%% There are some interesting quirks. There is something called
%% Broadcasting in TF that promotes, for example, a 1-D array to a 2-D
%% array on-demand at runtime. We account for this by implementing
%% similar broadcasting functions in EzPC.

%% Then there is also batch normalization (TODO: add details).
%% }

%% \fi


%% \subsection{Float-to-fixed}\label{subsec:athosquantizer}
%% We describe the compilation of a program over Reals written in SeeDot intermediate language
%% to a fixed-point program. 
%% %Athos source language is a high level
%% %language like MATLAB which succinctly expresses matrix computations. 
%% %To obtain a fixed-point program, we override the floating-point implementations with fixed-point procedures. These procedures can be implemented
%% %in any general purpose language like C++ or available languages for secure computation (ObliVM, OblivC, EzPC).
%% %We show the source code of these procedures in Figure~\ref{alg:proc}.
%% %auxilliary function
%% The float-to-fixed conversion is parametrized by the scale parameter $s$ that determines the precision.
%% We show how this scale is set in Section~\ref{subsec:athossettingscale}.
%% Given a scale $s$, we define a map $\rho_s:\mathbb{R}\rightarrow \mathbb{Z}_{b}$ that
%% maps Reals to $b$-bit integers: $\rho_s(r)=\lfloor r\cdot2^s\rfloor$. 
%% We abuse notation and
%% also apply $\rho_s$ to matrices of Reals where the result is a pointwise application of $\rho_s$ to each matrix element. 
%% In the output fixed-point code, every Real number $r$ is represented by a $b$-bit integer.
%% The Real representation of an integer $n$ is given by $\frac{n}{2^s}$.
%% %the conversion algorithm
%% The float-to-fixed conversion is described in the following algorithm:
%% \[
%% \begin{array}{lcl}
%% F(n) & = & n\\
%% F(r) & = & \rho_s(r)\\
%% F(M) & = & \rho_s(M)\\
%% F( \mathtt{let}\ x = e_1\ \mathtt{in}\  e_2) & = & \mathtt{let}\ x = F(e_1)\ \mathtt{in}\  F(e_2)\\
%% F(e_1+e_2) & = & F(e_1) + F(e_2)\\
%% F(e_1*e_2) & = & \frac{1}{2^s}*\left(F(e_1)*F(e_2)\right)\\
%% F(\mathit{unop}(e)) & = & \mathit{unop}(F(e)) 
%% \end{array}
%% \]
%% The converter $F$ takes a program over Reals as input and outputs a program over integers.
%% Here, $\mathit{unop}$ denotes a generic unary operator (e.g., $\mathtt{argmax},\mathtt{maxpool},\mathtt{relu},\dots$).
%% %In the first step, we run the a procedure that replaces all floating-point numbers in the program with fixed-point integers. 
%% % The only operation which needs to change because of float-to-fixed is matrix multiplication. Since convolution calls matrix multiplication (Figure~\ref{convtomatmul}), it needs to be modified to call the new matrix multiplication (Figure~\ref{newmatmul}).

%% %example
%% \
%% As an example of the conversion process, consider the program $M_1*M_2$ that multiplies
%% the row vector $M_1=[400.1,200.1]$ with the column vector $M_2=[0.3,0.1]^T$.
%% Then in infinite precision Real arithmetic the result of the computation
%% $400.1*0.3+200.1*0.1$
%% is 140.04. Floating-point
%% arithmetic only has a 23-bit mantissa and computes the approximately correct result 140.040009.
%% When using Athos, the computed result
%% can be much more precise than the floating-point result. 
%% Given $s=24$, $F(M_1*M_2)$ results 
%% into the following program over integers
%% \[
%% (\rho_{24}(400.1)*\rho_{24}(0.3)+\rho_{24}(200.1)*\rho_{24}(0.1))/{2^{24}}
%% \]
%% which results in the following computation with 64-bit integers
%% \[
%% (6712564224*3357121024+5033165*1677721)/2^{24}
%% \]
%% The final result is 2349481329 that represents the real number
%% $\frac{2349481329}{2^{24}}=140.040000021457672119140625$ which is a better approximation than
%% the floating-point result.


%% Athos, by design, assigns the same bitwidth $b$ and the same scale $s$ to all network parameters. 
%% Hence, no information about any individual parameter gets leaked. 
%% We keep the scale public for efficiency: division with $2^s$ when $s$ is secret is much more expensive than when $s$ is public.
%% Moreover, scaling down operations (division by $2^s$) cause loss of precision, as they lose significant bits, and hence need to be minimized.
%%  Therefore, Athos scales down only once per matrix multiplication and does not scale down matrix additions.
%% These design choices are unique to Athos and make it effective for secure machine learning (Section~\ref{subsec:athosexperiments}).
%% %Prior float-to-fixed converters assign different bitwidths/scales to different parameters and  scale down after every operation~\cite{}.
%% %Because IEEE754 floating-point numbers have a 23-bit mantissa, if $s$ is set to 23 and all intermediate values occuring at runtime are below 128 then the fixed-point code has very similar accuracy to floating-point code. 
%% %However, in practice, the intermediate values do exceed 128, causing $s$ being set to smaller values, and some accuracy is lost.

%% \subsection{Setting Scale}\label{subsec:athossettingscale}
%% %We try all possible $s$. If $s$ is large then overflows and grabage. If $s$ is small then imprecise %result. Autotuning gives a good $s$ that produces good results. 

%% Finding a good value of $s$ is difficult. We explain the various tradeoffs that
%% govern the choice of $s$ and then discuss our solution.

%% Suppose, in our example, $s$ is set to too low a value of 2.
%% Then the computed result is
%% \[
%% (1600*1+800*0)/4
%% \]
%% which represents the Real number $400/4=100$.
%% This, result is far from 140.04. Here, low scale values have lead to
%% loss of significant bits. In particular, 0.1 has been rounded to zero
%% causing an imprecise result. Hence, ideally we want to set the scale to a large value
%% so that the integers have many significant digits.

%% Next, suppose $s$ is set to too high a value, e.g., 60. Then, the
%% computation $\rho_{60}(400.1)*\rho_{60}(0.3)$ overflows 64-bit integers and
%% the result is garbage
%% (multiplication of these two large positive numbers would become a negative number).  
%% Usually, at compile time the operations are of the form $x*y$ where the values of $x$
%% and $y$ are known only at runtime. Hence, it is hard to decide whether a particular
%% scale value would lead to overflows or not at compile time. 

%% Thus, scale can neither be very low or very high; we need to find a sweet spot.
%% To determine an appropriate value of $s$, we sweep over all its possible values $\{0,1,\ldots,63\}$
%% and choose the value that leads to the best accuracy. For the example $400.1*0.3+200.1*0.1$,
%% the most accurate result is obtained at $s=24$. In general, machine learning algorithms have a 
%%  validation dataset which is used for hyperparameter tuning. We consider scale as a hyperparameter
%% and select the scale that leads to a fixed-point classifier implementation which performs the best on the validation set.
%% This scheme helps us select values of scales that result in minimal accuracy loss.
%% %The accuracy of the final classifier is reported on the test set. In particular, the test set
%% %is used only for evaluation and {\em not} to generate a classifier implementation. Athos only works with the validation
%% %set and does not have access to the test set. 
%% \end{comment}


%% %0   0   0   1   0
%% %1   0   0   2   0
%% %2   100   1600    4   400
%% %3   100   6400    8   800
%% %4   112.51953125    28805   16    1800
%% %5   131.28515625    134436    32    4201
%% %6   137.53662109375   563350    64    8802
%% %7   137.53662109375   2253400   128   17604
%% %8   138.31977844238281    9064925   256   35409
%% %9   139.49281311035156    36567204    512   71420
%% %10    139.88352966308594    146678518   1024    143240
%% %11    139.88352966308594    586714072   2048    286481
%% %12    139.93247985839844    2347677533    4096    573163
%% %13    140.00579833984375    9395630244    8192    1146927
%% %14    140.03022766113281    37589076214   16384   2294255
%% %15    140.03022766113281    150356317962    32768   4588510
%% %16    140.03327941894531    601438385602    65536   9177221
%% %17    140.03787231445312    2405832211824   131072    18355043
%% %18    140.03939819335938    9623433731112   262144    36710486
%% %19    140.03939819335938    38493734924448    524288    73420972
%% %20    140.03958129882812    153975149517856   1048576   146842145
%% %21    140.03987121582031    615901856782080   2097152   293684891
%% %22    140.03996276855469    2463609105269376    4194304   587370182
%% %23    140.03996276855469    9854436421077504    8388608   1174740364
%% %24    140.03999328613281    39417755753995264   16777216    2349481329
%% %25    140.04000854492188    157671029730223104    33554432    4698962859
%% %26    140.04000854492188    630684118920892416    67108864    9397925718
%% %27    140.04000854492188    2522736502540537856   134217728   18795851636
%% %28    140.04000854492188    10090946010162151424    268435456   37591703273
%% %29    12.040007591247559    3470295893229502464   536870912   6463929811
%% %30    12.040007591247559    13881183572918009856    1073741824    12927859623
%% %31    0.040007509291172028    184502070543384576    2147483648    85915471
%% %32    inf   738008282173538304    4294967296    171830943
%% %33    inf   2952033128694153216   8589934592    343661886
%% %34    inf   11808132514776612864    17179869184   687323773
%% %35    inf   10339041911687348224    34359738368   300905722
%% %36    inf   4462679499330289664   68719476736   64940533
%% %37    inf   17850717997321158656    137438953472    129881067
%% %38    inf   16062639768155979776    274877906944    58435543
%% %39    inf   8910326851495264256   549755813888    16207790
%% %40    inf   17194563332271505408    1099511627776   15638364
%% %41    inf   13438021107957366784    2199023255552   6110904
%% %42    inf   16858596284410363904    4398046511104   3833201
%% %43    inf   12094152916512800768    8796093022208   1374946
%% %44    inf   11483123518632099840    17592186044416    652740
%% %45    inf   9039005927109296128   35184372088832    256904
%% %46    inf   17709279634727632896    70368744177664    251664
%% %47    inf   15496886317781876736    140737488355328   110112
%% %48    inf   6647313049998852096   281474976710656   23616
%% %49    inf   8142508126285856768   562949953421312   14464
%% %50    inf   14123288431433875456    1125899906842624    12544
%% %51    inf   1152921504606846976   2251799813685248    512
%% %52    inf   4611686018427387904   4503599627370496    1024
%% %53    -nan    0   9007199254740992    0
%% %54    -nan    0   18014398509481984   0
%% %55    -nan    0   36028797018963968   0
%% %56    -nan    0   72057594037927936   0
%% %57    -nan    0   144115188075855872    0
%% %58    -nan    0   288230376151711744    0
%% %59    -nan    0   576460752303423488    0
%% %60    -nan    0   1152921504606846976   0
%% %61    -nan    0   2305843009213693952   0
%% %62    -nan    0   4611686018427387904   0
%% %140.040009


\section{Porthos}\label{sec:porthos}

We now describe Porthos, our improved secure 3PC protocol that
provides semi-honest security against one corrupted party
and privacy against one malicious corruption. The notion of privacy
against malicious corruption (introduced by Araki
\etal~\cite{maliciousprivacy}) informally guarantees that privacy of
inputs hold even against malicious party as long as none of the parties participating in the
protocol learn the output of the computation (this is relevant, for
example, when computation is offloaded to servers). Porthos builds
upon SecureNN~\cite{securenn} but makes crucial modifications to
reduce communication.
% and computation overheads. %, however we only rely on its full security against one semi-honest corruption here. 
We first describe our protocols that reduce communication and
summarize concrete improvements in Table~\ref{tab:protcomplexity}.
%We describe the compute optimizations of Porthos in Section \ref{subsec:porthoscomp}.
% In Section \ref{subsec:porthosexperiments}, we show our experimental results illustrating how Porthos outperforms relevant prior works.
% such as SecureNN~\cite{securenn}, ABY3~\cite{aby3}, EzPC~\cite{ezpc}, and CHET~\cite{chet}.

%\subsection{Reducing Communication}\label{subsec:porthoscomm}
%\dg{Need to polish this para.}
We reduce communication for both linear as well as non-linear layers
of DNNs. Linear layers  include fully connected layers as well as
convolutional layers.
We improve the communication for convolutional layers and our optimization gains get better with larger filter sizes.
%We modify how convolutional layers are computed in SecureNN. Our improvements are more pronounced when the network uses large filters in the convolution. 
With regards to non-linear layers (ReLU and MaxPool), we modify how
two of the protocols in SecureNN are used -- ComputeMSB and
ShareConvert. 
As we explain below, this directly translates to better communication for both ReLU and MaxPool computations. 
At a very high level, we trade communication with compute by modifying
the way certain shares are generated in the protocol.
\\\\
\noindent {\bf Convolution.} In SecureNN, secure computation of
convolutional layers is done by reducing them to a (larger) matrix
multiplication. As an example,  $2$-dimensional convolution of a
$3\times 3$ input matrix $X$ (with single input channel and stride 1)
with a filter $Y$ of size $2 \times 2$ reduces to a matrix
multiplication as follows:
%\vspace{-0.19in}
% \[ 
%\begin{small}
\begin{flalign*}
% \scriptsize{
&\mathsf{Conv2d}\left(\begin{bmatrix}
    x_{1}       & x_{2} & x_{3} \\
    x_{4}       & x_{5} & x_{6} \\
    x_{7}       & x_{8} & x_{9}
\end{bmatrix},
\begin{bmatrix}
    y_{1}       & y_{2} \\
    y_{3}       & y_{4}
\end{bmatrix}\right )
% }
= & 
% \]
\end{flalign*}
\[ %\hspace{-0.1cm} 
% \scriptsize{
\hspace{100pt}
\begin{bmatrix}
    x_{1}       & x_{2} & x_{4} & x_5 \\
    x_{2}       & x_{3} & x_{5} & x_{6}\\
    x_{4}       & x_{5} & x_{7} & x_{8}\\
    x_{5}       & x_{6} & x_{8} & x_{9}\\
\end{bmatrix}\times
\begin{bmatrix}
    y_{1}       \\
    y_2 \\
    y_3 \\
	y_4 \\
\end{bmatrix}
% \hfilneg
% }
\]
%\end{small}
%\vspace{-0.15in}
In the above matrix multiplication, we call the left matrix (derived from $X$) as the ``reshaped input'' (say, $X'$) and the right matrix (derived from $Y$) as the ``reshaped filter'' (say, $Y'$). 
%This can be generalized in a similar manner to other dimensions and parameters (such as padding, stride etc.) - see e.g.~\cite{convnotes}. 
The matrix multiplication  is computed securely using a matrix Beaver triple~\cite{beaver,secureml} based protocol. Later, the output can be reshaped to get the output of convolution in correct shape.
In this protocol, matrices being multiplied are masked by random matrices of same size and communicated and hence, the communication grows with the size of the matrices. 
We observe that this is quite wasteful for convolution because the reshaped input image (the first matrix in multiplication) has many duplicated entries (e.g., $x_2$ in row 1 and row 2) that get masked by independent random values. 
%For instance, value $x_2$ in row 1 and row 2 would be masked by independent random values and communicated
Let size of $X$ be $m\times m$ and size of $Y$ be $f\times f$. Then, the size of $X'$ is $q^2\times f^2$, where $q = m-f+1$.
%However, doing this is ``wasteful'' in the number of Beaver triples. To see this, in the above example, when the convolution is expressed as a matrix multiplication on the right side, the variable $x_2$ in row $2$ is treated as a ``new'' variable, different from the variable $x_2$ in row $1$. Hence, a ``fresh'' Beaver triple is utilized when computing the product of this $x_2$ with $k_1$. This can be avoided with the knowledge of the structure of the matrix multiplication. 
In Porthos, we optimize the size of matrix-based Beaver triples for convolution by exploiting the structure of re-use of elements as the filter moves across the image. 
At a high level, we pick random matrix of size matching $X$ for masking and communication only grows with size of $X$ (i.e., $m^2$) instead of $X'$ (i.e., $q^2f^2$) in SecureNN.

Before, we describe our optimized protocol, we set up some notation. Let $\share{x}{t}{0}$ and $\share{x}{t}{1}$ denote the two shares of a 2-out-of-2 additive secret sharing of $x$ over $\bbZ_t$ -- in more detail, pick $r \atrand \bbZ_t$, set $\share{x}{t}{0} = r$ and $\share{x}{t}{1} = x-r\ (\mathsf{mod}\ t)$. $\shareval{x}{t}$ denotes a sharing of $x$ over $\bbZ_t$. 
%The algorithm $\genshare{t}{x}$ generates the two shares of $x$ over $\bbZ_t$ and algorithm $\reconst{t}{x_0, x_1}$ reconstructs a value $x$ using $x_0$ and $x_1$ as the two shares (reconstruction is simply $x_0+x_1$ over $\bbZ_t$). 
Reconstruction of a value $x$ from its shares $x_0$ and $x_1$ is simply  $x_0+x_1$ over $\bbZ_t$.
This generalizes to larger dimensions - e.g. for the $m\times n$ matrix $X$, $\share{X}{t}{0}$ and $\share{X}{t}{1}$ denote the matrices that are created by secret sharing the elements of $X$ component-wise (other matrix notation such as $\reconst{t}{X_0,X_1}$ are similarly defined). 

Let $\mathsf{Conv2d}_{m,f}$ denote a convolutional layer with input $m\times m$, $1$ input channel, a filter of size $f\times f$, and $1$ output channel. 
Our protocol for $\mathsf{Conv2d}_{m,f}$ is described in Algorithm~\ref{algo:conv2d}, where $L = 2^\ell$, $\ell=64$. Algorithms  $\mathsf{ReshapeInput}$, $\mathsf{ReshapeFilter}$ and $\mathsf{ReshapeOutput}$ are used to reshape input, filter and output as described above and are formally described in Appendix \ref{appendix:porthos}. 
Parties $P_0$ and $P_1$ start with shares of input matrix $X$ and filter $Y$ over $\bbZ_L$ That is, $P_j$ holds $(\share{X}{L}{j}, \share{Y}{L}{j})$ for $j \in \{0,1\}$.
In SecureNN, $P_0$ first reshapes $\share{X}{L}{0}$ into $\share{X'}{L}{0}$ by running $\mathsf{ReshapeInput}$. Then, it picks a random matrix $\share{A'}{L}{0}$ of same size as $X'$ and sends $\share{E'}{L}{0} = \share{X'}{L}{0} - \share{A'}{L}{0}$ to $P_1$ that requires communicating $q^2f^2$ elements.
In Porthos, we optimize this as follows: $P_0$ picks a random matrix $\share{A}{L}{0}$ of same size as $X$ (Step 1) and sends $\share{E}{L}{0} = \share{X}{L}{0} - \share{A}{L}{0}$ to $P_1$ (Step 4) that requires communicating $m^2$ elements only. Later, parties can reshape $E$ locally to get $E'$. 
We reduce the communication by $P_1$ in a symmetric manner.
Concretely, we reduce communication from $(2q^2f^2+2f^2+q^2)\ell$ in SecureNN to $(2m^2+2f^2+q^2)\ell$.
This algorithm can  be easily generalized to the setting where there are $i$ input filters, $o$ output filters, and different stride and padding parameters. 

\begin{algorithm}
\KwIn{$P_0$  holds $(\share{X}{L}{0}, \share{Y}{L}{0})$ and $P_1$ holds $(\share{X}{L}{1}, \share{Y}{L}{1})$, where $X \in \bbZ_L^{m \times m}$, $Y \in \bbZ_L^{f \times f}$.}
\KwOut{$P_0$ gets $\share{\mathsf{Conv2d}_{m,f}(X,Y)}{L}{0}$ and $P_1$ gets $\share{\mathsf{Conv2d}_{m,f}(X,Y)}{L}{1}$.}
\textbf{Common Randomness}: $P_0$ \& $P_1$ hold shares 
%$\share{U}{L}{j}$, $j \in \{0,1\}$, 
of a zero matrix $U$ of dimension $q \times q$, $q = m-f+1$ . $P_0$ \& $P_2$ hold a common PRF key $k_0$, and $P_1$ \& $P_2$ hold a common PRF key $k_1$.
\begin{enumerate}
\item $P_0$ \& $P_2$ use PRF key $k_0$ to generate random matrices 
%$\share{A}{L}{0}$, $\share{B}{L}{0}$ and $\share{C}{L}{0}$ where 
$\share{A}{L}{0} \in \bbZ_L^{m \times m}$, $\share{B}{L}{0} \in \bbZ_L^{f \times f}$ and $\share{C}{L}{0} \in \bbZ_L^{q\times q}$.
\item $P_1$ \& $P_2$ use PRF key $k_1$ to generate random matrices 
%$\share{A}{L}{1}$ and $\share{B}{L}{1}$ where 
$\share{A}{L}{1} \in \mathbb{Z}_L^{m \times m}$ and  $\share{B}{L}{1} \in \mathbb{Z}_L^{f \times f}$.
%\end{itemize}
\item $P_2$ computes $A = \share{A}{L}{0} + \share{A}{L}{1}$ and $B = \share{B}{L}{0} + \share{B}{L}{1}$. 
Let
%\item $P_2$ now computes 
$A' = \mathsf{ReshapeInput}(A)$ and $B' = \mathsf{ReshapeFilter}(B)$. $P_2$ computes $\share{C}{L}{1} = A' \cdot B' - \share{C}{L}{0}$ and sends it to $P_1$.
\item For $j \in \{0,1\}$, $P_j$ computes $\share{E}{L}{j} = \share{X}{L}{j} - \share{A}{L}{j}$ and $\share{F}{L}{j} = \share{Y}{L}{j} - \share{B}{L}{j}$ and sends to $P_{j\oplus 1}$.
\item $P_0$ \& $P_1$ reconstruct $E$ and $F$ using exchanged shares.
\item For $j \in \{0,1\}$, $P_j$ computes $\share{X'}{L}{j} = \mathsf{ReshapeInput}(\share{X}{L}{j})$, $E' = \mathsf{ReshapeInput}(E)$,  $\share{Y'}{L}{j} = \mathsf{ReshapeFilter}(\share{Y}{L}{j})$, $F' = \mathsf{ReshapeFilter}(F)$.
%\item For $j \in \{0,1\}$, $P_j$ computes $\share{X'}{L}{j} = \mathsf{ReshapeInput}(\share{X}{L}{j})$ and $\share{Y'}{L}{j} = \mathsf{ReshapeFilter}(\share{Y}{L}{j})$, followed by $E' = \mathsf{ReshapeInput}(E)$ and $F' = \mathsf{ReshapeFilter}(F)$.
\item For $j \in \{0,1\}$, $P_j$ computes $\share{Z'}{L}{j} = -jE' \cdot F' + \share{X'}{L}{j} \cdot F' + E' \cdot \share{Y'}{L}{j} + \share{C}{L}{j} + \share{U}{L}{j}$.
\item For $j \in \{0,1\}$, $P_j$ outputs $\share{Z}{L}{j} = \mathsf{ReshapeOutput}(\share{Z'}{L}{j})$.
\end{enumerate}
    \caption{{3PC protocol for $\mathsf{Conv2d}_{m,f}$ } \label{algo:conv2d}}

\end{algorithm}

\noindent {\bf Activation Functions.} 
In SecureNN protocols for computing activations such as ReLU and MaxPool start with parties $P_0$ and $P_1$ having shares of values over $L = 2^{64}$. 
For both of these, parties run a protocol called $\computemsb$ to evaluate most significant bit (MSB) of secret values. This protocol require shares over $L-1$. So parties run a protocol called $\shareconvert$ to convert shares over $L$ to shares over $L-1$. Both protocols $\computemsb$ and $\shareconvert$ require $P_2$ to send fresh shares of a value to $P_0$ and $P_1$. In SecureNN, both of these shares were picked by $P_2$ and explicitly communicated to $P_0$ and $P_1$.  As mentioned before, shares of a value $x$ are $r$ and $x - r$, where $r$ is a appropriately picked uniformly random value. We observe that since one of the shares is truly random, it can be computed as the output of a shared PRF key between $P_2$ and one of the parties, say $P_0$. This cuts the communication of this step to {\em half}. Moreover, since many activations are computed in parallel, we can carefully ``load-balance'' this optimization between $P_0$ and $P_1$ to reduce the communication to half on the critical path. We implement this load-balance optimization and observe that this reduces the overall communication of $\shareconvert$, $\computemsb$, $\relu$ and $\maxpool$ by $25\%$. 

%When computing activation functions such as $\relu(x)$ and their derivatives, SecureNN goes through a series of protocols where parties $P_0$ and $P_1$ start with additive 2-out-of-2 shares of $x$ over $\bbZ_L$. They move to shares of the same $x$ over $\bbZ_{L-1}$ (with the help of $P_2$) using a protocol called $\shareconvert$, and then convert the problem of computing $\relu$ to computing the LSB of $x$ when $x$ is shared over $\bbZ_{L-1}$, using another protocol called $\computemsb$. In both $\shareconvert$ and $\computemsb$ protocols, $P_2$ must generate fresh shares of a value that is reconstructed in the protocol and send them to $P_0$ and $P_1$. In SecureNN, both these shares were computed by $P_2$ and sent to $P_0$ and $P_1$. We observe that since one of these shares is truly random, they can be computed through a PRF key shared between $P_2$ and (say) $P_0$, and hence only one of these shares must be communicated to (say) $P_1$. This roughly halves the communication of these protocols. Additionally, since these protocols are called in parallel several times, in order to ``load balance'' the communication, we have $P_2$ set the share to be given to $P_0$ as a PRF output in half of the invocations and the share to be given to $P_1$ as a PRF output in the other half of the invocations. 

The revised table with comparison of overall communication complexity of all protocols with improvements over SecureNN are provided in Table \ref{tab:protcomplexity}. 
%Only protocols whose communication complexity are an improvement over SecureNN are presented. 
$\mathsf{Conv2d}_{m,i,f,o}$ denotes a convolutional layer with input $m\times m$, $i$ input channels, a filter of size $f\times f$, and $o$ output channels. 
%$\drelu$ denotes the derivative of $\relu$. 
$\maxpool_n$ computes the maximum value out of a list of $n$ elements. $\prm$ denotes a prime value strictly larger than $65$ (set to $67$ in SecureNN), with $8$ bits being used to represent elements in $\bbZ_\prm$ (hence $\log \prm = 8$ in the table). %The round complexity of none of the protocols change and remain the same as in SecureNN.

\begin{table}
  \centering
  \resizebox{\columnwidth}{!}{
      \begin{tabular}{|l|l|l|l|}
    \hline
    Protocol & Communication (SecureNN) & Communication (Porthos) \\    \hline
 $\mathsf{Conv2d}_{m,i,f,o}$ & $(2q^2f^2i+2f^2oi+q^2o)\ell$ & $(2m^2i+2f^2oi+q^2o)\ell$\\ \hline
	$\shareconvert$ & $4\ell\log\prm+6\ell$ & $3\ell\log\prm+5\ell$ \\ \hline
    $\computemsb$ & $4\ell\log\prm+13\ell$ & $3\ell\log\prm+9\ell$ \\ \hline
   
    
$\relu$ & $8\ell\log\prm+24\ell$ & $6\ell\log\prm+19\ell$  \\ \hline
       $\maxpool_n$ & $(8\ell\log\prm+29\ell)(n-1)$ & $(6\ell\log\prm+24\ell)(n-1)$\\ \hline
 

\end{tabular}
}
 \caption{Communication complexity of protocols; $q = m-f+1$ and $\log \prm = 8$.}
\label{tab:protcomplexity}
%\tableup
	% \vspace{-0.8cm}
\end{table}

 


\begin{comment}
\subsection{Optimizing Computation}\label{subsec:porthoscomp}

In this section we describe some of the computation optimizations that are done in Porthos.
\\\\
\noindent\textbf{AES computation.} Porthos requires a lot of randomness (sometimes shared between pairs of parties) that are generated through AES NI~\cite{aesni} function calls (used as a pseudorandom function (PRF) with a shared key and counter). In larger networks like \resnet, Porthos requires of the order of a billion AES calls. While each AES NI call only requires tens of nanoseconds to compute, computing these values in a single-thread during the protocol is quite costly. Hence, at the beginning of the protocol, we compute all randomness required and store them in a large array. These values are read as and when required through specific \textsf{memcpy} instructions, so as to derive cache benefits.
\\\\
\noindent\textbf{Prime selection.} In SecureNN~\cite{securenn}, one of the main sub-protocols, called PrivateCompare, makes use of a finite field, whose size must be larger than $65$. They choose $\bbZ_{p}$ with $p = 67$ as the finite field. However, for compute efficiency reasons, the elements of $\bbZ_{67}$ are represented as $8-$bit integers (thereby resulting in slightly larger communication). Now, when sampling a random element from $\bbZ_{67}$, a random $8-$bit integer is sampled (using AES as the PRF) and the integer value $\mathrm{mod}~67$ is used if it is $< 3*67$, and discarded otherwise. This is done in order to ``uniformly'' sample from $\bbZ_{67}$. This results in many ``wasted'' AES calls as roughly $20\%$ of the AES calls will result in discarded values. To overcome this problem, we set $p = 127$. By doing this, we can use the value $\mathrm{mod}~127$ whenever the $8-$bit value is $< 254$ and AES values are discarded only with probability roughly $0.7\%$. It is to be noted that this optimization only makes sense on larger networks that require a very large number of AES calls for randomness and where compute also is a bottleneck.
\\\\
\noindent\textbf{Refactoring code.} We re-factored the SecureNN codebase~\cite{securenncode} in order to improve its performance in several places. For example, since matrix multiplication is a large cost in bigger benchmarks, we reduce the number of local matrix multiplication calls computed by each party during $\mathsf{MatMul}$ as well as $\mathsf{Convolution}$ protocols by restructuring code and ensuring that each party computes at most $2$ local matrix multiplications. Similarly, in these protocols, by carefully re-ordering operations, we ensure that parties are not idle and waiting for messages from other parties, and that all the local matrix multiplication computations by parties are in parallel.
\end{comment}

\section{Aramis}\label{sec:aramis}
%\vspace{-0.1cm}

In this section, we describe Aramis, a general technique to convert
any semi-honest secure MPC
protocol into a secure MPC protocol tolerating malicious
corruptions by relying on secure hardware. The threshold of corrupted parties tolerated by the
semi-honest protocol is retained in the malicious
secure protocol by our technique. 
\\\\
\noindent\textbf{Threat Model.} We consider a strong threat model where not only does the adversary control the operating system of the corrupted parties 
(i.e., the host operating system is outside the Trusted Computing Base)
but also observes the entire state of their secure hardware.
Aramis makes a very minimal
trust assumption of {\em integrity} on
hardware, namely that of code attestation (the outputs generated by the hardware are indeed from the code that it attested to).
This implicitly requires
the hardware to possess a trusted component that can produce
signatures and this signature scheme cannot be forged
by the adversary. 
However, the adversary can see the state  (i.e., all the code and the user data) of the hardware belonging to the corrupted parties, i.e., we {\em do not} assume {\em confidentiality} of state.
Prior works~\cite{vc3, obliviousmpml, GuptaFC16, BahmaniFC17,gcsgx,ndss1,ndss2,ndss3,slalom,opaque,chiron} that combine \mpc and hardware (SGX)  make stronger trust assumption on the hardware of both confidentiality and integrity,
 and hence, provide security only in a weaker threat model where the hardware hides the data residing in it from the adversary. 
%Aramis makes a very minimal
%trust assumption of {\em integrity} on
%hardware: the code and data residing in the hardware cannot be modified by the adversary.
%This implicitly requires
%the hardware to possess a trusted component that can produce
%signatures and this signature scheme cannot be forged
%by an adversary. 
%We do not assume confidentiality, that is, the adversary can see all the code and user data that resides in  the hardware belonging to the corrupted parties. 
%This significantly weakens the trust assumption on hardware compared to prior work that combines \mpc and SGX (\cite{vc3, obliviousmpml, GuptaFC16, BahmaniFC17,gcsgx,ndss1,ndss2,ndss3,slalom}).
\\\\
\noindent\textbf{Overview.} At a very high level, Aramis
exploits the following (well-known) observation: in order for a
semi-honest protocol to be made maliciously secure, one must ensure
that all messages sent by every party $P_i$ are computed honestly
according to the specification of the semi-honest protocol consistent
with $P_i$'s input and the transcript so far.
The next observation we make is that if party $P_i$ possesses hardware
whose code can be attested by party $P_j$ (and vice-versa), then $P_j$
can obtain guarantees on the correctness of protocol messages sent by
$P_i$ as long as these messages are computed and signed by $P_i$'s
hardware.
Using these observations, we can convert a semi-honest secure protocol into one that is maliciously secure by having every protocol message of $P_i$ be computed by the trusted hardware that $P_i$ executes. 
We shall now describe our techniques in more detail. 
We first describe the ideal functionality that is assumed out of the hardware in Section \ref{subsec:sgxideal}. 
We then describe our technique in Section~\ref{sec:shtomcompiler}.
%
Finally, we provide an implementation of Aramis using Intel SGX as the underlying secure hardware. 
We explain how Intel SGX can realize the ideal functionality
in Section~\ref{sec:fattest-sgx} and challenges in porting semi-honest
\mpc protocols to SGX in Section~\ref{sec:challenges-aramis}.
%The structure of the underlying semi-honest secure MPC protocol that we use is formalized in Section \ref{subsec:nextmessagefunction} and we present our compiler in Section \ref{subsec:shtomcompiler}. Finally, we describe some of the challenges in porting secure computation protocols to SGX in Section \ref{subsec:porting}.
%\vspace{-0.23cm}

\subsection{The attestation ideal functionality $\fattest$}
\label{subsec:sgxideal}
\noindent\textbf{Description.} We formally define the ideal functionality for attested executions 
%of deterministic functions 
in Figure \ref{fig:attestideal}.
% and later describe in Section~\ref{sgxtoideal} how Intel SGX can be used to realize this functionality.
The functionality is parameterized by a signing key pair $(\vk,\sksign)$. Let 
 $\sign_{\sksign}(m)$ denote the signing algorithm on message $m$ and $\verify_{\vk}(m,\sigma)$ denote verification of signature $\sigma$ on message $m$. 
At a high level, this functionality allows users to specify a function $g$ to the ideal functionality once using the $\gcommit$ command. 
The functionality returns a token $\token_g$ generated as $\sign_\sksign(H(g))$, where $H$ is a collision resistant hash function. 
Note that this token is publicly verifiable given $g$ and $\vk$.
Let $\st_\ctr$ be an internal state that the functionality maintains,
indexed by $\ctr$ -- this state can be maintained by signing it along
with $\ctr$ and verifying the signature of the state on every input
message.
When the functionality $\fattest$ is initialized, the initial state $\st_0$ is empty (or, $\epsilon$). % where $r$ denotes all the randomness that $g$ will ever use. 
Subsequent invocations of the functionality is done on input $w_\ctr$
% and program counter $\ctr$ 
using the $\compute$ command. 
The function $g$ is a deterministic mapping from $(\ctr,w_{\ctr},r_\ctr,\st_{\ctr-1})$ to
$(y_\ctr,\st_{\ctr})$, where $r_\ctr$ is the required randomness.
% picked by $\fattest$. 
The functionality picks randomness $r_\ctr$, evaluates $g$ and provide a signature on the function output $y_\ctr$ using the signing key $\sksign$.  
Furthermore, $(y_\ctr,\st_\ctr)$ is always given to party $P$ such that $\st_\ctr$ contains $r_\ctr$ in clear and 
%$g$ is a
%deterministic function mapping $(\ctr,w_{\ctr},r,\st_{\ctr-1})$ to
%$(y_\ctr,\st_{\ctr})$. 
%
%Note, that doing so 
this ensures that there is no information hidden from $P$ and we only assume correct execution of $g$.
That is, the ideal functionality can evaluate functions and provide signed outputs and these outputs could have anyway been computed by party $P$ given
knowledge of $g, w_\ctr, r_\ctr, \ctr, \st_\ctr$, which are all known to $P$. 
Thereby, we only assume that the functionality will sign the output of $g$ on the appropriate input and not hide any  data from $P$. This significantly weakens what is assumed from the trusted hardware.


%\vspace{0.3in}

%\aseem{Figure 8, shouldn't ideal functionality store g somewhere, or it is understood?}

\begin{tffbox}
\begin{mdframed}
\begin{center}
{\bf Functionality} $\fattest^{(\vk,\sksign)}$
\end{center}
%\vspace{.1in}
{\small

$\fattest$ interacts with a party $P$.

\begin{tiret}
       \item On input message $(\gcommit,g)$ from $P$,
% sample $r$ from $\zo^*$ at random, where $r$ is all the randomness that computation of $g$ will ever need.
       \begin{enumerate}
       \item Record $(\gcommit,\st_0)$, where $\st_0 = \epsilon$;
       \item Send $( \st_0, \token_g)$ to $P$, where $\token_g = \sign_\sksign(H(g))$.
       \item Ignore further $(\gcommit,g)$ messages.
       \end{enumerate}
	\item  On input message $(\compute,w_\ctr)$ from $P$, retrieve $\st_{\ctr-1}$, pick required randomness $r_\ctr$ and compute $g(\ctr,w_\ctr,r_\ctr,\st_{\ctr-1})$ to obtain $(y_\ctr,\st_{\ctr})$ such that $\st_\ctr$ contains $(y_\ctr, r_\ctr)$. Send $(y_\ctr,\ctr, \sign_{\sksign}(y_\ctr || \ctr),\st_\ctr)$ to $P$.
	       			
\end{tiret}
%\vspace{.1in}
} %SMALL
\end{mdframed}
\caption{\sl The Authentication functionality $\fattest^{(\vk,\sksign)}$.}
\label{fig:attestideal}
	%\vspace{-0.2cm}
\end{tffbox}


\subsection{Semi-honest security to malicious security}
\label{subsec:nextmessagefunction}

Our technique takes any semi-honest secure MPC protocol and converts
it into a malicious secure MPC protocol in the
$\fattest^{(\vk,\sksign)}-$hybrid model. The idea is to have messages
sent by every party $P_i$ to every other party $P_j$ in the
semi-honest protocol be computed by the corresponding
$\fattest^{(\vk_i,\sksign_i)}$ functionality interacting with $P_i$,
where $(\vk_i,\sksign_i)$ are keys used by the functionality.
These messages can be verified by functionality $\fattest^{(\vk_j,\sksign_j)}$ interacting with $P_j$. 
We assume that every $\fattest^{(\vk_i,\sksign_i)}$ knows the verification key $\vk_j$ used by functionalities of all other parties $P_j$ in a reliable manner. 
Later, we show how to achieve this through the use of remote attestation in the context of Intel SGX. We now set notation and describe the next message function of any semi-honest secure MPC protocol and how we modify it for our use.
\\\\
\noindent\textbf{Next message function.} Let $\pi(\cdot)$ be the next message function of any semi-honest secure MPC protocol. $\pi(\cdot)$ takes the following values as input - two party ids $i$ and $j$, input $x_i$, a round number $\ctr$, randomness $r_{i,j,\ctr}$ and $\transcript_i$, which includes the transcript of all messages sent and received by the party $P_i$ so far. 
Given these, $\pi(\cdot)$ outputs $y_{\ctr}^{i,j}$, which is the message that $P_i$ must send to $P_j$ in round $\ctr$ and also updates $\transcript_i$ appropriately. Additionally, $\pi(\cdot)$ takes message $y_{\ctr}^{j,i}$ sent by $P_j$ to $P_i$ at round $\ctr$ and update $\transcript_i$ with this message. We now describe how to modify $\pi(\cdot)$ to $\pi^*(\cdot)$ to incorporate checks to detect malicious behavior.
\\\\
\noindent\textbf{Modified next message function.} $\pi^*(\cdot)$, is the modified function that builds upon $\pi(\cdot)$ and we describe it for $P_i$.
%We will assume that every $\fattest^{g,(\vk_i,\sksign_i)}(P_i)$ knows and agrees upon $\pi(\cdot)$; we will show how to achieve this as well through attestation. 
%The modifications made to the next message function are now described:

\begin{enumerate}

\item For $\ctr = 1$, 
%$\pi^*(\cdot)$, picks an execution identity $\sid$. 
Let $x_i$ be the input of $P_i$ in $\pi(\cdot)$. Then,
%let $y^i_0$ denote $x_i$, and the randomness $r_i$ that will be used by $P_i$ in $\pi(\cdot)$. 
$(\ctr, x_i)$ is stored as $\st_1$ (also called as $\transcript^1_i$) and sent to $P_i$.
% after signing it with $\sksign_i$. 

\item When $\pi^*(\cdot)$ receives a message $M =
  (y_{\ctr}^{j,i},\ctr,\sigma)$ from party $P_j$, it runs
  $\verify_{\vk_j}((y_{\ctr}^{j,i},\ctr),\sigma)$. If verification
  succeeds, it appends $M$ to $\transcript_i$. Else, $P_i$ aborts.

\item $\pi^*(\cdot)$ on input $(\ctr, \st_{\ctr-1}, j)$ computes the
  next message from $P_i$ to $P_j$ as follows: It checks that
  $\st_{\ctr-1}$ contains a valid transcript of all messages computed
  so far. If it verifies, it picks randomness $r_{i,j,\ctr}$ and runs $\pi(\ctr, \st_{\ctr-1}, j, r_{i,j,\ctr})$ to compute next message $y^{i,j}_\ctr$ and updated state $\st_\ctr$ (containing $r_{i,j,\ctr}$). 
  Else it outputs $\bot$. 
Note that $\st_{\ctr-1}$ already contains
  input $x_i$, the input of party $P_i$.
\end{enumerate}~\\
\noindent\textbf{Malicious MPC in the $\fattest-$hybrid model.}
\label{sec:shtomcompiler}
The malicious \mpc protocol works as follows: Each party $P_i$ invokes $\fattest^{(\vk_i,\sksign_i)}$ with the command $\gcommit$ using function $\pi^*(\cdot)$ described above and sends the received token $\token^{(i)}_{\pi^*}$ to other parties $P_j$. It  receives similar tokens $\token^{(j)}_{\pi^*}$ from party $P_j$ and verifies it under $\vk_j$. Party $P_i$ aborts if any of these verifications fail. If all verifications succeed, it proceeds with running $\pi^*(\cdot)$ inside $\fattest^{(\vk_i,\sksign_i)}$ as described formally in Figure~\ref{fig:shtomprotocol}.

%whenever party $P_i$ must send a message according to $\pi(\cdot)$ to party $P_j$ in the underlying semi-honest secure protocol in round $\ctr$, $P_i$ invokes $\fattest^{\pi^*(\cdot)}$ along with the appropriate messages needed to compute the message $y_{\ctr}^{i,j}$. The functionality will check the validity of all messages and transcript and if correct, will produce $y_{\ctr}^{i,j}$ along with a signature on it (along with $\ctr,\sid$). This signed message is then passed on by $P_i$ to $P_j$ as the message in that round. 
%The complete protocol is in Figure \ref{fig:shtomprotocol}.

%\vspace{0.3in}

%%\aseem{(a) What is sid in Figure 9, is it just i? (b) In the fifth
%  step, it invokes F sub attest on (sid, Compute, j), is the last
%  argument j? Shouldn't it be the output of pi star? (c) The order
%  of ctr, sigma, y sub i,j is inconsistent, it won't typecheck :).}

\begin{tffbox}
\begin{mdframed}
\begin{center}
{\bf Protocol} $\protshtom$
\end{center}
%\vspace{.1in}
{\small

Party $P_i$ with input $x_i$ interacts with $\{P_j\}_{j\ne i}$ and $\fattest^{(\vk_i,\sksign_i)}$ and does the following:
\begin{tiret}
\item Invokes $\fattest^{(\vk_i,\sksign_i)}$ on $(\gcommit,\pi^*)$ to receive $( \st^{(i)}_0, \token^{(i)}_{\pi^*})$ and sends $\token^{(i)}_{\pi^*}$ to all parties $P_j$, $j \ne i$.

\item Receives $\token^{(j)}_{\pi^*}$ from $P_j$ and runs $\verify_{\vk_j}(H(\pi^*), \token^{(j)}_{\pi^*} )$ for all $j \in [n]\setminus i$. Aborts if one of these checks fail.

\item Invokes $\fattest^{(\vk_i,\sksign_i)}$ on $(\compute, x_i)$ to get $\transcript_i^1$ containing input $x_i$. %randomness $r_i$ that would be used to generate future messages from $P_i$.

\item When $P_i$ receives a message $M = (y_{\ctr}^{j,i},\ctr,\sigma)$ from party $P_j$, it invokes $\fattest^{(\vk_i,\sksign_i)}$ on  $( \compute, (y_{\ctr}^{j,i},\ctr,\sigma))$ and receives updated transcript or $\bot$ (and aborts).

\item When $P_i$ needs to send next message to $P_j$ it invokes $\fattest^{(\vk_i,\sksign_i)}$ on $( \compute, j)$ and receives $(y_{\ctr}^{i,j},\ctr, \sigma)$ along with updated transcript and randomness used. Here, $\sigma$ is a signature on $(y_{\ctr}^{i,j},\ctr)$ under $\sksign_i$. It sends $(y_{\ctr}^{i,j},\ctr, \sigma)$ to $P_j$.
   
   \item When $P_i$ has no more messages to send in $\pi(\cdot)$, it computes the output of the function from 
%transcript messages 
$\transcript_i$.
	       			
\end{tiret}
%\vspace{.1in}
} %SMALL
\end{mdframed}
\caption{\sl Malicious secure MPC $\protshtom$.}
\label{fig:shtomprotocol}
\end{tffbox}


%\vspace{0.3in}
\begin{tffbox}
\begin{mdframed}
\begin{center}
{\bf Functionality} $\fmpc^f(P_1,\cdots,P_n)$
\end{center}
%\vspace{.1in}
{\small

$\fmpc^f$ interacts with parties $\{P_1,\cdots,P_n\}$ and the adversary $\simu$.

\begin{tiret}
       \item On input message $x_i$ from $P_i$ record $x_i$ and ignore further $x_i$ from $P_i$
       \item Upon receiving $x_i$ from all $P_i, i\in [n]$, compute $y = f(x_1,\cdots,x_n)$ and send to $\simu$.
       \item Upon receiving $(i,\mathsf{Send})$ or $(i,\bot)$ from $\simu$, send $y$ or $\bot$ to $P_i$.
	       			

	       			\end{tiret}
%\vspace{.1in}
} %SMALL
\end{mdframed}
\caption{\sl The MPC functionality $\fmpc^f$.}
\label{fig:fideal}
%	\vspace{-0.5cm}
\end{tffbox}
%\vspace{-1cm}

\noindent\textbf{Malicious Security.} Next, we prove that if $\pi$ is secure against semi-honest adversaries, then the protocol described in Figure~\ref{fig:shtomprotocol} is an \mpc protocol secure against malicious adversaries with the same corruption threshold.  We prove the following result using the standard simulation paradigm in Appendix~\ref{app:proof}.

\begin{theorem}
\label{theorem:maliciousmpc}
Let $\pi(\cdot)$ be a semi-honest secure MPC protocol securely realizing $\fmpc^f$. Then, protocol $\protshtom$ described in Figure \ref{fig:shtomprotocol} securely realizes $\fmpc^f$ in the $\fattest^{(\vk_i,\sksign_i)}-$hybrid model (with $i \in [n]$) against malicious adversaries.
\end{theorem}

\subsection{Realizing $\fattest$}
\label{sec:fattest-sgx}
We note that the ideal functionality assumed out of the hardware can potentially be realized using various hardware platforms that provide code attestation and secure signing, e.g., STM32H7, MediaTek MT3620, CEC1702, ARMTrustZone, Intel SGX, etc. In this work, we provide an implementation of Aramis based on Intel SGX. 

SGX allows a host to create a protected region known as an enclave. Intel gives integrity guarantees, that is, the code and the data residing in the enclave, once attested, cannot be modified by the host or the operating system. 
When SGX receives a $\gcommit$ command (Figure~\ref{fig:attestideal}) for a function $g$, then it creates an enclave with code $g$.
Randomness $r_\ctr$ of Figure~\ref{fig:attestideal} can be sampled in SGX using {\tt sgx\_read\_rand} command. 
The attestation token $\token_g$ is generated by SGX communicating with Intel's Attestation Service (IAS) and this token is publicly verifiable given $g$ and  public verification key corresponding to Intel's Report Signing Key.
The key-pair $(\vk,\sksign)$ for ECDSA signature scheme is also generated inside the enclave and the verification key $\vk$ is sent as payload to IAS during the generation of the attestation token. 
The token $\token_g$ contains the verification key $\vk$ in the clear and this $\vk$ can be used to verify the signed outputs $y_\ctr$.
Now, on receiving the $\compute$ command, the enclave starts executing the code of $g$ and produces outputs signed under $\sksign$.

While running \mpc in the $\fattest$-hybrid, we require the enclave to reliably have verification keys used by enclaves of all other parties. This can be done by attaching the following prelude to $\pi^*$ (the code running inside SGX): Read the tokens of all parties, parse them to obtain the verification keys, and verify the signature on the tokens using verification key of Intel's Report Signing key. Note that since all the parties are running the same function $\pi^*$ (appended with this prelude), they can compute the hash of $\pi^*$ locally and compare it with the hash in the tokens (which has been signed by Intel's IAS) of all the other parties, proceeding only if they all match perfectly.

\subsection{Implementation challenges with Intel SGX}
\label{sec:challenges-aramis}

%\dg{SGX memory issues - [ReLU chunking, liveness ], ecall-ocall message passing payload optimizing, MAC}

We outline some of the key challenges in implementing \mpc between multiple SGX enclaves that involve multiple rounds of interaction and operate over large volumes of data.

 
\subsubsection{Memory constraints in SGX}
In SGX, all the enclave content, including code, and related data is stored in a special region of memory known as the Enclave Page Cache (EPC). The size of EPC is fixed in BIOS and can have a maximum size of 128MB. Typically, paging facilitates the execution of enclaves which cannot fit in EPC and any page that is evicted out is encrypted before storing it on unprotected memory \cite{intelsgxperf}. This additional overhead  has  detrimental effects on the overall performance of the enclave application. 
We reduce the working set of secure inference tasks to limit these overheads.
%To overcome this, we make changes to our code in the following way:
\begin{itemize}
	\item \textit{ReLU and MaxPool functions:} 
  % We split the computation of memory intensive non-linear functions into chunks  that fit  in EPC to avoid paging. For example, a secure ReLU computation that requires 120MB memory is split into 3 chunks  of 40MB each. Note that chunking increases the number of \mpc rounds and very small chunks are actually detrimental to performance.
  We split the computation of memory-intensive non-linear functions into chunks that fit in the EPC to avoid paging. However, lower chunk sizes  increase  the number of rounds, and so, the chunk sizes must be carefully selected. For \resnet, we  set the chunk sizes for ReLU and MaxPool layers to be 40 MB and 10 MB respectively. For our network configurations, the increase in rounds is justified by the elimination of paging costs and reduction in end-to-end runtimes.
	\item \textit{Convolution and Matrix Multiplication functions:} For the linear functions, we block the matrices into smaller ones, process the blocks, and aggregate them. We ensure that individual blocks fit in EPC. 
	\item \textit{Liveness Analysis:} Athos implements liveness analysis (Section~\ref{sec:athosopt}) which reduces the memory footprint of the compiled DNNs.  For example, the memory footprint of \resnet reduces from 1100 MB to 397 MB due to liveness analysis. When chunking and liveness analysis are done together, the memory footprint of \resnet comes down to 297MB.

\end{itemize}

%For the case of \resnet, we chose a chunksize of 40, 10 and 40 MiB for ReLU, MaxPool and Convolution, respectively. The number of rounds of a function increases linearly with the number of chunks needed to complete the function evaluation. For example, in \resnet, the ReLU with the maximum elements requires 5 chunks for 40 MiB each, making the rounds grow from 10 to 50. The values of chunksizes were chosen empirically to find a sweet-spot between increased number of rounds and decreased working set memory. In isolation to this chunking optimization, with liveness analysis, we observed that the memory footprint of \resnet comes down from 1100 MB to a mere 397 MB. Both these optimizations played a major role in drastically improving the performance of Aramis for our benchmarks.

\subsubsection{Porting Interactive Protocols to SGX}
To the best of our knowledge, we are the first work to implement highly interactive protocols in SGX and this comes with unique challenges. For example, whenever data is passed across the enclave's protected memory region, it has to be {\em marshalled} in/out of the region.
%\footnote{When pointers to memory are passed as parameters into the enclave via an $\mathsf{ecall}$, the referenced data block is {\em marshalled} into the enclave, specifically into the protected memory region that an enclave uses. Similarly, when a pointer to enclave data, residing in protected memory region, is passed outside an enclave via an $\mathsf{ocall}$, the referenced data block is marshalled out of the protected memory region.}. 
The performance of marshalling depends on the size of the parameters crossing the bridge. Larger parameters imply slower marshalling~\cite{intelsgxperf}, while smaller parameters increase the total numbers of cross-bridge calls (which have an overhead of their own). 
%For example, we observed that in order to send 1MB data in/out of the enclave, if 1048576 calls (OCALL-ECALL pair) are done with each carrying a payload of 1B, then it takes about 3.1 s for the calls alone, 16 calls with payload of 65KB each only take about 0.5 ms and 2 calls of 512KB each take 0.8 ms. Since we make calls of the order of 100000, 
Thus, we tune the payload size carefully. 
We also implement the techniques in~\cite{sealedglass}  for optimizing communication involving enclaves. 
%Apart from this, for efficiency purposes, we also employ other common optimizations for establishing symmetric key authenticated communication channels between enclaves as has already been suggested in \cite{sealedglass}.



\begin{comment}

\subsection{Porting crypto protocols to SGX}\label{subsec:porting}

In this section we describe how keys and code are exchanged and attested by the enclaves during the setup phase and also discuss other technical issues when porting crypto protocols to SGX. %Before doing so, we set some notation and provide some background on Intel SGX.
%\\\\
%\noindent\textbf{Background on Intel SGX.} %We present a very high level overview over here and defer a more detailed discussion to Appendix \ref{appendix:sgx}. 
%A {\em Trusted Execution Environment (TEE)} is an area of the processor that promises to provide security guarantees like data privacy and code integrity to a program that is loaded inside of it, even in the presence of a malicious operating system and hypervisor. With the Skylake series of processors, Intel introduced a new set of instructions, called {\em Intel Software Guard Extensions} (SGX), which provides a way to realize TEE on Intel chipsets. This is enforced through hardware access control mechanisms for the pages belonging to an application loaded with SGX. Intel SGX aims to provide the guarantee that code and data of the secure application can neither be read nor modified by even privileged software on the machine. Naturally, providing these guarantees are very hard and reducing the assumption needed of the TEE is of paramount importance. 

%Applications typically contain two parts: an {\em enclave} which contains all sensitive code and data, and the {\em standard} part, which handles all system calls related to input/output and sockets. A chunk of memory, called {\em Processor Reserved Memory} (around 128 MB) is set aside for holding the enclave data pages. There are two types of function calls in SGX - $\mathsf{ecall}$ and $\mathsf{ocall}$ with which the program control flow enters/exits an enclave - e.g., to make system calls. 

%Enclave Attestation is a protocol by which an enclave proves its identity to another application or enclave. There are two types of enclave attestations: a) Local, when both enclaves reside on the same machine; and b) Remote, when the enclaves reside on different machines. In the latter case, the enclave proving its identity (say A) first proves its identity locally to a special enclave known as the Quoting Enclave (QE) which has keys provisioned by Intel. Upon successful verification, QE queries Intel's Attestation Service (IAS) for an attestation certificate (sometimes referred to as a report). This report can then be used by the second enclave (B) to verify the identity of A using IAS's public key. Once attestation is completed successfully, A can set up a private and authenticated channel with B using standard mechanisms.

%\mayank{Changes made here.}
\noindent\textbf{Eastablishing Keys between SGX Enclaves.} Protocol setup involves the following steps:
\begin{itemize}
	\item To begin with, every enclave in the system already
          reliably possesses a public verification key
          $\vk_{\textrm{\tiny{irs}}}$ of the Intel Attestation Server
          (IAS), corresponding to a private key known as Intel's
          Report Signing Key~\cite{intel_sign}. Along with this, each
          enclave also has the MPC code that computes a function $f$
          (\aseem{Should we just call it pi (if it is same as pi in
            the formalization above)?}), which is the function
          that all the enclaves want to compute over their private
          inputs.
	\item Now, each enclave prepares a report which includes its
          identity $i$, a hash measurement (MRENCLAVE) (\aseem{What is
            MRENCLAVE? Can we briefly say?}) of the MPC
          code, a public key for encryption ($\pk_i$), and a
          verification key ($\vk_i$) for a signature scheme in it --
          these keys will be used for authenticated and encrypted
          communication with other enclaves (\aseem{Where does
            encryption come in? So far we have told the reader that we
          don't need encryption. Should we write a short note here
          that the encryption is to hide from a network attacker (not
          MPC attacker)? Or actually how about just let the host deal
          with the network attacker? I.e. have a TLS connection
          between parties that terminates at the host, if I understand
          correctly this is basically assumed by all MPC papers.}). The enclave sends this
          report to the TruCE server \cite{ibmsgx} which relays them
          to the Intel Attestation Server (IAS). The IAS replies back
          with an Attestation Verification Report that has been signed
          using the Report Signing Key. The TruCE server saves this
          report in a key value store that has been indexed by a
          unique ID (specific to each enclave), called TruCE ID.
	\item Every pair of enclaves share their TruCE IDs with each other. Using the TruCE ID of other enclaves, every enclave now queries the key value store on the TruCE server for the signed Attestation Verification Report of all other enclaves.
	\item Each enclave $i$ now verifies this report using
          $\vk_{\textrm{\tiny{irs}}}$ and checks that the MRENCLAVE
          value in the report is indeed correct by comparing it with
          the expected MRENCLAVE value which can be calculated locally
          given the MPC code (this guarantees that each enclave is
          running the unaltered and correct MPC code which computes
          the intended function $f$), and finally extracts the public
          key for encryption and authentication $(\pk_j, \vk_j)$, of
          other participating enclaves, from it. An important note to
          be made here is that this verification must be done inside
          the enclave (and not by the host party of the enclave) in
          order to prevent a man-in-the-middle attack. 
	\item At the end of this phase, every enclave will have a
          secure and authenticated communication channel with every
          other enclave. For efficiency reasons, we will use this
          secure and authenticated communication channel to set up a
          shared symmetric key between every pair of parties for
          secure authenticated communication. We do this using
          standard techniques -- for each $(i,j)$ pairs of enclaves,
          one of the enclaves (say $i$) samples a random MAC key
          $k_{ij}$ (AES GCM key in our case) and sends it to the other
          enclave via the newly established secure channel. This is
          done by encrypting $k_{ij}$ using $\pk_j$ (this is an RSA
          encryption key in our case) and then signing the ciphertext
          using $\sksign_i$ (which is an ECDSA signature key in our
          case). Enclave $j$ will verify the signature on the
          ciphertext received using $\vk_i$ and then decrypt the
          ciphertext using $\sk_j$ to get $k_{ij}$. This key will be
          used to authenticate any further communication between this
          pair of enclaves. \nc{Technically, shouldnt we be encrypting
            messages also between $i$ and $j$ so that an adversarial
            party $w$ cannot see the message? If so, why do we
            exchange only a MAC key?} \aseem{Please see my comment
            above.}
	\item During the main MPC protocol execution, every message
          sent by $i$ to $j$ is MACed using the key $k_{ij}$. $j$,
          upon receiving a message, will first check its MAC using
          $k_{ij}$. It proceeds with the execution of the MPC only if
          the MAC verifies. If it doesn't, the receiver
          aborts. \nc{So, we get security with abort? We dont discuss
            these issues - I dont know how much to go into it. Also,
            what about the attested copy of the code? We need to
            exchange that also in the beginning of the protocol to
            make sure everyone is running the same code, right?} 
\end{itemize}

%\noindent\textbf{SGX Verification Keys.} In order for participants to know the identities of various $\fattest^{g,(\vk_i,\sksign_i)}(P_i)$ functionalities in the protocol, the corresponding verification keys must be signed by a key registration authority (or reliably known to all other participants in the protocol). We implement this through the remote enclave attestation protocol to obtain verified verification keys $\vk_i$ as well as an attested copy of the hash of $\pi^*(\cdot)$ to ensure that all functionalities run the same semi-honest protocol. %The modified next message functionality $\pi(\cdot)$ will additionally also check that every $\vk_i$ has been properly attested when it is invoked with $\ctr = 0$ (and hence this message also contains these certificates). %For this purpose, every $\fattest^{g,(\vk_i,\sksign_i)}(P_i)$ contacts the Intel server once in order to verify the certificate provided.

%\noindent\textbf{Implementing Remote Attestation.} For the purpose of remote attestation, we used IBM's SGX Trust Management Framework~\cite{ibmsgx}. IBM's framework however supports only a setting where a non-SGX client  can verify that it is communicating with a genuine Intel SGX application on a cloud server and then set up a private communication channel with the enclave running on the cloud server. We modified this framework to support a setting where multiple SGX enclaves can verify the identity of other SGX enclaves and then set up a private MAC key with them, which can then be used for authenticating any messages. One important change that had to be done was to now run the report verification and public key extraction inside the enclave. If enclave A has to make sure that it is talking to enclave B, then, it first receives the attestation report, verifies it, makes sure that the hash of public keys in the report (signed by Intel's key) is consistent with the public keys sent along with it. It is crucial that the report is verified inside the enclave because the party whose report is being verified has to be sure that it is talking to the enclave and not an untrusted application -- otherwise, this would lead to a standard man-in-the-middle attack. %In case $P_0$ wants to make sure that it is talking to enclave of $P_1$. Suppose the report verification on $P_1$ was done by the application, then the application of $P_1$ can launch a man-in-the-middle attack with the enclave of $P_1$ and forge messages. 

\noindent\textbf{Assigning Protected Memory.} We observed that assigning more protected memory to the enclave than what it actually requires affects the performance of enclave code. Intel's SGX SDK provides a tool called \textsf{sgx\_emmt} (Enclave Memory Management Tool)~\cite{intelsgxdevref} which can be used to measure the peak protected memory required by an enclave at runtime. We use this tool to make sure that we tune the max heap and stack size in enclave configuration file to match what is reported by \textsf{sgx\_emmt}. This helps us to optimize the runtime of enclave code by making sure that the memory reserved for the enclave is no more than what it requires.

\noindent\textbf{Porting interactive protocols to SGX.} We are the first work to port heavily interactive protocols to SGX enclaves and this comes with unique challenges. For example, whenever data is passed across the enclave's protected memory region, it has to be {\em marshalled} in/out of the region\footnote{When pointers to memory are passed as parameters into the enclave via an $\mathsf{ecall}$, the referenced data block is {\em marshalled} into the enclave, specifically into the protected memory region that an enclave uses. Similarly, when a pointer to enclave data, residing in protected memory region, is passed outside an enclave via an $\mathsf{ocall}$, the referenced data block is marshalled out of the protected memory region.}. The performance of  marshalling mainly depends on the size of the parameters crossing the bridge. Larger parameters imply slower marshalling~\cite{intelsgxperf}, while smaller parameters would increase the total numbers of cross-bridge calls (which have an overhead of their own). When we port highly interactive protocols to SGX, calls in and out of the enclave become a major factor that affects the performance of the application. This calls for the careful tuning of parameter sizes in cross-bridge calls. %We also had to add some common functionalities to the SGX SDK - e.g., routines to get input from user at runtime, sockets, routines to read cleartext data from files, routines to print output coming from enclave, spawn and join threads, and so on. \nc{Last line a bit informal.}
%\paragraph{New functionalities added to SGX SDK} Since we run highly interactive protocols completely inside enclave, there are some common functionalities that had to be implemented to make sure the porting works. Some of them include -- routines to get input from user at runtime, sockets, routines to read cleartext data from files, routines to print output coming from enclave, spawn and join threads, and so on. For all of these functionalities, we make an \textsf{ocall} to run the functionality and then come back to the enclave execution with appropriate data.
\end{comment}

\section{Experiments}\label{sec:experiments}

\noindent\textbf{Overview.} In this section, we present our experimental results. First, in Section \ref{subsec:bigbenchmarks}, we use \tool to securely compute inference on the ImageNet dataset using the following \tensorflow programs: \resnet\footnote{\url{https://github.com/tensorflow/models/tree/master/official/r1/resnet}} and \densenet\footnote{\url{https://github.com/pudae/tensorflow-densenet}}. 
%The computation is performed as a three-party secure computation protocol, where the model and query is secret shared amongst the three parties.
%We stress that no prior work has run MPC on networks of this scale. 
We also show that the performance of semi-honest and malicious protocols generated by \tool scale linearly with the depth of DNNs.
 Second, in Section \ref{sec:prior-comparison}, we show that \cryptflow\ outperforms prior works on secure inference of DNNs. 
%These experiments are on smaller DNNs that perform prediction over the MNIST and CIFAR datasets as prior work could only handle such small benchmarks.
 Next, we evaluate each component of \tool in more detail. 
In Section \ref{sec:athosexperiments}, we show that the fixed-point code generated by Athos matches the accuracy of floating-point  \resnet and \densenet.  We show in Section \ref{subsec:porthosexperiments} how the optimizations in Porthos help it outperform prior works in terms of communication complexity and overall execution time. In Section \ref{subsec:aramisexperiments}, we show the overhead of obtaining malicious secure MPC (over semi-honest security) using Aramis for GMW~\cite{gmw} and Porthos.  
We show Aramis-based malicious secure inference outperforms pure crypto-based  malicious secure protocols by huge margins in Section~\ref{sec:concurrent-comparison}.
Finally, in section \ref{subsec:realworldimpact}, we discuss two case-studies of running \cryptflow\ on DNNs for healthcare.
 We begin by providing details of the systems used to run our experiments.
\\\\
\noindent\textbf{System Details.} All our large benchmark experiments are in a LAN setting on 3.7GHz machines, each with 4 cores and with 16 GB of RAM running Linux Ubuntu 16.04. The measured bandwidth between each of the machines was at most 377 MBps and the latency was sub-millisecond. 
Since we wanted to use the same machines to benchmark both our semi-honest as well as our malicious secure protocols, we were constrained to use machines that had Intel SGX enabled on them - this led to machines that had considerably lower bandwidth between them (377 MBps) than those normally used by prior works in the area (e.g. \cite{aby3, quantizednn} used networks with bandwidth of 1.5 GBps). For Aramis, we used Intel SGX SDK version 2.4.
The compilation time of \tool is around 5 sec for \resnet, 35 sec for \densenet and 2 minutes for {{\textsc{ResNet200}}\xspace}.
% remains under 40 seconds for our benchmarks.
% \nc{A line here about the code being available...}


%Through this evaluation, we want to demonstrate the following:
%\begin{itemize}
%\item Section 6.1 shows that we can run state of the art neural networks securely using cryptflow.
%We show imagenet scale predictions using some of the most famous DNNs (ResNet50, VGG, MobileNets).
%No prior published work has run MPC on networks of this scale.
%\item Section 6.2 shows that cryptflow is far superior to other alternatives performance wise. 
%These experiments are on smaller MNIST/CIFAR DNNs as prior work has handled only those.
%Maybe we also show that cryptflow is much easier to use than other frameworks.
%\item Section 6.3 shows that Athos' scale exploration is critical for obtaining a good fixed-point network.
%We also show a sanity check that fixed-point runs much faster than floating-point in MPC.
%\item Section 6.4 uses microbenchmarks to evaluate Porthos against state-of-the-art 2PC and 3PC crypto protocols.
%\item Section 6.5 shows overhead of malicious security using SGX. 
%\end{itemize}
\subsection{Secure Inference on ImageNet}\label{subsec:bigbenchmarks}
%In this section, we show the power of \cryptflow, by demonstrating secure inference of \resnet and \densenet over the ImageNet dataset with over 1000 classes. 
We briefly describe our  benchmarks and then present performance results.
\begin{enumerate}
\item \resnet~\cite{resnet} is a network that follows the residual neural network architecture. 
The residual nodes employ ``skip connections'' or short cuts between layers.
% in order to avoid the problem of vanishing gradients while training.
 It consists of 53 convolution layers with filters of size up to $7\times 7$, and 1 fully connected layer of size $2048\times 1001$. The activation function between most layers is batch normalization (Appendix~\ref{appendix:batchnorm}) followed by ReLU. After the first convolutional layer, the activation function also includes a MaxPool.

\item \densenet~\cite{densenet} is a form of residual neural network that employs several parallel skips. It consists of 121 convolutional layers with filters of size up to $7\times 7$. The activation function between these layers is usually batch normalization, followed by ReLU. Some layers also use MaxPool or AvgPool. 

%\item \squeezenet~\cite{squeezenet} is a notoriously hard to train network and no prior work has considered evaluating this architecture on ImageNet with %fixed-point arithmetic.
%It consists of 26 convolutional layers with filters of size up to $3\times 3$. The activation function between these layers is usually ReLU and MaxPool. 
\end{enumerate}~\\
\noindent\textbf{Performance.} Table \ref{tab:bigbenchmarks} shows performance of \cryptflow\ on these benchmarks.
% within a range of 25--36 seconds with semi-honest security and 75--112 seconds with malicious security.
 We measure communication as total communication between all $3$ parties - each party roughly communicates a third of this value. The communication in semi-honest secure and malicious secure inference is almost the same. Thus demonstrating that ImageNet scale inference can be performed in about 30 seconds with semi-honest security and in under two minutes with malicious security. The malicious protocol of \tool is about 3x slower than the semi-honest version.



%\nc{We probably need to say something about QuantizedNN and differences}
\begin{table}
  \centering
%  \resizebox{0.7\columnwidth}{!}{

      \begin{tabular}{|c|c|c|c|c|}
    \hline
    Benchmark & Semi-Honest (s) &  Malicious (s) & Comm. (GB)  \\
    \hline
    $\resnet$ & $25.9$ & 75.4 &$6.9$\\ \hline
    $\densenet$ & $36.0$ & 112.9 &$10.5$\\ 	\hline
%    $\squeezenet$ & $11.3$ & 29.0 &$2.6$\\ 	\hline

\end{tabular}
%}
 \caption{\cryptflow: ImageNet scale benchmarks.}
\label{tab:bigbenchmarks}
%\tableup
	%\vspace{-0.5cm}
\end{table}~\\
\noindent\textbf{Scalability.} We show that the running time of \tool-based protocols increases linearly with the depth of DNNs. We compile \textsc{ResNet-$n$} (where $n$, the approximate number of convolutional layers, varies from 18 to 200) with \tool and evaluate with both semi-honest (Porthos) and malicious secure protocols (Aramis) in Figure \ref{fig:scalingResnet}. Our largest benchmark here is \textsc{ResNet-$200$}, the deepest version of \textsc{ResNet} on the ImageNet dataset \cite{he2016identity}, which has 65 million parameters. Other \textsc{ResNet}-$n$ benchmarks have between 11 to 60 million parameters \footnote{Specifically, 11, 22, 25, 44 and 60 million parameters for \textsc{ResNet}-$n$ for $n=$ 18, 34, 50, 101, and 152 respectively.}. We observe that the communication and runtime increase linearly with depth. Even with increasing depth, the overhead of malicious security (over semi-honest security) remains constant at about 3X. 
\begin{figure}
  \includegraphics[width=\linewidth]{scale-plot-resnet.pdf}
	\caption{Scalability of {\cryptflow} on {\sc ResNet}-$n$.}
  \label{fig:scalingResnet}
\end{figure}

\subsection{Comparison with prior work}
\label{sec:prior-comparison}
In this section, we show that \tool outperforms prior works on secure inference of DNNs
 on the benchmarks they consider, i.e., tiny 2--4 layer DNNs over the MNIST and CIFAR-10 datasets. We stress that these benchmarks are very small compared to the ImageNet scale DNNs discussed above. % and hence may not be an accurate reflection of the power of \cryptflow. 
In order to provide a fair comparison, for these experiments, we use a network with similar bandwidth as prior works (1.5 GBps) and machines with similar compute (2.7 GHz). 
%For protocols whose code is publicly available (e.g. SecureNN~\cite{securenncode}), we ran the protocol on our benchmark machines and the times reported are those; for others, the times are from the respective papers that use similar machines. 

Table~\ref{tab:porthosvspriormnist} shows that Porthos outperforms prior (ABY$^3$ \cite{aby3}, Chameleon\footnote{Chameleon is a 2PC protocol in the online phase but requires a trusted third party in the offline phase. We report overall time here.}~\cite{chameleon}, and SecureNN \cite{securenn}) and concurrent (QuantizedNN~\cite{quantizednn}) semi-honest secure 3PC works on the MNIST dataset. 
It is well-known that 3PC-based techniques like Porthos are much faster than techniques based on 2PC and FHE. We relegate comparison between Porthos and 2PC/FHE
works to Appendix~\ref{unfaircomp}. We omit comparisons with~\cite{trident,astraccsw} as their published MSB protocol was incorrect~\cite{astraeprint}.
%
%Table \ref{tab:porthosvspriorcifar10} compares Porthos with prior 2PC work.  We omit some prior works (e.g., \cite{secureml,hycc,chameleon}, etc.) in these tables as they are slower than %Gazelle and do not provide additional insights. 
%We note that other than ABY$^3$ \cite{aby3}, QuantizedNN \cite{quantizednn}, and SecureNN \cite{securenn}, other works are for 2-party secure computation. 
%As can be seen from the tables, the 2PC systems are much slower than the 3PC systems and Porthos performs better than other 3PC systems. For instance, for a 4-layer CNN for MNIST %from MiniONN~\cite{minionn}, the best 2PC-backend Gazelle takes 810 ms, prior best 3PC work takes 47 ms, and Porthos takes 34 ms. 
%Further, some of these benchmarks employ much higher bandwidth than our system  -- e.g. ABY$^3$~\cite{aby3} and QuantizedNN~\cite{quantizednn} make use of a network with %1.5GBps bandwidth which is {\em more than 3 times faster than our network}; however the code of these works are not public and hence we could not reproduce their results on our %benchmark machines. Even then, our performance numbers either beat prior times or are only marginally slower that accounting for bandwidth difference would make our protocol faster. 

\begin{table}
  \centering
%  \resizebox{\columnwidth}{!}{

      \begin{tabular}{|c|c|c|c|c|c|}
    \hline
    Benchmark & \cite{aby3}  & \cite{quantizednn} & \cite{securenn} & \cite{chameleon} & Porthos \\
    \hline
	Logistic Regression & $4$ & $-$ & $-$ & -  & $2.7$\\
    \hline
	SecureML (3 layer) & $8$ & $20$ & $17$ & - & $8$\\
	\hline
	MiniONN (4 layer)  & -  & $80$ & $47$ & 2240 & $34$\\
	\hline
   LeNet (4 layer) & - & $120$ & $79$ & - & $58$ \\
	\hline
\end{tabular}
%}
 \caption{Comparison with 3PC on MNIST dataset with ABY$^3$ \cite{aby3}, QuantizedNN~\cite{quantizednn}, SecureNN \cite{securenn}, and Chameleon~\cite{chameleon}. All times in milliseconds.}
\label{tab:porthosvspriormnist}
%\tableup
	%\vspace{-0.5cm}
\end{table}



%To put Aramis in comparison with concurrent actively secure inference works, we compare Aramis with QuantizedNN \cite{quantizednn} in table \ref{tab:aramisvsspdz} on the 4 networks described in SecureNN \cite{securenn}. We used MP-SPDZ framework \cite{mpspdz} to run QuantizedNN on these 4 benchmarks with malicious security in an honest majority setting. MP-SPDZ repository already has these 4 benchmarks written for QuantizedNN which we used for this comparison. However, it only implements 8-bit quantization with networks stripped out of all ReLU layers, while for Aramis we ran full networks with 64-bit quantization. In doing so we are being unfair to Aramis and the execution time of MP-SPDZ with ReLUs would be much worse
%than what is shown here.

\begin{comment}
\begin{table}
  \centering
  \resizebox{0.7\columnwidth}{!}{

      \begin{tabular}{|l|c|c|c|c|}
    \hline
    Benchmark & Aramis Gains over Mal. QuantizedNN\\
    \hline
	SecureNN N/W A & $9$x \\
	\hline
	SecureNN N/W B & $32.7$x \\
	\hline
	SecureNN N/W C & $34.1$x \\
	\hline
	SecureNN N/W D & $7.9$x \\
	\hline

\end{tabular}
}
 \caption{MNIST dataset -- Aramis vs Malicious QuantizedNN}
\label{tab:aramisvsspdz}
%\tableup
	%\vspace{-0.5cm}
\end{table}
\end{comment}




%these benchmarks are at a regime where computation is not an issue (as the matrix multiplications performed by parties are small) and are only bandwidth constrained. This is in contrast to larger benchmarks where both compute and communication are overheads. 
\subsection{Athos experiments}
\label{sec:athosexperiments}
\noindent\textbf{Accuracy of Float-to-Fixed.} 
We show that Athos generated fixed-point code matches the accuracy of floating-code on \resnet and \densenet in Table~\ref{tab:fixed-accuracy}.
The table also shows the precision or the scale that is selected by Athos (Section~\ref{subsec:athosquantizer}). We observe that different benchmarks require different precision to maximize the classification accuracy and that the technique of ``sweeping'' through various precision levels is effective.
We show how accuracy varies with precision in Appendix~\ref{appendix:accuracies}.
Evaluating accuracy also helps validate the correctness of our compilation~\cite{frigate}. 

\begin{table}
\centering
% \resizebox{0.7\columnwidth}{!}{
\begin{tabular}{|c|c|c|c|c|c|}
\hline
Benchmark & Float & Fixed & Float & Fixed & Scale \\
          & Top 1 & Top 1 & Top 5 & Top 5 & \\
\hline
$\resnet$     & 76.47 & 76.45 & 93.21 & 93.23 & 12 \\ \hline
$\densenet$   & 74.25 & 74.33 & 91.88 & 91.90 & 11 \\ \hline
%$\squeezenet$ & 55.86 & 55.92 & 79.18 & 79.24 & 10 \\ \hline
\end{tabular}
%}
\caption{Accuracy of fixed-point vs floating-point.}
\label{tab:fixed-accuracy}
%\tableup
%\vspace{-0.8cm}
\end{table}~\\
\noindent\textbf{Modularity.} 
Since \tool is modular, we can compile it to various \mpc backends. To demonstrate this ability, we also add a 2PC semi-honest secure protocol ABY~\cite{aby} to \tool. 
The performance with this backend is in Table \ref{tab:2pcnumbers}.
We ran logistic regression (LR) as well as a small LeNet network~\cite{lenet} which comprises of 2 convolutional layers (with maximum filter size of $5\times 5$) and 2 fully connected layers, with ReLU and MaxPool as the activation functions.
% We see that performance of 2PC is much worse than 3PC Porthos.. 
 This evaluation shows that \cryptflow\ can be easily used for a variety of backends. 
 % - however the current state-of-the-art performance of 2PC makes it difficult to execute the large DNNs described in Section \ref{subsec:bigbenchmarks}. 
%One could potentially implement the functions in Table~\ref{tab:smf} with
% state-of-the-art 2PC backends such as Delphi~\cite{delphi}. The code of Delphi is not publicly available and hence we could not do the same.

\begin{table}
  \centering
%  \resizebox{0.7\columnwidth}{!}{

      \begin{tabular}{|c|c|c|}
    \hline
    Benchmark & \cryptflow\ (s) & Communication (MB) \\
    \hline
	$\mathsf{Logistic Regression}$ & $0.227$ & $25.5$\\
	\hline
    $\mathsf{LeNet}$ $\mathsf{Small}$ & $47.4$ & $2939$ \\ 	\hline

\end{tabular}
%}
 \caption{\cryptflow\ compilation to 2PC on MNIST.}
\label{tab:2pcnumbers}
%\tableup
%	\vspace{-0.8cm}
\end{table}

%A few other interesting observations follow. After a certain threshold in scaling factor, accuracies drop dramatically. This is expected as beyond this level of precision, we lose information %due to the overflow of underlying values when performing computation such as matrix multiplication. Interestingly, accuracy doesn't always improve with larger bits of precision even until %this threshold - e.g., the best Top 1 accuracy on the \resnet\ network is obtained at 12 bits of precision.

\subsection{Porthos experiments}\label{subsec:porthosexperiments}
Since Porthos builds on SecureNN, we compare them in mode detail.
As described earlier, Porthos improves over the communication complexity of SecureNN~\cite{securenn} both for convolutional layers as well as for non-linear activation functions.
We have already compared SecureNN and Porthos on benchmarks considered in SecureNN in Table~\ref{tab:porthosvspriormnist}.
Additionally, we also compare Porthos and SecureNN on ImageNet scale benchmarks in Table \ref{tab:porthosvssecurenn}. For this purpose, we add the code of SecureNN available at~\cite{securenncode} as another backend to \tool. These results show that Porthos improves upon the communication of SecureNN by a factor of roughly 1.2X--1.5X and the runtime by a factor of roughly 1.4X--1.5X.
% on these benchmarks.

\begin{table}
  \centering
  \resizebox{\columnwidth}{!}{

      \begin{tabular}{|c|c|c|c|c|}
    \hline
    Benchmark & SecureNN & Porthos& SecureNN & Porthos \\
     & (s)  & (s) & Comm. (GB) & Comm. (GB) \\
	\hline
	$\resnet$ & $38.36$ & $25.87$& $8.54$& $6.87$ \\
	\hline
    $\densenet$ & $53.99$ & $36.00$& $13.53$ & $10.54$ \\ 	\hline
  %  $\squeezenet$ & $16.55$ & $11.28$& $3.88$ & $2.63$ \\ 	\hline

\end{tabular}
}
 \caption{Porthos vs SecureNN.}
\label{tab:porthosvssecurenn}
%	\vspace{-0.5cm}
%\tableup
\end{table}

\subsection{Aramis experiments}\label{subsec:aramisexperiments}
We applied Aramis to both the 2-party GMW protocol~\cite{gmw} (using the codebase~\cite{gmwcode}, based on~\cite{gmwpaper}) as well as Porthos. 
The results for different functions using the GMW protocol are presented in Table \ref{tab:gmwport}. $\mathsf{IP}_n$ denotes the inner product of two $n$-element vectors over $\bbF_2$, $\mathsf{Add}_{32}$ and $\mathsf{Mult}_{32}$ denote addition and multiplication over $32$ bits respectively, and $\mathsf{Millionaire}_{32}$ denotes the millionaires problem that compares two $32-$bit integers $x$ and $y$ and outputs a single bit denoting whether $x>y$.  The overheads of Aramis-based malicious security, are within $54\%$ of the semi-honest protocol. Table~\ref{tab:bigbenchmarks} and Figure~\ref{fig:scalingResnet} evaluate Aramis with Porthos.
%Aramis in this table denotes the semi-honest GMW protocol ported into Intel SGX to provide malicious security. Evaluation of Aramis applied to Porthos 

% As can be seen, all overheads in this case, are within $54\%$ of the semi-honest protocol. The evaluation of Aramis on

%Since these benchmarks are small, all of the code and data fit inside the SGX enclave without any requirement for paging and hence overheads are also minimal. For ImageNet scale %benchmarks, the overheads are higher (Table~\ref{tab:bigbenchmarks}).
%Since these benchmarks are much larger, we have to deal with well-known paging issues that occur when using Intel SGX with large data~\cite{intelsgxperf}; here we incur overheads of %about 3X over semi-honest protocols. 

\subsubsection{Comparison with crypto-only malicious \mpc}
\label{sec:concurrent-comparison}
We demonstrate that Aramis based malicious secure protocols are better suited for large scale inference tasks compared to pure cryptographic solutions. 
We compare the performance of Porthos compiled with Aramis and the concurrent work of QuantizedNN~\cite{quantizednn} that uses the MP-SPDZ~\cite{mpspdz} framework to also provide a malicious secure variant of their protocol. Both these approaches provide security for the same  setting of 3PC with 1 corruption. On the four MNIST inference benchmarks A/B/C/D in the MP-SPDZ repository, Aramis is 10X/46X/44X/15X faster.
% Furthermore, in these performance measurements, we are being unfair to Aramis: MP-SPDZ strips out all the ReLUs from its performance estimates and its %performance would be much worse in their presence. However, the Aramis time measurements include the time to compute ReLUs, which can be 80\% of the %total execution time. If one were to evaluate ReLUs with MP-SPDZ then the speedups of Aramis would be even higher.
% \nc{Should we clarify somewhere that earlier mentions of QuantizedNN were for semihonest security and here alone we are referring to their malicious protocol? To prevent against a reviewer thinking that all are malicious protocols?}

\begin{table}
  \centering
 % \resizebox{0.7\columnwidth}{!}{

      \begin{tabular}{|c|c|c|c|}
    \hline
    Benchmark & GMW (ms) & Aramis (ms) & Overhead \\
    \hline
	$\mathsf{IP}_{10,000}$ & $464$ & $638$ & $1.37$x\\
	\hline
    $\mathsf{IP}_{100,000}$ & $2145$ & $3318$ & $1.54$x \\ 	\hline
    $\mathsf{Add}_{32}$ & $279$ & $351$ & $1.25$x \\ 	\hline
    $\mathsf{Mult}_{32}$ & $354$ & $461$ & $1.30$x \\ 	\hline
    $\mathsf{Millionaire}_{32}$ & $292$ & $374$ & $1.28$x \\ 
	\hline

\end{tabular}
%}
 \caption{Semi-honest GMW vs Malicious Aramis.}
\label{tab:gmwport}
	%\vspace{-0.5cm}

%\tableup
\end{table}

%Table \ref{tab:porthosport} shows the overheads of Aramis over Porthos for the benchmarks in Section \ref{subsec:bigbenchmarks}. 
%Figure~\ref{fig:aramisreluplot} shows how overhead of Aramis over Porthos varies with memory usage. In this landscape, the plot also shows where the benchmarks in Section \ref{subsec:bigbenchmarks} lie.

\begin{comment}
\begin{table}
  \centering
	\resizebox{0.7\columnwidth}{!}{

      \begin{tabular}{|l|c|c|c|l|}
    \hline
    Benchmark & Porthos (s) & Aramis (s) & Overhead \\
    \hline
	$\resnet$ & $25.87$ & $75.4$ & $2.9$x\\
	\hline
    $\densenet$ & $36.00$ & $112.9$ & $3.1$x \\ 	\hline
    $\squeezenet$ & $11.28$ & $29.01$ & $2.5$x \\ 	\hline
\end{tabular}
}
 \caption{Semi-honest Porthos vs Malicious Aramis.}
\label{tab:porthosport}
	%\vspace{-0.5cm}
%\tableup
\end{table}
\end{comment}

%\mayank{Adding new SGX plots here. NOTE: the overheads of SqueezeNet, ResNet and DenseNet will change when Athos is updated. Take new readings before submission. NOT FINAL.}
%We also benchmark the performance of a single ReLU layer and plot its trend with increasing ReLU complexity interpreted as memory usage in our plot. The cryptographic protocol that %securely realizes ReLU incurs an 8x increase in the memory footprint of a ReLU operation which is of concern in Aramis given the fact that the usable EPC page size is below 100MB. Figure %\ref{fig:aramisreluplot} shows that overhead of Aramis over Porthos scales well even when we run ReLUs securely requiring upto 2GB of memory. In this memory usage landscape, the plot %also shows where the benchmarks in Section \ref{subsec:bigbenchmarks} lie.
%
%\begin{figure}
%  \includegraphics[width=\linewidth]{plot-sgx-relu.pdf}
%  \caption{Aramis overheads over Porthos vs Memory usage}
%  \label{fig:aramisreluplot}
%\end{figure}


\subsection{Real world impact}\label{subsec:realworldimpact}
%Our evaluation thus far has been based on ML models and datasets in image recognition. In this section, 
We discuss our experience with using {\cryptflow} to compile and run DNNs used in healthcare. These DNNs are available as pre-trained Keras models. We converted them into \tensorflow using~\cite{kttf} and compiled the automatically generated \tensorflow code with \tool.

\paragraph{Chest X-Ray} In \cite{chestxray2018}, the authors train a {\densenet} to predict lung diseases from chest X-ray images. They use the publicly available NIH dataset of chest X-ray images and end up achieving an average AUROC score of 0.845 across 14 possible disease labels.
% We took their publicly available pretrained keras model, converted it to tensorflow~\cite{kttf}, and compiled the \tensorflow code with {\cryptflow}.
During secure inference, we observed no loss in accuracy and the runtime is similar to the runtime of \densenet for ImageNet.
%Both the latency and accuracy of this inference is similar to the one we report for {\densenet} and . We also end up achieving a similar AUROC score.
%\nishant{Todo on me to find this AUROC score for compiled code and double check that the performance is indeed identical to \densenet}

\paragraph{Diabetic Retinopathy CNN} Diabetic Retinopathy (DR), one of the major causes of blindness, is a medical condition that leads to damage of retina due to diabetes \cite{janakirammsv2017}. In recent times, major tech companies have taken an interest in using DNNs for diagnosing DR from retinal images \cite{janakirammsv2017,googleDRPaper}. 
%We used one such DNN pretrained in keras, converted the same to \tensorflow~\cite{kttf}, and then ran {\cryptflow} on it to run it securely using Porthos as our backend. 
Predicting whether a retina image has DR or not can be done securely in about 30 seconds with \tool.
%\nishant{Add more on this: performance, accuracy metric? but we only had 100 images, however google paper has auroc score.}

\section{Related Work}
% adv
% blackbox
% query efficient score-based blackbox
% a. more info; b. not guarantee success c. #query 

\paragraph{Boundary-based Attack}
% A line of work that is most related to ours is using decision-based attacks.
Boundary Attack~\cite{brendel2017decision} is one of the first work that uses final decisions of a classifier to perform blackbox attacks. The attack process starts from the \sourceimage, which is classified as the adversarial \maliciousclass. Then it employs a reject sampling mechanism to find a \boundaryimage that still belongs to the \maliciousclass by performing random walk along the boundary. The goal is to minimize the distance between the \boundaryimage and the \targetimage. However, as the steps taken are randomly sampled, the convergence of this method is slow and the query number is large.

Several techniques have been proposed to improve the performance of Boundary Attack. \cite{brunner2019guessing,srinivasan2019black,guo2018low} propose to choose the random perturbation in each step more wisely instead of Gaussian perturbation, using Perlin noise, alpha distribution and DCT respectively. \cite{ilyas2018black,khalid2019red,liu2019geometry,chen2019hopskipjumpattack} propose a similar idea - approximating the gradient around the boundary using Monte Carlo algorithm. 
% In our work, we mainly follow the pipeline in \cite{chen2019hopskipjumpattack} because it establishes a good theoretic analysis for the model.

There are two other blackbox attacks which are not based on the boundary. \cite{cheng2018query} proposes to transform the boundary-based output into a continuous metric, so that the score-based attack techniques can be adopted. \cite{dong2019efficient} adopts evolution algorithm to achieve the decision-based attack against face recognition system.
% \Xiaojun{remove redundancy in the background part; mention other gradient estimation on BA} HopSkipJumpAttack~\cite{chen2019hopskipjumpattack} improves upon the basic random walk idea in the original Boundary Attack paper. It proposes an unbiased estimate of the gradient of the model on the decision boundary via a Monte-Carlo algorithm. The \queryimages are generated by adding the \boundaryimage with a set of noise vectors that are randomly sampled from the whole image space. Using the estimated gradient, the paper is able to perform more efficient update to get to a \boundaryimage that is closer to the \targetimage by taking a step towards the gradient direction and then performing binary search back to the decision boundary iteratively. This method reduces the query number compared with Boundary Attack, but since it is sampling from an extremely high-dimensional space(for example, ImageNet data samples lie in $224\times 224\times 3$ dimensional space), the number of queries required to get a fair estimation for updating is still large.
\vspace{-5mm}
\paragraph{Dimension Reduction in Score-based Attack}
Another line of work involves the dimension reduction techniques only for the score-based attacks, which requires access to the prediction of confidence for each class.
In~\cite{guo2018low}, the authors draw intuition from JPEG codec~\cite{wallace1992jpeg} image compression techniques and propose to use discrete cosine transform (DCT) for generating low frequency adversarial perturbations to assist score-based adversarial attack.
AutoZoom~\cite{tu2019autozoom} trains an auto-encoder offline with natural images and uses the decoder network as a dimension reduction tool. Constrained perturbations in the latent space of the auto-encoder are generated and passed through the decoder. The resulting perturbation in the image space is added to the benign one to obtain a query sample. 
\section{Conclusion}\label{sec:conclusion}
\cryptflow\ is the first end-to-end system that translates high-level
\tensorflow inference code to MPC protocols. It has 3 components - a) compiler from \tensorflow to \mpc, b) an improved semi-honest 3PC protocol for DNNs, and c) a generic technique to convert semi-honest secure protocols to malicious secure ones.
% that makes a minimal trust assumption in hardware. 
Using \cryptflow, we demonstrate the first instance of secure inference on large benchmarks such as \resnet\ and \densenet\ on the ImageNet dataset
with both semi-honest (in about thirty seconds) and malicious security (in less than two minutes).
 \cryptflow's modular design supports a variety of backends, and we hope that it can serve as a testbed for benchmarking new MPC protocols in the area. 

Going forward, we would like to plugin protocols like SPDZ~\cite{mpspdz} and Delphi~\cite{delphi} in \cryptflow.
%hope that protocol developers like~\cite{aby3,delphi} would integrate their work as backends to \cryptflow.
Our more ambitious goal is to extend \cryptflow\ to support
\tensorflow training.
%% We would like to consider compiling \tensorflow training code as
%% well.
It is a challenging problem since in the absence of the GPU support, the
overheads of MPC protocols for secure training can be prohibitive.
%% However, none of the existing  \mpc protocols~\cite{secureml,securenn,aby3} for training have GPU support and thus the overheads of secure training are huge.

\section{Acknowledgements}\label{sec:ack}
We thank our shepherd Xiao Wang, and anonymous reviewers for their valuable feedback.
We also thank  Sridhar Gopinath, Aayan Kumar,  Wonyeol Lee,
Sundararajan Renganathan, and Kapil Vaswani for helpful discussions.

%
% The next two lines define the bibliography style to be used, and the bibliography file.
\bibliographystyle{IEEEtranS}
\bibliography{main}

%\newpage
% 
% If your work has an appendix, this is the place to put it.
\appendix
\onecolumn
\section{Appendix}

\subsection{Proof in {\assnname}}
\label{app:assn}

\implictrule*
\begin{proof} 
	For the first rule, note that  $(s_{e0}s_{e1})\rho = s_{e0}(s_{e1}\rho ) = s_{e0}\rho  = \rho$. 
	Also, $\forall \svar \in \sigma$, $s_{e0}s_{e1} \svar = s_{e0}\svar s_{e1} = \svar s_{e0}s_{e1}$. Thus, $(\rho, \sigma) \models s_{e0}s_{e1}$. $(\rho, \sigma) \models \lambda_0 s_{e0} + \lambda_1 s_{e1} $ can be proved similarly. \\
	For the second rule, note that $s_{e1}\rho = s_{e1}(s_{e0}\rho ) =
	(s_{e1}s_{e0})\rho = \rho$, and $\forall \svar \in \sigma$, $(\svar s_{e_1})s_{e_0} = s_{e_1}s_{e_0}\svar = (s_{e_1}\svar) s_{e_0}$. Since $s_{e_0}$ is not singular, we have $\svar s_{e_1}= s_{e_1}\svar$, thus $(\rho, \sigma) \models s_{e1}$. \\
	For the final rule, notice that $(a s_{e0} + bs_{e1}s_{e2})\rho = a s_{e0}\rho + bs_{e1}s_{e2}\rho = a s_{e0}\rho + bs_{e1}\rho = \rho$.\\
	Finally, it is easy to see in all these rules, the stabilizer in $\sigma$ is commutable with the target stabilizer expressions. 
\end{proof}

\boolassn*
\begin{proof}
	We first prove the conjunction rule. Since $A_0\wedge A_1 \Rightarrow A_0$, $A_0\wedge A_1 \Rightarrow A_1$, then by the consequence rule, we have $\{A_0\wedge A_1\}\prog\{B_0\}$ and $\{A_0\wedge A_1\}\prog\{B_1\}$, i.e., $\{A_0\wedge A_1\}\prog\{B_0\wedge B_1\}$. For the disjunction rule, notice that if $(\rho,\sigma)\models (A_0\vee A_1)$, then either $(\rho,\sigma)\models A_0$ or $(\rho,\sigma)\models A_1$. 
	Finally,
	$\{I\}\prog\{I\}$ always holds since any state $(\rho,\sigma)$ satisfies $I$. $\{0\}\prog\{B\}$ is true because $(\rho,\sigma)\models 0 \Rightarrow \denot{P}(\rho,\sigma)\models B$.
\end{proof}

\decodecorrect*
\begin{proof}
First, any valid correction function will project the state into one quiescent state of the QEC code. It's the definition of QEC code error correction. \\
Second, note that the assertion $A \wedge A_S$  represents error-free states in the QEC code, thus any valid \textbf{correct} protocol will place a \textbf{skip} statement for correcting the error-free state. 
Assume the \textbf{correct} protocol is implemented based on the look-up table, 
since $A \wedge A_S \wedge -\svar_i = 0$,
then by the condition rule and $\{0\}\prog\{A \wedge A_S\}$ (Lemma~\ref{lem:bool-assn}), we directly get $\{A \wedge A_S\}\textbf{correct}(\svar_0, \svar_1, \cdots) \{A \wedge A_S\}$.
\end{proof}



\soundness*
\begin{proof}
\setcounter{cnt}{0}
(\showcnt) Skip. Note than the skip rule does not change the program state.\\
(\showcnt) Initialization.
By the definition of the substitution rule, $(\rho, \sigma) \models A[\ket{0}/\rho]$ is equivalent to $(\rho_0^q, \sigma) \models A$, then the state after initialization $(\rho',\sigma) = (\rho_0^q,\sigma)$ also satisfies $A$. \\
(\showcnt) Unitary. Note that $(UAU^\dagger)(U\rho U^\dagger) = U A \rho U^\dagger$, so  \\$(UAU^\dagger)(U\rho U^\dagger) = (U\rho U^\dagger) \Leftrightarrow A\rho = \rho$.
\\
(\showcnt) Assignment. For the first rule, assume $(\rho, \sigma) \models A$, then $A$ is commutable with $\svar$. Then, $A$ is also commutable with $-\svar$. Thus, $(\rho,\sigma') = (\rho,\sigma[-\svar/\svar])$ also satisfies $A$. \\
The second rule is obviously correct, but it limits the selection of $A$. \\
(\showcnt) Sequencing. Assume $(\rho, \sigma) \models A$, then $\denot{P_0}(\rho, \sigma)\models C$ by the hypothesis $\{A\}P_0\{C\}$. On the other hand $\denot{P_0;P_1}(\rho,\sigma) = \denot{P_1}(\denot{P_0}(\rho, \sigma)) \models B$ by the hypothesis $\{C\}P_1\{B\}$.
\\
(\showcnt) Condition. 
First, $\sum A_i M_i$ is a legal stabilizer expression because $M_1 = \frac{I+\svar}{2}$ and $M_0 = \frac{I-\svar}{2}$ are legal stabilizer expressions. 
Assume $(\rho, \sigma) \models A$, then $\sigma(\svar)$ is commutable with $A$, so is $M_1$ and $M_0$. Thus, $A M_1\rho M_1^\dagger = M_1 A\rho M_1^\dagger = M_1\rho M_1^\dagger$. Likewise, we have $A M_0\rho M_0^\dagger = M_0\rho M_0^\dagger$. Let $A = \sum_i A_i M_i$, then $A M_1\rho M_1^\dagger = A_1M_1(M_1\rho M_1^\dagger) + A_0M_0(M_1\rho M_1^\dagger) = A_1(M_1\rho M_1^\dagger)$ since $M_1M_1 = M_1$, $M_1M_0 = 0$. Thus, we have $A_1M_1\rho M_1^\dagger = M_1\rho M_1^\dagger$. 
Since $\svar$ is commutable with both $A_1$ and $A_0$, we have $(M_1\rho M_1^\dagger,\sigma) \models A_1$ and $(M_0\rho M_0^\dagger,\sigma[-\svar/\svar]) \models A_0$. Also, $(M_1\rho M_1^\dagger,\sigma) \models \svar$ and $(M_0\rho M_0^\dagger,\sigma[-\svar/\svar]) \models -\svar$. Thus, if $(\rho,\sigma)\models \sum_i A_iM_i$, we have $(M_1\rho M_1^\dagger,\sigma) \models A_1\wedge \svar$ and $(M_0\rho M_0^\dagger,\sigma) \models A_0\wedge -\svar$.
Since $\{A_1 \wedge \svar\}P_1\{B\}$ and $\{A_0 \wedge -\svar\}P_0\{B\}$, by the semantics of the condition statement, we have $\{\sum A_i M_i\}\textbf{if}\,M[\svar, \bar{q}]\,\textbf{then}\, P_0\,\textbf{else}\, P_1\,\textbf{end}\{B\}$. \\
(\showcnt) While. The proof of the While rule is quite similar to that of the Condition rule. $\sum A_iM_i$ is called the invariant of the loop. If the execution enters the loop body, then by $\{ A_1 \wedge \svar \}P_0\{\sum A_i M_i\}$, we still have $(\rho, \sigma) \models \sum A_i M_i$ for the next loop. So, when the while loop terminates, we always have $(\rho, \sigma) \models  A_0 \wedge -\svar$. \\
To prove the While rule more formally, we only need to show the partial correctness holds for $\textbf{while}^{(k)}$, as $\textbf{while}$ is the disjunction of $\textbf{while}^{(k)}$, $k=0,1,2,\cdots$.
\\
(\showcnt) Consequence. Assume $(\rho,\sigma) \models A$, then $ (\rho,\sigma) \models A'$ by $\{A\Rightarrow A'\}$. Since $\{A'\}\prog\{B'\}$, we have $\denot{P}(\rho,\sigma) \models B'$. Then $\denot{P}(\rho,\sigma) \models B$ by $B'\Rightarrow B$. Thus, $\{A\}\prog\{B\}$.
\end{proof}

\subsection{Verification of Quantum Repetition Code}
\label{app:rep}
\repcnot*
\begin{proof}
First, for control qubit $a$ and target qubit $b$, $\text{CNOT}_{ab} = \frac{1}{2}(I + X_b + Z_a - Z_aX_b)$. Then\\
(1) $\{Z_{L0} I_{L1}\}\prog\{Z_{L0}I_{L1}\}$. Note that both $\text{CNOT}_{03}$, $\text{CNOT}_{14}$ and $\text{CNOT}_{25}$ are commutable with $Z_{L0}$, so $\text{CNOT}_{03}Z_{L0}\text{CNOT}_{03} = Z_{L0}\text{CNOT}_{03}\text{CNOT}_{03} = Z_{L0}$, , $\text{CNOT}_{14}Z_{L0}\text{CNOT}_{14}= Z_{L0}$ and $\text{CNOT}_{25}Z_{L0}\text{CNOT}_{25}= Z_{L0}$. \\
(2) $\{X_{L0} I_{L1}\}\prog\{X_{L0}X_{L1}\}$. Note that $\text{CNOT }_{03}X_{L0}\text{CNOT}_{03} = X_{L0}X_3$. Since $X_3$ is commutable with $\text{CNOT}_{14}$, $\text{CNOT}_{14}X_{L0} X_3 \text{CNOT}_{14} = (\text{CNOT }_{14}X_{L0}\text{CNOT}_{14})X_3 = X_{L0}X_4X_3$. Finally, $\text{CNOT}_{25}X_{L0}X_4X_3\text{CNOT}_{25} = X_{L0}X_5X_4X_3 = X_{L0}X_{L1}$. \\
(3) $\{I_{L0} X_{L1}\}\prog\{I_{L0}X_{L1}\}$. Note that both $\text{CNOT}_{03}$, $\text{CNOT}_{14}$ and $\text{CNOT}_{25}$ are commutable with $X_{L1}$. \\
(4) $\{I_{L0} Z_{L1}\}\prog\{Z_{L0}Z_{L1}\}$. Note that  $\text{CNOT}_{03}Z_{L1}\text{CNOT}_{03} = Z_{0}Z_{L1}$, $\text{CNOT}_{14}Z_{0}Z_{L1}\text{CNOT}_{14} = Z_{0}\text{CNOT}_{14}Z_{L1}\text{CNOT}_{14} = Z_{0}Z_{1}Z_{L1}$, and $\text{CNOT}_{25}Z_{0}Z_{1}Z_{L1}\text{CNOT}_{25} = Z_{0}Z_{1}\text{CNOT}_{25}Z_{L1}\text{CNOT}_{25} = Z_{0}Z_{1}Z_{2}Z_{L1} = Z_{L0}Z_{L1}$. \\
Finally, We can prove that $\{Z_0Z_1\}\prog\{Z_0Z_1\}$, $\{Z_1Z_2\}\prog\{Z_1Z_2\}$, $\{Z_3Z_4\}\prog\{Z_0Z_1Z_3Z_4\}$, $\{Z_4Z_5\}\prog\{Z_1Z_2Z_4Z_5\}$ in a similar way. Combing all these facts, we can prove the desired partial correctness on the logical CNOT gate.
\end{proof}

\subsection{Verification of the Surface Code}
\label{app:surf}


\begin{program}[Initialize $\ket{0_L}$]
For the initialization operation in the  figure below, which initializes an X-cut logical qubit to $\ket{0_L}$,\\
\includegraphics[width=0.5\textwidth]{figure/initplus6.pdf}\\
\label{prog:initzero}
we have $\prog \Coloneqq
\bar{q} \coloneqq \ket{0};
\svar_0 \coloneqq X_0X_1X_2X_4;
\svar_1 \coloneqq X_4X_6X_7X_9;
\svar_2 \coloneqq X_9X_{11}X_{12}X_{14};
\svar_3 \coloneqq X_{14}X_{16}X_{17}X_{18}; \\
\svar_4 \coloneqq Z_1Z_3Z_4Z_6;
\svar_5 \coloneqq Z_2Z_4Z_5Z_7;
\svar_6 \coloneqq Z_{6}Z_{8}Z_{9}Z_{11};
\svar_7 \coloneqq Z_{11}Z_{13}Z_{14}Z_{16};
\svar_8 \coloneqq Z_{7}Z_{9}Z_{10}Z_{12};
\svar_{9} \coloneqq Z_{12}Z_{14}Z_{15}Z_{17};
\svar_{10} \coloneqq \cdots \\
\textbf{correct}(\svar_0,\svar_1,\cdots);
\svar_0 \coloneqq I; 
\svar_1 \coloneqq I;
\svar_2 \coloneqq I; 
\svar_3 \coloneqq I;
\svar_4 \coloneqq Z_1Z_3Z_6; 
\svar_5 \coloneqq Z_2Z_5Z_7; 
\svar_6 \coloneqq Z_{6}Z_{8}Z_{11}; 
\svar_7 \coloneqq Z_{11}Z_{13}Z_{16}; \\
\svar_8 \coloneqq Z_{7}Z_{10}Z_{12}; 
\svar_{9} \coloneqq Z_{12}Z_{15}Z_{17}; 
\svar_{10} \coloneqq \cdots \\
\svar_{\stabnum +1} \coloneqq Z_{4}; 
\svar_{\stabnum +2} \coloneqq Z_{9};
\svar_{\stabnum +3} \coloneqq Z_{14}; \\
\text{// set }q_4, q_9, q_{14}\text{ to }\ket{0}; \\
\qif{\svar_{\stabnum +1}, q_4 }{\textbf{skip}}{\bar{q} \coloneqq X_4X_6X_7X_9\bar{q}; \svar_{\stabnum +1} \coloneqq Z_{4}} %
\\
\textbf{if}\ M[\svar_{\stabnum +2}, q_9]\ \textbf{then}
\myquad \textbf{skip}\ 
\textbf{else}
\myquad \bar{q} \coloneqq X_9X_{11}X_{12}X_{14}\bar{q}; 
\svar_{\stabnum +2} \coloneqq Z_{9}\ 
\textbf{end} \\
\qif{\svar_{\stabnum +3}, q_{14}}{\textbf{skip}}{\bar{q} \coloneqq X_{14}X_{16}X_{17}X_{18}\bar{q}; \svar_{\stabnum +1} \coloneqq Z_{14}}; \\
\svar_1 \coloneqq X_4X_6X_7X_9;
\svar_2 \coloneqq X_9X_{11}X_{12}X_{14}$;
$\textbf{correct}(\svar_0,\svar_1,\cdots)$.
\end{program}

\begin{restatable}[Initialize $\ket{0_L}$]{proposition}{surfinitzerol}
For the program $\prog$ in Program~\ref{prog:initzero} which initializes a X-cut logical qubit to $\ket{0_L}$, \\
$\{I\}\prog\{Z_4Z_9Z_{14}\}$. Here $Z_4Z_9Z_{14}$ is the logical Z operator $Z_{L}$.
\end{restatable}
\begin{proof}
By Proposition~\ref{prop:decodecorrect}, after \textbf{correct} function, $(\rho,\sigma)\models (\svar_0 \wedge \svar_1 \wedge \svar_2 \cdots)$. The following stabilizer assignments which turn off X-stabilizers will just forward the precondition. \\
For simplicity, assume there are {\stabnum} stabilizers in the surface code array. let $\Lambda = \{0,\cdots,w-1\}$, then $(\svar_0 \wedge \svar_1 \wedge \svar_2 \cdots) = \wedge_{i\in \Lambda}\svar_i$. Since $\svar_0 \wedge \svar_1 \Rightarrow X_0X_1X_2X_6X_7X_9$ and $X_0X_1X_2X_6X_7X_9$ is commutable with $Z_4$, $\{\wedge_{i\in \Lambda}\svar_i\}\svar_{\stabnum +1} \coloneqq Z_{4}\{(\wedge_{i\in \Lambda\setminus \{0,1\}}\svar_i) \wedge X_0X_1X_2X_6X_7X_9\}$. Likewise, we know that after $\svar_{\stabnum +2} \coloneqq Z_{9}$, the precondition will become \\
$\{(\wedge_{i\in \Lambda\setminus \{0,1,2,3\}}\svar_i) \wedge X_0X_1X_2X_6X_7X_{11}X_{12}X_{16}X_{17}X_{18}\}$. \\
Note that $(\wedge_{i\in \Lambda\setminus \{0,1,2,3\}}\svar_i) \wedge X_0X_1X_2X_6X_7X_{11}X_{12}X_{16}X_{17}X_{18} \Rightarrow X_0X_1X_2X_6X_7X_{11}X_{12}X_{16}X_{17}X_{18} $. \\
Let $A = X_0X_1X_2X_6X_7X_{11}X_{12}X_{16}X_{17}X_{18}$, \\
$c = \qif{\svar_{\stabnum +1}, q_4}{\textbf{skip}}{\bar{q} \coloneqq X_4X_6X_7X_9\bar{q}; \svar_{\stabnum +1} \coloneqq Z_{4}}$.
It's easy to see that $\{A \wedge Z_4\} \textbf{skip} \{A \wedge Z_4\}$, and $\{A \wedge -Z_4\} q_4 \coloneqq Xq_4; \svar_{\stabnum +1}\coloneqq Z_4 \{A \wedge Z_4\}$. Thus, $\{A\}c\{A \wedge Z_4\}$. Then, after reset $q_{14}$ to $\ket{0}$, the precondition will become: \\
$\{A \wedge Z_4 \wedge Z_9 \wedge Z_{14}\}$.
Again, the following stabilizer assignments will just forward the precondition. By the implication rule, we have that $A \wedge Z_4 \wedge Z_9 \wedge Z_{14} \Rightarrow A \wedge Z_4Z_9Z_{14}$. Since $Z_4Z_9Z_{14}$ and all assertions in $A$ are commutable with stabilizers $\svar_0, \svar_1, \cdots$, we have $\{ A \wedge Z_4Z_9Z_{14}\} \textbf{correct}(\svar_0,\svar_1,\cdots) \{\wedge_{i\in \Lambda} \svar_i \wedge Z_4Z_9Z_{14}  \wedge X_0X_1X_2X_6X_7X_{11}X_{12}X_{16}X_{17}X_{18} \}$. Then by applying the consequence rule, we get $\{I\}\prog\{Z_4Z_9Z_{14}\}$.
\end{proof}

\begin{program}[Logical X gate] For the logical X gate $X_L$ in the Figure below:\\
\includegraphics[width=0.2\textwidth]{figure/surflogx6.pdf} \\
we have $\prog \Coloneqq q_0q_1q_2q_4 \coloneqq X_{0}X_{1}X_{2}X_{4}q_0q_1q_2q_4$.
\end{program}

\begin{proposition}[Logical X gate]
For program $\prog$ in Figure~\ref{fig:surfcode}(d), we have  $\{Z_L\}\prog\{-Z_L\}$ and $\{-Z_L\}\prog\{Z_L\}$, \\
where $Z_L = Z_4Z_9Z_{14}$.
\end{proposition}
\begin{proof}
	Notice that $(X_L)(Z_L)(X_L)^\dagger = -Z_L$.
\end{proof}

\statestabilizer*
\begin{proof}
For the first part, we can get $a = \frac{\alpha^2 - \beta^2}{\alpha^2 + \beta^2}$ and $b = \frac{2\alpha\beta}{\alpha^2 + \beta^2}$ simply by solving the equation $(a Z_L + b X_L)\ket{\psi} = \ket{\psi}$. \\
For the second part, assume $\ket{\psi_0} = \alpha_0 Z_L + \beta_0 X_L$ and $\ket{\psi_1} = \alpha_1 Z_L + \beta_1 X_L$, if there is a $aZ_L + bX_L$ s.t. $(aZ_L + bX_L)\ket{\psi_0} = \ket{\psi_0}$ and $(aZ_L + bX_L)\ket{\psi_1} = \ket{\psi_1}$. Then, we have $a = (\frac{\alpha_0^2 - \beta_0^2}{\alpha_0^2 + \beta_0^2} = (\frac{\alpha_1^2 - \beta_1^2}{\alpha_1^2 + \beta_1^2}$, which is equivalent to $1 - \frac{2}{1 + (\frac{\alpha_0}{\beta_0})^2} = 1 - \frac{2}{1 + (\frac{\alpha_1}{\beta_1})^2}$. Thus, $(\frac{\alpha_0}{\beta_0})^2 = (\frac{\alpha_1}{\beta_1})^2$. On the other hand, $b = \frac{2\alpha_0\beta_0}{\alpha_0^2 + \beta_0^2} = \frac{2\alpha_1\beta_1}{\alpha_1^2 + \beta_1^2}$, which is equivalent to $\frac{\frac{\alpha_0}{\beta_0}}{1 + (\frac{\alpha_0}{\beta_0})^2} = \frac{\frac{\alpha_1}{\beta_1}}{1 + (\frac{\alpha_1}{\beta_1})^2}$. Thus, $\frac{\alpha_0}{\beta_0} = \frac{\alpha_1}{\beta_1}$, i.e., $\ket{\psi_0} = \ket{\psi_1}$ up to a global phase.
\end{proof}
\surfvqmov*
\begin{proof}
After the first \textbf{correction} function, the precondition is transformed into: $(a Z_L + b X_L)\wedge_i \svar_i$.
The three following stabilizer assignments will forward the precondition. Then by the implication rule, $(a Z_L + b X_L)\wedge_i \svar_i \Rightarrow (aZ_L + bX_{2}X_{3}X_{4}X_{8}X_{9}X_{10})\wedge_{i \ne 1}\svar_i$. so for the next stabilizer assignment $\svar_{\stabnum +1} = Z_6$, precondition $(aZ_L + bX_{2}X_{3}X_{4}X_{8}X_{9}X_{10}) \wedge_{i \ne 1}\svar_i$ will be forwarded. Note that $(aZ_L + b X_{2} X_{3} X_{4} X_{8} X_{9}X_{10}) \wedge_{i \ne 1}\svar_i \Rightarrow aZ_L + b X_{2} X_{3} X_{4} X_{8} X_{9}X_{10}$,
let $A = aZ_L + b X_{2} X_{3} X_{4} X_{8} X_{9}X_{10}$, \\
$c = \qif{\svar_{\stabnum +1}, \bar{q}}{\textbf{skip}}{\bar{q}\coloneqq X_6X_{8}X_{9}X_{10}\bar{q}; \svar_{\stabnum +1}=Z_6}$. \\
For the if statement, $\{A \wedge \svar_{w+1}\}\textbf{skip}\{A \wedge \svar_{w+1}\}$ and $\{A \wedge -\svar_{w+1}\}\bar{q}\coloneqq X_6X_{8}X_{9}X_{10}\bar{q}; \svar_{\stabnum +1}=Z_6\{A \wedge \svar_{w+1}\}$, then $\{A\}c\{A \wedge \svar_{w+1} \}$.
By implication rule, $A \wedge \svar_{w+1} \Rightarrow aZ_LZ_6 + bX_{2} X_{3} X_{4} X_{8} X_{9}X_{10}$. The next three stabilizer assignment will forward $aZ_LZ_6 + bX_{2} X_{3} X_{4} X_{8} X_{9}X_{10}$. Then with the \textbf{correct} function and the consequence rule, we get that $\{aZ_L + bX_L\}\prog\{aZ_L' + bX'_L\}$, i.e., the logical state is not changed by the qubit moving operation.
\end{proof}

A braiding operation involves many data qubits, and at least 51 data qubits will be referenced in the problem. To simplify the program, we will use the qubit moving as primitive. 
$\textbf{qmov}(X_L, X_L')$ means to move the defect that changes the logical X operation of a X-cut qubit from $X_L$ to $X_L'$, and $\textbf{qmov}(Z_L, Z_L')$ to move the defect that changes the logical Z operation of a Z-cut qubit from $Z_L$ to $Z_L'$.

\begin{program}[Braiding]\label{prog:surf-braid} In the figure below, we braid a Z-cut qubit with a X-cut qubit: \\
\includegraphics[width=0.43\textwidth]{figure/surfbraid1.pdf}
\\
The associated program is $\prog \Coloneqq$ \\
$
\textbf{qmov}(Z_{5}Z_{9}Z_{10}Z_{15}, Z_{15}Z_{20}Z_{21}Z_{26}) \\
\textbf{qmov}(Z_{15}Z_{20}Z_{21}Z_{26}, Z_{26}Z_{31}Z_{32}Z_{37}) \\
\textbf{qmov}(Z_{26}Z_{31}Z_{32}Z_{37}, Z_{37}Z_{42}Z_{43}Z_{47}) \\
\textbf{qmov}(Z_{37}Z_{42}Z_{43}Z_{47}, Z_{38}Z_{43}Z_{44}Z_{48}) \\
\textbf{qmov}(Z_{38}Z_{43}Z_{44}Z_{48}, Z_{39}Z_{44}Z_{45}Z_{49}) \\
\textbf{qmov}(Z_{39}Z_{44}Z_{45}Z_{49}, Z_{40}Z_{45}Z_{46}Z_{50}) \\
\textbf{qmov}(Z_{40}Z_{45}Z_{46}Z_{50}, Z_{29}Z_{34}Z_{35}Z_{40}) \\
\textbf{qmov}(Z_{29}Z_{34}Z_{35}Z_{40}, Z_{18}Z_{23}Z_{24}Z_{29}) \\
\textbf{qmov}(Z_{18}Z_{23}Z_{24}Z_{29}, Z_{8}Z_{12}Z_{13}Z_{18}) \\
\textbf{qmov}(Z_{8}Z_{12}Z_{13}Z_{18}, Z_{7}Z_{11}Z_{12}Z_{17}) \\
\textbf{qmov}(Z_{7}Z_{11}Z_{12}Z_{17}, Z_{6}Z_{10}Z_{11}Z_{16}) \\
\textbf{qmov}(Z_{6}Z_{10}Z_{11}Z_{16}, Z_{5}Z_{9}Z_{10}Z_{15})
$
\end{program}

According to Fowler, the verification of the braiding operation only need to focus on four configurations of logical states on a pair of logical qubits: $X_{L1}\otimes I_{L2}$, $I_{L1}\otimes X_{L2}$, $I_{L1}\otimes Z_{L2}$ and $Z_{L1}\otimes I_{L2}$.
\begin{restatable}[Braiding]{proposition}{surfbraid} 
For the program $\prog$ in Program~\ref{prog:surf-braid}, \\
$\{X_{L1} I_{L2}\}\prog\{X_{L1}X_{L2}\}$, $\{I_{L1} Z_{L2}\}\prog\{Z_{L1} Z_{L2}\}$, $\{I_{L1}X_{L2}\}\prog\{I_{L1} X_{L2}\}$ and $\{Z_{L1} I_{L2}\}\prog\{Z_{L1} I_{L2}\}$.
\end{restatable}
\begin{proof} To simplify the proof, we let $A_S = \wedge_i \svar_i$, i.e., the assertion generated by current active stabilizers in the surface code array. Note that $A_S$ may change at different time-step. The proof of the braiding operation involves tedious computation and we only give a sketch of the proof here. \\
(1) Prove $\{X_{L1} I_{L2}\}\prog\{X_{L1} X_{L2}\}$. Since $X_{L1} I_{L2} = X_{L1}$, we only need to focus on the reasoning on $X_{L1}$ only. From the verification of the qubit moving, 
$\{X_{L1} I_{L2}\} \textbf{qmov}(Z_{5}Z_{9}Z_{10}Z_{15}, Z_{15}Z_{20}Z_{21}Z_{26}) \{X_{L1}X_{15}  I_{L2}\}$ (after \textbf{correct} function, $\{X_{L1}I_{L2} \wedge X_{15} \wedge A_S$ becomes $\{X_{L1}X_{15}I_{L2} \wedge A_S$). Then, after all these qubit moving operations, we will get \\ $\{X_{L1}I_{L2}\}\prog\{X_{L1} X_{15}X_{26}X_{37}X_{43}X_{44}X_{45}X_{29}X_{18}X_{12}X_{11}X_{10}I_{L2} \wedge A_S\}$. Apply implication rule on $A_S$, we get \\ $A_S \Rightarrow (X_{10}X_{15}X_{16}X_{21}) (X_{21}X_{26}X_{27}X_{32})(X_{32}X_{37}X_{38}X_{43})(X_{33}X_{38}X_{39}X_{44})(X_{34}X_{39}X_{40}X_{45})\\
(X_{23}X_{28}X_{29}X_{34})(X_{12}X_{17}X_{18}X_{23})(X_{11}X_{16}X_{17}X_{22}) = (X_{15}X_{26}X_{37}X_{43}X_{44}X_{45}X_{29}X_{18}X_{12}X_{11}X_{10})(X_{27}X_{33}X_{28}X_{22})
$. Then, by the consequence rule, we have $\{X_{L1}I_{L2}\}\prog\{X_{L1}X_{L2}\}$.
\\
(2) Prove $\{I_{L1} Z_{L2}\}\prog\{Z_{L1} Z_{L2}\}$.
Before the qubit moving operation involves qubits in $Z_{L2}$, the precondition $\{I_{L1}Z_{L2}\}$ will be forwarded by the qubit moving operation. So, we only need to elaborate on $\textbf{qmov}(Z_{38}Z_{43}Z_{44}Z_{48}, Z_{39}Z_{44}Z_{45}Z_{49})$. \\
Before measuring $q_{44}$ in X basis, the assignment statement about $X_{44}$ will turn the precondition $\{I_{L1} Z_{L2}\}$ into \\
$Z_{L2}(Z_{39}Z_{44}Z_{45}Z_{49})$, following the previous verification steps of qubit moving. The if statement on $X_{44}$ and $q_{44}$ will then transform the precondition into $Z_{L2}(Z_{39}Z_{44}Z_{45}Z_{49}) \wedge X_{44}$. The following assignment statement about $Z_{38}Z_{43}Z_{44}Z_{48}$ will turn the precondition $Z_{L2}(Z_{39}Z_{44}Z_{45}Z_{49}) \wedge X_{44}$ into  $Z_{L2}(Z_{39}Z_{44}Z_{45}Z_{49})$. Likewise, the remaining qubit moving operations will change the precondition $Z_{L2}(Z_{39}Z_{44}Z_{45}Z_{49})$ to $Z_{L2}(Z_{40}Z_{45}Z_{46}Z_{50})$, $\cdots$, until $Z_{L2}(Z_{5}Z_{9}Z_{10}Z_{15})$, which is just $Z_{L1}Z_{L2}$. Thus, $\{I_{L1} Z_{L2}\}\prog\{Z_{L1} Z_{L2}\}$.
\\
(3) Prove $\{I_{L1} X_{L2}\}\prog\{I_{L1} X_{L2}\}$. Recall the verification of the qubit moving operation. It is easy to see that $\{I_{L1}\}\textbf{qmov}\{I_{L1}\}$ for any qubit moving operation in $P$. On the other hand, the qubit moving operations in $P$ does not involve any qubits in $X_{L2}$, so precondition $I_{L1}X_{L2}$ will be forwarded by all qubit moving operations, i.e.,
$\{I_{L1} X_{L2}\}\prog\{I_{L1} X_{L2}\}$.
\\
(4) Prove $\{Z_{L1} I_{L2}\}\prog\{Z_{L1}I_{L2}\}$. Since $Z_{L1} I_{L2} = Z_{L1}$, we only focus on the reasoning of $Z_{L1}$ here. It is obvious that starting from $Z_5Z_9Z_{10}Z_{15}$, the logical Z operator finally returns to $Z_5Z_9Z_{10}Z_{15}$ by a series of qubit moving operations. Thus, $\{Z_{L1} I_{L2}\}\prog\{Z_{L1}I_{L2}\}$.
\end{proof}


\end{document}
