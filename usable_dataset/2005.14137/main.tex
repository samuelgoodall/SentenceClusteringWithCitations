\documentclass[10pt,twocolumn,letterpaper]{article}

\usepackage{cvpr}
\usepackage{times}
\usepackage{epsfig}
\usepackage{graphicx}
\usepackage{amsmath}
\usepackage{amssymb}

% Include other packages here, before hyperref.
\usepackage{algorithm,algorithmic}
\usepackage{amsmath}
\usepackage{bbm}
\usepackage{amssymb}
\usepackage{amsthm}
\usepackage{url}
\usepackage{makecell}

\usepackage{multirow}
\usepackage{multicol}
\usepackage{booktabs}
\usepackage{rotating}
\usepackage{threeparttable}

% subfigure
\usepackage{graphicx}
\usepackage{subcaption}

\newtheorem{theorem}{Theorem}
% \newtheorem{corollary}{Corollary}
\newtheorem{lemma}{Lemma}

% notations
\DeclareMathOperator*{\argmax}{arg\,max}
\DeclareMathOperator*{\argmin}{arg\,min}
\newcommand{\xadv}[1]{{\bf x}_{adv}^{(#1)}}

\newcommand{\sourceimage}{{source-image}\xspace}
\newcommand{\Sourceimage}{{Source-image}\xspace}
\newcommand{\targetimage}{{target-image}\xspace}
\newcommand{\Targetimage}{{Target-image}\xspace}
\newcommand{\advimage}{{adv-image}\xspace}
\newcommand{\Advimage}{{Adv-image}\xspace}

% TODO: remove the below commands.
\newcommand{\queryimage}{{\it query image}\xspace}
\newcommand{\Queryimage}{{\it Query image}\xspace}
\newcommand{\queryimages}{{\it query image}s\xspace}
\newcommand{\Queryimages}{{\it Query image}s\xspace}


\newcommand{\boundaryimage}{{\it boundary-image}\xspace}
\newcommand{\Boundaryimage}{{\it Boundary-image}\xspace}

\newcommand{\benignclass}{{\it benign-class}\xspace}
\newcommand{\Benignclass}{{\it Benign-class}\xspace}

\newcommand{\maliciousclass}{{\it malicious-class}\xspace}
\newcommand{\Maliciousclass}{{\it Malicious-class}\xspace}

\newcommand{\decisionlabel}{{\it decision-label}\xspace}
\newcommand{\Decisionlabel}{{\it Decision-label}\xspace}

\newcommand{\expertlabel}{{expert-label}\xspace}
\newcommand{\Expertlabel}{{Expert-label}\xspace}

\newcommand{\name}{{QEBA}\xspace}


% Comments Section----------------------------------------
\newcount\Comments  % 1 suppresses notes to selves in text
\Comments=1   % TODO: set to 1 for final version
\usepackage{color}
\definecolor{darkgreen}{rgb}{0,0.5,0}
\definecolor{darkblue}{rgb}{0,0,0.5}
\definecolor{purple}{rgb}{1,0,1}
\newcommand{\kibitz}[2]{\ifnum\Comments=0\textcolor{#1}{#2}\fi}
\newcommand{\Huichen}[1]{\kibitz{blue}      {[HL: #1]}}
\newcommand{\Xiaojun}[1]{\kibitz{purple}         {[XX: #1]}}
\newcommand{\Xiaolu}[1]{\kibitz{green}         {[#1]}}
\newcommand{\bo}[1]{\kibitz{darkblue}{Bo: #1}}

%----------------------------------------------------------

% \newcommand{\CameraReady}[1]{{\textcolor{red}{#1}}}

% If you comment hyperref and then uncomment it, you should delete
% egpaper.aux before re-running latex.  (Or just hit 'q' on the first latex
% run, let it finish, and you should be clear).
\usepackage[pagebackref=true,breaklinks=true,letterpaper=true,colorlinks,bookmarks=false]{hyperref}

\cvprfinalcopy % *** Uncomment this line for the final submission

\def\cvprPaperID{9582} % *** Enter the CVPR Paper ID here
\def\httilde{\mbox{\tt\raisebox{-.5ex}{\symbol{126}}}}

% Pages are numbered in submission mode, and unnumbered in camera-ready
\ifcvprfinal\pagestyle{empty}\fi
\begin{document}

%%%%%%%%% TITLE
\title{QEBA: Query-Efficient Boundary-Based Blackbox Attack}
\author{Huichen Li$^{1*}$\quad Xiaojun Xu$^{1}$
\thanks{The first two authors contribute equally. This work was done while they were interns at Ant Financial. To appear at CVPR 2020. The code is available at \url{https://github.com/AI-secure/QEBA}} % last sentences added for arxiv version
\quad Xiaolu Zhang$^{2}$\quad Shuang Yang$^{2}$\quad Bo Li$^{1}$
\\
$^{1}$University of Illinois at Urbana-Champaign\quad $^{2}$ Ant Financial
}

% \author{First Author\\
% Institution1\\
% Institution1 address\\
% {\tt\small firstauthor@i1.org}
% % For a paper whose authors are all at the same institution,
% % omit the following lines up until the closing ``}''.
% % Additional authors and addresses can be added with ``\and'',
% % just like the second author.
% % To save space, use either the email address or home page, not both
% \and
% Second Author\\
% Institution2\\
% First line of institution2 address\\
% {\tt\small secondauthor@i2.org}
% }
\maketitle
\thispagestyle{empty}

%%%%%%%%% ABSTRACT
\begin{abstract}
   Machine learning (ML), especially deep neural networks (DNNs) have been widely used in various applications, including several safety-critical ones (e.g. autonomous driving). As a result, recent research about \emph{adversarial examples} has raised great concerns. Such adversarial attacks can be achieved by adding a small magnitude of perturbation to the input to mislead model prediction. While several whitebox attacks have demonstrated their effectiveness, which assume that the attackers have full access to the machine learning models; blackbox attacks are more realistic in practice. 
%   In this paper, we propose a Query-Efficient Boundary-based Blackbox Attack (\name) to estimate model decision boundary given classification output by leveraging the intrinsic dimensionalities of inputs.
In this paper,  we propose a Query-Efficient Boundary-based blackbox Attack (\name) based only on model's final prediction labels.
%   We theoretically show why estimating model gradient is not efficient in terms of blackbox attack, and provide the optimality analysis for our dimension reduced decision boundary estimation.
We theoretically show why previous boundary-based attack with gradient estimation on the whole gradient space is not efficient in terms of query numbers, and provide optimality analysis for our dimension reduction-based gradient estimation.
   On the other hand, we conducted extensive experiments on ImageNet and CelebA datasets to evaluate \name. We show that compared with the state-of-the-art blackbox attacks, \name is able to use a smaller number of queries to achieve a lower magnitude of perturbation with 100\% attack success rate. We also show case studies of attacks on real-world APIs including MEGVII Face++ and Microsoft Azure.
%   \bo{Azure?}.\bo{do I miss any exp results here? if you have numbers want to add, it is also good}
   
\end{abstract}


%%%%%%%%% BODY TEXT
\section{Introduction}


Quantum error correcting (QEC) codes~\cite{Fowler2012SurfaceCT, Chamberland2020BuildingAF, Chamberland2020TopologicalAS} are vital for implementing fault-tolerant quantum computation and overcoming the noise present in quantum hardware~\cite{Preskill2018QuantumCI, Holmes2020NISQBQ}. Quantum device vendors are exploring various quantum error correction codes to boost the error tolerance of quantum computation. For example, Google exploits the repetition code~\cite{nielsen2002quantum} to suppress errors in their Sycamore device~\cite{google50296}, IBM extends the surface code~\cite{Fowler2012SurfaceCT} to their low-degree superconducting quantum computers~\cite{Chamberland2020}, and Amazon utilizes the concatenated cat code~\cite{Chamberland2020BuildingAF} to build a fault-tolerant qubit.



A central concept in QEC code design is that of a ``stabilizer''~\cite{Gottesman1997StabilizerCA}. The term refers to a quantum operator which expresses the correlations present among the physical qubits forming the logical qubit. Operations to encode logical states or detect and correct errors can be derived once the stabilizers of the QEC code are provided.
In a stabilizer code, these primitive operations over logical qubits consist of quantum programs, one for each primitive.
As an example, Fowler et al.~\cite{Fowler2012SurfaceCT} developed a series of programs on the surface code to implement the primitive operations (e.g., a logical X gate, H gate, and CNOT gate) necessary for universal fault-tolerant quantum computation.
While executing these primitives, any stabilizer code implementation requires frequent measurements of the physical qubits to detect possible hardware errors and, thus, apply the appropriate correction operation.




When analyzing the correctness of a stabilizer code, there are two key aspects that need to be considered:
1) \textit{the correctness of the logical operation:} The stabilizer code must implement the desired logical operation over the logical qubits by applying several physical operations over the consituent physical qubits. %
2) \textit{the capability of error correction:} 
When hardware errors happen, there exist protocols for error decoding and correction which are based on the information extracted by measurements in QEC codes. %



To the best of our knowledge, there is no formal verification framework for QEC codes yet.
Previous works on quantum error correction~\cite{Fowler2012SurfaceCT, Chamberland2020TopologicalAS, Lao2020FaulttolerantQE, Chao2019FlagFE, Noh2020FaulttolerantBQ} demonstrate the correctness of the proposed QEC protocol by numerical simulation on QEC programs. 
However, this approach does not provide formal proof for the correctness of QEC codes.
We asked ourselves the question:












\begin{center}
    \textit{Can one formally verify QEC codes}\\
    \textit{using existing verification frameworks}\\
    \textit{for general quantum programs since QEC codes} \\
	\textit{are effectively a kind of quantum program?}
\end{center}


In this vein, one well-developed method for quantum program verification~\cite{ Wu2019FullstateQC,Zhou2019AnAQ, Li2020ProjectionbasedRA} is to use dynamic techniques such as quantum simulation. 
This category of methods can accurately characterize the quantum state evolution of small quantum programs but can not scale up to large quantum programs with more than 50 qubits due to the exponential computation overhead~\cite{Wu2019FullstateQC}.
This poor scalability of dynamic methods makes it inefficient for the verification of QEC codes since  
a reasonably fault-tolerant logical qubit would inevitably involve many physical qubits~\cite{Fowler2012SurfaceCT}.

Another type of verification works~\cite{Ying2012FloydhoareLF, Ying2018ReasoningAP, Unruh2019QuantumHL, DHondt2006QuantumWP,Selinger2004TowardsAQ, Feng2020QuantumHL, Feng2021VerificationOD}, exploits static analysis techniques to reason about quantum programs. 
These works all naturally incur exponential computation overhead since they need to track the evolution of some Hermitian matrices, which are of dimension $O(4^n)$ for a $n$-qubit system. 
Yu and Palsberg~\cite{Yu2021QuantumAI} recently proposed a computationally efficient quantum abstract interpretation technique to reason about the correctness for certain kinds of assertions.
Yet, trading in accuracy is not suitable for the verification of QEC codes which requires exact correctness.















Thus, our answer to the question above is:

\begin{center}
    \textit{No, adopting general verification frameworks }\\
    \textit{sacrifices either scalability or accuracy. } 
\end{center}
%

To this end, we build a formal verification framework crafted for quantum stabilizer codes to squeeze out the best verification efficiency without compromising accuracy.
Our approach rests on a central idea: while realizing scalable verification for a general quantum program is hard, it might be possible 
to efficiently verify the QEC codes
with a delicate separation between the hard and easy parts in the verification process.    
We observe
that most parts in verifying QEC codes turn out to fall into the easy region 
because they can be efficiently processed by preserving the high-level stabilizer information. 
In particular, stabilizers provide a compact description for QEC codes~\cite{Gottesman1997StabilizerCA}. Major components in QEC codes, e.g., error channels, Clifford gates, and parity measurements can all be described within the stabilizer formalism. Besides, it only takes $O(n^2)$ complexity to emulate Clifford operations on stabilizers~\cite{nielsen2002quantum}. Using stabilizers as predicates, we can potentially avoid unnecessary exponential computation overhead  in general quantum program reasoning. 







We first propose a concise QEC programming language, {\langname}, where stabilizers are treated as first-class objects. 
This allows {\langname} to represent in an intuitive and compact form different operations in the QEC implementation, ranging from encoding to decoding and to error correction.
We develop operational semantics and denotational semantics for {\langname}, which lays the foundation for building up the syntax-directed verification system. 
One key enabler for our semantics design is the separation between quantum states of the physical qubits and the information captured by the stabilizers. 
It describes the former with partial density matrices and treats the latter as a classic program state. 
It significantly simplifies the computations associated with the stabilizers by avoiding the direct description of how the stabilizers are measured and instead focusing on how the high-level information is used in the decoding stage.

We further develop a new assertion language, named {\assnname}, in which the predicates are defined by stabilizers.
To enhance the logical expressive power, we introduce not only the standalone stabilizers, but also the arithmetic and logical expressions of them for expressing assertions.
The key insight behind such a design is to form a universal state space for verification, as a standalone stabilizer could not represent the whole state space. 
We also remark that, despite our {\langname} and {\assnname} are crafted for QEC codes, their design allows for broader applicability. Any quantum programs and quantum predicates that could be expressed in the {\qwhilelang}~\cite{Ying2012FloydhoareLF} and Hermitian-based predicates~\cite{DHondt2006QuantumWP} can also be %
expressed in our languages. With such an elegant property, our languages could potentially serve as the common foundation of both general quantum programs and QEC designs. It will therefore avoid the dilemma of choosing
between a general but less effective language or a domain-specific but more effective language. 


Together with {\langname} and {\assnname}, we further establish a sound quantum Hoare logic for QEC programs. 
This proof system can demonstrate exponential time and space saving for most QEC operations (e.g., state preparation, Pauli gates, and error detection) when, in a real QEC program, the predicate formulae is commutable with the stabilizer variable in our quantum Hoare logic.
Even for the most challenging verification of the logical T gate implementation, our proof system may still have this strong advantage, depending on the actual T gate implementation of the target QEC code.






We give both theoretical analysis and detailed case studies for evaluating the proposed framework. 
We first compare {\myFrameworkName} with the vanilla quantum Hoare logic~\cite{Ying2012FloydhoareLF} and quantum \textbf{while}-language~\cite{Ying2012FloydhoareLF} in terms of the complexity when describing and verifying the quantum stabilizer codes. 
We then give very detailed, step-by-step case studies of two well-known QEC codes. This allows the reader to familiarize the concepts behind our framework and its usage for verifying the correctness of quantum stabilizer codes.


To summarize, our major contributions are as follows:
\begin{itemize}
    \item We propose {\langname}, a concise  programming language for QEC codes and give a full specification of its syntax and operational/denotational semantics. 
    \item We formulate a new assertion language {\assnname}. It is the first effort that exploits stabilizers for building universal assertions on quantum states.
    \item We develop a sound quantum Hoare logic framework based on {\assnname} and {\langname} with a set of inference rules to verify the correctness of QEC programs. 
    \item  We demonstrate the effectiveness of our framework with both theoretical complexity analysis and in-depth case studies of two well-known stabilizer QEC codes.
\end{itemize}





% \section{Background}


\section{Problem Definition}
Consider a $k$-way image classification model $f({\bf x})$ where ${\bf x}\in\mathbb{R}^m$ denotes the input image with dimension $m$, and $f({\bf x})\in\mathbb{R}^k$ represents the vector of confidence scores of the image belonging to each classes. 
In boundary-based black-box attacks, the attacker can only inquire the model with queries $\{ {\bf x}_i \}$ (a series of updated images) and get the predicted labels % $\tilde{y}=F({\bf x}) = \argmax_j [f({\bf x})]_j$
$\tilde{y}_i=F({\bf x_i}) = \argmax_j [f({\bf x_i})]_j,$
where $[f]_j$ represents the score of the $j$-th class. The parameters in the model $f$ and the score vector $\bf s$ are not accessible.

There is a \targetimage ${\bf x}_{tgt}$ with a \emph{benign label} $y_{ben}$. Based on the \emph{malicious label} $y_{mal}$ of their choice, the adversary will start from a \sourceimage ${\bf x}_{src}$ selected from the category with label $y_{mal}$, and move ${\bf x}_{src}$ towards ${\bf x}_{tgt}$ on the pixel space while keeping $y_{mal}$ to guarantee the attack. 
An image that is on the decision boundary between the two classes (e.g. $y_{ben}$ and $y_{mal}$) and is classified as $y_{mal}$ is called \boundaryimage. 

The adversary's goal is to find an \emph{adversarial image}(\advimage) ${\bf x}_{adv}$ such that $F({\bf x}_{adv}) = y_{mal}$ and $D({\bf x}_{tgt}, {\bf x}_{adv})$ is as small as possible, where $D$ is the distance metric (usually $L_2$-norm or $L_\infty$-norm distance). By definition, \advimage is a \boundaryimage with an optimized (minimal) distance from the \targetimage. 
In the paper we focus on targeted attack and the approaches can extend to untargeted scenario naturally. 



% \paragraph{Boundary-based Attacks}


% Boundary attack (BA)\Xiaojun{cite} is an iterative algorithm for decision-based attack. In BA, the adversary initializes its \advimage with \sourceimage $\xadv{0} = {\bf x}_s$ so that $F(\xadv{0}) = y_{mal}$. At each iterative step $t$, a random small perturbation $\Delta_t \in \mathbb{R}^m$ will be sampled from a chosen distribution to modify 

% HopSkipJumpAttack~\cite{chen2019hopskipjumpattack} improves upon the basic random walk idea in the original Boundary Attack paper. It proposes an unbiased estimate of the gradient of the model on the decision boundary via a Monte-Carlo algorithm. The queries are generated by adding the \boundaryimage with a set of noise vectors that are randomly sampled from the whole image space. Using the estimated gradient, the paper is able to perform more efficient update to get to a \boundaryimage that is closer to the \targetimage by taking a step towards the gradient direction and then performing binary search back to the decision boundary iteratively. This method reduces the query number compared with BA, but since it is sampling from an extremely high-dimensional space(for example, ImageNet data samples lie in $224\times 224\times 3$ dimensional space), the number of queries required to get a fair estimation for updating is still large.

% Boundary attack (BA)\Xiaojun{cite} is an iterative algorithm for decision-based attack. In BA, the adversary initializes its \advimage with \sourceimage $\xadv{0} = {\bf x}_{src}$ so that $F(\xadv{0}) = y_{mal}$. At each iterative step $t$, a small random perturbation $\Delta_t \in \mathbb{R}^m$ is sampled such that:
% \begin{align}
%     D(\xadv{t}+\Delta_t, {\bf x}_{tgt}) < D(\xadv{t}, {\bf x}_{tgt}).
% \end{align}
% and a rejective sampling procedure is used to update the \advimage:
% \begin{align}
%     \xadv{t+1} = \begin{cases}
%     \xadv{t}+\Delta_t, & \text{if } F(\xadv{t}+\Delta_t) = y_{mal}\\
%     \xadv{t}, & \text{otherwise}
%     \end{cases}
% \end{align}

% Experiment results show that BA can generate adversarial examples comparable with that generated by white-box or score-based attack.
% \Xiaojun{may add more explanation if we have space}
% \subsection{HopSkipJumpAttack}
% The disadvantage in applying BA in practice is that it requires a large number of queries. 
% Many follow-up works have been proposed to reduce the required number of queries. We focus on one of the most effective approaches named `HopSkipJumpAttack'(HSJA)\cite{chen2019hopskipjumpattack}. In HSJA, the random sampling process is replaced with a smarter way to update the \advimage. 
% The core idea is that the adversary can use multiple queries to estimate the gradient of \advimage if it lies on the decision boundary. Thus, each update breaks down into two sub-steps: first, move the \advimage towards the estimated gradient direction so that its score of recognized as the malicious class increases; second, project the \advimage back to the decision boundary to reduce its distance with the \targetimage.

% HopSkipJumpAttack (HSJA)~\cite{chen2019hopskipjumpattack} reduces the query number in this process by estimating gradient directions at the decision boundary to assist updates. Define the attack loss and the indicator function:
% \begin{align}
%     S_{{\bf x}_{tgt}}({\bf x}) &= [f({\bf x})]_{y_{mal}} - \max_{y \neq y_{mal}} [f({\bf x})]_y\\
%     \phi_{{\bf x}_{tgt}}({\bf x}) &= \text{sign}(S_{{\bf x}_{tgt}}({\bf x})) =
%     \begin{cases}
%     1 & \text{if } S_{{\bf x}_{tgt}}({\bf x})\geq 0\\
%     -1 & \text{otherwise}
%     \end{cases}
% \end{align}
% \begin{align}
%     \phi({\bf x}) &=
%     \begin{cases}
%     1 & \text{if } F(x) \text{ equals } y_{mal} \\
%     -1 & \text{otherwise}.
%     \end{cases}
% \end{align}
% We abbreviate the two functions as $S({\bf x})$ and $\phi({\bf x})$ if it does not cause confusion. 
% % Then $\bf x$ is adversarial if and only if $S({\bf x})\geq 0$, and the adversary can know the value of $\phi$ in the decision-based black-box setting. Thus, 
% Then the gradient of $S$ w.r.t. an \advimage which lies on the decision boundary can be estimated via Monte Carlo method:
% \begin{align}
%     \widetilde{\nabla S} = \frac1B \sum_{i=1}^B \phi({\bf x}+\delta {\bf u}_b) {\bf u}_b
%     \label{eq:MC_gradient_estimation}
% \end{align}
% where $\{{\bf u}_b\}$ are $B$ random perturbations uniformly sampled from the unit sphere in $\mathbb{R}^m$ and $\delta$ is a small weighting constant.

% Each iterative step $t$ starts with an adv-image $\xadv{t}$ on the decision boundary. The update breaks down into two steps:
% \begin{enumerate}
%     \item Move the \advimage towards the gradient direction:
%     \begin{align}
%         \label{eqn:grad-est}
%         \hat{\bf x}_{t+1} = \xadv{t} + \xi_t \cdot \frac{\widetilde{\nabla S}}{||\widetilde{\nabla S}||_2}
%     \end{align}
%     \item Project the new image back to the decision boundary:
%     \begin{align}
%         \label{eqn:binary}
%         \xadv{t+1} = \alpha_t {\bf x}_{tgt} + (1-\alpha_t) \hat{\bf x}_{t+1}
%     \end{align}
%     where the projection is achieved by a binary search over $\alpha_t$.
% \end{enumerate}
% Note that the \sourceimage does not lie on the boundary, so we need to first project it onto boundary using Eqn.\ref{eqn:binary} to get $\xadv{0}$.


% In \cite{chen2019hopskipjumpattack}, the authors show that with appropriate choice of $B$, $\xi$ and $\delta$, the pipeline can achieve good attack performance both theoretically and empirically. 








% stale part for problem definition

% In traditional adversarial attacks, the adversary starts at ${\bf x}^*$ and gradually perturbs it to make it misclassified by the model. In contrast, in a decision-based attack the adversary usually starts with an image ${\bf x}_s$ where $F({\bf x}_s) = y_{mal}$, and gradually perturb it to minimize its distance with ${\bf x}^*$. Therefore, we call ${\bf x}^*$ the \emph{\targetimage} and ${\bf x}_s$ the \emph{\sourceimage}.

% Consider in the image classification scenario, the attacker only has access to the final predicted label of the system (the \decisionlabel) by inquiring it with a image of their choice (the \queryimage). 
% % The output label of the system is called \decisionlabel and the image for inquiring is denoted as \queryimage. 
% The label given by an human expert(thus used as ground truth for the true label) for the same image is called \expertlabel. 

% The attacker has two images: the \sourceimage with \expertlabel of \maliciousclass, and the \targetimage with \expertlabel of \benignclass.
% An image that is very close to the decision boundary of the classifier with the \decisionlabel of \maliciousclass, is called \boundaryimage.

% The attacker would like to find an \advimage that has an \expertlabel of \benignclass but the system would give a \decisionlabel of \maliciousclass. In other words, the \advimage fools the system.

% \subsection{Boundary Attack}
% Boundary attack (BA)\Xiaojun{cite} is an iterative algorithm for decision-based attack. In BA, the adversary initializes its \advimage with \sourceimage $\xadv{0} = {\bf x}_s$ so that $F(\xadv{0}) = y_{mal}$. At each iterative step $t$, a random small perturbation $\Delta_t \in \mathbb{R}^m$ will be sampled from a chosen distribution such that:
% \begin{align}
%     D(\xadv{t}+\Delta_t, {\bf x}^*) < D(\xadv{t}, {\bf x}^*)
% \end{align}
% And the \advimage is updated by:
% \begin{align}
%     \xadv{t+1} = \begin{cases}
%     \xadv{t}+\Delta_t, & \text{if } F(\xadv{t}+\Delta_t) = y_{mal}\\
%     \xadv{t}, & \text{otherwise}
%     \end{cases}
% \end{align}
% Experiment results show that BA can generate adversarial examples comparable with that generated by white-box or score-based attack.
% \Xiaojun{may add more explanation if we have space}

% \subsection{HopSkipJumpAttack}
% The disadvantage in applying BA in practice is that it requires a large number of queries. Many follow-up works have been proposed to reduce the required number of queries. We focus on one of the most effective approaches named `HopSkipJumpAttack'(HSJA)\cite{chen2019hopskipjumpattack}. In HSJA, the random sampling process is replaced with a smarter way to update the \advimage. The core idea is that the adversary can use multiple queries to estimate the gradient of \advimage if it lies on the decision boundary. Thus, each update breaks down into two sub-steps: first, move the \advimage towards the estimated gradient direction so that its score of recognized as the malicious class increases; second, project the \advimage back to the decision boundary to reduce its distance with the \targetimage.

% Formally speaking, we define the following functions:
% \begin{align}
%     S_{{\bf x}^*}({\bf x}) &= [f({\bf x})]_{y_{mal}} - \max_{y \neq y_{mal}} [f({\bf x})]_y\\
%     \phi_{{\bf x}^*}({\bf x}) &= \text{sign}(S_{{\bf x}^*}({\bf x})) =
%     \begin{cases}
%     1 & \text{if } S_{{\bf x}^*}({\bf x})\geq 0\\
%     -1 & \text{otherwise}
%     \end{cases}
% \end{align}
% We abbreviate the two functions as $S({\bf x})$ and $\phi({\bf x})$ if it does not cause confucsion. Then $\bf x$ is adversarial if and only if $S({\bf x})\geq 0$, and the adversary can know the value of $\phi$ in the decision-based black-box setting. Thus, given an \advimage which lies on the decision boundary (i.e. $S({\bf x})=0$), the gradient of $S$ w.r.t. the input can be estimated via Monte Carlo:
% \begin{align}
%     \widetilde{\nabla S} = \frac1B \sum_{i=1}^B \phi({\bf x}+\delta {\bf u}_b) {\bf u}_b
% \end{align}
% where $\{{\bf u}_b\}$ are random perturbations uniformly sampled from the unit sphere in $\mathbb{R}^m$.

% \Xiaojun{Fig. or Alg.} In iterative step $t$, suppose the adv-image $\xadv{t}$ is already on the decision boundary. Then the update breaks down into two steps:
% \begin{enumerate}
%     \item Move the \advimage towards the gradient direction:
%     \begin{align}
%         \label{eqn:grad-est}
%         \hat{\bf x}_{t+1} = \xadv{t} + \xi_t \cdot \frac{\widetilde{\nabla S}}{||\widetilde{\nabla S}||_2}
%     \end{align}
%     \item Project the new image back to the decision boundary:
%     \begin{align}
%         \label{eqn:binary}
%         \xadv{t+1} = \alpha_t x^* + (1-\alpha_t) \hat{\bf x}_{t+1}
%     \end{align}
%     where the projection is achieved by a binary search over $\alpha_t$.
% \end{enumerate}
% Note that the \sourceimage does not lie on the boundary, so we need to first project it onto boundary using Eqn.\ref{eqn:binary} to get $\xadv{0}$.
% In \cite{chen2019hopskipjumpattack}, the authors show that with appropriate choice of $B$, $\xi$ and $\delta$, the pipeline can achieve good attack performance both theoretically and empirically. 




\section{Query-Efficient Boundary-based blackbox Attack (\name)}
\label{sec:subspaces}
In this section we first introduce the pipeline of \name which is based on HopSkipJumpAttack (HSJA)~\cite{chen2019hopskipjumpattack}. We then illustrate the three proposed query reduction approaches in detail. We provide the theoretic justification of \name in Section~\ref{sec:dimred-theory}.
The pipeline of the proposed Query-Efficient Boundary-based blackbox Attack (QEBA) is shown in Figure \ref{fig:pipeline} as an illustrative example. The goal is to produce an \advimage that looks like ${\bf x}_{tgt}$ (cat) but is mislabeled as the malicious label (fish) by the victim model.
% The \sourceimage is the starting point of the attack instance.
First, the attack initializes the \advimage with ${\bf x}_{src}$.
% Then it performs an iterative algorithm which takes the \targetimage and a chosen representative subspace as input.
Then it performs an iterative algorithm consisting of three steps: \textbf{estimate gradient at decision boundary} which is based on the proposed representative subspace, \textbf{move along estimated gradient}, and \textbf{project to decision boundary} which aims to move towards ${\bf x}_{tgt}$.
% There are two inputs to the attack algorithm: the \targetimage and the optimized representative subspace. The \targetimage is fixed and used during the attack. The optimized representative subspace is generated before the attack and also fixed. The attack itself is an iterative algorithm consisting of three steps: estimating gradient at decision boundary, moving towards estimated gradient, and projecting to decision boundary.

% Before we introduce the three steps in our pipeline in detail, 
% we first 
First, define the adversarial prediction score $S$ and the indicator function $\phi$ as:
\begin{align}
    S_{{\bf x}_{tgt}}({\bf x}) &= [f({\bf x})]_{y_{mal}} - \max_{y \neq y_{mal}} [f({\bf x})]_y,\\
    \phi_{{\bf x}_{tgt}}({\bf x}) &= \text{sign}(S_{{\bf x}_{tgt}}({\bf x})) =
    \begin{cases}
    1 & \text{if } S_{{\bf x}_{tgt}}({\bf x})\geq 0;\\
    -1 & \text{otherwise}.
    \end{cases}
\end{align}
% \begin{align}
%     \phi({\bf x}) &=
%     \begin{cases}
%     1 & \text{if } F(x) \text{ equals } y_{mal} \\
%     -1 & \text{otherwise}.
%     \end{cases}
% \end{align}
We abbreviate the two functions as $S({\bf x})$ and $\phi({\bf x})$ if it does not cause confusion. In boundary-based attack, the attacker is only able to get the value of $\phi$ but not $S$.
% Then $\bf x$ is adversarial if and only if $S({\bf x})\geq 0$, and the adversary can know the value of $\phi$ in the decision-based black-box setting.

In the following, we first introduce the three interative steps in the attack in Section~\ref{sec:iterative_attack}, then introduce three different methods for generating the optimized representative subspace in Section~\ref{sec:name-S}-\ref{sec:name-I}. 

\subsection{General framework of \name}
\label{sec:iterative_attack}

\begin{figure}
    \centering
    % \includegraphics[width=\linewidth]{figs/estimation.pdf}
    \includegraphics[width=\linewidth]{figs/arxiv_ver/estimation.jpeg}
    \caption{Query model and estimate gradient near the decision boundary.}
    \label{fig:gradient_estimation}
    % \vspace{-0.5cm}
\end{figure}

\paragraph{Estimate gradient at decision boundary}
Denote $\xadv{t}$ as the \advimage generated in the $t$-th step. The intuition in this step is that we can estimate the gradient of $S(\xadv{t})$ using only the access to $\phi$ if $\xadv{t}$ is at the decision boundary. This gradient can be sampled via Monte Carlo method:
% Therefore, if we assume that $\xadv{t}$ is at the decision boundary, then the gradient of $S$ w.r.t. an \advimage which lies on the decision boundary can be estimated via Monte Carlo method:
\begin{align}
    \widetilde{\nabla S} = \frac1B \sum_{i=1}^B \phi(\xadv{t}+\delta {\bf u}_b) {\bf u}_b
    \label{eq:MC_gradient_estimation}
\end{align}
where $\{{\bf u}_b\}$ are $B$ randomly sampled perturbations with unit length and $\delta$ is a small weighting constant. An example of this process is shown in Figure \ref{fig:gradient_estimation}. The key point here is how to sample the perturbation ${\bf u}_b$'s and we propose to draw from a representative subspace in $\mathbb{R}^n$.

Formally speaking, let $W=[w_1, \ldots, w_n] \in \mathbb{R}^{m\times n}$ be $n$ orthonormal basis vectors in $\mathbb{R}^m$, meaning $W^\intercal W = I$. Let $\text{span}(W) \subseteq \mathbb{R}^m$ denote the $n$-dimensional subspace spanned by $w_1, \ldots, w_n$. We would like to sample random perturbations from $\text{span}(W)$ instead of from the original space $\mathbb{R}^m$. In order to do that, we sample ${\bf v}_b \in \mathbb{R}^n$ from unit sphere in $\mathbb{R}^n$ and let ${\bf u}_b = W {\bf v}_b$. The detailed gradient estimation algorithm is shown in Alg.\ref{alg:grad-approx}. 
Note that if we let $\text{span}(W)=\mathbb{R}^m$, this step will be the same as in \cite{chen2019hopskipjumpattack}. However, we will sample from some representative subspace so that the gradient estimation is more efficient, and the corresponding theoretic justification is discussed in Section \ref{sec:dimred-theory}.
% In \cite{chen2019hopskipjumpattack} the authors sample uniformly in the unit ball in the entire input space $\mathbb{R}^m$. 

\begin{algorithm}
\caption{Gradient Approximation Based \name}
\label{alg:grad-approx}
\begin{algorithmic}[1]
\renewcommand{\algorithmicrequire}{\textbf{Input:}}
 \renewcommand{\algorithmicensure}{\textbf{Output:}}
 \REQUIRE a data point on the decision boundary ${\bf x} \in \mathbb{R}^m$, basis of the subspace $W \in \mathbb{R}^{m\times n}$, number of random sampling $B$, access to query the decision of victim model $\phi$.
 \ENSURE the approximated gradient $G$
 \STATE sample $B$ random Gaussian vectors of the lower dimension: $V_{rnd} \in \mathbb{R}^{B \times n}$.
 \STATE project the random vectors onto the gradient basis to get the perturbation vectors: $U_{rnd} = V_{rnd} \cdot W^\intercal$.
 \STATE get query points by adding perturbation vectors with the original point on the decision boundary: ${\bf x}_q[i] = {\bf x} + U_{rnd}[i]$.
%  \STATE query the victim model to get the binary decisions(1 for success, -1 otherwise): $D = g_{victim}(X_{query})$
 \STATE Monte Carlo approximation for the gradient: $G = \frac{1}{B}\sum_{i=1}^{B} \phi({\bf x}_q[i]) \cdot U_{rnd}[i]$
 \RETURN $G$
\end{algorithmic}
\end{algorithm}
\vspace{-0.5cm}


% \textbf{Difference on Gradient Estimation in Boundary-based Attack compared with Score-based Attack.} Note that both score-based and gradient-assisted boundary-based attacks do gradient estimation, but both the usage of estimated gradients and the capability of the attacker are different. 
% In score-based attacks, the attacker keep moving the image (which has the benign label) towards the estimated gradient direction until it is predicted as the adversarial label. In boundary-based attacks, the \advimage on the boundary is moved toward the gradient direction so that its adversarial prediction score is increased, and then project back to the boundary so that its distance towards the target-image is decreased.
% In score-based attacks, gradient estimation can be done at each data point with model's confidence scores, but the boundary-based attackers are only able to estimate gradients at the decision boundary using Monte Carlo methods. 
% \Huichen{todo:move}

\paragraph{Move along estimated gradient}
After we have estimated the gradient of adversarial prediction score $\nabla S$, we will move the $\xadv{t}$ towards the gradient direction:
\begin{align}
    \label{eqn:move-grad}
    \hat{\bf x}_{t+1} = \xadv{t} + \xi_t \cdot \frac{\widetilde{\nabla S}}{||\widetilde{\nabla S}||_2}
\end{align}
where $\xi_t$ is the step size at the $t$-th step. Hence, the prediction score of the adversarial class will be increased.

\vspace{-2mm}
\paragraph{Project to decision boundary}
% Now that the adversarial prediction score is larger than 0\bo{score or $\phi$? I think we should not talk about score any more in our method sinc it won't use it right? otherwise it would be confusing. we can introduce score for score based attack and be done. never mention it again in our method.}
Current $\hat{\bf x}_{t+1}$ is beyond the boundary, we can move the \advimage towards the target image so that it is projected back to the decision boundary:
\begin{align}
    \label{eqn:binary}
    \xadv{t+1} = \alpha_t \cdot {\bf x}_{tgt} + (1-\alpha_t) \cdot \hat{\bf x}_{t+1}
\end{align}
where the projection is achieved by a binary search over $\alpha_t$.

Note that we assume $\xadv{t}$ lies on the boundary while ${\bf x}_{src}$ does not lie on the boundary. Therefore, in the initialization step we need to first apply a project operation as in Eqn. \ref{eqn:binary} to get $\xadv{0}$.

In the following sections, we will introduce three exploration for the representative subspace optimization from spatial, frequency, and intrinsic component perspectives.
% In each iterative step we estimate gradient by sampling from a representative subspace and move the \advimage towards the estimated gradient direction. Then we move the \advimage towards the target-image and project it back to the boundary.

% To achieve query efficiency, we aim to optimize a subspace for the gradient matrix with a $\rho$ as large as possible to sample from based on Corollary \ref{tho:dimred}.
% This is equivalent to optimize a representative subspace to reduce the query  space, and then leverage linear mapping to project the subspace back to the original higher-dimensional space. 
% In particular, we propose three approaches to optimize the subspace:
% spatial transformation, low-frequency optimization, and intrinsic components optimization. In the following discussion we assume the image is an $N\times N$ RGB image, which means the dimension of entire image space is $m=3\times N \times N$.

% \subsection{Our pipeline}
%  Define the attack loss and the indicator function:
% \begin{align}
%     S_{{\bf x}_{tgt}}({\bf x}) &= [f({\bf x})]_{y_{mal}} - \max_{y \neq y_{mal}} [f({\bf x})]_y\\
%     \phi_{{\bf x}_{tgt}}({\bf x}) &= \text{sign}(S_{{\bf x}_{tgt}}({\bf x})) =
%     \begin{cases}
%     1 & \text{if } S_{{\bf x}_{tgt}}({\bf x})\geq 0\\
%     -1 & \text{otherwise}
%     \end{cases}
% \end{align}
% % \begin{align}
% %     \phi({\bf x}) &=
% %     \begin{cases}
% %     1 & \text{if } F(x) \text{ equals } y_{mal} \\
% %     -1 & \text{otherwise}.
% %     \end{cases}
% % \end{align}
% We abbreviate the two functions as $S({\bf x})$ and $\phi({\bf x})$ if it does not cause confusion. 
% % Then $\bf x$ is adversarial if and only if $S({\bf x})\geq 0$, and the adversary can know the value of $\phi$ in the decision-based black-box setting. Thus, 
% Then the gradient of $S$ w.r.t. an \advimage which lies on the decision boundary can be estimated via Monte Carlo method:
% \begin{align}
%     \widetilde{\nabla S} = \frac1B \sum_{i=1}^B \phi({\bf x}+\delta {\bf u}_b) {\bf u}_b
%     \label{eq:MC_gradient_estimation}
% \end{align}
% where $\{{\bf u}_b\}$ are $B$ random perturbations uniformly sampled from the unit sphere in $\mathbb{R}^m$ and $\delta$ is a small weighting constant.

% Each iterative step $t$ starts with an adv-image $\xadv{t}$ on the decision boundary. The update breaks down into two steps:
% \begin{enumerate}
%     \item Move the \advimage towards the gradient direction:
%     \begin{align}
%         \label{eqn:grad-est}
%         \hat{\bf x}_{t+1} = \xadv{t} + \xi_t \cdot \frac{\widetilde{\nabla S}}{||\widetilde{\nabla S}||_2}
%     \end{align}
%     \item Project the new image back to the decision boundary:
%     \begin{align}
%         \label{eqn:binary}
%         \xadv{t+1} = \alpha_t {\bf x}_{tgt} + (1-\alpha_t) \hat{\bf x}_{t+1}
%     \end{align}
%     where the projection is achieved by a binary search over $\alpha_t$.
% \end{enumerate}
% Note that the \sourceimage does not lie on the boundary, so we need to first project it onto boundary using Eqn.\ref{eqn:binary} to get $\xadv{0}$.

\subsection{Spatial Transformed Subspace (\name-S)}
\label{sec:name-S}
First we start with the spatial transformed query reduction approach.
The intuition comes from the observation that the gradient of input image has a property of local similarity\cite{ilyas2018prior}. Therefore, a large proportion of the gradients lies on the low-dimensional subspace spanned by the bilinear interpolation operation\cite{spath1993two}.
% The intuition comes from image downsampling where a group of image pixels in a rectangle space is merged into one `hyper pixel' in the new image, and we use the inverse process here.
In order to sample random perturbations for an image, we first sample a lower-dimensional random perturbation $Q$ of shape $\lfloor \frac Nr \rfloor \times \lfloor \frac Nr \rfloor$, where $r$ is the hyperparameter of dimension reduction factor. Then we use bilinear-interpolation to map it back the original image space, $X = \text{Bil\_Interp}(Q)$.

The basis of this spatial transformed subspace is the images transformed from unit perturbations in the lower space:
\begin{align*}
    w^{(i,j)} = \text{Bil\_Interp}(e^{(i,j)}),\quad 0\leq i,j \leq \lfloor N/r \rfloor
\end{align*}
where $e^{(i,j)}$ represents the unit vector that has 1 on the $(i,j)$-th entry and 0 elsewhere.
% In order to sample random perturbations for an image, we sample a smaller vector $v_h$ where each value is the perturbation for a group of pixels in the image. Let the group size be $r_g > 0$, $v_h \sim \mathbb{R}^{\frac{m}{r_g}}$. Then we map $v_h$ into $\mathbb{R}^{m}$ by copying each value for $r_g$ times to fill the pixels in the group.
% \bo{remember to cite the algorithm if you put in the paper}

\subsection{Low Frequency Subspace (\name-F)}
\label{sec:name-F}
In general the low frequency subspace of an image contains the most of the critical information, including the gradient information\cite{guo2018low}; while the high frequency signals contain more noise than useful content.
Hence, we would like to sample our perturbations from the low frequency subspace via Discrete Cosine Transformation(DCT)\cite{ahmed1974discrete}. Formally speaking, define the basis function of DCT as:
\begin{align}
    \phi(i,j) = \cos \bigg(\frac{(i+\frac12)j}{N} \pi \bigg)
\end{align}
The inverse DCT transformation is a mapping from the frequency domain to the image domain $X=\text{IDCT}(Q)$:
\begin{align}
    X_{i_1,i_2}=\sum_{j_1=0}^{N-1}\sum_{j2=0}^{N-1} N_{j_1} N_{j_2} Q_{j_1,j_2} \phi(i_1,j_1) \phi(i_2,j_2)
\end{align}
where $N_j=\sqrt{1/N}$ if $j=0$ and otherwise $N_j=\sqrt{2/N}$.

We will use the lower $\lfloor N/r \rfloor$ part of the frequency domain as the subspace, i.e.
\begin{align}
    w^{(i,j)} = \text{IDCT}(e^{(i,j)}),\quad 0\leq i,j \leq \lfloor N/r \rfloor
\end{align}
where hyperparameter $r$ is the dimension reduction factor.
% Here $r$ denotes the hyperpameter of dimension reduction factor. For example, if we use $r=4$, we get a subspace whose dimension is $1/16$ of that from the original space.

% The basis of the frequency domain is the images transformed from unit vectors in the frequency space:
% \begin{align}
%     w^{(i,j)} = \text{IDCT}(e^{(i,j)})
% \end{align}
% where $e^{(i,j)}$ represents the unit vector that has 1 on the $(i,j)$-th entry and 0 elsewhere. We use the first $\lfloor N/r \rfloor$ part as the basis of the low frequency subspace:
% \begin{align}
%     \{w^{(i,j)}\},\quad 0\leq i,j \leq \lfloor N/r \rfloor
% \end{align}

% Discrete Cosine Transformation (DCT) can transform an image to frequency domain and is used for image compression by only keeping the low-frequency components~\cite{wallace1992jpeg}. Here we first sample a low-dimension random vector $v_f \in \mathbb{R}^{r_f}$ as the perturbations for low-frequency components. To map this vector to the original space, we append 0's to it as the perturbations for higher-frequency components, then do an inverse DCT to get back to image domain.

\subsection{Intrinsic Component Subspace (\name-I)}
\label{sec:name-I}
% \Xiaojun{gradient space, transferbility, randomized, disk}
\begin{figure}
    \centering
    % \includegraphics[width=\linewidth]{figs/subspace.pdf}
    \includegraphics[width=\linewidth]{figs/arxiv_ver/subspace.jpeg}
    \caption{Generate representative subspace from the original high-dimensional gradient space.}
    \label{fig:bases_generation}
    \vspace{-0.5cm}
\end{figure}

Principal Component Analysis (PCA)\cite{wold1987principal} is a standard way to perform dimension reduction in order to search for the intrinsic components of the given instances. Given a set of data points in high dimensional space, PCA aims to find a lower dimensional subspace so that the projection of the data points onto the subspace is maximized. 

Therefore, it is possible to leverage PCA to optimize the subspace for model gradient matrix. However, in order to perform PCA we will need a set of data points. In our case that should be a set of gradients of $S({\bf x})$ w.r.t. different $\bf x$. This is not accessible under black-box setting. Hence, we turn to a set of `reference models' to whose gradient we have access. As shown in Figure \ref{fig:bases_generation}, we will use a reference model to calculate a set of image gradients ${\bf g}_1, {\bf g}_2, \ldots, {\bf g}_K \in \mathbb{R}^m$ Then we perform a PCA to extract its top-$n$ principal components - ${\bf w}_1, \ldots, {\bf w}_n \in \mathbb{R}^m$. These $w$'s are the basis of the Intrinsic Component Subspace.
Note that different from transferability, we do not restrict the reference models to be trained by the same training data with the original model, since we only need to search for the intrinsic components of the give dataset which is relatively stable regarding diverse models.
% By fitting an optimal subspace over gradients of different model, we can find the subspace that captures the the gradient space on the task for all the models. 
% The detailed algorithm is shown in Alg. \ref{alg:basis-gen}.


In practice, the calculation of PCA may be challenging in terms of time and memory efficiency based on large high-dimensional dataset (the data dimension on ImageNet is over 150k and we need a larger number of data points, all of which are dense). Therefore, we leverage the randomized PCA algorithms\cite{halko2011finding} which accelerates the speed of PCA while achieving comparable performance.
% In addition, we can further improve time efficiency by noticing that we can calculate the subspace of the top-$K$ components of PCA without calculating the exact value of the top-$K$ components. The detailed algorithm is as follows\Xiaojun{todo}.

An additional challenge is that the matrix $X$ may be too large to be stored in memory. Therefore, we store them by different rows since each row (i.e. gradient of one image) is calculated independently with the others. The multiplication of $X$ and other matrices in memory are then implemented accordingly.


% \begin{algorithm}
% \caption{Basis Generation Algorithm}
% \label{alg:basis-gen}
% \begin{algorithmic}[1]
%  \renewcommand{\algorithmicrequire}{\textbf{Input:}}
%  \renewcommand{\algorithmicensure}{\textbf{Output:}}
%  \REQUIRE attacker-owned data $X\in \mathbb{R}^{n\times m}$, substitute model ${f}_{sub}$, number of basis $n_b$
%  \ENSURE  orthonormal basis $B$
%  \STATE calculate the gradient of $X$ with respect to the model ${f}_{sub}$: $\Delta {f}_{sub}(X) \in \mathbb{R}^{n\times m}$.
% %  \STATE 
%  \IF {the size of gradient matrix is small} 
%     \STATE compute the singular value decomposition of gradient matrix: $\Delta {f}_{sub}(X) = U_f \Sigma_f V_f^T$. 
%  \ELSE 
%     \STATE approximate the range of gradient matrix with Algorithm \ref{alg:col_space_approx}.
%  \ENDIF
%  \RETURN the basis is the first $b$ rows of matrix $V_f$: $B = V_f[:n_b, :]$.
% \end{algorithmic}
% \end{algorithm}

% \begin{algorithm}
% \caption{Matrix Column Space Approximation}
% \label{alg:col_space_approx}
% \begin{algorithmic}[1]
%  \renewcommand{\algorithmicrequire}{\textbf{Input:}}
%  \renewcommand{\algorithmicensure}{\textbf{Output:}}
%  \REQUIRE matrix $X\in \mathbb{R}^{n\times m}$, the rank to keep from the column space of the matrix $n_r$, a hyperparameter $t$ representing the trade-off between approximation precision and running time
%  \ENSURE column space $R \in {n\times n_r}$
%  \STATE sample a random Gaussian matrix $V_{randn} \mathbb{R}^{m\times n_r}$.
%  \STATE multiply target matrix with random matrix and get $Y_0 = X V_{randn} \in \mathbb{R}^{n\times n_r}$.
%  \FOR{$i = 1$ to $t$}
%     \STATE $Y_i = XX^T Y_{i-1}$.
%  \ENDFOR
%  \STATE do QR decomposition on $Y_t$ to get $Y_t = Q_t R_t$.
%  \RETURN $R = R_t \in \mathbb{R}^{n_r\times n_r}$
% \end{algorithmic}
% \end{algorithm}

% \begin{figure*}
%     \centering
%     \includegraphics[width=\textwidth]{figs/pipeline_v1.pdf}
%     \caption{Overall pipeline of the attack version 1}
%     \label{fig:pipeline_v1}
% \end{figure*}

% \begin{figure*}
%     \centering
%     \includegraphics[width=\textwidth]{figs/pipeline_v2.pdf}
%     \caption{Overall pipeline of the attack version 2}
%     \label{fig:pipeline_v2}
% \end{figure*}


% \section{Dimension Reduction-assisted Boundary Attack} \Huichen{change title to something related to `theory' blablabla?}
\section{Theoretic Analysis on \name}
% \subsection{Dimension Reduction Theorem}
% \section{Dimension Reduction Helps Gradient Estimation}
\label{sec:dimred-theory}

% We propose to apply dimension reduction techniques to improve the gradient estimation quality. The intuition is straightforward: if the dimension of the space from which we sample the $u$'s is smaller, then the estimation quality will be better. Hence, we will choose a subspace in $\mathbb{R}^m$ from which we sample the random perturbations.
% We theoretically show that the gradient estimation quality will be better if we can sample from an informative subspace rather than from the entire space.
% In this section, w
We theoretically analyze how dimension reduction helps with the gradient estimation in \name. We show that the gradient estimation bound is tighter by sampling from a representative subspace rather than the original space.

We consider the gradient estimation as in Eqn. \ref{eq:MC_gradient_estimation} and let $\rho = \frac{||\text{proj}_{\text{span}(W)}(\nabla S)||_2}{||\nabla S||_2}$ denote the proportion of $\nabla S$ that lies on the chosen subspace $\text{span}(W)$. Then we have the following theorem on the expectation of the cosine similarity between $\nabla S$ and estimated $\widetilde{\nabla S}$:

\begin{theorem}
\label{tho:dimred}
Suppose 1) $S({\bf x})$ has $L$-Lipschitz gradients in a neighborhood of $\bf x$, 2) the sampled ${\bf v}_1, \ldots, {\bf v}_B$ are orthogonal to each other, and 3) $W^\intercal W = I$, then the expected cosine simliarity between $\widetilde{\nabla S}$ and $\nabla S$ can be bounded by:
% For a boundary point $x$, suppose that 1) $S(x)$ has $L$-Lipschitz gradients in a neighborhood of $x$, 2) the sampled ${\bf v}_1, \ldots, {\bf v}_B$ are orthogonal to each other, 3) $W^\intercal W = I$. Let constant $\rho = \frac{||\text{proj}_{\text{span}(W)}(\nabla S)||_2}{||\nabla S||_2}$ denote the proportion of $\nabla S$ that lies on $\text{span}(W)$. Then the expected cosine simliarity between $\widetilde{\nabla_W y}$ and $\nabla y$ can be bounded by:
\begin{align}
    &\bigg( 2\bigg(1-(\frac{L\delta}{2||\nabla S||_2})^2\bigg)^{\frac{n-1}{2}} - 1 \bigg)c_n\rho\sqrt{\frac B n} \\
    \leq & \mathbb{E}\big[\cos (\widetilde{\nabla S}, \nabla S) \big]\\
    \leq & c_n\rho\sqrt{\frac B n}
\end{align}
where $c_n$ is a coefficient related with the subspace dimension $n$ and can be bounded by $c_n \in (2/\pi, 1)$. In particular:
\begin{equation}
    \label{eqn:grad-est-qual-dr}
    \lim_{\delta\rightarrow 0}\mathbb{E}\big[\cos (\widetilde{\nabla_W S}, \nabla S) \big] = c_n\rho 
    \sqrt{\frac B n}.
\end{equation}
\end{theorem}
The theorem proof is in Appendix \ref{sec:tho-proof}. 
If we sample from the entire space (i.e. $\text{span}(W) = \mathbb{R}^m$), the expected cosine similarity is $c_m\sqrt{\frac{B}{m}}$.
If we let $m=3\times224\times224$ and $B=100$, the similarity is only around 0.02.
% Hence, we claim that gradient estimation by sampling from the entire space is not an efficient estimation approach.

On the other hand, if the subspace basis $w$'s are randomly chosen, then $\rho \approx \sqrt{\frac{n}{m}}$ and the estimation quality is low. With larger $\rho$, the estimation quality will be better than sampling from the entire space.
Therefore, we further explore three approaches to optimize the representative subspace that contains a larger portion of the gradient as discussed in Section~\ref{sec:subspaces}.
For example, in the experiments we see that when $n=m/16$, we can reach $\rho=0.5$ and the expected cosine similarity increase to around 0.06.
This improves the gradient estimation quality which leads to more efficient attacks.
% with a carefully chosen subspace with more information, we can have a larger $\rho$ to achieve better estimation quality of the gradient. Thus, the key is to choose a subspace that maximizes $\rho$.




\section{Experiments}
In this section, we introduce our experimental setup and quantitative results of the proposed methods \name-S, \name-F, and \name-I, compared with the HSJA attack\cite{chen2019hopskipjumpattack}, which is the-state-of-the-art boundary-based blackbox attack.
Here we focus on the strongest baseline HSJA, which outperforms all of other Boundary Attack~\cite{brendel2017decision}, Limited Attack~\cite{ilyas2018black} and Opt Attack~\cite{cheng2018query} by a substantial margin.
We also show two sets of qualitative results for attacking two real-world APIs with the proposed methods.
\subsection{Datasets and Experimental Setup}
\paragraph{Datasets}
We evaluate the attacks on two offline models on ImageNet\cite{deng2009imagenet} and CelebA\cite{liu2018large} and two online face recognition APIs Face++\cite{facepp-compare-api} and Azure\cite{azure-detect-api}. We use a pretrained ResNet-18 model as the target model for ImageNet and fine-tune a pretrained ResNet-18 model to classify among 100 people in CelebA. We randomly select 50 pairs from the ImageNet/CelebA validation set that are correctly classified by the model as the source and target images. 
\paragraph{Attack Setup}
Following the standard setting in \cite{chen2019hopskipjumpattack}, we use $\xi_t=||\xadv{t-1}-{\bf x}_{tgt}||_2/\sqrt{t}$ as the size in each step towards the gradient. We use $\delta_t=\frac1m ||\xadv{t-1}-{\bf x}_{tgt}||_2$ as the perturbation size and $B=100$ queries in the Monte Carlo algorithm to estimate the gradient, where $m=3\times224\times224$ is the input dimension in each Monte Carlo step.

We provide two \textbf{evaluation metrics} to evaluate the attack performance. The first is the average Mean Square Error (MSE) curve between the target image and the adversarial example in each step, indicating the magnitude of perturbation. The smaller the perturbation is, the more similar the adversarial example is with the \targetimage, thus providing better attack quality.
The second is the attack success rate based on a limited number of queries, where the `success' is defined as reaching certain specific MSE threshold. The less queries we need in order to reach a certain perturbation threshold, the more efficient the attack method is.

% In CIFAR, we use a factor of 2 in resize and DCT, which gives a 768 dimensional subspace. We extract the top 768 major components in PCA.
As for the dimension-reduced subspace, we use the dimension reduction factor $r=4$ 
% \bo{define this thing? give a notation maybe? check the statements to be rigrous!!}\Xiaojun{defined this in section 4.1, 4.2}
in \emph{spatial transformed} and \emph{low frequency} subspace, which gives a 9408 dimensional subspace.
In order to generate the \emph{Intrinsic Component Subspace}, we first generate a set of image gradient vectors on the space. We average over the gradient of input w.r.t. five different pretrained substitute models - ResNet-50\cite{he2016deep}, DenseNet-121\cite{huang2017densely}, VGG16\cite{simonyan2014very}, WideResNet\cite{zagoruyko2016wide} and GoogleNet\cite{szegedy2015going}. We use part of the ImageNet validation set (280000 images) to generate the gradient vectors. Finally we adopt the scalable approximate PCA algorithm\cite{halko2011finding} to extract the top 9408 major components as the intrinsic component subspace.

% \bo{in this above section, please separately discuss the setting for the three methods. this is an important part.}

% We use the approximate PCA algorithm \Xiaojun{cite} to extract the top 9408 major components to improve efficiency.
% In order to generate the gradient vectors for PCA, we will average over the gradient of input w.r.t. five different pretrained substitute models - ResNet-50, DenseNet-121, VGG16, WideResNet and GoogleNet\Xiaojun{cite}. We use part of the ImageNet validation set (280000 images) to generate the gradient vectors.
% Other attack parameter setting is the same as that in \cite{chen2019hopskipjumpattack}.

\subsection{Commercial Online APIs}
Various companies provide commercial APIs (Application Programming Interfaces) of trained models for different tasks such as face recognition. Developers of downstream tasks can pay for the services and integrate the APIs into their applications. Note that although typical platform APIs provide the developers the confidence score of classes associated with their final predictions, the end-user using the final application would not have access to the scores in most cases. For example, some of Face++'s partners use the face recognition techniques for log-in authentication in mobile phones~\cite{facepp-partner}, where the user only knows the final decision (whether they pass the verification or not).

We choose two representative platforms for our real-world experiments based on only the final prediction. The first is Face++ from MEGVII\cite{facepp-compare-api}, and the second is Microsoft Azure\cite{azure-detect-api}. Face++ offers a `compare' API~\cite{facepp-compare-api} with which we can send an HTTPS request with two images in the form of byte strings, and get a prediction confidence of whether the two images contain the same person. In all the experiments we consider a confidence greater than 50\% meaning the two images are tagged as the same person. Azure has a slightly more complicated interface. To compare two images, each image first needs to pass a `detect' API call~\cite{azure-detect-api} to get a list of detected faces with their landmarks, features, and attributes. Then the features of both images are fed into a `verify' function~\cite{azure-verify-api} to get a final decision of whether they belong to the same person or not. The confidence is also given, but we do not need it for our experiments since we only leverage the binary prediction for practical purpose.

In the experiments, we use the examples in Figure~\ref{fig:src_tgt_imgs} as \sourceimage and \targetimage. More specifically, we use a man-woman face as the source-target pair for the `compare' API Face++, and we use a cat-woman face as the pair for the `detect' API Azure face detection.


\begin{figure}
\centering
\begin{subfigure}[t]{.28\linewidth}
  \centering
  \includegraphics[width=\linewidth]{figs/resized_f6.png}
  \caption{Person 1}
  \label{fig:tgt_img}
\end{subfigure}
\begin{subfigure}[t]{.28\linewidth}
  \centering
  \includegraphics[width=\linewidth]{figs/resized_m3.png}
  \caption{Person 2}
  \label{fig:src_img_facepp}
\end{subfigure}
\begin{subfigure}[t]{.28\linewidth}
  \centering
  \includegraphics[width=\linewidth]{figs/resized_n3.png}
  \caption{No-face}
  \label{fig:src_img_azure}
\end{subfigure}
\caption{
The source and target images for online API experiments. All images are resized to $3\times 224\times 224$. 
Image~\ref{fig:tgt_img} is the \targetimage for both APIs. Image~\ref{fig:src_img_facepp} is the \sourceimage for attacking Face++ `compare' API, and \ref{fig:src_img_azure} the \sourceimage for Azure `detect' API. }
\label{fig:src_tgt_imgs}
% \vspace{-0.5cm}
\end{figure}


\begin{figure*}[htpb]
\begin{subfigure}[t]{0.2465\linewidth}
    \centering
    \includegraphics[width=\textwidth]{results/multi_imagenet_mean.pdf}
    \caption{The MSE vs. query number on ImageNet.}
    \label{fig:result-imagenet}
\end{subfigure}
\hspace{1mm}
\begin{subfigure}[t]{0.231\linewidth}
    \centering
    \includegraphics[width=\textwidth]{results/multi_imagenet_success.pdf}
    \caption{The attack success rate with threshold $10^{-3}$ on ImageNet.}
    \label{fig:result-imagenet-sr}
\end{subfigure}
\hspace{1mm}
\begin{subfigure}[t]{0.2465\linewidth}
    \centering
    \includegraphics[width=\textwidth]{results/multi_celeba_mean.pdf}
    \caption{The MSE vs. query number on CelebA.}
    \label{fig:result-celeba}
\end{subfigure}
\hspace{1mm}
\begin{subfigure}[t]{0.235\linewidth}
    \centering
    \includegraphics[width=\textwidth]{results/multi_celeba_success.pdf}
    \caption{The attack success rate with threshold $10^{-5}$ on CelebA.}
    \label{fig:result-celeba-sr}
\end{subfigure}
\caption{The attack results on ImageNet and CelebA datasets.}
\label{fig:results}
\end{figure*}

% \subsection{Optimizations for Experiments}
\paragraph{Discretization Optimization for Attacking APIs}
The attack against online APIs suffers from the problem of `discretization'. That is, in the attack process we assume the pixel values to be continuous in $[0,1]$, but we need to round it into 8-bit floating point in the uploaded RGB images when querying the online APIs.
% The pixel values of each channel in RGB images are 8-bit floating point numbers and the pixel array of an image takes the value of $\mathbb{D}^m \in \{0,\frac{1}{255},\ldots,\frac{254}{255},1\}^m$. There are at most $256^3$ colors possible.
% A typical computer, on the other hand, has 32-bit or 64-bit system. So the querying samples generated by boundary-based attacks with small perturbation around the decision boundary for gradient estimation are likely to contain values that cannot be represented with only 8 bits. We refer to these 32-bit or 64-bit pixel values as `continuous' and the 8-bit images as `discrete' in our discussion for simplicity.
% This is not a problem for offline experiments, since a local machine learning model for image recognition does not set constraint on the input range and it can work as normal on a `continuous image'.
% However, the difference between value ranges incurs a problem for attacking online APIs. The commercial platforms require the uploaded images to have valid pixel values, so the perturbed \queryimages have to be rounded.
This would cause error in the Monte Carlo gradient estimation format in Equation~\ref{eq:MC_gradient_estimation} since the real perturbation between the last \boundaryimage and the new \queryimage after rounding is different from the weighted perturbation vector $\delta {\bf u}_b$. 

In order to mitigate this problem, we perform discretization locally. Let $P_{rd}$ be a projection from a continuous image $\bf x_c$ to a discrete image $\bf x_d = P_{rd}(\bf x_c)$. Let $\delta {\bf u'}_b = P_{rd}(\bf x+\delta {\bf u}_b) - x$, the new gradient estimation format becomes:
\begin{align}
    \widetilde{\nabla f} &= \frac1B \sum_{i=1}^B \phi(P_{rd}({\bf x}+\delta {\bf u}_b)) {\bf u'}_b.
\end{align}

% \bo{this section should be moved to sub of commercial APIs maybe}\Huichen{Ok. Done.}

% One obstacle in applying \Xiaojun{our approach} to attack real-world machine learning system is the problem of \emph{discretization}. That is to say, during our attack process the adversarial examples are considered in the continuous space $\mathbb{R}^m$ (or $[0,1]^m$ if we bound the pixel values), but in reality image pixel values are rounded to an 8-bit float number $\mathbb{D}^m = \{0,\frac{1}{255},\ldots,\frac{254}{255},1\}^m$. 
% Boundary-based attack relies on querying with small perturbation around the decision boundary to estimate the gradient, so it is sensitive to the problem of discretization.

% In order to mitigate the effect of rounding, we propose a pipeline to discretize the subspace from which we are sampling. In particular, suppose the \advimage at the current step is $\xadv{t}$ and we originally want to sample from the subspace ${\bf u} \sim \text{span}(W)$. In a discretized setting, we will discretize $\text{span}(W)$ into $Discr(\text{span}(W)\big|\xadv{t})$ such that ${\bf u} \sim Discr(\text{span}(W)\big|\xadv{t})$ will satisfy:
% \begin{align}
%     \label{eqn:discr}
%     \xadv{t}+\delta{\bf u} \in \mathbb{D}^m
% \end{align}
% The intuition of this approach is that Eqn.\ref{eqn:discr} ensures that our query to the model ($\phi(\xadv{t}+\delta{\bf u}_b$) will not be changed in the rounding process, so we can get a more accurate gradient estimation compared with that when we do not use the discretized subspace.

\subsection{Experimental Results on Offline Models}
% \begin{figure}
%     \centering
%     \includegraphics[width=0.8\linewidth]{results/multi_cifar_mean.pdf}
%     \caption{CIFAR}
%     \label{fig:result-cifar}
% \end{figure}
% \begin{figure}
%     \centering
%     \includegraphics[width=0.8\linewidth]{results/multi_cifar_success.pdf}
%     \caption{CIFAR-success rate}
%     \label{fig:result-cifar-sr}
% \end{figure}

% \begin{table*}[h]
%     \centering
%     \begin{tabular}{|c|c|c|c|}
%         \hline
%         naive/Resize768/DCT768/PCA768train & 0.01 & 0.001 & 0.0001 \\
%         \hline
%         5000 & 1.00 / 1.00 / 1.00 / 1.00 & 1.00 / 1.00 / 1.00 / 1.00 & 0.94 / 0.60 / 0.68 / 0.94 \\
%         10000 & 1.00 / 1.00 / 1.00 / 1.00 & 1.00 / 1.00 / 1.00 / 1.00 & 1.00 / 1.00 / 1.00 / 1.00 \\
%         20000 & 1.00 / 1.00 / 1.00 / 1.00 & 1.00 / 1.00 / 1.00 / 1.00 & 1.00 / 1.00 / 1.00 / 1.00 \\
%         \hline
%     \end{tabular}
%     \caption{cifar}
%     \label{tab:result-cifar}
% \end{table*}

% \begin{table*}[h]
%     \centering
%     \begin{tabular}{|c|c|c|c|}
%         \hline
%         naive/Resize9408/DCT9408/PCA9408 & 0.01 & 0.001 & 0.0001 \\
%         \hline
%         5000 & 0.76 / 0.86 / 0.86 / 0.86 & 0.16 / 0.40 / 0.42 / 0.36 & 0.02 / 0.08 / 0.06 / 0.04 \\
%         10000 & 0.98 / 1.00 / 0.96 / 0.98 & 0.50 / 0.74 / 0.76 / 0.74 & 0.06 / 0.32 / 0.30 / 0.20 \\
%         20000 & 1.00 / 1.00 / 1.00 / 1.00 & 0.84 / 0.98 / 0.96 / 0.98 & 0.28 / 0.70 / 0.66 / 0.68 \\
%         \hline
%     \end{tabular}
%     \caption{imagenet}
%     \label{tab:result-imagenet}
% \end{table*}
% \begin{table*}[h]
%     \centering
%     \begin{tabular}{|c|c|c|c|}
%         \hline
%         naive/Resize9408/DCT9408/PCA9408 & 0.01 & 0.001 & 0.0001 \\
%         \hline
%         5000 & 0.96 / 1.00 / 1.00 / 0.96 & 0.90 / 1.00 / 1.00 / 0.94 & 0.76 / 0.96 / 0.96 / 0.90 \\
%         10000 & 1.00 / 1.00 / 1.00 / 1.00 & 0.98 / 1.00 / 1.00 / 1.00 & 0.90 / 1.00 / 1.00 / 1.00 \\
%         20000 & 1.00 / 1.00 / 1.00 / 1.00 & 1.00 / 1.00 / 1.00 / 1.00 & 1.00 / 1.00 / 1.00 / 1.00 \\
%         \hline
%     \end{tabular}
%     \caption{celeba}
%     \label{tab:result-celeba}
% \end{table*}

% \begin{table}[!t]
% 	\centering
% 	\caption{imagenet}
%     % \renewcommand\tabcolsep{2.9pt} % adjust the space between each column 
%     \begin{threeparttable}
% 	\begin{tabular}{clcccc}
% 		\toprule
% 	     & \textbf{MSE} & 0.01 & 0.001 & 0.0001 \\
% 		\midrule
% 		\multirow{4}[0]{*}{\begin{sideways}5000\end{sideways}} & naive & 0.76  & 0.16 & 0.02 \\
% 		& Resize9408 & 0.86  & 0.40 & 0.08 \\
% 		& DCT9408 & 0.86  & 0.42 & 0.06 \\
% 		& PCA9408 & 0.86  & 0.36 & 0.04\\
% 		\midrule
% 		\multirow{4}[0]{*}{\begin{sideways}10000\end{sideways}} & naive & 0.98  & 0.50 & 0.06 \\
% 		& Resize9408 & 1.00  & 0.74 & 0.32 \\
% 		& DCT9408 & 0.96  & 0.76 & 0.30 \\
% 		& PCA9408 & 0.98  & 0.74 & 0.20 \\
% 		\midrule
% 		\multirow{4}[0]{*}{\begin{sideways}20000\end{sideways}} & naive & 1.00 & 0.84 & 0.28 \\
% 		& Resize9408 & 1.00 & 0.98 & 0.70 \\
% 		& DCT9408 & 1.00 & 0.96 & 0.66 \\
% 		& PCA9408 & 1.00 & 0.98 & 0.68 \\
% 		\bottomrule
% 	\end{tabular}%
% 	\end{threeparttable}
% 	\label{tab:result-imagenet}%
% \end{table}%


% \begin{table}[!t]
% 	\centering
% 	\caption{celeba}
%     % \renewcommand\tabcolsep{2.9pt} % adjust the space between each column 
%     \begin{threeparttable}
% 	\begin{tabular}{clcccc}
% 		\toprule
% 	     & \textbf{MSE} & 0.01 & 0.001 & 0.0001 \\
% 		\midrule
% 		\multirow{4}[0]{*}{\begin{sideways}5000\end{sideways}} & naive & 0.96  & 0.90 & 0.76 \\
% 		& Resize9408 & 1.00  & 1.00 & 0.96 \\
% 		& DCT9408 & 1.00  & 1.00 & 0.96 \\
% 		& PCA9408 & 0.96  & 0.94 & 0.90 \\
% 		\midrule
% 		\multirow{4}[0]{*}{\begin{sideways}10000\end{sideways}} & naive & 1.00  & 0.98 & 0.90 \\
% 		& Resize9408 & 1.00  & 1.00 & 1.00 \\
% 		& DCT9408 & 1.00  & 1.00 & 1.00 \\
% 		& PCA9408 & 1.00  & 1.00 & 1.00 \\
% 		\midrule
% 		\multirow{4}[0]{*}{\begin{sideways}20000\end{sideways}} & naive & 1.00 & 1.00 & 1.00 \\
% 		& Resize9408 & 1.00 & 1.00 & 1.00 \\
% 		& DCT9408 & 1.00 & 1.00 & 1.00 \\
% 		& PCA9408 & 1.00 & 1.00 & 1.00 \\
% 		\bottomrule
% 	\end{tabular}
% 	\end{threeparttable}
% 	\label{tab:result-celeba}
% \end{table}%

\begin{table*}[ht]
  \centering
  \caption{Attack success rate using different number of queries and different MSE thresholds.}
%   \vspace{-0.2cm}
    \renewcommand\tabcolsep{5pt} % adjust the space between each column 
    \begin{threeparttable}
    \begin{tabular}{l|l|cccc|cccc|cccc}
    \toprule
    &  & \multicolumn{4}{c|}{\# Queries = 5000} & \multicolumn{4}{c|}{\# Queries = 10000} & \multicolumn{4}{c}{\# Queries = 20000} \\
    \cmidrule{2-6} \cmidrule{7-10} \cmidrule{11-14} 
    & \makecell{MSE\\ threshold} & \makecell{HJSA} & \makecell{\\-S} & \makecell{\name\\-F} & \makecell{\\-I} & \makecell{HJSA} & \makecell{\\-S} & \makecell{\name\\-F} & \makecell{\\-I} & \makecell{HJSA} & \makecell{\\-S} & \makecell{\name\\-F} & \makecell{\\-I} \\ 
    % \midrule
    \cmidrule{1-2}\cmidrule{3-6} \cmidrule{7-10} \cmidrule{11-14} 
%     \multirow{3}[0]{*}{Cifar10} 
%     & 0.01 & 1.00 & 1.00 & 0.94 & 0.76 & 0.16 & 0.02 & 0.96 & 0.90 & 0.76  \\
% 	& 0.001 & 1.00 & 1.00 & 0.60 & 0.86 & 0.40 & 0.08 & 1.00 & 1.00 & 0.96\\
% 	& 0.0001 & 1.00 & 1.00 & 0.68 & 0.86 & 0.42 & 0.06 & 1.00 & 1.00 & 0.96\\
% 	\midrule
    \multirow{3}[0]{*}{ImageNet} 
    & 0.01 & 0.76 & \bf 0.86 & \bf 0.86 & \bf 0.86 & 0.98 & \bf 1.00 & 0.96 & 0.98 & \bf 1.00 & \bf 1.00 & \bf 1.00 & \bf 1.00 \\
	& 0.001 & 0.16 & 0.40 & \bf 0.42 & 0.36 & 0.50 & 0.74 & \bf 0.76 & 0.74 & 0.84 & \bf 0.98 & 0.96 & \bf 0.98 \\
	& 0.0001 & 0.02 & \bf 0.08 & 0.06 & 0.04 & 0.06 & \bf 0.32 & 0.30 & 0.20 & 0.28 & \bf 0.70 & 0.66 & 0.68\\
	\midrule
    \multirow{3}[0]{*}{CelebA} 
    & 0.01 & 0.96 & \bf 1.00 & \bf 1.00 & 0.96 & \bf 1.00 & \bf 1.00 & \bf 1.00 & \bf 1.00 & \bf 1.00 & \bf 1.00 & \bf 1.00 & \bf 1.00 \\
	& 0.001 & 0.90 & \bf 1.00 & \bf 1.00 & 0.94 & 0.98 & \bf 1.00 & \bf 1.00 & \bf 1.00 & \bf 1.00 & \bf 1.00 & \bf 1.00 & \bf 1.00 \\
	& 0.0001 & 0.76 & \bf 0.96 & \bf 0.96 & 0.90 & 0.90 & \bf 1.00 & \bf 1.00 & \bf 1.00 & \bf 1.00 & \bf 1.00 & \bf 1.00 & \bf 1.00 \\
    \bottomrule
    \end{tabular}%
    \end{threeparttable}
  \label{tab:results}%
\end{table*}

\begin{figure}
    \centering
    \includegraphics[width=0.95\linewidth]{results/process.pdf}
    \caption{An example of attacking ImageNet trained model based on different subspaces.}
    \label{fig:result-process}
    \vspace{-0.5cm}
\end{figure}
% \Xiaojun{Talk about cifar}CIFAR: Figure\ref{fig:result-cifar}; maybe remove it?

To evaluate the effectiveness of the proposed methods, we first show the average MSE during the attack process of ImageNet and CelebA using different number of queries in Figure~\ref{fig:result-imagenet} and Figure \ref{fig:result-celeba} respectively. 
We can see that all the three proposed query efficient methods outperform HSJA significantly.
We also show the attack success rate given different number of queries in Table \ref{tab:results} using different MSE requirement as the threshold. In addition, we provide the attack success rate curve in Figure \ref{fig:result-imagenet-sr} and \ref{fig:result-celeba-sr} using $10^{-3}$ as the threshold for ImageNet and $10^{-5}$ for CelebA to illustrate convergence trend for the proposed \name-S, \name-F, and \name-I, comparing with the baseline HJSA.

We observe that sampling in the optimized subspaces results in a better performance than sampling from the original space. The spatial transforamed subspace and low-frequency subspace show a similar behaviour since both of them rely on the local continuity. The intrinsic component subspace does not perform better than the other two approaches, and the potential reason is that we are only using 280000 cases to find intrinsic components on the 150528-dimensional space. Therefore, the extracted components may not be optimal. We also observe that the face recognition model is much easier to attack than the ImageNet model, since the face recognition model has fewer classes (100) rather than 1000 as of ImageNet. 

A qualitative example process of attacking the ImageNet model using different subspaces is shown in Figure \ref{fig:result-process}. In this example, the MSE (shown as $d$ in the figures) reaches below $1\times10^{-3}$ using around 2K queries when samlping from the subspaces, and it is already hard to tell the adversarial perturbations in the examples. When we further tune the \advimage using 10K queries, it reaches lower MSE.

% The attack result on ImageNet and CelebA is shown in Figure \ref{fig:results} and Table \ref{tab:results}. We see that sampling in all the subspaces results in a better performance than sampling from the entire one. The hyper-pixel subspace and frequency subspace show a similar behaviour since both of them rely on the local continuity. The intrinsic component subspace does not perform better than the other two simpler approaches. We owe it to the reason that we are only using 280000 cases to find intrinsic components on the 150528-dimensional space. Therefore, the used components may not be optimal. An example process of attacking the ImageNet model using different subspaces is shown in Figure \ref{fig:result-process}.


\subsection{Results of Attacking Online APIs}
The results of attacking online APIs Face++ and Azure are shown in Figure~\ref{fig:facepp_atk} and Figure~\ref{fig:azure_atk} respectively.
% against Face++ and Azure face detection API. 
The labels on the y-axis indicate the methods. Each column represents successful attack instances with increasing number of API calls. \Huichen{I unify the two (.) and move front}
% (the perturbation magnitude $d$ is also shown). 
As is the nature of boundary-based attack, all images are able to produce successful attack. The difference lies in the quality of attack instances. 
% (perturbation magnitude $d$). 

For attacks on Face++ `compare' API, the \sourceimage is a man and the \targetimage is a woman as shown in Figure~\ref{fig:src_tgt_imgs}. Notice the man's eyes appear in a higher position in the \sourceimage than the woman in the \targetimage because of the pose. 
% The first column in Figure~\ref{fig:facepp_atk} show the results with a few queries, and we can see all methods produce images with four eyes in them at the beginning.
All the instances on the first row in Figure~\ref{fig:facepp_atk} based on HJSA attacks contain two pairs of eyes. The MSE scores ($d$ in the figures) also confirm that the distance between the attack instance and the \targetimage does not go down much even with more than 6000 queries. On the other hand, our proposed methods \name- can optimize attack instances with smaller magnitude of perturbation more efficiently. The perturbations are also smoother.

The attack results on Azure `detect' API show similar observations. The \sourceimage is a cat and the \targetimage is the same woman. Sampling from the original high-dimensional space (HJSA) gives us attack instances that presents two cat ear shapes at the back of the human face as shown in the first row in Figure~\ref{fig:azure_atk}. With the proposed query efficient attacks, the perturbations are smoother. The distance metric ($d$) also demonstrates the superiority of the proposed methods.

\begin{figure}[t]
    \centering
    \includegraphics[width=0.91\linewidth]{results/facepp_atk.pdf}
    % \vspace{-0.1cm}
    \caption{Comparison of attacks on Face++ `compare' API. Goal: obtain an image that is tagged as `same person' with the \sourceimage person 2 (Figure~\ref{fig:src_img_facepp}) by the API when humans can clearly see person 1 here.}
    \label{fig:facepp_atk}
    \vspace{-0.3cm}
\end{figure}

\begin{figure}[t]
    \centering
    \includegraphics[width=0.91\linewidth]{results/azure_atk.pdf}
    % \vspace{-0.1cm}
    \caption{Comparison of attacks on Azure `detect' API. Goal: get an image that is tagged as `no face' by the API when humans can clearly see a face there. The \sourceimage is a cat as shown in Figure~\ref{fig:src_img_azure}.}
    \label{fig:azure_atk}
    \vspace{-0.3cm}
\end{figure}

% this is for the arxiv version
% the cvpr version does not have the two \vspace commands and the width is 0.95 for both images

\section{Related Work}
% adv
% blackbox
% query efficient score-based blackbox
% a. more info; b. not guarantee success c. #query 

\paragraph{Boundary-based Attack}
% A line of work that is most related to ours is using decision-based attacks.
Boundary Attack~\cite{brendel2017decision} is one of the first work that uses final decisions of a classifier to perform blackbox attacks. The attack process starts from the \sourceimage, which is classified as the adversarial \maliciousclass. Then it employs a reject sampling mechanism to find a \boundaryimage that still belongs to the \maliciousclass by performing random walk along the boundary. The goal is to minimize the distance between the \boundaryimage and the \targetimage. However, as the steps taken are randomly sampled, the convergence of this method is slow and the query number is large.

Several techniques have been proposed to improve the performance of Boundary Attack. \cite{brunner2019guessing,srinivasan2019black,guo2018low} propose to choose the random perturbation in each step more wisely instead of Gaussian perturbation, using Perlin noise, alpha distribution and DCT respectively. \cite{ilyas2018black,khalid2019red,liu2019geometry,chen2019hopskipjumpattack} propose a similar idea - approximating the gradient around the boundary using Monte Carlo algorithm. 
% In our work, we mainly follow the pipeline in \cite{chen2019hopskipjumpattack} because it establishes a good theoretic analysis for the model.

There are two other blackbox attacks which are not based on the boundary. \cite{cheng2018query} proposes to transform the boundary-based output into a continuous metric, so that the score-based attack techniques can be adopted. \cite{dong2019efficient} adopts evolution algorithm to achieve the decision-based attack against face recognition system.
% \Xiaojun{remove redundancy in the background part; mention other gradient estimation on BA} HopSkipJumpAttack~\cite{chen2019hopskipjumpattack} improves upon the basic random walk idea in the original Boundary Attack paper. It proposes an unbiased estimate of the gradient of the model on the decision boundary via a Monte-Carlo algorithm. The \queryimages are generated by adding the \boundaryimage with a set of noise vectors that are randomly sampled from the whole image space. Using the estimated gradient, the paper is able to perform more efficient update to get to a \boundaryimage that is closer to the \targetimage by taking a step towards the gradient direction and then performing binary search back to the decision boundary iteratively. This method reduces the query number compared with Boundary Attack, but since it is sampling from an extremely high-dimensional space(for example, ImageNet data samples lie in $224\times 224\times 3$ dimensional space), the number of queries required to get a fair estimation for updating is still large.
\vspace{-5mm}
\paragraph{Dimension Reduction in Score-based Attack}
Another line of work involves the dimension reduction techniques only for the score-based attacks, which requires access to the prediction of confidence for each class.
In~\cite{guo2018low}, the authors draw intuition from JPEG codec~\cite{wallace1992jpeg} image compression techniques and propose to use discrete cosine transform (DCT) for generating low frequency adversarial perturbations to assist score-based adversarial attack.
AutoZoom~\cite{tu2019autozoom} trains an auto-encoder offline with natural images and uses the decoder network as a dimension reduction tool. Constrained perturbations in the latent space of the auto-encoder are generated and passed through the decoder. The resulting perturbation in the image space is added to the benign one to obtain a query sample. 

% \section{Discussion and conclusions}
% % talk about potential defense

\section{Conclusion}
Overall we propose \name, a general query-efficient boundary-based blackbox attack framework. We in addition explore three novel subspace optimization approaches to reduce the number of queries from spatial, frequency, and intrinsic components perspectives. 
Based on our theoretic analysis, we show the optimality of the proposed subspace based gradient estimation compared with the estimation over the original space. 
Extensive results show that the proposed \name significantly reduces the required number of queries and yields high quality adversarial examples against both offline and onlie real-world APIs.

\subsection*{Acknowledgement}
We would like to thank Prof. Yuan Qi and Prof. Le Song for their comments and advice in this project. This work is partially supported by NSF grant No.1910100.

\newpage

{\small
\bibliographystyle{ieee_fullname}
\bibliography{bibliography}
}

\clearpage
\appendix
\onecolumn
\section{Appendix}

\subsection{Proof in {\assnname}}
\label{app:assn}

\implictrule*
\begin{proof} 
	For the first rule, note that  $(s_{e0}s_{e1})\rho = s_{e0}(s_{e1}\rho ) = s_{e0}\rho  = \rho$. 
	Also, $\forall \svar \in \sigma$, $s_{e0}s_{e1} \svar = s_{e0}\svar s_{e1} = \svar s_{e0}s_{e1}$. Thus, $(\rho, \sigma) \models s_{e0}s_{e1}$. $(\rho, \sigma) \models \lambda_0 s_{e0} + \lambda_1 s_{e1} $ can be proved similarly. \\
	For the second rule, note that $s_{e1}\rho = s_{e1}(s_{e0}\rho ) =
	(s_{e1}s_{e0})\rho = \rho$, and $\forall \svar \in \sigma$, $(\svar s_{e_1})s_{e_0} = s_{e_1}s_{e_0}\svar = (s_{e_1}\svar) s_{e_0}$. Since $s_{e_0}$ is not singular, we have $\svar s_{e_1}= s_{e_1}\svar$, thus $(\rho, \sigma) \models s_{e1}$. \\
	For the final rule, notice that $(a s_{e0} + bs_{e1}s_{e2})\rho = a s_{e0}\rho + bs_{e1}s_{e2}\rho = a s_{e0}\rho + bs_{e1}\rho = \rho$.\\
	Finally, it is easy to see in all these rules, the stabilizer in $\sigma$ is commutable with the target stabilizer expressions. 
\end{proof}

\boolassn*
\begin{proof}
	We first prove the conjunction rule. Since $A_0\wedge A_1 \Rightarrow A_0$, $A_0\wedge A_1 \Rightarrow A_1$, then by the consequence rule, we have $\{A_0\wedge A_1\}\prog\{B_0\}$ and $\{A_0\wedge A_1\}\prog\{B_1\}$, i.e., $\{A_0\wedge A_1\}\prog\{B_0\wedge B_1\}$. For the disjunction rule, notice that if $(\rho,\sigma)\models (A_0\vee A_1)$, then either $(\rho,\sigma)\models A_0$ or $(\rho,\sigma)\models A_1$. 
	Finally,
	$\{I\}\prog\{I\}$ always holds since any state $(\rho,\sigma)$ satisfies $I$. $\{0\}\prog\{B\}$ is true because $(\rho,\sigma)\models 0 \Rightarrow \denot{P}(\rho,\sigma)\models B$.
\end{proof}

\decodecorrect*
\begin{proof}
First, any valid correction function will project the state into one quiescent state of the QEC code. It's the definition of QEC code error correction. \\
Second, note that the assertion $A \wedge A_S$  represents error-free states in the QEC code, thus any valid \textbf{correct} protocol will place a \textbf{skip} statement for correcting the error-free state. 
Assume the \textbf{correct} protocol is implemented based on the look-up table, 
since $A \wedge A_S \wedge -\svar_i = 0$,
then by the condition rule and $\{0\}\prog\{A \wedge A_S\}$ (Lemma~\ref{lem:bool-assn}), we directly get $\{A \wedge A_S\}\textbf{correct}(\svar_0, \svar_1, \cdots) \{A \wedge A_S\}$.
\end{proof}



\soundness*
\begin{proof}
\setcounter{cnt}{0}
(\showcnt) Skip. Note than the skip rule does not change the program state.\\
(\showcnt) Initialization.
By the definition of the substitution rule, $(\rho, \sigma) \models A[\ket{0}/\rho]$ is equivalent to $(\rho_0^q, \sigma) \models A$, then the state after initialization $(\rho',\sigma) = (\rho_0^q,\sigma)$ also satisfies $A$. \\
(\showcnt) Unitary. Note that $(UAU^\dagger)(U\rho U^\dagger) = U A \rho U^\dagger$, so  \\$(UAU^\dagger)(U\rho U^\dagger) = (U\rho U^\dagger) \Leftrightarrow A\rho = \rho$.
\\
(\showcnt) Assignment. For the first rule, assume $(\rho, \sigma) \models A$, then $A$ is commutable with $\svar$. Then, $A$ is also commutable with $-\svar$. Thus, $(\rho,\sigma') = (\rho,\sigma[-\svar/\svar])$ also satisfies $A$. \\
The second rule is obviously correct, but it limits the selection of $A$. \\
(\showcnt) Sequencing. Assume $(\rho, \sigma) \models A$, then $\denot{P_0}(\rho, \sigma)\models C$ by the hypothesis $\{A\}P_0\{C\}$. On the other hand $\denot{P_0;P_1}(\rho,\sigma) = \denot{P_1}(\denot{P_0}(\rho, \sigma)) \models B$ by the hypothesis $\{C\}P_1\{B\}$.
\\
(\showcnt) Condition. 
First, $\sum A_i M_i$ is a legal stabilizer expression because $M_1 = \frac{I+\svar}{2}$ and $M_0 = \frac{I-\svar}{2}$ are legal stabilizer expressions. 
Assume $(\rho, \sigma) \models A$, then $\sigma(\svar)$ is commutable with $A$, so is $M_1$ and $M_0$. Thus, $A M_1\rho M_1^\dagger = M_1 A\rho M_1^\dagger = M_1\rho M_1^\dagger$. Likewise, we have $A M_0\rho M_0^\dagger = M_0\rho M_0^\dagger$. Let $A = \sum_i A_i M_i$, then $A M_1\rho M_1^\dagger = A_1M_1(M_1\rho M_1^\dagger) + A_0M_0(M_1\rho M_1^\dagger) = A_1(M_1\rho M_1^\dagger)$ since $M_1M_1 = M_1$, $M_1M_0 = 0$. Thus, we have $A_1M_1\rho M_1^\dagger = M_1\rho M_1^\dagger$. 
Since $\svar$ is commutable with both $A_1$ and $A_0$, we have $(M_1\rho M_1^\dagger,\sigma) \models A_1$ and $(M_0\rho M_0^\dagger,\sigma[-\svar/\svar]) \models A_0$. Also, $(M_1\rho M_1^\dagger,\sigma) \models \svar$ and $(M_0\rho M_0^\dagger,\sigma[-\svar/\svar]) \models -\svar$. Thus, if $(\rho,\sigma)\models \sum_i A_iM_i$, we have $(M_1\rho M_1^\dagger,\sigma) \models A_1\wedge \svar$ and $(M_0\rho M_0^\dagger,\sigma) \models A_0\wedge -\svar$.
Since $\{A_1 \wedge \svar\}P_1\{B\}$ and $\{A_0 \wedge -\svar\}P_0\{B\}$, by the semantics of the condition statement, we have $\{\sum A_i M_i\}\textbf{if}\,M[\svar, \bar{q}]\,\textbf{then}\, P_0\,\textbf{else}\, P_1\,\textbf{end}\{B\}$. \\
(\showcnt) While. The proof of the While rule is quite similar to that of the Condition rule. $\sum A_iM_i$ is called the invariant of the loop. If the execution enters the loop body, then by $\{ A_1 \wedge \svar \}P_0\{\sum A_i M_i\}$, we still have $(\rho, \sigma) \models \sum A_i M_i$ for the next loop. So, when the while loop terminates, we always have $(\rho, \sigma) \models  A_0 \wedge -\svar$. \\
To prove the While rule more formally, we only need to show the partial correctness holds for $\textbf{while}^{(k)}$, as $\textbf{while}$ is the disjunction of $\textbf{while}^{(k)}$, $k=0,1,2,\cdots$.
\\
(\showcnt) Consequence. Assume $(\rho,\sigma) \models A$, then $ (\rho,\sigma) \models A'$ by $\{A\Rightarrow A'\}$. Since $\{A'\}\prog\{B'\}$, we have $\denot{P}(\rho,\sigma) \models B'$. Then $\denot{P}(\rho,\sigma) \models B$ by $B'\Rightarrow B$. Thus, $\{A\}\prog\{B\}$.
\end{proof}

\subsection{Verification of Quantum Repetition Code}
\label{app:rep}
\repcnot*
\begin{proof}
First, for control qubit $a$ and target qubit $b$, $\text{CNOT}_{ab} = \frac{1}{2}(I + X_b + Z_a - Z_aX_b)$. Then\\
(1) $\{Z_{L0} I_{L1}\}\prog\{Z_{L0}I_{L1}\}$. Note that both $\text{CNOT}_{03}$, $\text{CNOT}_{14}$ and $\text{CNOT}_{25}$ are commutable with $Z_{L0}$, so $\text{CNOT}_{03}Z_{L0}\text{CNOT}_{03} = Z_{L0}\text{CNOT}_{03}\text{CNOT}_{03} = Z_{L0}$, , $\text{CNOT}_{14}Z_{L0}\text{CNOT}_{14}= Z_{L0}$ and $\text{CNOT}_{25}Z_{L0}\text{CNOT}_{25}= Z_{L0}$. \\
(2) $\{X_{L0} I_{L1}\}\prog\{X_{L0}X_{L1}\}$. Note that $\text{CNOT }_{03}X_{L0}\text{CNOT}_{03} = X_{L0}X_3$. Since $X_3$ is commutable with $\text{CNOT}_{14}$, $\text{CNOT}_{14}X_{L0} X_3 \text{CNOT}_{14} = (\text{CNOT }_{14}X_{L0}\text{CNOT}_{14})X_3 = X_{L0}X_4X_3$. Finally, $\text{CNOT}_{25}X_{L0}X_4X_3\text{CNOT}_{25} = X_{L0}X_5X_4X_3 = X_{L0}X_{L1}$. \\
(3) $\{I_{L0} X_{L1}\}\prog\{I_{L0}X_{L1}\}$. Note that both $\text{CNOT}_{03}$, $\text{CNOT}_{14}$ and $\text{CNOT}_{25}$ are commutable with $X_{L1}$. \\
(4) $\{I_{L0} Z_{L1}\}\prog\{Z_{L0}Z_{L1}\}$. Note that  $\text{CNOT}_{03}Z_{L1}\text{CNOT}_{03} = Z_{0}Z_{L1}$, $\text{CNOT}_{14}Z_{0}Z_{L1}\text{CNOT}_{14} = Z_{0}\text{CNOT}_{14}Z_{L1}\text{CNOT}_{14} = Z_{0}Z_{1}Z_{L1}$, and $\text{CNOT}_{25}Z_{0}Z_{1}Z_{L1}\text{CNOT}_{25} = Z_{0}Z_{1}\text{CNOT}_{25}Z_{L1}\text{CNOT}_{25} = Z_{0}Z_{1}Z_{2}Z_{L1} = Z_{L0}Z_{L1}$. \\
Finally, We can prove that $\{Z_0Z_1\}\prog\{Z_0Z_1\}$, $\{Z_1Z_2\}\prog\{Z_1Z_2\}$, $\{Z_3Z_4\}\prog\{Z_0Z_1Z_3Z_4\}$, $\{Z_4Z_5\}\prog\{Z_1Z_2Z_4Z_5\}$ in a similar way. Combing all these facts, we can prove the desired partial correctness on the logical CNOT gate.
\end{proof}

\subsection{Verification of the Surface Code}
\label{app:surf}


\begin{program}[Initialize $\ket{0_L}$]
For the initialization operation in the  figure below, which initializes an X-cut logical qubit to $\ket{0_L}$,\\
\includegraphics[width=0.5\textwidth]{figure/initplus6.pdf}\\
\label{prog:initzero}
we have $\prog \Coloneqq
\bar{q} \coloneqq \ket{0};
\svar_0 \coloneqq X_0X_1X_2X_4;
\svar_1 \coloneqq X_4X_6X_7X_9;
\svar_2 \coloneqq X_9X_{11}X_{12}X_{14};
\svar_3 \coloneqq X_{14}X_{16}X_{17}X_{18}; \\
\svar_4 \coloneqq Z_1Z_3Z_4Z_6;
\svar_5 \coloneqq Z_2Z_4Z_5Z_7;
\svar_6 \coloneqq Z_{6}Z_{8}Z_{9}Z_{11};
\svar_7 \coloneqq Z_{11}Z_{13}Z_{14}Z_{16};
\svar_8 \coloneqq Z_{7}Z_{9}Z_{10}Z_{12};
\svar_{9} \coloneqq Z_{12}Z_{14}Z_{15}Z_{17};
\svar_{10} \coloneqq \cdots \\
\textbf{correct}(\svar_0,\svar_1,\cdots);
\svar_0 \coloneqq I; 
\svar_1 \coloneqq I;
\svar_2 \coloneqq I; 
\svar_3 \coloneqq I;
\svar_4 \coloneqq Z_1Z_3Z_6; 
\svar_5 \coloneqq Z_2Z_5Z_7; 
\svar_6 \coloneqq Z_{6}Z_{8}Z_{11}; 
\svar_7 \coloneqq Z_{11}Z_{13}Z_{16}; \\
\svar_8 \coloneqq Z_{7}Z_{10}Z_{12}; 
\svar_{9} \coloneqq Z_{12}Z_{15}Z_{17}; 
\svar_{10} \coloneqq \cdots \\
\svar_{\stabnum +1} \coloneqq Z_{4}; 
\svar_{\stabnum +2} \coloneqq Z_{9};
\svar_{\stabnum +3} \coloneqq Z_{14}; \\
\text{// set }q_4, q_9, q_{14}\text{ to }\ket{0}; \\
\qif{\svar_{\stabnum +1}, q_4 }{\textbf{skip}}{\bar{q} \coloneqq X_4X_6X_7X_9\bar{q}; \svar_{\stabnum +1} \coloneqq Z_{4}} %
\\
\textbf{if}\ M[\svar_{\stabnum +2}, q_9]\ \textbf{then}
\myquad \textbf{skip}\ 
\textbf{else}
\myquad \bar{q} \coloneqq X_9X_{11}X_{12}X_{14}\bar{q}; 
\svar_{\stabnum +2} \coloneqq Z_{9}\ 
\textbf{end} \\
\qif{\svar_{\stabnum +3}, q_{14}}{\textbf{skip}}{\bar{q} \coloneqq X_{14}X_{16}X_{17}X_{18}\bar{q}; \svar_{\stabnum +1} \coloneqq Z_{14}}; \\
\svar_1 \coloneqq X_4X_6X_7X_9;
\svar_2 \coloneqq X_9X_{11}X_{12}X_{14}$;
$\textbf{correct}(\svar_0,\svar_1,\cdots)$.
\end{program}

\begin{restatable}[Initialize $\ket{0_L}$]{proposition}{surfinitzerol}
For the program $\prog$ in Program~\ref{prog:initzero} which initializes a X-cut logical qubit to $\ket{0_L}$, \\
$\{I\}\prog\{Z_4Z_9Z_{14}\}$. Here $Z_4Z_9Z_{14}$ is the logical Z operator $Z_{L}$.
\end{restatable}
\begin{proof}
By Proposition~\ref{prop:decodecorrect}, after \textbf{correct} function, $(\rho,\sigma)\models (\svar_0 \wedge \svar_1 \wedge \svar_2 \cdots)$. The following stabilizer assignments which turn off X-stabilizers will just forward the precondition. \\
For simplicity, assume there are {\stabnum} stabilizers in the surface code array. let $\Lambda = \{0,\cdots,w-1\}$, then $(\svar_0 \wedge \svar_1 \wedge \svar_2 \cdots) = \wedge_{i\in \Lambda}\svar_i$. Since $\svar_0 \wedge \svar_1 \Rightarrow X_0X_1X_2X_6X_7X_9$ and $X_0X_1X_2X_6X_7X_9$ is commutable with $Z_4$, $\{\wedge_{i\in \Lambda}\svar_i\}\svar_{\stabnum +1} \coloneqq Z_{4}\{(\wedge_{i\in \Lambda\setminus \{0,1\}}\svar_i) \wedge X_0X_1X_2X_6X_7X_9\}$. Likewise, we know that after $\svar_{\stabnum +2} \coloneqq Z_{9}$, the precondition will become \\
$\{(\wedge_{i\in \Lambda\setminus \{0,1,2,3\}}\svar_i) \wedge X_0X_1X_2X_6X_7X_{11}X_{12}X_{16}X_{17}X_{18}\}$. \\
Note that $(\wedge_{i\in \Lambda\setminus \{0,1,2,3\}}\svar_i) \wedge X_0X_1X_2X_6X_7X_{11}X_{12}X_{16}X_{17}X_{18} \Rightarrow X_0X_1X_2X_6X_7X_{11}X_{12}X_{16}X_{17}X_{18} $. \\
Let $A = X_0X_1X_2X_6X_7X_{11}X_{12}X_{16}X_{17}X_{18}$, \\
$c = \qif{\svar_{\stabnum +1}, q_4}{\textbf{skip}}{\bar{q} \coloneqq X_4X_6X_7X_9\bar{q}; \svar_{\stabnum +1} \coloneqq Z_{4}}$.
It's easy to see that $\{A \wedge Z_4\} \textbf{skip} \{A \wedge Z_4\}$, and $\{A \wedge -Z_4\} q_4 \coloneqq Xq_4; \svar_{\stabnum +1}\coloneqq Z_4 \{A \wedge Z_4\}$. Thus, $\{A\}c\{A \wedge Z_4\}$. Then, after reset $q_{14}$ to $\ket{0}$, the precondition will become: \\
$\{A \wedge Z_4 \wedge Z_9 \wedge Z_{14}\}$.
Again, the following stabilizer assignments will just forward the precondition. By the implication rule, we have that $A \wedge Z_4 \wedge Z_9 \wedge Z_{14} \Rightarrow A \wedge Z_4Z_9Z_{14}$. Since $Z_4Z_9Z_{14}$ and all assertions in $A$ are commutable with stabilizers $\svar_0, \svar_1, \cdots$, we have $\{ A \wedge Z_4Z_9Z_{14}\} \textbf{correct}(\svar_0,\svar_1,\cdots) \{\wedge_{i\in \Lambda} \svar_i \wedge Z_4Z_9Z_{14}  \wedge X_0X_1X_2X_6X_7X_{11}X_{12}X_{16}X_{17}X_{18} \}$. Then by applying the consequence rule, we get $\{I\}\prog\{Z_4Z_9Z_{14}\}$.
\end{proof}

\begin{program}[Logical X gate] For the logical X gate $X_L$ in the Figure below:\\
\includegraphics[width=0.2\textwidth]{figure/surflogx6.pdf} \\
we have $\prog \Coloneqq q_0q_1q_2q_4 \coloneqq X_{0}X_{1}X_{2}X_{4}q_0q_1q_2q_4$.
\end{program}

\begin{proposition}[Logical X gate]
For program $\prog$ in Figure~\ref{fig:surfcode}(d), we have  $\{Z_L\}\prog\{-Z_L\}$ and $\{-Z_L\}\prog\{Z_L\}$, \\
where $Z_L = Z_4Z_9Z_{14}$.
\end{proposition}
\begin{proof}
	Notice that $(X_L)(Z_L)(X_L)^\dagger = -Z_L$.
\end{proof}

\statestabilizer*
\begin{proof}
For the first part, we can get $a = \frac{\alpha^2 - \beta^2}{\alpha^2 + \beta^2}$ and $b = \frac{2\alpha\beta}{\alpha^2 + \beta^2}$ simply by solving the equation $(a Z_L + b X_L)\ket{\psi} = \ket{\psi}$. \\
For the second part, assume $\ket{\psi_0} = \alpha_0 Z_L + \beta_0 X_L$ and $\ket{\psi_1} = \alpha_1 Z_L + \beta_1 X_L$, if there is a $aZ_L + bX_L$ s.t. $(aZ_L + bX_L)\ket{\psi_0} = \ket{\psi_0}$ and $(aZ_L + bX_L)\ket{\psi_1} = \ket{\psi_1}$. Then, we have $a = (\frac{\alpha_0^2 - \beta_0^2}{\alpha_0^2 + \beta_0^2} = (\frac{\alpha_1^2 - \beta_1^2}{\alpha_1^2 + \beta_1^2}$, which is equivalent to $1 - \frac{2}{1 + (\frac{\alpha_0}{\beta_0})^2} = 1 - \frac{2}{1 + (\frac{\alpha_1}{\beta_1})^2}$. Thus, $(\frac{\alpha_0}{\beta_0})^2 = (\frac{\alpha_1}{\beta_1})^2$. On the other hand, $b = \frac{2\alpha_0\beta_0}{\alpha_0^2 + \beta_0^2} = \frac{2\alpha_1\beta_1}{\alpha_1^2 + \beta_1^2}$, which is equivalent to $\frac{\frac{\alpha_0}{\beta_0}}{1 + (\frac{\alpha_0}{\beta_0})^2} = \frac{\frac{\alpha_1}{\beta_1}}{1 + (\frac{\alpha_1}{\beta_1})^2}$. Thus, $\frac{\alpha_0}{\beta_0} = \frac{\alpha_1}{\beta_1}$, i.e., $\ket{\psi_0} = \ket{\psi_1}$ up to a global phase.
\end{proof}
\surfvqmov*
\begin{proof}
After the first \textbf{correction} function, the precondition is transformed into: $(a Z_L + b X_L)\wedge_i \svar_i$.
The three following stabilizer assignments will forward the precondition. Then by the implication rule, $(a Z_L + b X_L)\wedge_i \svar_i \Rightarrow (aZ_L + bX_{2}X_{3}X_{4}X_{8}X_{9}X_{10})\wedge_{i \ne 1}\svar_i$. so for the next stabilizer assignment $\svar_{\stabnum +1} = Z_6$, precondition $(aZ_L + bX_{2}X_{3}X_{4}X_{8}X_{9}X_{10}) \wedge_{i \ne 1}\svar_i$ will be forwarded. Note that $(aZ_L + b X_{2} X_{3} X_{4} X_{8} X_{9}X_{10}) \wedge_{i \ne 1}\svar_i \Rightarrow aZ_L + b X_{2} X_{3} X_{4} X_{8} X_{9}X_{10}$,
let $A = aZ_L + b X_{2} X_{3} X_{4} X_{8} X_{9}X_{10}$, \\
$c = \qif{\svar_{\stabnum +1}, \bar{q}}{\textbf{skip}}{\bar{q}\coloneqq X_6X_{8}X_{9}X_{10}\bar{q}; \svar_{\stabnum +1}=Z_6}$. \\
For the if statement, $\{A \wedge \svar_{w+1}\}\textbf{skip}\{A \wedge \svar_{w+1}\}$ and $\{A \wedge -\svar_{w+1}\}\bar{q}\coloneqq X_6X_{8}X_{9}X_{10}\bar{q}; \svar_{\stabnum +1}=Z_6\{A \wedge \svar_{w+1}\}$, then $\{A\}c\{A \wedge \svar_{w+1} \}$.
By implication rule, $A \wedge \svar_{w+1} \Rightarrow aZ_LZ_6 + bX_{2} X_{3} X_{4} X_{8} X_{9}X_{10}$. The next three stabilizer assignment will forward $aZ_LZ_6 + bX_{2} X_{3} X_{4} X_{8} X_{9}X_{10}$. Then with the \textbf{correct} function and the consequence rule, we get that $\{aZ_L + bX_L\}\prog\{aZ_L' + bX'_L\}$, i.e., the logical state is not changed by the qubit moving operation.
\end{proof}

A braiding operation involves many data qubits, and at least 51 data qubits will be referenced in the problem. To simplify the program, we will use the qubit moving as primitive. 
$\textbf{qmov}(X_L, X_L')$ means to move the defect that changes the logical X operation of a X-cut qubit from $X_L$ to $X_L'$, and $\textbf{qmov}(Z_L, Z_L')$ to move the defect that changes the logical Z operation of a Z-cut qubit from $Z_L$ to $Z_L'$.

\begin{program}[Braiding]\label{prog:surf-braid} In the figure below, we braid a Z-cut qubit with a X-cut qubit: \\
\includegraphics[width=0.43\textwidth]{figure/surfbraid1.pdf}
\\
The associated program is $\prog \Coloneqq$ \\
$
\textbf{qmov}(Z_{5}Z_{9}Z_{10}Z_{15}, Z_{15}Z_{20}Z_{21}Z_{26}) \\
\textbf{qmov}(Z_{15}Z_{20}Z_{21}Z_{26}, Z_{26}Z_{31}Z_{32}Z_{37}) \\
\textbf{qmov}(Z_{26}Z_{31}Z_{32}Z_{37}, Z_{37}Z_{42}Z_{43}Z_{47}) \\
\textbf{qmov}(Z_{37}Z_{42}Z_{43}Z_{47}, Z_{38}Z_{43}Z_{44}Z_{48}) \\
\textbf{qmov}(Z_{38}Z_{43}Z_{44}Z_{48}, Z_{39}Z_{44}Z_{45}Z_{49}) \\
\textbf{qmov}(Z_{39}Z_{44}Z_{45}Z_{49}, Z_{40}Z_{45}Z_{46}Z_{50}) \\
\textbf{qmov}(Z_{40}Z_{45}Z_{46}Z_{50}, Z_{29}Z_{34}Z_{35}Z_{40}) \\
\textbf{qmov}(Z_{29}Z_{34}Z_{35}Z_{40}, Z_{18}Z_{23}Z_{24}Z_{29}) \\
\textbf{qmov}(Z_{18}Z_{23}Z_{24}Z_{29}, Z_{8}Z_{12}Z_{13}Z_{18}) \\
\textbf{qmov}(Z_{8}Z_{12}Z_{13}Z_{18}, Z_{7}Z_{11}Z_{12}Z_{17}) \\
\textbf{qmov}(Z_{7}Z_{11}Z_{12}Z_{17}, Z_{6}Z_{10}Z_{11}Z_{16}) \\
\textbf{qmov}(Z_{6}Z_{10}Z_{11}Z_{16}, Z_{5}Z_{9}Z_{10}Z_{15})
$
\end{program}

According to Fowler, the verification of the braiding operation only need to focus on four configurations of logical states on a pair of logical qubits: $X_{L1}\otimes I_{L2}$, $I_{L1}\otimes X_{L2}$, $I_{L1}\otimes Z_{L2}$ and $Z_{L1}\otimes I_{L2}$.
\begin{restatable}[Braiding]{proposition}{surfbraid} 
For the program $\prog$ in Program~\ref{prog:surf-braid}, \\
$\{X_{L1} I_{L2}\}\prog\{X_{L1}X_{L2}\}$, $\{I_{L1} Z_{L2}\}\prog\{Z_{L1} Z_{L2}\}$, $\{I_{L1}X_{L2}\}\prog\{I_{L1} X_{L2}\}$ and $\{Z_{L1} I_{L2}\}\prog\{Z_{L1} I_{L2}\}$.
\end{restatable}
\begin{proof} To simplify the proof, we let $A_S = \wedge_i \svar_i$, i.e., the assertion generated by current active stabilizers in the surface code array. Note that $A_S$ may change at different time-step. The proof of the braiding operation involves tedious computation and we only give a sketch of the proof here. \\
(1) Prove $\{X_{L1} I_{L2}\}\prog\{X_{L1} X_{L2}\}$. Since $X_{L1} I_{L2} = X_{L1}$, we only need to focus on the reasoning on $X_{L1}$ only. From the verification of the qubit moving, 
$\{X_{L1} I_{L2}\} \textbf{qmov}(Z_{5}Z_{9}Z_{10}Z_{15}, Z_{15}Z_{20}Z_{21}Z_{26}) \{X_{L1}X_{15}  I_{L2}\}$ (after \textbf{correct} function, $\{X_{L1}I_{L2} \wedge X_{15} \wedge A_S$ becomes $\{X_{L1}X_{15}I_{L2} \wedge A_S$). Then, after all these qubit moving operations, we will get \\ $\{X_{L1}I_{L2}\}\prog\{X_{L1} X_{15}X_{26}X_{37}X_{43}X_{44}X_{45}X_{29}X_{18}X_{12}X_{11}X_{10}I_{L2} \wedge A_S\}$. Apply implication rule on $A_S$, we get \\ $A_S \Rightarrow (X_{10}X_{15}X_{16}X_{21}) (X_{21}X_{26}X_{27}X_{32})(X_{32}X_{37}X_{38}X_{43})(X_{33}X_{38}X_{39}X_{44})(X_{34}X_{39}X_{40}X_{45})\\
(X_{23}X_{28}X_{29}X_{34})(X_{12}X_{17}X_{18}X_{23})(X_{11}X_{16}X_{17}X_{22}) = (X_{15}X_{26}X_{37}X_{43}X_{44}X_{45}X_{29}X_{18}X_{12}X_{11}X_{10})(X_{27}X_{33}X_{28}X_{22})
$. Then, by the consequence rule, we have $\{X_{L1}I_{L2}\}\prog\{X_{L1}X_{L2}\}$.
\\
(2) Prove $\{I_{L1} Z_{L2}\}\prog\{Z_{L1} Z_{L2}\}$.
Before the qubit moving operation involves qubits in $Z_{L2}$, the precondition $\{I_{L1}Z_{L2}\}$ will be forwarded by the qubit moving operation. So, we only need to elaborate on $\textbf{qmov}(Z_{38}Z_{43}Z_{44}Z_{48}, Z_{39}Z_{44}Z_{45}Z_{49})$. \\
Before measuring $q_{44}$ in X basis, the assignment statement about $X_{44}$ will turn the precondition $\{I_{L1} Z_{L2}\}$ into \\
$Z_{L2}(Z_{39}Z_{44}Z_{45}Z_{49})$, following the previous verification steps of qubit moving. The if statement on $X_{44}$ and $q_{44}$ will then transform the precondition into $Z_{L2}(Z_{39}Z_{44}Z_{45}Z_{49}) \wedge X_{44}$. The following assignment statement about $Z_{38}Z_{43}Z_{44}Z_{48}$ will turn the precondition $Z_{L2}(Z_{39}Z_{44}Z_{45}Z_{49}) \wedge X_{44}$ into  $Z_{L2}(Z_{39}Z_{44}Z_{45}Z_{49})$. Likewise, the remaining qubit moving operations will change the precondition $Z_{L2}(Z_{39}Z_{44}Z_{45}Z_{49})$ to $Z_{L2}(Z_{40}Z_{45}Z_{46}Z_{50})$, $\cdots$, until $Z_{L2}(Z_{5}Z_{9}Z_{10}Z_{15})$, which is just $Z_{L1}Z_{L2}$. Thus, $\{I_{L1} Z_{L2}\}\prog\{Z_{L1} Z_{L2}\}$.
\\
(3) Prove $\{I_{L1} X_{L2}\}\prog\{I_{L1} X_{L2}\}$. Recall the verification of the qubit moving operation. It is easy to see that $\{I_{L1}\}\textbf{qmov}\{I_{L1}\}$ for any qubit moving operation in $P$. On the other hand, the qubit moving operations in $P$ does not involve any qubits in $X_{L2}$, so precondition $I_{L1}X_{L2}$ will be forwarded by all qubit moving operations, i.e.,
$\{I_{L1} X_{L2}\}\prog\{I_{L1} X_{L2}\}$.
\\
(4) Prove $\{Z_{L1} I_{L2}\}\prog\{Z_{L1}I_{L2}\}$. Since $Z_{L1} I_{L2} = Z_{L1}$, we only focus on the reasoning of $Z_{L1}$ here. It is obvious that starting from $Z_5Z_9Z_{10}Z_{15}$, the logical Z operator finally returns to $Z_5Z_9Z_{10}Z_{15}$ by a series of qubit moving operations. Thus, $\{Z_{L1} I_{L2}\}\prog\{Z_{L1}I_{L2}\}$.
\end{proof}


\end{document}
