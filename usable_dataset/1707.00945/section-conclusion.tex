\section{Conclusion\label{sec:conclusion}}
Although the verification of SPARK~2014 programs is very close to
execution semantics and therefore mostly intuitive, we believe that
developers still need some basic training to avoid common mistakes as
described in this paper, which otherwise could lead to a false confidence in
the software being developed. 
%Unless a software is testified AoRTE, there may be missed
%error and defect due to the deductive approach. We think this needs an
%impact analysis and quantification of sort, to not create false
%confidence in the verification progress and thereby to avoid
%underestimating the remaining risk that may reside in the software.
Overall, the language forces developers to address boundary
cases % here it's "boundary case"
of a system explicitly, which eventually helps understanding the
system better, and usually reveals missing requirements for
boundary cases. As a downside, SPARK~2014 programs are often longer than
(approximately) equivalent Ada programs, since in the latter case a
general exception handler can be installed to handle all pathological
cases at once, without differentiating them. Furthermore, static analysis is ready
to replace unit tests, but integration tests have still been found
necessary. 

Regarding the shortcomings of the GNAT dimensionality system, we
can report that as a consequence of our experiments, a solution for
generic operations on dimensioned has been found and will be part of future GNAT releases.

Our remaining criticism to SPARK~2014 and its tools is as follows:
next to some minor tooling enhancements to avoid the mistakes
mentioned earlier and adding some more knowledge to the analyzer, it
is necessary to support object-oriented features in a better
way. All in all, SPARK~2014 raises the bar for formal
verification and its tools, but developers still have to be aware of
limitations.

%%% Local Variables: ***
%%% mode:latex ***
%%% TeX-master: "paper.tex"  ***
%%% End: ***