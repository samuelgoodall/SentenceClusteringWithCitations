
%%%%%%%%%%%%%%%%%%%%%%%%%%%%%%%%%%%%%%%%%%%%%%%%%%%%%%%%%%%%%%%%%%%%%
\chapter[\QKD Against Bounded Eavesdroppers]
{\QKD Secure Against Quantum-Memory-Bounded Eavesdroppers} \label{chap:qkd}
%%%%%%%%%%%%%%%%%%%%%%%%%%%%%%%%%%%%%%%%%%%%%%%%%%%%%%%%%%%%%%%%%%%%%
\index{quantum key distribution|(} In this chapter, we present another
application for the uncertainty relation derived in
Section~\ref{sec:morerelation}. This illustrates that these relations
are useful in scenarios beyond the simple two-party setting.

In Quantum Key Distribution (\QKD), two honest players Alice and Bob
want to agree on a secure key, using only completely insecure quantum
and authentic classical communication. The computationally unbounded
eavesdropper Eve should not get any information about the key. A major
difficulty when implementing \QKD schemes is that they require a
low-noise quantum channel.  The tolerated noise level depends on the
actual protocol and on the desired security of the key.  Because the
quality of the channel typically decreases with its length, the
maximum tolerated \index{noise level}noise level
 is an important parameter limiting the
\index{maximum distance}maximum distance between Alice and Bob.

We consider a model in which the adversary has a limited amount of
quantum memory to store the information she intercepts during the
protocol execution. In this model, we show that the maximum
tolerated noise level is larger than in the standard scenario where
the adversary has unlimited resources.  

For simplicity, we restrict ourselves to {\em one-way \QKD protocols}
which are protocols where error-correction is performed
non-interactively, i.e., a single classical message is sent from one
party to the other.

The results in this chapter appeared in~\cite{DFRSS07}.

%% The quantum-memory-bounded eavesdropper model studied here is not
%% comparable to other restrictions on adversaries considered in the
%% literature (e.g. \emph{individual attacks}, where the eavesdropper is
%% assumed to apply independent measurements to each qubit sent over the
%% quantum channel\cite{FGGNP97,lutkenhaus00}).  In fact, these
%% assumptions are generally artificial and their purpose is to simplify
%% security proofs rather than to relax the conditions on the quality of
%% the communication channel from which secure key can be generated.  We
%% believe that the quantum-memory-bounded eavesdropper model is more
%% realistic.

\section{Derivation of the Maximum Tolerated Noise Level}
Let $\bset$ be a set of orthonormal bases of a $d$-dimensional Hilbert
space $\cH_d$. For each basis $\vartheta \in \bset$, we assume that
the $d$ basis vectors are parametrized by the elements of the fixed
set $\cX$ of size $|\cX|=d$. We then consider \QKD protocols consisting
of the steps described in Figure~\ref{fig:QKDShape}.
\begin{myfigure}{h}
\begin{myprotocol}[let $N \in \naturals$ be arbitrary]{One-Way \QKD}
\item \emph{Preparation:} For $i=1 \ldots N$, Alice chooses at random
  a basis $\vartheta_i \in \bset$ and a random element $X_i \in \cX$.
  She encodes $X_i$ into the state of a quantum system according to
  the basis $\vartheta_i$ and sends this system to Bob.  Bob measures
  each of the states he receives according to a randomly chosen basis
  $\vartheta'_i$ and stores the outcome $Y_i \in \cX$ of this
  measurement. \index{preparation}
\item \emph{Sifting:} Alice and Bob publicly announce their choices of
  bases and keep their data at position $i$ only if $\vartheta_i =
  \vartheta'_i$. In the following, we denote by $X$ and $Y$ the
  concatenation of the remaining data $X_i$ and $Y_i$, respectively.
  $X$ and $Y$ are sometimes called the \emph{sifted raw key}. \index{sifting}
\item \emph{Error correction:} Alice computes some error correction
  information $C$ depending on $X$ and sends $C$ to Bob.  Bob computes
  a guess $\hat{X}$ for Alice's string $X$, using $C$ and
  $Y$. \index{error correction}
\item \emph{Privacy amplification:} Alice chooses at random a function
  $f$ from a two-universal family of hash functions and announces $f$
  to Bob. Alice and Bob then compute the final key by applying $f$ to
  their strings $X$ and $\hat{X}$, respectively. \index{privacy amplification}
\end{myprotocol}
\caption{General form for {\em one-way} \QKD protocols.}\label{fig:QKDShape}
\end{myfigure} \index{quantum key distribution!one-way}

Note that the quantum channel is only used in the preparation step.
Afterwards, the communication between Alice and Bob
is only classical (over an authentic channel).

As shown in~\cite[Lemma~6.4.1]{Renner05}, the length $\ell$ of the
secret key that can be generated by the protocol described above is
given by\footnote{The approximation in this and the following
  equations holds up to some small additive value which depends
  logarithmically on the desired security $\eps$ of the final key.}
\[
\ell \approx \hminee(\rho_{X \regE} \mid \regE) - \hmax(C)  \, ,
\]
where the cq-state $\rho_{X \regE}$ is the state of the quantum system
with the property that $\regE$ contains all the information Eve has
gained during the preparation step of the protocol and where
$\hmax(C)$ is the number of error correction bits sent from Alice to
Bob. Note that this formula can be seen as a generalization of the
well-known expression by Csisz\'{a}r and K\"{o}rner for classical key
agreement~\cite{CK78}.

Let us now assume that Eve's system $\regE$ can be decomposed into a
classical part $U$ and a purely quantum part $\regE'$. Then, by the
same derivation as in the proof of Corollary~\ref{thm:pasmooth}, we find
\[
  \ell 
\approx 
  \hminee(\rho_{X U \regE'} \mid U \regE') - \hmax(C) 
\geq
  \hiee{X \mid U} - \qhmax(\rho_{\regE'}) - \hmax(C) \ .
\]
As, during the preparation step, Eve does not know the encoding
bases which are chosen at random from the set $\bset$, we can apply our
uncertainty relation (Theorem~\ref{thm:genrel}) to get a lower bound for
the min-entropy of $X$ conditioned on Eve's classical information $\Theta$,
i.e.,
\[
\hiee{X \mid \Theta} \geq M h,
\]
where $M$ denotes the length of the sifted \index{raw key}raw key $X$ and $h$ is the
\index{average entropic uncertainty bound}average entropic uncertainty
bound for $\bset$. \comment{write much more!} Let $q$ be the bound
on the size of Eve's quantum memory $\qhmax(\rho_{\regE'}) \leq q$.
Moreover, let $e$ be the average amount of error correction
information that Alice has to send to Bob per symbol of the sifted raw
key $X$. Then
\[
  \ell 
\gtrapprox 
  M (h-e) - q \ .
\]
Hence, if the memory bound only grows sublinearly in the length $M$ of
the sifted raw key, then the \emph{key rate}, i.e., the number of key
bits generated per bit of the sifted raw key, is lower bounded by
\[
  \mathrm{rate}
\geq
  h-e \ .
\]

\section{The Binary-Channel Setting}
For a binary channel (with a two-dimensional Hilbert space $\cH_2$), the average
amount of error correction information $e$ is given by the binary Shannon
entropy\footnote{This value of $e$ is only achieved if an optimal
  error-correction scheme is used. In practical implementations, the
  value of $e$ might be slightly larger.} $h(p)$, where $p$ is
the bit-flip probability (for classical bits encoded according to some
orthonormal basis as described above). The achievable key rate of a
\QKD protocol using a binary quantum channel is thus given by
\[
  \mathrm{rate}_{\mathrm{binary}} \geq h - h(p) \ .
\]
%(as long as the adversary's memory bound is sublinear in the length of
%the raw key). 
Summing up, we have derived the following theorem.

\begin{theorem}
  Let $\bset$ be a set of orthonormal bases of $\cH_2$ with average
  entropic uncertainty bound $h$. Then, a \emph{one-way} \QKD protocol
  as in Figure~\ref{fig:QKDShape} produces a secure key against
  eavesdroppers whose quantum-memory size is sublinear in the length
  of the raw key (i.e., sublinear in the number of qubits sent from
  Alice to Bob) at a positive rate as long as the bit-flip probability
  $p$ fulfills
\begin{equation} \label{eq:noiselevel}
  h(p) < h \ .
\end{equation}
\end{theorem}

For the BB84 protocol~\cite{BB84}, we have $h = \frac{1}{2}$ (cf.
Inequality~\eqref{eq:maassenuffink}).  Inequality~\eqref{eq:noiselevel} is
thus satisfied as long as $p \leq 11\%$.  This bound coincides with
the known bound for one-way \QKD in the standard model (with an
unbounded eavesdropper). So, using our analysis here, the memory-bound
does not give an advantage.

The situation is different for the six-state protocol where $h =
\frac{2}{3}$. According to~\eqref{eq:noiselevel}, security against
memory-bounded adversaries is guaranteed (i.e. $h(p) < \frac{2}{3}$)
as long as $p \leq 17\%$. If one requires security against an
unbounded adversary, the threshold for the same protocol lies below
$13\%$ as shown by Lo~\cite{Lo01}, and even the best known QKD
protocol on binary channels with one-way classical post-processing can
only tolerate noise up to roughly $14.1\%$~\cite{RGK05}. It has also
been shown that, in the unbounded model, no such protocol can tolerate
an error rate of more than~$16.3\%$.

The performance of \QKD protocols against quantum-memory bounded
eavesdroppers can be improved further by making the choice of the
encoding bases more random. For example, they might be chosen from the
set of all possible orthonormal bases on a two-dimensional Hilbert
space.  As shown in Section~\ref{sec:uncertbound}, the overall average
entropic uncertainty bound is then given by $h \approx 0.72$
and~\eqref{eq:noiselevel} is satisfied if $p \lessapprox 20\%$.  For an
unbounded adversary, the thresholds are the same as for the six-state
protocol (i.e., $14.1\%$ for the best known one-way protocol).

\section{Possible Extensions}
It is an interesting open problem to consider protocols using
higher-dimensional quantum systems. The results described in
Section~\ref{sec:uncertbound} show that for high-dimensional systems,
the average entropic \index{average entropic uncertainty bound}
uncertainty bound converges to its theoretical maximum. The maximal
tolerated channel noise might thus be higher for such protocols
(depending on the noise model for higher-dimensional quantum
channels).

\index{quantum key distribution!one-way}Another interesting problem 
is to derive completely one-way
quantum-key-distribution schemes, i.e.~to eliminate the interactive
\index{sifting}sifting phase from the protocol in Figure~\ref{fig:QKDShape}. The idea
is to let the honest parties use a pre-shared secret key to determine
the bases of the encoding. If a key of size linear in the number of
qubits is used, the scheme has to guarantee that a big portion of the
key can be reused several times in order to yield a reasonable amount
of fresh key. Quantifying the amount of information an eavesdropper can
learn about the pre-shared key by interfering in the preparation step
and eavesdropping on the following classical communication is an open
problem. 

Another approach consists of expanding a pre-shared key of
size only logarithmic in the number of qubits into a pseudo-random linear-size key
to determine the bases of the encoding. It is an open question how to
extend our uncertainty relation from Section~\ref{sec:morerelation}
to the case of only pseudo-random bases.

%% \subsection{Open question}
%% For practical protocols, it might be desirable to use less randomness
%% for the choice of the bases. For example, one might want to choose the
%% bases according to a short initial key. (In this case, no sifting step
%% is needed, which improves the key rate of the protocol.) It is an open
%% question whether our analysis can be applied to such protocols as
%% well.


% \section{Conclusion --- WORK OUT OR REMOVE !!!}
% We have proven a new tight high-order quantum uncertainty relation and
% illustrated its usefulness in proving secure cryptographic two-party
% protocols in the bounded quantum-memory setting. It is a promising open question to investigate how to apply
% our techniques in other settings, for instance to show the security of quantum key distribution
% schemes against coherent attacks solely based on a quantum
% uncertainty relation.

\index{quantum key distribution|)}


%%%%%%%%%%%%%%%%%%%%%%%%%%%%%%%%%%%%%%%%%%%%%%%%%%%%%%%%%%%%%%%%%%
\chapter{Conclusion} \label{chap:conclusions}
%%%%%%%%%%%%%%%%%%%%%%%%%%%%%%%%%%%%%%%%%%%%%%%%%%%%%%%%%%%%%%%%%%

%% In this last chapter, we have a closer look at practical aspects of
%% the models and protocols presented in the last chapters. 


%%%%%%%%%%%%%%%%%%%%%%%%%%%%%%%%%%%%%%%%%%%%%%%%%%%%%%%%%%%%%%%%%%
\section{Towards Practice} \label{sec:techpractice}
%%%%%%%%%%%%%%%%%%%%%%%%%%%%%%%%%%%%%%%%%%%%%%%%%%%%%%%%%%%%%%%%%%
In the following two sections, we elaborate on the question how close
to practice our systems are. First, we argue that imperfections
occurring in practice like \emph{dark counts} and \emph{empty pulses}
are covered by our $(\phi,\eta)$-weak quantum model used in
Sections~\ref{sec:weakass}, \ref{sec:weakmodel12ot}, and~\ref{sec:weakassumptioncomm}. Second,
we sketch how our techniques can be extended to the more realistic
setting of \emph{noisy quantum memory}.

%%%%%%%%%%%%%%%%%%%%%%%%%%%%%%%%%%%%%%%%%%%%%%%%%%%%%%%%%%%%%%%%%%%%%%%%%%%%%
\subsection{More Imperfections} \label{sec:moreimperfect}
%%%%%%%%%%%%%%%%%%%%%%%%%%%%%%%%%%%%%%%%%%%%%%%%%%%%%%%%%%%%%%%%%%%%%%%%%%%%%
A natural approach for implementing two-party protocols like \BBqot,
\Randlqot,\ and \comm\ is to use the polarization of photons governed
by the laws of quantum optics. Such systems are nowadays at the stage
where they can be built in a optical physics lab. \index{laser} Besides the
already modeled bit errors and multi-pulse emissions, more
imperfections of the physical apparatus such as \emph{empty pulses}
and \emph{dark counts} need to be taken into account.
\index{empty pulse} \index{dark count} \index{weak quantum model}

The players have synchronized clocks and in every predefined time
\index{time slot} slot, the sender is supposed to send out a single qubit. In practice,
\index{weak coherent pulse} \index{single-photon source}
weak coherent pulses are used to approximate single-photon sources by
producing in average only a small fraction of one qubit per pulse.
This means that most of the pulses are \emph{empty}, but on the other
hand, there is also a small probability for a multi-qubit pulse. The
\index{multi-qubit emission}receiver reports to the sender in which 
time slots he received pulses.

Empty pulses also occur when the quantum channel lets a transmitted
qubit escape or when it is absorbed. It is realistic that a good
estimate on the rate at which empty pulses are produced (when no
adversary is present) is known, e.g., from the hardware specifications
and by measuring and calibrating the experimental setup. In this case,
the adversary can only take advantage of empty pulses caused by
absorption in the fiber. The best the adversary can do is to
substitute the fiber for one that preserves all qubits sent and to
report empty pulses when a single pulse has been received. The effect
is to increase the rate at which multi-qubit pulses occur. This attack
\index{photon-number-splitting attack} 
is known as \emph{Photon-Number-Splitting attack} as first noted by
Huttner, Imoto, Gisin, and Mor~\cite{HIGM95} and for instance
explained in \cite{BLMS00a,BLMS00b} in the setting of quantum key
distribution. It follows that empty pulses can also be included in the
$(\phi,\eta)$-weak quantum model by an appropriate adjustment of
parameter $\eta$.

Furthermore, thermal fluctuation in the detector hardware might result
in detection even though no qubit was received. This is called a
\emph{dark count}. In this time slot, the receiver will report the
\index{dark count}reception of a qubit and as the outcome is random, it agrees with the
actual bit sent with probability $\frac12$.


Formally, assume that a practical implementation of \BBqot,
\Randlqot,\ or \comm\ takes place in a setting where $\phix$ is the
probability for a bit error caused by the channel, $\phidc$ is the
probability for a dark count in a specific time slot, $\etamq$ is the
probability for a multi-qubit transmission in a non-empty pulse, and
$\etaab$ is the probability for an empty
pulse caused by absorption of a non-empty pulse. %% These parameters are defined under the
%% condition that the source is sending out a signal.
In these terms, dark counts contribute $\frac{\phidc}{2}$ to the
bit-error rate $\phix$. If the adversary is able to get perfect
transmission, she can suppress single-qubit pulses up to a rate of
$\etaab$, thereby increasing the rate $\etamq$ of multi-photon pulses
by $\frac{1}{1-\etaab}$.  It follows that if \BBqot, \comm,\ and
\Randlqot\ 
are secure in the $(\phix+\frac{\phidc}{2},
\frac{\etamq}{1-\etaab})$-weak quantum model, then their
implementation is also secure, provided it is accurately modeled by
these four parameters.  \index{weak quantum model}
\index{empty pulse} \index{dark count}


%% In practice, 

%% quantum transmissions are subject to other
%% imperfections: dark counts and empty pulses. Dark counts occur due to
%% thermal fluctuation in the detector hardware which results
%% in  detection even though no
%% qubit was received. Dark counts contribute to the
%% error-rate (i.e. each dark count accounts for a bit error with
%% probability $\frac{1}{2}$) of the channel. This imperfection can
%% therefore be included in the 
%% $(\phi,\eta)$-weak quantum model by an appropriate choice 
%% of parameter $\phi$ without the need for any further modification.  

%% Empty pulses occur in two cases: when the quantum channel lets a
%% transmitted qubit escape (or when it is absorbed) and when the source
%% did not produce any qubit for a given time slot. The latter is
%% unavoidable for sources using weak coherent pulses as it is the case
%% in most experimental settings. Weak coherent pulses approximate a
%% single-qubit source by producing in average only a small fraction of
%% one qubit per pulse. It means that although most of the pulses are
%% empty, there is a very small probability for a multi-qubit pulse. In
%% this setting, the receiver must report to the sender the positions of all
%% pulses detected.  Assuming the honest sender knows a tight upper bound
%% on the rate at which the source produces empty pulses, the adversary
%% can only take advantage of empty pulses caused by absorption in the
%% fiber. The best the adversary can do is to substitute the fiber for
%% one that preserves all qubits sent and to report empty pulses when a
%% single pulse has been received. The effect is to increase the rate at
%% which multi-qubit pulses occur. This attack is known as the {\em
%%   Photon-Number-Splitting attack}\cite{BLMS00a, BLMS00b, HIGM95} in
%% quantum-key-distribution applications. It follows that empty pulses
%% can also be included in the $(\phi,\eta)$-weak quantum model by an
%% appropriate adjustment of parameter $\eta$.

%% Assume that a practical implementation of \BBqot, \comm,\ or \qot\ takes
%% place in a setting where $\phi_{\mbox{\scriptsize{\sc x}}}$ is the
%% probability for a bit error caused by the channel,
%% $\phi_{\mbox{\scriptsize{\sc dc}}}$ is the probability for a dark
%% count, $\eta_{\mbox{\scriptsize{\sc mq}}}$ is the probability for a
%% multi-qubit transmission, and $\eta_{\mbox{\scriptsize{\sc ab}}}$ is
%% the probability for an empty pulse caused by absorption. These
%% parameters are defined under the condition that the source is sending
%% out a signal. It follows that if \BBqot, \comm,\ and \qot\ are secure
%% in the $(\phi_{\mbox{\scriptsize{\sc
%%       x}}}+\frac{\phi_{\mbox{\scriptsize{\sc dc}}}}{2},
%% \frac{\eta_{\mbox{\scriptsize{\sc mq}}}}{1-\eta_{\mbox{\scriptsize{\sc
%%         ab}}}})$-weak quantum model, then their implementation is also
%% secure provided it is accurately modeled by these four parameters.

Likewise, a variety of imperfections specific to particular
implementations may be adapted to the weak quantum model.
 
%%%%%%%%%%%%%%%%%%%%%%%%%%%%%%%%%%%%%%%%%%%%%%%%%%%%%%%%%%%%%%%%%%
\subsection{Generalizing the Memory Model} \label{sec:noisymem}
%%%%%%%%%%%%%%%%%%%%%%%%%%%%%%%%%%%%%%%%%%%%%%%%%%%%%%%%%%%%%%%%%%
The \index{bounded-quantum-storage model}bounded-quantum-storage model
 limits the number of physical qubits
the adversary's memory can contain. A more realistic model would
rather address the noise process the adversary's memory undergoes.
For instance, it is not hard to build a very large, but unreliable
memory device containing a large number of qubits. It is reasonable to
expect that our protocols remain secure also in a scenario where the
adversary's memory is of arbitrary size, but where some quantum
operation (modeling noise) applies to it. If we do not substitute
$\qhmax(\rho_{\regE})$ with the number of qubits $q$ in
Term~(\ref{eq:lemma}) in the privacy-amplification
Section~\ref{sec:pa}, then our constructions can cope with slightly
more general memory models. In particular, all our protocols that are
secure against adversaries with memory of no more than $\gamma n$
qubits are also secure against any noise model that reduces the rank
$\qhmax(\rho_{\regE})$ of the mixed state $\rho_{\regE}$ held by the
adversary to at most $2^{\gamma n}$. \index{noisy-memory model}
%Let $\regE$ be the the random state received by the adversary. 

An example of a noise process resulting in a reduction of $\qhmax(\rho_{\regE})$
is an \index{erasure channel}erasure channel. Assuming the $n$ initial qubits are each erased
with probability larger than $1-\gamma$ when the memory bound applies,
it holds except with negligible probability in $n$ that
$\qhmax(\rho_{\regE})<\gamma n$.  The same applies if the noise process is
modeled by a \index{depolarizing channel}depolarizing channel with error probability
$p=1-\gamma$. Such a depolarizing channel replaces each qubit by a
random one with probability $p$ and does nothing with probability
$1-p$.

The technique we have developed does not allow to deal with depolarizing channels
with $p < 1-\gamma$ although one would expect that some $0< p < 1-\gamma$
should be sufficient to ensure security against such adversaries.
The reason being that not knowing the positions where the errors occurred
should make it more difficult for the adversary than when the noise process
is modeled by an erasure channel. However, it seems that our uncertainty 
relations 
%(i.e. Theorems~\ref{thm:hadamard} and \ref{thm:mub}) 
are not strong enough to address this case. Generalizing the
bounded-quantum-storage model to more realistic \index{noisy-memory model}noisy-memory models is
an interesting open question.


%%%%%%%%%%%%%%%%%%%%%%%%%%%%%%%%%%%%%%%%%%%%%%%%%%%%%%%%%%%%%%%%%%%%%%%%%%%%%
\section{Conclusion} \label{sec:conclusion}
%%%%%%%%%%%%%%%%%%%%%%%%%%%%%%%%%%%%%%%%%%%%%%%%%%%%%%%%%%%%%%%%%%%%%%%%%%%%%
The \index{bounded-quantum-storage model}bounded-quantum-storage model
presented in this thesis is an attractive model, in both the
theoretical and practical sense. On the theoretical side, it allows
for very simple protocols implementing basic two-party primitives such
as oblivious transfer and bit commitment. New high-order entropic
\index{uncertainty relation}uncertainty relations
 have been established in order to show the
security with the help of techniques such as purification and privacy
amplification by two-universal hashing. These uncertainty
relations can also be applied in different settings like quantum key
distribution.

On the practical side, the protocols do not require any quantum memory
for honest players and remain secure provided the adversary has a
quantum memory of size bounded by a constant fraction of all
transmitted qubits.  Such a gap between the amount of storage required
for honest players and adversaries is not achievable by classical
means.  The protocols can be adapted to tolerate various kinds of
errors and in fact, they can be implemented with today's technology. A
collaboration of people from the computer science and physics
departments of the University of Aarhus is currently working on the
implementation of these protocols\footnote{See
\texttt{http://www.brics.dk/{\textasciitilde}salvail/qusep.html}
for further information on the QUSEP project.}.

In summary, one can say that the bounded-quantum-storage model has
passed its first tests by proving its power (the possibility of
oblivious transfer) and by inspiring beautiful theoretical results
(quantum uncertainty relations). It is a good sign that the protocols
for the basic primitives are simple in structure. In principle, enough
instances of these protocols could be used to implement more involved
cryptographic tasks like secure identification, which reduces
essentially to securely checking whether two inputs are equal (without
revealing more than this mere bit of information). However, it is a
natural next step to find more efficient, direct protocols for those
tasks, secure in the bounded-quantum-storage model. Such a direct
approach gives a better ratio between storage-bound and
communication-complexity and is the topic of a recent paper
\cite{DFSS07}.

A major open problem is the optimality of the bounds on the
adversary's quantum memory. The bit-commitment protocol \comm\ for
instance appears to be secure against any adversary with memory less
than $n$ qubits, but our analysis requires the memory to be smaller
than $n/2$ (or $n/4$ for strong binding).  Also, finding protocols
secure against adversaries in more general \index{noisy-memory model}
noisy-memory models, as discussed in the last
Section~\ref{sec:noisymem}, would certainly be a natural and
interesting extension of this work to more practical settings~
\cite{DSTW07privcom}. Furthermore, there is still a lack of simple and
intuitive security definitions for primitives like \OT etc.\ with
rigorous composability results (like universal composability) in the
quantum setting. Very recent results in this direction have been
established in~\cite{WW07privcom}.


%%% Local Variables: 
%%% mode: latex
%%% TeX-master: "diss"
%%% End: 
