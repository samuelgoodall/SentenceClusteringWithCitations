% This is LLNCS.DEM the demonstration file of
% the LaTeX macro package from Springer-Verlag
% for Lecture Notes in Computer Science,
% version 2.4 for LaTeX2e as of 16. April 2010
%
\documentclass[oneside,a4paper,10pt]{llncs}

\usepackage[english]{babel}
\usepackage{amssymb}
\usepackage{graphicx}
\usepackage{units}
\usepackage[usenames,dvipsnames]{xcolor}
\usepackage{fixltx2e}
\usepackage{booktabs}
\usepackage{setspace}
\usepackage{listings}
\usepackage{footmisc}
\usepackage{wrapfig}
\usepackage{paralist}
\usepackage{caption}
\usepackage{subcaption}
%\usepackage{mdwlist}
%\usepackage{natbib}
%\usepackage[fixlanguage]{babelbib}
\usepackage{url}
%\usepackage{showframe}
\usepackage{color}

%%%%%%%%%%%%%%%%%%%%%%%%%%%%%%%%%%%%%%%%%%%%%%%%%%
%% Color definitions (acc. to style guide 2008)
%%%%%%%%%%%%%%%%%%%%%%%%%%%%%%%%%%%%%%%%%%%%%%%%%%
\definecolor{TUMblue}{RGB}   {  0, 101, 189} % "Pantone300"
%% additional colors (*not* in logo!!!)
\definecolor{Pantone540}{RGB}{  0,  51,  89}
\definecolor{Pantone301}{RGB}{  0,  82, 147}
\definecolor{Pantone285}{RGB}{  0, 115, 207}
\definecolor{Pantone542}{RGB}{100, 160, 200}
\definecolor{Pantone283}{RGB}{152, 198, 234}
\definecolor{TUMdarkgray}{RGB}{ 88, 88, 90}
\definecolor{TUMgray}{RGB}{ 156, 157, 159}
\definecolor{TUMlightgray}{RGB}{ 217, 218, 219}
%% emph colors (use rarely!)
\definecolor{TUMgreen}{RGB}{162, 173, 0} % Pantone383
\definecolor{TUMorange}{RGB}{227, 114, 34} % Pantone158
\definecolor{TUMelfenbein}{RGB}{218, 215, 203} %Pantone7527
%% for presentations only: emph colors
\definecolor{TUMpyellow}{RGB}{255, 180, 0}
\definecolor{TUMporange}{RGB}{255, 128, 0}
\definecolor{TUMpred}{RGB}{229, 52, 24}
\definecolor{TUMpdarkred}{RGB}{202, 33, 63}
\definecolor{TUMpblue}{RGB}{0, 153, 255}
\definecolor{TUMplightblue}{RGB}{65, 190, 255}
\definecolor{TUMpgreen}{RGB}{145, 172, 107}
\definecolor{TUMplightgreen}{RGB}{181, 202, 130}
% this is similar to style guide
%\definecolor{TUMpbluethemeupper}{RGB}{0, 82, 147} %=Pantone301
%\definecolor{TUMpbluethemelower}{RGB}{  0, 115, 207}
% this is in powerpoint POT, but looks ugly
\definecolor{TUMpbluethemelower}{RGB}{0, 82, 147} %=Pantone301
\definecolor{TUMpbluethemeupper}{RGB}{0, 40, 72}

%%%%%%%%%%%%%%%%%%%%%%%%%%%%%%%%%%%%%%%%%%%%%%%%%%


%%%%%%%%%%%%%%%%%%%%%%%%%
% SUBTLE SQUEEZE MARGINS
%%%%%%%%%%%%%%%%%%%%%%%%%
% original Springer margins: 
\usepackage[a4paper,top=1.62in,bottom=2.5in,left=1.87in,right=1.59in]{geometry}
% slightly squeezed margins:
%\usepackage[a4paper,asymmetric,top=1.62in,bottom=1.8in,left=1.57in,right=1.39in]{geometry}

%% Springer's llncs is incompatible to titlesec, but we can fix it
% Save the class definition of \subparagraph
\let\llncssubparagraph\subparagraph
% Provide a definition to \subparagraph to keep titlesec happy
\let\subparagraph\paragraph
\usepackage{titlesec}
\usepackage{chngcntr}
\usepackage{hyperref}
\usepackage{siunitx}
\usepackage{tikz}
\usetikzlibrary{shapes,arrows,shadows,fit}

%%%%%%%%%%%%%%%%%%%%%%%%%%%%%%%%%%%%%%%%%%%%%%%%%%
%    BEGIN SQUEEZE MAGIC
%%%%%%%%%%%%%%%%%%%%%%%%%%%%%%%%%%%%%%%%%%%%%%%%%%
\renewcommand\footnotelayout{\small}
%\addtolength{\textfloatsep}{-5mm}
%\addtolength{\itemsep}{-0.05in}
%\renewcommand{\topfraction}{0.85}
% %\renewcommand{\textfraction}{0.1}
% \renewcommand{\floatpagefraction}{0.75}
\renewcommand{\baselinestretch}{.98} % hands away until right before submission, if necessary!
% \setlength{\jot}{0pt} % tighter math
% \AtBeginDocument{%
%   \setlength\abovedisplayskip{6pt}
%   \setlength\belowdisplayskip{6pt}}
% \renewcommand{\bibsep}{.0em}
%\titlespacing*{\section}{0pt}{2ex}{1ex}
%\titlespacing*{\subsection}{0pt}{2ex}{1ex}
%\titlespacing*{\subsubsection}{0pt}{0.5ex}{0ex}
%\parskip 0pt
%%%%%%%%%%%%%%%%%%%%%%%%%%%%%%%%%%%%%%%%%%%%%%%%%%
%    END SQUEEZE MAGIC
%%%%%%%%%%%%%%%%%%%%%%%%%%%%%%%%%%%%%%%%%%%%%%%%%%

\definecolor{shadecolor}{gray}{.95}                                                           %farbe für listings


% listings default format
\lstset{captionpos=t, 
  xleftmargin=.6cm,
  basicstyle=\ttfamily\scriptsize,%
  language=Ada,
  commentstyle=\color{TUMdarkgray},
  keywordstyle=[1]\bfseries\color{Pantone301},
  stringstyle=\color{black},%\color{TUMpred},
  morekeywords={Dim_Type,Angle_Type,Time_Type,Angular_Velocity_Type,Lat_Type},
  identifierstyle=\color{black},
  extendedchars=true,%
  numbers=left,%
  numbersep=5pt,%
  numberstyle={\tiny\color{TUMdarkgray}},%
 % backgroundcolor=\color{shadecolor},%
  stepnumber=1, % Z.nr in 5er schritten
  numberfirstline=true,
  breaklines=true, % Zeilenumbruch
  breakautoindent=true, % Bei Zeilenumbruch einrücken
  tabsize=2, % Breite eines Tabulators
  postbreak=\space,
  showspaces=false, % Keine Leerzeichensymbole
  showtabs=false, % Keine Tabsymbole
  showstringspaces=false,% Leerzeichen in Strings         
  rulecolor=\color{TUMdarkgray},
  frame=l}

%\renewcommand*{\lstlistingname}{Code \arabic{lstlisting} de \textsl{Mathematica}}


% handy todo
\providecommand{\todo}[1]{\marginpar{\color{red}{\textbf{ToDo!}}}\textcolor{red}{\textbf{ToDo: #1}}}

%
%
\begin{document}

%%%%%%%%%%%%%%%%%%%%%%%%%%%%%%%%%
% EDITOR ONLY:
%%%%%%%%%%%%%%%%%%%%%%%%%%%%%%%%%


% \chapter*{Preface}
% %
% This textbook is intended for use by students of physics, physical
% chemistry, and theoretical chemistry. The reader is presumed to have a
% basic knowledge of atomic and quantum physics at the level provided, for
% example, by the first few chapters in our book {\it The Physics of Atoms
% and Quanta}. The student of physics will find here material which should
% be included in the basic education of every physicist. This book should
% furthermore allow students to acquire an appreciation of the breadth and
% variety within the field of molecular physics and its future as a
% fascinating area of research.

% For the student of chemistry, the concepts introduced in this book will
% provide a theoretical framework for that entire field of study. With the
% help of these concepts, it is at least in principle possible to reduce
% the enormous body of empirical chemical knowledge to a few basic
% principles: those of quantum mechanics. In addition, modern physical
% methods whose fundamentals are introduced here are becoming increasingly
% important in chemistry and now represent indispensable tools for the
% chemist. As examples, we might mention the structural analysis of
% complex organic compounds, spectroscopic investigation of very rapid
% reaction processes or, as a practical application, the remote detection
% of pollutants in the air.

% \vspace{1cm}
% \begin{flushright}\noindent
% April 1995\hfill Walter Olthoff\\
% Program Chair\\
% ECOOP'95
% \end{flushright}
% %
% \chapter*{Organization}
% ECOOP'95 is organized by the department of Computer Science, Univeristy
% of \AA rhus and AITO (association Internationa pour les Technologie
% Object) in cooperation with ACM/SIGPLAN.
% %
% \section*{Executive Commitee}
% \begin{tabular}{@{}p{5cm}@{}p{7.2cm}@{}}
% Conference Chair:&Ole Lehrmann Madsen (\AA rhus University, DK)\\
% Program Chair:   &Walter Olthoff (DFKI GmbH, Germany)\\
% Organizing Chair:&J\o rgen Lindskov Knudsen (\AA rhus University, DK)\\
% Tutorials:&Birger M\o ller-Pedersen\hfil\break
% (Norwegian Computing Center, Norway)\\
% Workshops:&Eric Jul (University of Kopenhagen, Denmark)\\
% Panels:&Boris Magnusson (Lund University, Sweden)\\
% Exhibition:&Elmer Sandvad (\AA rhus University, DK)\\
% Demonstrations:&Kurt N\o rdmark (\AA rhus University, DK)
% \end{tabular}
% %
% \section*{Program Commitee}
% \begin{tabular}{@{}p{5cm}@{}p{7.2cm}@{}}
% Conference Chair:&Ole Lehrmann Madsen (\AA rhus University, DK)\\
% Program Chair:   &Walter Olthoff (DFKI GmbH, Germany)\\
% Organizing Chair:&J\o rgen Lindskov Knudsen (\AA rhus University, DK)\\
% Tutorials:&Birger M\o ller-Pedersen\hfil\break
% (Norwegian Computing Center, Norway)\\
% Workshops:&Eric Jul (University of Kopenhagen, Denmark)\\
% Panels:&Boris Magnusson (Lund University, Sweden)\\
% Exhibition:&Elmer Sandvad (\AA rhus University, DK)\\
% Demonstrations:&Kurt N\o rdmark (\AA rhus University, DK)
% \end{tabular}
% %
% \begin{multicols}{3}[\section*{Referees}]
% V.~Andreev\\
% B\"arwolff\\
% E.~Barrelet\\
% H.P.~Beck\\
% G.~Bernardi\\
% E.~Binder\\
% P.C.~Bosetti\\
% Braunschweig\\
% F.W.~B\"usser\\
% T.~Carli\\
% A.B.~Clegg\\
% G.~Cozzika\\
% S.~Dagoret\\
% Del~Buono\\
% P.~Dingus\\
% H.~Duhm\\
% J.~Ebert\\
% S.~Eichenberger\\
% R.J.~Ellison\\
% Feltesse\\
% W.~Flauger\\
% A.~Fomenko\\
% G.~Franke\\
% J.~Garvey\\
% M.~Gennis\\
% L.~Goerlich\\
% P.~Goritchev\\
% H.~Greif\\
% E.M.~Hanlon\\
% R.~Haydar\\
% R.C.W.~Henderso\\
% P.~Hill\\
% H.~Hufnagel\\
% A.~Jacholkowska\\
% Johannsen\\
% S.~Kasarian\\
% I.R.~Kenyon\\
% C.~Kleinwort\\
% T.~K\"ohler\\
% S.D.~Kolya\\
% P.~Kostka\\
% U.~Kr\"uger\\
% J.~Kurzh\"ofer\\
% M.P.J.~Landon\\
% A.~Lebedev\\
% Ch.~Ley\\
% F.~Linsel\\
% H.~Lohmand\\
% Martin\\
% S.~Masson\\
% K.~Meier\\
% C.A.~Meyer\\
% S.~Mikocki\\
% J.V.~Morris\\
% B.~Naroska\\
% Nguyen\\
% U.~Obrock\\
% G.D.~Patel\\
% Ch.~Pichler\\
% S.~Prell\\
% F.~Raupach\\
% V.~Riech\\
% P.~Robmann\\
% N.~Sahlmann\\
% P.~Schleper\\
% Sch\"oning\\
% B.~Schwab\\
% A.~Semenov\\
% G.~Siegmon\\
% J.R.~Smith\\
% M.~Steenbock\\
% U.~Straumann\\
% C.~Thiebaux\\
% P.~Van~Esch\\
% from Yerevan Ph\\
% L.R.~West\\
% G.-G.~Winter\\
% T.P.~Yiou\\
% M.~Zimmer\end{multicols}
% %
% \section*{Sponsoring Institutions}
% %
% Bernauer-Budiman Inc., Reading, Mass.\\
% The Hofmann-International Company, San Louis Obispo, Cal.\\
% Kramer Industries, Heidelberg, Germany
% %
% \tableofcontents
% %
% \mainmatter              % start of the contributions
% %

%%%%%%%%%%%%%%%%%%%%%%%%%%%%%%%%%
% TITLE, AUTHORS ETC OF PAPERS
%%%%%%%%%%%%%%%%%%%%%%%%%%%%%%%%%

\title{Development and Verification of a Flight Stack\\for a High-Altitude Glider in Ada/SPARK~2014\thanks{The source code for this project is available at \url{github.com/tum-ei-rcs/StratoX}. This is a pre-print. The final publication will be available at Springer via \url{http://dx.doi.org/TODO} in \emph{Computer Safety, Reliability and Security, 36th International Conference SAFECOMP~2017}, S. Tonetta and E. Schoitsch and F. Bitsch (Eds.), Trento, Italy, 2017.}}
%
\titlerunning{A SPARK~2014 Flight Stack for a High-Altitude Glider}  % abbreviated title (for running head)
%                                     also used for the TOC unless
%                                     \toctitle is used
%
\author{\vspace*{-1em}Martin Becker \and Emanuel Regnath \and Samarjit Chakraborty}
%
\authorrunning{Becker et al.} % abbreviated author list (for running head)
%
%%%% list of authors for the TOC (use if author list has to be modified)
\tocauthor{Martin Becker, Emanuel Regnath, and Samarjit Chakraborty}
%
\institute{Chair of Real-Time Computer Systems, Technical University of Munich\\80333 Munich, Germany\\
%\email{\{becker,regnath\}@rcs.ei.tum.de}
}

\maketitle              % typeset the title of the contribution

\begin{abstract}\vspace*{-1em}
%The abstract should summarize the contents of the paper
%using at least 70 and at most 150 words. It will be set in 9-point
%font size and be inset 1.0 cm from the right and left margins.
%There will be two blank lines before and after the Abstract. \dots
%
% explore capabilities and weaknesses of latest technology in software verification
% co-verification to influence design (and not be stuck in the end)
% we derive a set of rules
% we give design recommendations
  SPARK~2014 is a modern programming language and a new state-of-the-art
  tool set for development and verification of high-integrity
  software. In this paper, we explore the capabilities and limitations
  of its latest version in the context of building a flight stack for
  a high-altitude unmanned glider. Towards that, we deliberately
  applied static analysis early and continuously during
  implementation, to give verification the possibility to steer
  the software design. 
  %
  In this process we have identified several limitations and pitfalls
  of software design and verification in SPARK, for which we give
  workarounds and protective actions to avoid them. Finally, we give
  design recommendations that have proven effective for verification, and
  summarize our experiences with this new language.
  % perfect length of abstract
\keywords{Ada/SPARK, formal verification, limitations, rules}
\end{abstract}
%
\section{Introduction}
\label{sec:intro}

% Wording:
% - VC vs. check: UNCLEAR
%   - VC: what is to be proved. overflow check, etc.
%   - check: evaluation of a property during run-time. AdaCore uses it as synonym for 
% - ...

% \begin{itemize}
% \item here is a problem, make it specific and not too generic
%   \begin{itemize}
%   \item show that it is interesting
%   \item say that it is unsolved
%   \end{itemize}
% \item state contribution, clearly.
%   \begin{itemize}
%   \item make contributions refutable (to keep it interesting)
%   \item briefly explain idea
%   \end{itemize}
% \end{itemize}

% How do we start? No blabla, we need to get there straight away because no space.
% Formal verification of software has the reputation of being
% time-consuming, complicated and having limitations, especially in
% terms of scalability.  In our earlier work we concluded that a small
% program of a microcontroller can be fully analyzed with Model
% Checking, but only with a lot of additional work and if the software
% has a specific structure.~\cite{Becker2015}. In this paper use
% state-of-the-art deductive verification tools to verify a much more
% complex software system, and we show what the manual effort is
% comparably low, if some rules are followed.

The system under consideration is a novel kind of weather balloon
which is actively controlled, and thus requires verification to ensure
it is working properly in public airspace.  As any normal weather
balloon, the system climbs up to the stratosphere (beyond an altitude of $\SI{10}{\kilo\meter}$),
while logging weather data such as temperature, pressure,
NO\textsubscript{2}-levels and so on. Eventually the balloon bursts,
and the sensors would be falling back to the ground with a parachute,
drifting away with prevailing wind conditions. However, our system is
different from this point onwards: the sensors are placed in a
light-weight glider aircraft which is attached to the balloon. At a
defined target altitude, the glider separates itself from the balloon,
stabilizes its attitude and performs a controlled descent back to the
take-off location, thus, bringing the sensors back home. In this paper
we focus on the development and formal verification of the glider's
onboard software.

% \begin{figure}[htbp]
%   \centering
%   \includegraphics[width=\textwidth]{fig/flight-phases}
%   \caption{Mission Profile of high-altitude glider.}
%   \label{fig:mission}
%  \end{figure}

% this is obvious from the earlier description:
%   \begin{enumerate}
%   \item Detecting take-off and memorizing location
%   \item Monitoring of ascend, actively unhitch at predefined
%     target altitude
%   \item Stabilize glider's attitude in the prevailing wind conditions
%   \item Navigate back to take-off location
%   \item Logging of weather data during the entire mission
%   \end{enumerate}
% requirements

 % motivate the criticality: TODO: reduce
The requirements for such a system are challenging already because of the
extreme environmental conditions; temperatures range from $\SI{+30}{\celsius}$
down to $\SI{-50}{\celsius}$, winds may exceed $\SI{100}{kph}$, and GPS devices may
yield vastly different output in those altitudes due to decreasing precision and the
wind conditions. The combination of
those extreme values is likely to trigger corner cases % here its "corner case"
in the software, and thus should be covered by means of extensive testing or
by analysis. 
 % Problems due to extreme conditions are aggravated by the complexity
 % of such a system: Since we have to stabilize and control the attitude
 % of the glider, this means implementing a full flight stack including
 % complex algorithms for pose estimation (i.a., a Kalman Filter), but
 % also logging of weather data to an SD card using a file system. This
 % is a complex piece of multi-threaded software, which under all these
 % circumstances must work properly. Clearly, such a high-altitude flyer
 % has many ways to fail, yet, being technically a drone in public
 % airspace, we need to keep the vehicle in a controlled flight even in
 % the presence of some hardware failures that may occur.

 We use this opportunity of a safety-critical, yet hardly testable
 system to explore the new state-of-the art verification tools of
 Ada/SPARK~2014~\cite{GNATprove}, especially to identify limitations,
 pitfalls and applicability in practice.  To
 experience this new SPARK release to its full extent, we applied a
 \emph{co-verification} approach. That is, we did not perform
 verification on a finished product, but instead in parallel to the
 software development (the specific strategy is not of relevance for this paper, but only the effect that this enabled us to identify 
 code features that pose challenges in verification, and find workarounds for them). The implementation could therefore be shaped by
 verification needs. Moreover, since the high-altitude glider was a
 research project, we allowed ourselves to modify the initial
 software design to ease verification when needed.

% Our contributions are as follows:
% \begin{enumerate}
% \item We give a brief introduction to the SPARK~2014 language in Section~\ref{sec:verification-spark},
% \item we evaluate the application of a dimensionality checking system
%   and object-oriented features in verification,
% \item we explain verification and design problems and workarounds in
%   Section~\ref{sec:workarounds},
% \item we propose a daily workflow for co-verification in Section~\ref{sec:verification} and
% \item we generalize the experienced problems and propose a set of
%   verification rules in Section~\ref{sec:results}.
% % there is not enough work/evidence to claim that we explored OOP capabilities
% \end{enumerate}

% NOT: "the rest of this paper is organized as follows"

%%% Local Variables: ***
%%% mode:latex ***
%%% TeX-master: "paper.tex"  ***
%%% End: ***

\section{Verification in Ada/SPARK}
\label{sec:verification-spark}

% \subsection{Introduction to SPARK}
% SPARK is a formally defined, imperative and strongly typed programming
% language, founded specifically for use in high-integrity and
% safety-critical applications and to leverage state-of-the-art
% verification tools. SPARK~2014 is the latest release, which is a
% complete redesign that adopts wide parts of Ada 2012, in particular a
% deterministic subset thereof.  As a result, an Ada compiler can build
% an executable from SPARK source code, and moreover, it also becomes
% possible to write software which mixes both languages (though, there
% are rules what kind of mixing is allowed). Since only the SPARK subset
% is intended for formal verification, this means that a safety-critical
% software can still make use of Ada features, if necessary, but this
% should be minimized because those parts cannot be verified.  At the
% present moment, there exists only one implementation of SPARK~2014,
% maintained by AdaCore and Altran~\cite{GNATprove}.

% We now briefly explain language features relevant for this paper. For
% a more comprehensive overview we refer the reader to~\cite{SPARKUG}
% and further~\cite{Trojanek2014}.

% \textbf{Restrictions. } There are only few Ada features that are
% restricted in the SPARK subset and worth mentioning in practice. Once
% of them are pointers (called \emph{access} in Ada), which would make
% formal verification hard and oftentimes impossible, as well known from
% other verification tools. In extension to this, SPARK 2014 also
% forbids the use of aliasing and allocators. Most of the remaining
% restrictions (such as no \texttt{goto} statements, or expressions and
% functions must be free of side effects) should be familiar to
% developers concerned with safety-critical software, and can be largely
% avoided by appropriate design and coding guidelines. However, the most
% severe restriction is exclusion of exception handling.
% %A SPARK program must only have one
% %global handler that catches all exceptions, and subsequently
% %terminates the program. 
% As a consequence, SPARK programs first and foremost must be shown to
% be free of run-time exceptions (called \emph{AoRTE} - absence of
% run-time exceptions), which is going to be the main verification task.


SPARK~2014 is a major redesign of the original SPARK language, which was intended for formal verification. SPARK~2014 now adopts Ada~2012 syntax, and covers a large subset of Ada. As a result, the GNAT Ada compiler can build
an executable from SPARK~2014 source code, and even compile a program which mixes both languages. Compared to
Ada, the most important exclusions are
pointers (called \emph{access}), aliasing and allocators, as well as a
ban of exception handling.
As a consequence, SPARK programs first and foremost must be shown to
be free of run-time exceptions (called \emph{AoRTE} - absence of
run-time errors), which constitutes the main verification task.

The SPARK language -- for the rest of this paper we refer to SPARK~2014 simply as SPARK -- is built
on functional contracts and data flow contracts. Subprograms (procedures and
functions) can be annotated with pre- and postconditions, as well as with
data dependencies. GNATprove, the (only) static analyzer for SPARK~2014, aims to prove subprograms in a
modular way, by analyzing each of them individually.
The effects of callees are summarized by their
post-condition when the calling subprogram is analyzed, and the
precondition of the callee is imposing a proof obligation on the
caller, i.e., the need to verify that the caller respects the callee's
precondition. Further proof obligations arise from each
language-defined check that is executed on the target, such as
overflow checks, index checks, and so on. If all proofs are successful, 
then the program is working according to its contracts 
and no exceptions will be raised during execution, i.e., AoRTE is established. % no parentheses at end of sentence


Internally, GNATprove~\cite{GNATprove} builds on the
Why3 platform~\cite{why3}, which performs deductive verification on
the proof obligations to generate verification conditions (VCs), and
then passes them to a theorem solver of user's choice, e.g., cvc4, alt-ergo or z3. Note that there exists also a tool for abstract interpretation, which is, however, not discussed here.

%mentioned already. not necessary:
% \textbf{Multi-Threading.} Since its 2016 update, the SPARK tools support
% the analysis of a deterministic subset of Ada's tasking facilities,
% known as the \emph{Ravenscar Profile}~\cite{Ravenscar}. With this,
% SPARK programs have become significantly more flexible and powerful
% for their multi-thread support. The analyzer is able to understand the
% data flow of multi-threaded applications, and point out possible race
% conditions.

% \textbf{State Abstraction. FIXME: do we need this? }The last feature that we mention here
% is \emph{state abstraction} for packages. Packages with internal
% states may hide these to the outside world, but for analysis is is
% often required to express properties over the states from the outside
% view. For example, whether a package is being initialized or not can
% depend on a number of internal variables, but for the user of the
% package who expects a usage contract from his point of view, this is
% only an atomic proposition. In such cases, internal states can be
% summarized and represented by an abstract state. This enables to
% provide an outside-view specification of a package's behavior, and yet
% perform the proofs on the internals of the package, which may be one
% of many implementation choices.

% \subsection{Deductive Verification Backend}
% Internally, the SPARK tools build on the Why3 platform~\cite{why3},
% which implements deductive verification by means of Weakest
% Precondition (WP) calculus and passing the goal (precondition implies
% WP) to a theorem solver of user's choice (e.g., cvc4, z3,
% mathsat). Therefore, SPARK programs are first translated to WhyML,
% where further axioms are available for lists, arithmetic operations
% etc. Then, WP calculus is used to generate verification conditions
% (VC), which subsequently are translated to the input language of the
% selected prover, and hopefully discharged. Since provers have their
% particular strength and weaknesses, Why3 (and therefore the SPARK
% tools) allows to invoke as many solvers as needed to prove a
% property. Consequently, each proof obligation in the SPARK program may
% have multiple prover runs associated
% % \footnote{In fact, complex proofs
% %  could also be decomposed into smaller sub-proofs, and cause even
% %  more runs.}. 
% Some may run out of memory, others may be unable to
% decide, and again others might produce a counterexample or discharge
% the VC.
% % we don't talk about the back-mapping here, that gnat2why performs. This is for another paper.

% % \begin{itemize}
% % \item 2-valued vs 3-valued logic~\cite{Chalin2005}
% % \end{itemize}

\subsection{The GNAT Dimensionality Checking System}
We also want to introduce a feature that is not part of the SPARK
language itself, but an implementation-defined extension of the GNAT compiler, and thus available for SPARK programs. 
% Facts:
% Dimension_System is an implementation defined aspect -> unique for GNAT     [GNAT_RM]
% Implementation defined aspects possible since Ada2012
% Aspect Dimension_System allows to specify up to 7 dimensions on a numeric derived type
% GNAT provides definition of the International MKS system in the runtime package System.Dim.Mks.   [GNAT_RM]
% Aspect Dimesion specifies the dimensionsof a \emph{subtype} of a dimensioned numeric type.    [GNAT_RM]
% "when the dimensioned type is an integer type, then any dimension value must be an integer literal"  [GNAT_RM]
%\begin{itemize}
%\item GNAT's system, what does it do?
%\item What use it is for the user?
%\item What must the user do to use it?
%\end{itemize}
%
%references: \cite{Polderman2013,Grein2014,Grein2003,Schonberg:2012:ISD:2402676.2402692}
%
% One Sentence Motivation
%Checking the dimensions of physical calculations is an established practice in scientific disciplines but mostly ignored in software implementations.
% History?
%Previous attempts to implement a dimension system involved significant overhead either by operator overloading or due to additional run-time checks~\cite{Grein2003}.
% Features
Since Ada 2012, the GNAT compiler offers a dimension system for numeric types through implementation-defined aspects~\cite{Schonberg:2012:ISD:2402676.2402692}. 
The dimension system can consist of up to seven base dimensions, % <-- oxford comma 
and physical quantities are declared as subtypes, annotated with the exponents of each dimension.
Expressions using such variables are statically analyzed by the compiler for their dimensional consistency.
Furthermore, the dimensioned variables contribute to readability and documentation of the code. Inconsistencies such as the following are found (dividend and divisor are switched in the calculation of rate):

\vspace{-1mm}\begin{lstlisting}[name=units]
angle : Angle_Type := 20.0 * Degree;         
dt    : Time_Type  := 100.0 * Milli * Second;
rate  : Angular_Velocity_Type := dt / angle; -- compiler error
\end{lstlisting}\vspace{-1mm}
Note that scaling prefixes like \lstinline$Milli$ can be used, and
that common conversions, such as between Degree and Radian in line 1, can be governed in a similar way.

%
% unfortunately too much detail:
%To use the dimensionality checking system, the aspect \texttt{Dimension\_System} must be applied on a numeric derived type.
%This aspect allows to specify the dimensions and the name and the symbol of each corresponding unit.
%Afterwards, subtypes of the dimensioned type are declared with the aspect \texttt{Dimension} to assign the exponent of each dimension.
%To reduce specification effort, the GNAT runtime also provides predefined types of the International MKS system in the package \texttt{System.Dim.%Mks}.

In our project, we specified a unit system with the dimensions \emph{length}, \emph{mass}, \emph{time}, \emph{temperature}, \emph{current}, and \emph{angle}. 
Adding angle as dimension provides better protection against assignments of dimensionless types, as proposed in~\cite{Xiang2015}.




% Maybe just "verify dimensional consistency of physical computations".
% How to setup
% make it more explicit that you now talk about usage: "To make use of the dimensionality checking system, ..."

%UNDO: The implementation defined aspect \texttt{Dimension\_System} can be applied on a numeric derived type and allows to specify up to 7 dimensions of that type. Subtypes of the dimensioned type define possible quantities by assigning the exponent of each dimension with the aspect \texttt{Dimension}.


%%% Local Variables: ***
%%% mode:latex ***
%%% TeX-master: "paper.tex"  ***
%%% End: ***
\section{Initial System Design \& Verification Goals}
\label{sec:context}

\textbf{Target Hardware.} We have chosen the ``Pixhawk''
autopilot~\cite{pixhawk2011}. It comprises two ARM processors; one
Cortex-M4F (STM32F427) acting as flight control computer, and one
Cortex-M3 co-processor handling the servo outputs. We implemented our
flight stack on the Cortex-M4 from the ground up, thus completely
replacing the original PX4/NuttX firmware that is installed when
shipped. 

%Besides being suitable for our environmental requirements,
%another reason for this choice was that there existed an Ada Run-Time
%System (RTS) for a very similar processor, the STM32F409.

\textbf{Board Support, Hardware Abstraction Layer \& Run-Time System.}
We are hiding the specific target from the application
layer by means of a board support package (not to be confused with an Ada package). This package contains an
hardware abstraction layer (HAL) and a run-time system (RTS). The RTS
is implementing basic functionality such
as tasking and 
%string handling, math operations, 
memory management. %removed "required by Ada"; because SPARK does require it as well.
The HAL is our extension of AdaCore's Drivers
Library~\cite{AdaDriverLib}, and the RTS is our port of the Ada RTS for the
 STM32F409 target.  Specifically, we have ported the
Ravenscar Small Footprint variant~\cite{Ravenscar},   % Abbr. SFP is never used
which restricts Ada's and SPARK's tasking facilities to a deterministic and
analyzable subset, but meanwhile forbids exception handling, which anyway is not permitted in SPARK.

% \begin{figure}[htb]
%   \centering
%   \includegraphics[width=.8\textwidth]{fig/software-parts}
%   \caption{Software Layering}
%   \label{fig:layers}
% \end{figure}
% Figure~\ref{fig:X} shows the layers of our flight stack: The RTS, the
% board support package and the hardware abstraction layer (HAL), and on
% top the application. 


\textbf{Separating Tasks by Criticality\label{sec:separ-tasks-crit}}
has been one goal, since multi-threading is supported in
SPARK. In particular, \begin{inparaenum}
\item termination of low-critical tasks must not cause termination of
  high-critical tasks,
\item higher-criticality tasks must not be blocked by lower-critical tasks and,
\item adverse effects such as deadlocks,
priority inversion and race condition must not occur. 
\end{inparaenum}
We partitioned our glider software into two tasks (further concurrency arises
from interrupt service routines):
\begin{enumerate}
\item The \emph{Flight-Critical Task} includes all execution flows
  required to keep the glider in a controlled and
  navigating flight, thus including sensor reads and actuator
  writes. It is time-critical for control reasons. High-criticality.
\item The \emph{Mission-Critical Task} includes all execution flows that
  are of relevance for recording and logging of weather data to an SD
  card. Low-priority task, only allowed to run when the
  flight-critical task is idle. Low-criticality.
\end{enumerate}
%
The latter task requires localization data from the former one, to
annotate the recorded weather data before writing it to the SD card.
Additionally, it takes over the role of a flight logger, saving data
from the flight-critical task that might be of interest for a
post-flight analysis. The interface between these two tasks would
therefore be a protected object with a message queue that must be
able to hold different types of messages. 

\textbf{Verification Goals.} First and foremost, AoRTE shall be
established for all SPARK parts, since exceptions would result in task
termination. Additionally, the application shall make use of as many
contracts and checks as possible, and perform all of its computations using
dimension-checked types. Last but not least, a few functional high-level
requirements related to the homing functionality have been encoded in
contracts.
%
Overall, the focus of verification was the application, not the
BSP. The BSP has been written in SPARK only as far as necessary to 
support proofs in the application. The rationale was that the RTS was assumed to
be well tested, and the HAL was expected to be hardly verifiable
due to direct hardware access involving pointers and restricted types.

%%% Local Variables: ***
%%% mode:latex ***
%%% TeX-master: "paper.tex"  ***
%%% End: ***
%\section{Daily Development \& Verification Workflow}
\label{sec:verification}
\todo{cross-verification: how to}
We have developed and verified the onboard software of the glider in parallel for two reasons: 
\begin{enumerate}
\item To let verification needs steer design refinement and
  implementation decisions, and 
\item to get an early grasp on verification efforts, and avoid that
  verification would be done in the last minute, possibly posing
  difficulties or even exposing an unverifiable implementation.
\end{enumerate}
We used two mechanisms were to realize such a co-verification
approach. First, the developers were also responsible for
verification. Besides taking and realizing implementation choices,
their job was to ensure that a minimal set of annotations were
provided (as required per the SPARK language, e.g., state
abstraction), such that static analysis can process their packages.
They also had the option to run the static analysis locally on their
machines. However, analyzing the complete flight stack with sufficient
depth would have taken too long, so local runs were usually limited to
the subprogram they were working on, and with short timeouts
only. This helped them to identify simple defects early, such as
out-of-bounds access to arrays.

Second, we had set-up a Jenkins server to perform nightly analysis
runs on the whole flight stack, with a longer
timeout. One such run could take up to one hour on an octa-core machine
(\SI{2.7}{GHz} Intel Xeon E5-2680 processor with \SI{16}{GB} RAM); note that the
analysis makes full use of all cores. Every time the server had
finished analyzing the project, artifacts summarizing the verification
were emailed to the developers, most importantly:
\begin{inparaenum}
\item the log files of the analysis, showing detailed outputs such as
  warnings, errors and counterexamples and
\item a tabular verification summary.% as shown in Tab.~\ref{tab:cov}. 
\end{inparaenum}
%
% The verification summary table is showing all packages, sorted
% descendingly first by \emph{coverage}, then by \emph{success}. We
% denote as \emph{coverage}, the percentage of SPARK code relative to
% all code in a package. A coverage of 100\% can only be reached if all
% entities in the package body are in SPARK, which includes instances of
% generics, and if all callees of other packages have at least a SPARK
% specification. We denote as \emph{success} the percentage of
% successfully verified properties over all properties.  
%
~These artifacts were used every morning to decide where development
and verification efforts should continue. The goal was reach foll coverage and maximum verification success in the flight-critical packages.

% \bgroup
% \scriptsize
% \setlength\tabcolsep{.45em}
% \begin{table}[btp]
% \begin{minipage}{\linewidth}
%   \centering
%   \caption{Example of a nightly verification summary.}
%     \small
%   \begin{tabular}{lrrrrrr}
%     \toprule
%     Unit & \#entities & \#skip & coverage \% & \#properties & \#proven & success \%\\
%     \midrule
% px4io.protocol & 1 & 0 &  100 & 54 & 54 & 100.0\\
% ublox8.driver & 17 & 0 &  100 & 205 & 202 & 98.5\\
% units.navigation & 16 & 0 &  100 & 109 & 93 & 85.4\\
% \vdots&\vdots&\vdots&\vdots&\vdots&\vdots\\
% mystrings & 8 & 1 & 88 & 18 & 18 & 100.0\\
% controller & 25 & 3 &  88 & 139 & 119 & 85.6\\
% estimator & 25 & 3 &  88 & 175 & 148 & 84.6\\
% \vdots&\vdots&\vdots&\vdots&\vdots&\vdots\\
% fat\_filesystem.directories & 9 & 9 &  0 & 0 & 0 & --\\
%     \bottomrule
%   \end{tabular}\label{tab:cov}
%   \textbf{Totals:} entities: 1062, success: 89.9\%, proven: 2274, coverage:  32.5\%, props: 2531, units: 106
% \end{minipage}
% \end{table}
% \egroup

\textbf{Failing checks} have been investigated by placing additional
assertions to probe the knowledge of the solver, and with that
identify the underlying reason and propose a rewrite that could be
verified. If a failing property depended on a fact that was not
visible to the solver (e.g., the implicitly known length of integer
images), then \texttt{pragma Assume} has been used to provide
knowledge to the analyzer, or, alternatively, a wrapper function was
introduced with a post-condition stating the new fact. If a result was
identified as False Positive, then we applied \texttt{pragma Annotate}
to suppress the warning. Finally, for unclear cases we placed an
\texttt{pragma Assert}, which has the same effect as \texttt{pragma
  Assume}, but leaves a failing VC marking such cases, and potentially
produces a run-time exception which then can be used to understand
such cases.

%%% Local Variables: ***
%%% mode:latex ***
%%% TeX-master: "paper.tex"  ***
%%% End: ***
\section{Problems and Workarounds}
\label{sec:workarounds}
In this section, we describe the perils and difficulties that we
identified during verification of SPARK programs. We use the following nomenclature:
\begin{itemize}
\item \textbf{False Positive} denotes a failing check (failed VC) in static analysis % no comma
  which would not fail in any execution on the target, i.e., a false alarm.
\item \textbf{False Negative} denotes a successful check (discharged VC) in static
  analysis which % no comma
  would fail in at least one execution on the target, i.e., a missed
  failure.
\end{itemize}

\subsection{How to Miss Errors}%not bugs. failure=when it shows up, infection=when it deviates from what it should be, defect=where it is created, the code 
There are a few situations in which static analysis can miss run-time
exceptions, which in a SPARK program inevitably ends in abnormal
program termination. Before we show these unwanted situations, we have to
point out one important property of a deductive verification approach: Proofs
build on each other. Consider the following example (results of static
analysis given in comments):
\begin{lstlisting}[name=proofdep]
  a := X / Z; -- medium: division check might fail
  b := Y / Z; -- info: division check proved
\end{lstlisting}  
The analyzer reports that the check in line 2 cannot fail, although it
suffers from the same defect as line 1. However, when the run-time
check at line 1 fails, then line 2 cannot be reached with the
offending value of \lstinline$Z$, therefore line 2 is
not a False Negative, unless exceptions have been wrongfully disabled.

\textbf{Mistake 1: Suppressing False Positives.} When a developer
comes to the conclusion that the analyzer has generated a False
Positive (e.g., due to insufficient knowledge on something that is
relevant for a proof), then it might be justified to suppress the
failing property. However, we experienced cases where this has generated
False Negatives which where hiding (critical) failures.  Consider the
following code related to % the - intentional "Russian bug". Like break must to
the GPS: % protocol parser:
%\vspace*{-3mm}
\begin{lstlisting}[name=missedbug,escapechar=\$]
function toInt32 (b : Byte_Array) return Int_32 with Pre => b'Length = 4;
procedure Read_From_Device (d : out Byte_Array) is begin
  d := (others => 0); -- False Positive
  pragma Annotate (GNATprove, False_Positive, "length check might fail", ...);
end Read_From_Device;

procedure Poll_GPS is
  buf    : Byte_Array(0..91) := (others => 0);
  alt_mm : Int_32;
begin
  Read_From_Device (buf);
  alt_mm := toInt32(buf(60..64)); -- False Negative, guaranteed exception
end Poll_GPS;
\end{lstlisting}
Static analysis found that the initialization of the array \lstinline$d$
in line 3 could fail, but this is not possible in this context, and
thus a False Positive\footnote{This particular case has been fixed in recent versions of GNATprove.}. The developer
was therefore suppressing this warning with an annotation
pragma. However, because proofs build on each other, a severe defect
in line 12 was missed. The array slice has an off-by-one error which
\emph{guaranteed} failing the precondition check of
\lstinline$toInt32$.  The reason for this False Negative is that
everything after the initialization of \lstinline$d$ became virtually
\emph{unreachable} and that all following VCs consequently have been discharged.  
In general, a False Positive may exclude some or all execution paths
for its following statements, and thus
hide (critical) failure. We therefore recommend to avoid suppressing False Positives, and either
leave them visible for the developer as warning signs, or even better, rewrite the code in a prover-friendly manner following the tips in Section~\ref{sec:recommendations}.

\textbf{Mistake 2: Inconsistent Contracts.} Function contracts act as
barriers for propagating proof results (besides inlined
functions), that is, the result of a VC in one subprogram cannot affect the result of another in a different subprogram. However, these barriers can be broken when function
contracts are inconsistent, producing False Negatives by our definition. One way to obtain inconsistent contracts, is writing a postcondition which itself contains a failing VC (line 2):
% in the following code, we cannot write an expression function to save space, since the example does not work anymore, then.
\begin{lstlisting}[name=postvc]
function f1 (X : Integer) return Integer 
  with Post => f1'Result = X + 1 is -- overflow check might fail
begin 
  return X;
end f1;

procedure Caller is
   X : Integer := Integer'Last;
begin
   X := X + 1; -- overflow check proved.
   X := f1(X);
end Caller;
\end{lstlisting}
Clearly, an overflow must happen at line 10, resulting in
an exception. The analyzer, however, proves absence of overflows in
\lstinline$Caller$. The reason is that in the Why3 backend, the
postcondition of \lstinline$f1$ is used as an axiom in the analysis of
\lstinline{Caller}. The resulting theory for \lstinline{Caller} is an
inconsistent axiom set, from which (\emph{principle of explosion})
anything can be proven, including that false VCs are true. In such
circumstances, the solver may
%when not given enough time to decide the goal, 
also produce a \emph{spurious} counterexample. 
%Therefore, this is a situation which is not obvious, but can
%refute results and therefore must be avoided.

In the example above, the developer gets a warning for the
inconsistent postcondition and can correct for it, thus keep barriers
intact and ensure that the proofs in the caller are not
influenced. However, if we change line 4 to \lstinline{return X+1}, then the
failing VC is now indicated in the body of \lstinline{f1}, and -- since the
proofs build on each other -- the postcondition is verified and a
defect easily missed. Therefore, failing VCs within callees may also
refute proofs in the caller (in contrast to execution semantics) and
have to be taken into account. Indeed, the textual report of GNATprove (with flag
\texttt{--assumptions}) indicates that AoRTE in \lstinline{Caller} depends on both
the body and the postcondition of \lstinline{f1}, and therefore the reports have to be studied with great care to judge the verification output.
%Not a fix: let GNATprove use 
%mathematical semantics as overflow mode, which might be more intuitive for
%some developers and avoids the above problem. The failing VC disappears at all,
%but the false negative remains.
Finally, note that the same principle applies for assertions and loop invariants.

\textbf{Mistake 3: Forgetting the RTS.} Despite proven AoRTE, one
procedure which rotates the frame of reference of the gyroscope
measurements was sporadically triggering an exception after a
floating-point multiplication. The situation was eventually captured
in the debugger as follows:
\begin{lstlisting}[name=subnormal]
-- angle = 0.00429, vector (Z) = -2.023e-38
result(Y) := Sin (angle) * vector(Z);
-- result(Y) = -8.68468736e-41 => Exception
\end{lstlisting}
%The reader who is familiar with floating-point implementations, might
% notice that 
Variable \lstinline$result$ was holding a \emph{subnormal} floating-point number,
roughly speaking, an ``underflow''. GNATprove models floating-point
computations according to IEEE-754, which requires support for
subnormals on the target processor. Our processor's FPU indeed
implements subnormals, but the RTS, part of which describes
floating-point capabilities of the target processor, was incorrectly
indicating the opposite\footnote{This also has been fixed in recent
  versions of the embedded ARM RTS.}. As a result, the language-defined
float validity check occasionally failed (in our case when the glider
was resting level and motionless at the ground for a longer period of
time). 
%In other words, the RTS was incorrectly
%  describing the processor capabilities, which created a discrepancy
%  between what is statically analyzed and what is being executed. 
  Therefore, the RTS must be carefully configured and checked manually
  for discrepancies, otherwise proofs can be refuted since static
  analysis works with an incorrect premise.

% \subsection{Bad Patterns}
% We now reflect on two situations that may occur when developers try to
% achieve successful verification with inappropriate methods.
\textbf{Mistake 4: Bad Patterns.} % good move. Indirectly we miss errors by that.
\emph{Saturation} may seem like an effective workaround to ensure
overflows, index checks and so on cannot fail, but it usually hides
bigger flaws. Consider the following example, also from the GPS protocol parser:
\begin{lstlisting}[name=saturate]
subtype Lat_Type is Angle_Type range -90.0 * Degree .. 90.0 * Degree;
Lat : Lat_Type := Dim_Type (toInt32 (data_rx(28..31))) * 1.0e-7 * Degree;
\end{lstlisting}
The four raw bytes in \lstinline$data_rx$ come from the GPS device and represent a scaled float, which could in
principle carry a value exceeding the latitude range of $[-90,90]$
Degree. To protect against this sort of error, it is tempting to
implement a function (even a generic) of the form
\lstinline$if X > Lat_Type'Last then X := Lat_Type'Last else...$ that limits the value to the available range, and apply it to all
places where checks could be failing. However, we found that almost every
case where saturation was applied, was masking a
boundary case that needs to be addressed. In this example, we needed
handling for a GPS that yields faulty values. 
In general, such cases usually indicate a missing software requirement.

% \textbf{Misinterpreting Positives} can lead to wrong counteractions.
% The analyzer indicated a possible failure in $\sqrt{1-\sin(a)\cos(b)}$
% when the argument of \texttt{Sqrt} becomes negative. However, this is
% not possible, since sine and cosine are always $\in[-1,1]$. The solver
% could only prove that $\sin()\cdot\cos()\in[-2,2]$. The developer has
% therefore incorrectly assumed that either the solver did not have
% enough precision, or that indeed a value beyond $[-1,1]$ can be
% reached due to rounding errors, and fixed it like follows:
% $\sqrt{1-Sat(\sin(a)\cos(b))}$, where \texttt{Sat} was again
% saturating to $[-1,1]$ and stating this fact as a postcondition. After
% this fix, the analyzer could prove AoRTE. But the reason was not
% imprecision or rounding errors, but not giving the solver enough time.
% This particular change reduced the complexity for the solver, which is
% why it could be proven. In summary, the developer was not considering
% that Positives may be False Positives due to resource limitations, and
% made unnecessary changes to a working code base, which itself could
% introduce new problems. A better approach is to query the solver
% knowledge and add user lemmas to add the missing knowledge or speed up
% the analysis.


\subsection{Design Limitations}
We now describe some cases where the current version of the SPARK~2014
\emph{language} -- not the static analysis tool -- imposes limitations.

% class-wide in_out "Driver.Coefficients" must also be a class-wide input of overridden subprogram "read_Measurement" at generic_sensor.ads:53, instance at line 15 (SPARK RM 6.1.6); See: http://docs.adacore.com/spark2014-docs/html/lrm/subprograms.html

\textbf{Access types} (pointers) are forbidden in SPARK, however, low-level
drivers heavily rely on them. One workaround is to hide those types in
a package body that is not in SPARK, and only provide a SPARK specification. 
Naturally, the body cannot be verified, but at least its subprograms 
can be called from SPARK subprograms. Sometimes it
is not possible to hide access types, in particular when packages use
them as interface between each other. This is the case for our SD card
driver, which is interfaced by an implementation of the FAT filesystem through access types. Both
are separate packages, but the former one exports restricted types and
access types which are used by the FAT package, thus requiring that wide parts of the FAT package 
are written in Ada instead of SPARK.


As a consequence, access types are sometimes demanding
to form larger monolithic packages, here to combine SD card driver and FAT filesystem
into one (possibly nested) package.

\textbf{Polymorphism} is available in SPARK, but its use is
limited as a result of the access type restriction. Our message
queue between flight-critical and mission-critical task was planned to
hold messages of a polymorphic type. However, without access types the
only option to handover messages would be to take a deep copy and
store it in the queue. However, the queue itself 
is realized with an array and can hold only objects of the
same type. This means a copy would also be an upcast to the base
type. This, in turn, would loose the components specific to the derived
type, and therefore render polymorphism useless. As a workaround, we
used mutable variant records.

\textbf{Interfaces.} Closely related to polymorphism, we intended to implement sensors as polymorphic types. That is, specify an abstract sensor interface that must be overridden by each sensor implementation. 
%The idea was that the estimator could start sensor measurements by iterating over a list of sensors without knowing the details of each sensor implementation.
Towards that, we declared an abstract tagged type %\texttt{Sensor\_Tag} 
with abstract primitive methods denoting the interface that a specific sensor must implement. %\texttt{start\_Measurement}. 
However, when we override the method for a sensor implementation, such as the IMU, SPARK requires specifying the global dependencies of the overriding IMU implementation as class-wide global dependencies of the abstract %\texttt{start\_Measurement} 
method (SPARK RM 6.1.6). This happens even without an explicit \texttt{Global} aspect.
%
%\todo{Q: why does it want that? Is it because we have added aspect 'Global' to the abstract type? If yes, than we have a point, because we can say that flow contracts cannot be used in a sensible way with polymorphism, i.e., a conceptual incompatibility in the language. If this happens even without us wanting this, then the language must be questioned for this decision. However, if this comes from the LSP (well...but data flows are not strictly within the scope of LSP...maybe AdaCore went overboard then), then this means we question LSP. And that's probably not a good idea.} This would break the OOP concept of abstraction, since the abstract sensor type would require knowing the details of all possible implementations. 
As workaround, we decided to avoid polymorphism and used simple inheritance without overriding methods.
% Wish: For the future, SPARK could automatically detect class-wide dependencies without the need of specifying them explicitly.

\textbf{Dimensioned Types.}
Using the GNAT dimensionality checking system in SPARK, had revealed two missing features. Firstly, in the current stable version of the GNAT compiler, it is not possible to specify general operations on dimensioned types that are resolved to specific dimensions during compilation.
% preserve their specific dimensionality.
% View conversion of types: https://www2.adacore.com/gap-static/GNAT_Book/html/aarm/AA-4-6.html
%The parameters of subprograms must be specified explicitly 
%MBe: Something is missing here. Why does it say "class-wide"? 
% The dimensionality system is not laid out as derived types (they would be incompatible then), therefore class-wide types don't work (I think).
% If we would re-write everything, then class-wide types would be available, but dispatching the only way I can see to get this working. 
% Then, however, we would need a separate overriding integrate operation for each new type. 
For example, we could not write a generic time integrator function for the PID controller that multiplies any dimensioned type with a time value and returns the corresponding unit type. Therefore, we reverted to dimensionless and unconstrained floats within the generic PID controller implementation. 
% an attempt to rewrite this a bit:
Secondly, it is not possible to declare vectors and matrices with mixed subtypes, which would be necessary to retain the dimensionality information throughout vector calculations (e.g., in the Kalman Filter). As a consequence, we either have split vectors into their components, or reverted to dimensionless and unconstrained floats. %Computations involving the state vector were split into their individual components to make use of the dimensioned types.\todo{are they verified, then?}
As a result of these workarounds, %of reverting to dimensionless and unlimited Floats, 
numerous overflow checks related to PID control and Kalman Filter could not be proven (which explains more than \SI{70}{\percent} of our failed floating-point VCs). %\todo{We could have used limited dimensionless floats.}



\subsection{Solver Weaknesses}
We now summarize some frequent problems introduced by the current
state of the tooling. %Many of them can be mitigated by
%refactoring, or by providing more information to the solver. 
%We mention the most frequent here, and how they can be addressed. 
%A complete list of current tooling limitations is given here:
%\url{http://docs.adacore.com/spark2014-docs/html/ug/en/appendix/gnatprove_limitations.html}

The \texttt{'Position} attribute of a record allows evaluating the
position of a component in that record.
% This is useful for low-level protocols or device drivers. 
However, GNATprove has no precise information
about this position, and therefore proofs building on that might fail.

Another feature that is used in driver code, are
\emph{unions}, which provide different views on the
same data. GNATprove does
not know about the overlay and may generate False Positives for
initialization, as well as for proofs which build on the relation
between views.

% When \emph{mutable} variant records are copied (e.g., used as
% replacement for polymorphism), then the target object must also be
% mutable.  Currently, GNATprove does not infer mutability of the target
% object, when it is an output parameter of a subprogram call. Instead,
% a precondition (\lstinline$not msg'Constrained$) has to be added to
% the subprogram.

We had several False Positives related to possibly uninitialized
variables.  SPARK follows a strict data initialization policy. Every
(strict) output of a subprogram must be initialized. In the current version,
GNATprove only considers initialization of arrays as complete when done in a single
statement. This generates warnings when an array is initialized in
multiple steps, e.g., through loops, which we have suppressed.

  % not discussed here: loop invariants.
  % \item Loop invariants: not used; generated frame conditions seem strong enough
  %   \begin{itemize}
  %   \item TODO: inspect all our loops; see if there are failing checks around
  %   \end{itemize}  
  

% \subsection{Others }
%  \textbf{Things to test:}
%   \begin{itemize}
%   \item Type invariant: TODO, see if we could have used it. outside of the immediate scope of a type with an invariant, only values of this type are allowed as given by its invariant
%     \begin{itemize}
%     \item not used. 
%     \end{itemize}
%   \end{itemize}


%%% Local Variables: ***
%%% mode:latex ***
%%% TeX-master: "paper.tex"  ***
%%% End: ***
%
\section{A Rule Set for Co-Verification}
These rules ensure: 
\begin{itemize}
\item consistency of the prover result it itself
\item no masking of errors occurs
\item ...
\end{itemize}

\subsection{Consistency Rules}

\begin{itemize}
\item \textbf{rule:} code must not be modified to reach AoRTE based on proof attempts that ran into timeout/memout/stepout (see \texttt{Lim} function). Instead, increase steps/time or apply \texttt{Assert\_and\_cut};
\item \textbf{rule:} necessary: contracts must be proven AoRTE,
  otherwise we get false negatives \textbf{upstream}, which is
  unintuitive and likely to cause big problems. sufficient: body must
  also be AoRTE, otherwise RTE in post can be masked.


\item \textbf{rule:} sufficient: we can only rely on the proofs in a
  subprogram, if all its dependencies are met, which in particular
  includes the bodies and postcondition of all callees.
\item \textbf{false negatives} skew the picture and can hide dangerous run-time exceptions
  \begin{itemize}
  \item cause exceptions that are not anticipated
  \item \textbf{rule:} we can only trust any verified CV within a subprogram if the subprogram itself is completely free of RTE and has a verified pre- and Postcondition
  \item if Post is not verified and exceptions are off, then errors propagate wildly (obviously), which should be avoided at all cost. 
  \item \textbf{rule:} assuming exceptions are not turned off before \textbf{all} VCs are verified: in contrast to Model Checking with CBMC, a subprogram with non-verified VCs cannot generate false negatives in other subprograms (exception to this rule: inlining in the analysis). Therefore, deductive verification - or more general, assume-guarantee reasoning at pre- and post-condition of functions - limits propagation of error conditions and gives a more certain picture on ``verification success''.
  \item \textbf{false positives} (developer knows that the prover has insufficient knowledge) due to insufficient solver knowledge can also hide dangerous run-time exceptions (GPS failure)
    \begin{itemize}
    \item \textbf{rule:} suppressing false positives should be avoided. If necessary, then include an \texttt{Assert\_And\_Cut} after the suppressed statement, to reset solver knowledge on the specific item that cause the check to fail. E.g., when length check fails, then \texttt{Assert\_And\_Cut(arr'Length >= 0)}, to remove all possibly incorrect assumptions on the array length and thereby unhide downstream false negatives.
    \end{itemize}
  \end{itemize}
  \item RTS requires careful configuration and cannot be abstracted completely
  \item E.g., if architecture details are incorrect (S'Denorm) then
    some runtime checks may fail unexpectedly because target does not
    behave as claimed
  \item can refute the soundness of analysis

\end{itemize}



%%% Local Variables: ***
%%% mode:latex ***
%%% TeX-master: "paper.tex"  ***
%%% End: ***
\section{Results}
\label{sec:results}

In general, verification of SPARK~2014 programs is accessible and mostly
automatic.  Figure~\ref{fig:stats} shows
the results of our launch release. As it can be seen, we could not
prove all properties during the time of this project (three months). The non-proven
checks have largely been identified as ``fixable'', following our design recommendations given below.

\begin{figure}[hbtp]\vspace*{-4mm}
    %\centering
    \begin{subfigure}[t]{0.48\textwidth}
        \includegraphics[width=\textwidth]{fig/vc_success}
        %\caption{VC type vs. success ratio}
        %\label{fig:usucc}
    \end{subfigure}
    ~
    \begin{subfigure}[t]{0.48\textwidth}
      \includegraphics[width=\textwidth]{fig/total_time_per_VC}
      %\caption{Total CPU time per VC type}
      %\label{fig:props}
    \end{subfigure}%\vspace*{-5mm}
    % \begin{subfigure}[t]{0.48\textwidth}
    %   \includegraphics[width=\textwidth]{fig/props}
    %   \caption{Property Occurrence}
    %   \label{fig:props}
    % \end{subfigure}
    \caption{Statistics on Verification Conditions (VCs) by type.%: Floating-Point VCs have the lowest success ratio, meanwhile they take most analysis time.
\vspace*{-3mm}}\label{fig:stats}
  \end{figure}

The complexity of our flight stack and verification progress are
summarized in Table~\ref{tab:complexity}. It can be seen that our
focus on the application part is reflected in the SPARK coverage that we
have achieved (\SI{82}{\percent} of all bodies in SPARK, and even
\SI{99}{\percent} of all specifications), but also that considerably more work has to
be done for the BSP (currently only verified by testing). In
particular, the HAL (off-chip device drivers, bus configuration, etc.)
is the largest part and thus needs a higher SPARK coverage.
% SVD: 14248 loc -> 43\% of HAL
However, we should add that \SI{43}{\percent} of the HAL is consisting of
specifications generated from CMSIS-SVD files, which do not contain any
subprograms, but only definitions of peripheral addresses and record definitions to access
them, and therefore mostly cannot be covered in SPARK. Last
but not least, a completely verified RTS would be desirable, as well.

\bgroup
\setlength\tabcolsep{.5em}
\begin{table}[htbp]\vspace*{-9mm}
\begin{minipage}{\linewidth}
  \centering
  \caption{Metrics and verification statistics of our Flight Stack.}
  \small
  %\scriptsize % looks odd
  \begin{tabular}{lrrrr}
    \toprule
    &               & \multicolumn{2}{c}{\scriptsize{Board Support Package}} & \\
                         \cmidrule{3-4}
    Metric                      & {Application} & {HAL}         & RTS          & {All}\\
    \midrule
    lines of code (GNATmetric)  & 6,750         & 32,903        & 15,769       & 55,422   \\
    number of packages          & 49            & 100           & 121          & 270      \\
    cyclomatic complexity       & 2.03          & 2.67          & 2.64         & 2.53      \\
    SPARK body/spec             & 81.9/99.4\,\% & 15.5/23.5\,\% & 8.6/11.8\,\% & 30.0/38.5\,\%  \\ 
    %SPARK spec                     & 99.4\%        & 23.5\%  & 11.8\%  & 38.5\%  \\ 
    \midrule
    number of VCs               & 3,214         & 765      & 2       & 3,981      \\
    VCs proven                  & 88.1\,\%      & 92.5\,\% & 100\,\% & 88.8\,\%   \\
    analysis time\footnote{Intel Xeon E5-2680 Octa-Core with \SI{16}{GB} RAM, timeout=\SI{120}{\second}, steps=inf.}  & --        & --        & -- & \SI{19}{\minute} \\
    \bottomrule
  \end{tabular}\label{tab:complexity}
\end{minipage}
\vspace*{-4mm}
\end{table}
\egroup


\textbf{Floats are expensive}. Statistically, we have spent most
  of the analysis time (\SI{65}{\percent}) for proving absence of floating-point
  overflows, although these amount to only \SI{21}{\percent} of all VCs. This is
  because discharging such VCs is in average one
  magnitude slower than discharging most other VC types. In particular, one has
  to allow a high step limit (roughly the number of
  decisions a solver may take, e.g., deciding on a literal) and a
  high timeout. Note that at some point an increase of either of them does not improve 
  the result anymore.% because: I don't know why. If the solver's cannot decide, then they
 % should terminate early. But they don't. So this means they are trying something. How
 % do we know that more time/steps do not lead to solution?

% workaround there
  % \textbf{Nonlinear operations, such as trigonometric functions are
  %   uninterpreted}. There are many arithmetic functions which are not
  % interpreted by the solvers, such as sine, square root, and so on.
  % Analyzers have to assume the worst when no precise information is
  % available, and therefore may produce False Positives. This is one of
  % the few places where the solvers benefit from manual annotations in
  % the form of user lemmas. These are provided by ghost subprograms
  % with null body, where the post-condition expresses the lemma, and
  % the pre-condition captures the circumstances in which the lemma is
  % valid (and becomes a proof obligation). SPARK includes a lemma
  % library which has been proven useful for nonlinear operations, but
  % also we had to add our own lemmas. In particular, we added a lemma
  % to state the fact that $|a\cdot b| \leq |b|$ (postcondition) when
  % $|a| \leq 1.0$ (precondition) which holds for IEE754 single
  % precision floats. This was necessary to verify the \emph{haversine
  %   formula}, which computes the distance between two location
  % coordinates as part of the homing functionality. This formula
  % includes a dozen of nonlinear operations applied in
  % sequence. Furthermore, some basic postconditions about the ranges of
  % sine, cosine and arctan were added, as well as for the sign of
  % arctan ($sgn(arc\tan(y,x) = sgn(y)$). Providing this information
  % enabled to prove AoRTE in the entire subprogram in a few seconds, as
  % well as the post-condition that distance is positive and at most
  % half the circumference of the earth.
  
% \begin{figure}[htbp]
%   \includegraphics[width=\textwidth]{fig/units_cov}
%   \caption{SPARK coverage in units. Hardware interfaces (\texttt{hil},
%     \texttt{hal}) tend to have less coverage, whereas data processing
%     and high-level implementation (left side) have full coverage. Exception: \texttt{estimator} (array casts in generic queue), \texttt{units.vectors} (RTS not in SPARK).}
%   \label{fig:ucov}
% \end{figure}


%First: how set search depth (by time, because nightly build). In general, more properties could be proven with more analysis time. However, practically this saturates quickly, (we are not sure whether spending much more time would help there anyway). 

%We show the data for the latest stable with timeout=30s/property and unlimited steps (roughly number of decisions be taken by the solver per VC, i.e., choosing literals). 

%Fig.~TODO shows the 

\textbf{Multi-Threading} could be proven to follow our goals. By using the Ravenscar RTS, 
our goals related to deadlock, priority inversion and blocking, hold true by design.
Several
race conditions and non-thread-safe subprograms have been identified by GNATprove, which otherwise would have refuted task separation. To ensure that
termination of low-criticality tasks cannot terminate the flight-critical task, we
provided a custom implementation for GNAT's last chance handler 
(outside of the SPARK language and therefore not being analyzed) which
reads the priority of the failing task %(assumed to be proportional
%to its criticality) 
and acts accordingly: If the priority is lower than
that of the flight-critical task (i.e., the mission-critical
task had an exception), then we prevent a system reset by sending the
low-priority task into an infinite null loop (thus keeping it busy
executing \texttt{nop}s, and keeping the flight-critical task alive). 
If the flight-critical task is failing, then our handler allows a system reset. 
Multi-threading is therefore easy to
implement, poses no verification problems, and can effectively
separate tasks by their criticality.

\textbf{High-Level Behavioral Contracts} related to the homing functionality could
be expressed and proven with the help of ghost functions, although this is beyond the main purpose of SPARK contracts. For example, we could prove the overall behavior in case of loosing the GPS fix, or missing home coordinates. 




% This stuff should go in the conclusion:
% \begin{itemize}
% \item general: ``how hard is it'' to write zero defect code?
%   \begin{itemize}
%   \item becomes more practical now
%   \end{itemize}
% \item ``bang for buck'': How much of verification works automatically, how much needs writing contracts?
%   \begin{itemize}
%   \item a lot of checks are proven w/o annotations, such as invariants
%   \item TODo: where do we have the most contracts, and why?
%   \end{itemize}
% \item investigating remaining fails: float. propose interval arithmetic or fixed-point as solution?
% \item what cannot be verified and why (pointer stuff; can only be hidden so much with spec functions)
% \end{itemize}

\subsection{Design Recommendations}\label{sec:recommendations}
The following constructs and strategies have been found amenable to verification:\vspace{-1mm}
\begin{enumerate}
\item Split long expressions into multiple statements $\rightarrow$ discharges more VCs. % TODO: find out why. It's not a time limit thing, and not a steps thing either.
\item Limit ranges of data types, especially floats $\rightarrow$ better analysis of overflows.
\item Avoid saturation $\rightarrow$ uncovers missing error handling and requirements.
\item Avoid interfaces $\rightarrow$ annotations for data flows break concept of abstraction.
\item Emulate polymorphic objects that must be copied with mutable variant records.
\item Separation of tasks by criticality using a custom last chance handler $\rightarrow$ abnormal termination of a low-criticality task does not cause termination of high-criticality tasks.
%\item Avoid the following constructs:
%  \begin{itemize}
%  \item TODO
%  \end{itemize}
\end{enumerate}


%%% Local Variables: ***
%%% mode:latex ***
%%% TeX-master: "paper.tex"  ***
%%% End: ***

\section{Related Work}
\label{sec:related-work}

Only a small number of experience reports about SPARK~2014 have been
published before. A look back at (old) SPARK's history
and its success, as well as an initial picture of SPARK~2014 is given
by Chapman and Schanda in~\cite{Chapman2014}. We can report that the
mentioned difficulties with floating-point numbers are solved in SPARK~2014, 
and that the goal to make verification more accessible, has been reached.  A small case study with SPARK~2014
is presented in~\cite{Trojanek2014}, but at that point multi-threading
(Ravenscar) was not yet supported, and floating point numbers have
been skipped in the proof. We can add to the conclusion given there,
that both are easily verified in ``real-world'' code, although
float proofs require more (computational and mental) effort.  Larger
case studies are summarized by Dross et al. in \cite{Dross2014}, with
whom we share the opinion of minor usability issues, and that some
small amount of developer training is required.  Finally, SPARK~2014
with Ravenscar has recently been announced to be used in the Lunar
IceCube~\cite{Brandon2016} satellite, a successor of the successful CubeSat project that was implemented in SPARK 2005. It will be a message-centric software, conceptually similar to NASA's cFE/CFS, but 
fully verified and striving to become an open source platform for spacecraft software.
In contrast to all the above publications, this paper is not focused on the application or case studies, but  pointing out typical sources of errors in SPARK programs, which a developer has to
know in order to get correct verification results.

%%% Local Variables: ***
%%% mode:latex ***
%%% TeX-master: "paper.tex"  ***
%%% End: ***
\section{Conclusion\label{sec:conclusion}}
Although the verification of SPARK~2014 programs is very close to
execution semantics and therefore mostly intuitive, we believe that
developers still need some basic training to avoid common mistakes as
described in this paper, which otherwise could lead to a false confidence in
the software being developed. 
%Unless a software is testified AoRTE, there may be missed
%error and defect due to the deductive approach. We think this needs an
%impact analysis and quantification of sort, to not create false
%confidence in the verification progress and thereby to avoid
%underestimating the remaining risk that may reside in the software.
Overall, the language forces developers to address boundary
cases % here it's "boundary case"
of a system explicitly, which eventually helps understanding the
system better, and usually reveals missing requirements for
boundary cases. As a downside, SPARK~2014 programs are often longer than
(approximately) equivalent Ada programs, since in the latter case a
general exception handler can be installed to handle all pathological
cases at once, without differentiating them. Furthermore, static analysis is ready
to replace unit tests, but integration tests have still been found
necessary. 

Regarding the shortcomings of the GNAT dimensionality system, we
can report that as a consequence of our experiments, a solution for
generic operations on dimensioned has been found and will be part of future GNAT releases.

Our remaining criticism to SPARK~2014 and its tools is as follows:
next to some minor tooling enhancements to avoid the mistakes
mentioned earlier and adding some more knowledge to the analyzer, it
is necessary to support object-oriented features in a better
way. All in all, SPARK~2014 raises the bar for formal
verification and its tools, but developers still have to be aware of
limitations.

%%% Local Variables: ***
%%% mode:latex ***
%%% TeX-master: "paper.tex"  ***
%%% End: ***

\section*{Acknowledgements}
Thanks to the SPARK~2014 team of AdaCore for their guidance and insights.

% \selectbiblanguage{english}
{
% \bibliographystyle{plain}
 \scriptsize
 \bibliographystyle{splncs03}
 \bibliography{literature}
}

\end{document}
