%%%%%%%%%%%%%%%%%%%%%%%%%%%%%%%%%%%%%%%%%%%%%%%%%%%%%%%%%%%%%%%%%%%%%%%%%%%%%
\chapter{Quantum Bit Commitment} \label{chap:qbc}
%%%%%%%%%%%%%%%%%%%%%%%%%%%%%%%%%%%%%%%%%%%%%%%%%%%%%%%%%%%%%%%%%%%%%%%%%%%%%
\index{bit commitment|(}
This chapter is about quantum Bit Commitment (\BC) schemes. In \BC, a
committer $\C$ commits himself to a choice of a bit $b \in \set{0,1}$
by exchanging information with a verifier $\V$. We want that $\V$ does
not learn $b$ (we say the commitment is \emph{hiding}), yet $\C$ can later
choose to reveal $b$ in a convincing way, i.e., only the value fixed at
commitment time will be accepted by $\V$ (we say the commitment is
\emph{binding}).

In the next section, we present a \BC scheme from a committer $\C$ with
bounded quantum memory to an unbounded receiver $\V$. The scheme is
peculiar since in order to commit to a bit, the committer does not
send anything. During the committing stage, information only goes from
$\V$ to $\C$.  Therefore, there is no way for the verifier to get
information about the committed bit, i.e.~the scheme is perfectly
hiding.

In Section~\ref{sec:defbinding}, we define two notions of the binding
property and show our scheme secure against quantum-memory-bounded
committer in both of these senses. Similar techniques as in the two previous
chapters for the analysis of the oblivious-transfer protocols are
used.

The results in this chapter appeared in \cite{DFSS05, DFRSS07}.

%%%%%%%%%%%%%%%%%%%%%%%%%%%%%%%%%%%%%%%%%%%%%%%%%%%%%%%%%%%%%%%%%%%%%%%%%%%%%
\section{The Protocol} \label{sec:commprotocol}
%%%%%%%%%%%%%%%%%%%%%%%%%%%%%%%%%%%%%%%%%%%%%%%%%%%%%%%%%%%%%%%%%%%%%%%%%%%%%
The protocol is given in Figure~\ref{fig:comm}. Intuitively, a
commitment to a bit $b$ is made by measuring random BB84-states in
basis $\{+,\times\}_{[b]}$.
  
\begin{myfigure}{h}
\begin{myprotocol}{\comm$(b)$}
\item $\V$ picks $x \in_R \nbit$ and $\theta \in_R \{+,\times \}^n$ and
  sends $x_i$ in the corresponding bases $\ket{x_1}_{\theta_1},
  \ket{x_2}_{\theta_2}, \ldots, \ket{x_n}_{\theta_n}$ to~$\C$.
 \item $\C$ commits to the bit $b$ by measuring all qubits in basis
   $\{+,\times \}_{[b]}$. Let $x' \in \nbit$ be the result.
 \item\label{step:open} To open the commitment, $\C$ sends $b$ and $x'$ to $\V$.
 \item $\V$ verifies that $x_i= x_i'$ for those $i$ where $\theta_i =
   \{+,\times \}_{[b]}$. $\V$ accepts if and only if this is the case.
\end{myprotocol}
\caption{Protocol for quantum bit commitment}\label{fig:comm}
\end{myfigure}

As for the oblivious-transfer protocols in the two previous chapters, we present an
equivalent EPR-version of the protocol that is easier to analyze (see
Figure~\ref{fig:eprcomm}). \index{purification}
\begin{myfigure}{h}
\begin{myprotocol}{\eprcomm$(b)$}
\item $\V$ prepares $n$ EPR pairs each in state
  $\ket{\Omega}=\frac{1}{\sqrt{2}}(\ket{00}+\ket{11})$. $\V$ sends one half of each pair to $\C$ and keeps the other halves.
\item $\C$ commits to the bit $b$ by measuring all received qubits in basis
  $\{+,\times \}_{[b]}$. Let $x' \in \nbit$ be the result.
\item To open the commitment, $\C$ sends $b$ and $x'$ to $\V$.\label{step:epropen}
\item $\V$ measures all his qubits in basis $\{+,\times \}_{[b]}$ and
  obtains $x \in \nbit$.  He chooses a random subset $I \subseteq \{1,
  \ldots ,n\}$. $\V$ verifies that
  $x_i= x_i'$ for all $i \in I$ and accepts if and only if this is the
  case.\label{step:last}
\end{myprotocol}
\caption{Protocol for EPR-based quantum bit commitment}\label{fig:eprcomm}
\end{myfigure}

\begin{lemma}\label{lem:commeprcomm}
  \comm\ is secure against dishonest committers $\dC$ if and only if \eprcomm\ is.
\end{lemma}
\begin{proof}
The proof uses similar reasoning as the one for Lemma~\ref{lem:seqequiv}. 
First, it clearly makes no difference, if we change Step~\ref{step:last} to the
following:
\begin{itemize}
\item[\ref{step:last}'.] $\V$ chooses the subset $I$, measures all
  qubits with index in $I$ in basis $\{+,\times \}_{[b]}$ and all
  qubits not in $I$ in basis $\{+,\times \}_{[1-b]}$. $\V$ verifies
  that $x_i= x_i'$ for all $i\in I$ and accepts if and only if this is
  the case.
\end{itemize}
Finally, we can observe that the view of $\dC$ does not change if $\V$
would have done his choice of $I$ and his measurement already in
Step~1. Doing the measurements at this point means that the qubits to
be sent to $\dC$ collapse to a state that is distributed identically to
the state prepared in the original scheme. The EPR-version is
therefore equivalent to the original commitment scheme from $\dC$'s
point of view.
\end{proof}

It is clear that \eprcomm\ is \index{bit commitment!hiding}hiding, i.e., that the commit phase
reveals no information on the committed bit, since no information is
transmitted to $\V$ at all. Hence we have
\begin{lemma}\label{lem:sec:hiding}
\eprcomm\ is perfectly hiding.
\end{lemma}
%% \begin{proof}                   
%% %  which $\C$ committed to is leaked to any verifier $\dV$, because $\C$ does
%% %  not send anything, i.e.~\eprcomm\ is non-interactive! 
%% \end{proof}



%%%%%%%%%%%%%%%%%%%%%%%%%%%%%%%%%%%%%%%%%%%%%%%%%%%%%%%%%%%%%%%%%%%%%%%%%%%%%
\section{Modeling Dishonest Committers}\label{sec:dishonestcomm}
%%%%%%%%%%%%%%%%%%%%%%%%%%%%%%%%%%%%%%%%%%%%%%%%%%%%%%%%%%%%%%%%%%%%%%%%%%%%%
\index{dishonest committer|(} A dishonest committer $\dC$ with bounded
memory of at most $\gamma n$ qubits in \eprcomm\ can be modeled very
similarly to the dishonest oblivious-transfer receivers $\dB$ from
Section~\ref{sec:modeldishonestreceiversrabin} and
\ref{sec:modeldishonestreceivers}: $\dC$ consists first of a circuit
acting on all $n$ qubits received, then of a measurement of all but at
most $\gamma n$ qubits, and finally of a circuit that takes the
following input: a bit $b$ that $\dC$ will attempt to open, the
$\gamma n$ qubits in memory, and some ancilla in a fixed state. The
output is a string $x' \in \nbit$ to be sent to $\V$ at the opening
stage.
\begin{definition}
  We define $\mathfrak{C}_{\gamma}$ to be the class of all committers
  $\{\dC_n\}_{n>0}$ in \comm\ or \eprcomm\ that, at the start of the opening phase
  (i.e. at Step \ref{step:epropen}), have a quantum memory of size at most
  $\gamma n$ qubits.
\end{definition}
\index{dishonest committer|)}

%%%%%%%%%%%%%%%%%%%%%%%%%%%%%%%%%%%%%%%%%%%%%%%%%%%%%%%%%%%%%%%%%%%%%
\section{Defining the Binding Property} \label{sec:defbinding}
%%%%%%%%%%%%%%%%%%%%%%%%%%%%%%%%%%%%%%%%%%%%%%%%%%%%%%%%%%%%%%%%%%%%%
\index{bit commitment!binding|(}
\subsection{The ``Standard'' Binding Condition}
%%%%%%%%%%%%%%%%%%%%%%%%%%%%%%%%%%%%%%%%%%%%%%%%%%%%%%%%%%%%%%%%%%%%%
In the context of unconditionally secure \emph{quantum} bit
commitment, it is widely accepted that ``the right way'' of defining
the \emph{binding property} is to require that the probability of
opening a commitment successfully to 0 plus the probability of opening
it successfully to 1 is essentially upper bounded by one, put forward by
Dumais, Mayers, and Salvail \cite{DMS00}. We call this notion
\emph{weakly binding}, as opposed to the new notion of \emph{strongly
  binding} defined in the next section below.
\begin{definition} \label{def:weakbinding}
  A (quantum) bit-commitment scheme is \emph{weakly binding}
  against $\mathfrak{C}$ if for all $\{\dC_n\}_{n>0}\in \mathfrak{C}$,
  the probability $p_b(n)$ that $\dC_n$ opens $b\in\{0,1\}$ with
  success satisfies
\[ p_0(n)+p_1(n) \leq 1+\negl{n}.
\] 
\end{definition} \index{bit commitment!weak binding}
In the next Section~\ref{sec:weakbinding}, we show that \eprcomm\ is
weakly binding against $\mathfrak{C}_{\gamma}$ for any
$\gamma<\frac{1}{2}$.

Note that the binding condition given here in
Definition~\ref{def:weakbinding} is weaker than the classical one,
where one would require that a bit $b$ exists such that $p_b(n)$ is
negligible.  For a general quantum adversary though who can always
commit to 0 and 1 in superposition, this is a too strong requirement;
thus, it is typically argued that Definition~\ref{def:weakbinding} is the best
one can hope for. 

However, we argue now that this weaker notion is not really
satisfactory, and we show that there exists a stronger notion, which
still allows the committer to commit to a superposition and thus is
not necessarily impossible to achieve in a quantum setting, but which
is closer to the classical standard way of defining the
binding property. 

%%%%%%%%%%%%%%%%%%%%%%%%%%%%%%%%%%%%%%%%%%%%%%%%%%%%%%%%%%%%%%%%%%%%%
\subsection{A Stronger Binding Condition}\label{sec:strongerbinding}
%%%%%%%%%%%%%%%%%%%%%%%%%%%%%%%%%%%%%%%%%%%%%%%%%%%%%%%%%%%%%%%%%%%%%
A shortcoming of Definition~\ref{def:weakbinding} is that committing
bit by bit is not guaranteed to yield a secure string commitment---the
argument that one is tempted to use requires independence of the
$p_{b}$'s between the different executions, which in general does not
hold.
%% Because of these shortcoming, we propose the
%% following \emph{stronger} definition.
%% one expects that if one commits to a string
%% bit by bit using a secure bit commitment scheme, then this gives a
%% secure string commitment scheme. However, this weaker notion does not
%% seem to allow a proof for this reduction - the argument that
%%   one is tempted to use requires independency of the $p_{b}$'s between
%%   the different executions, which in general does not hold. 

We now argue that this notion is {\em unnecessarily} weak, at least in
some cases, and in particular in the case of commitments in the
bounded-quantum-storage model where the dishonest committer is forced
to do some partial measurement and where we assume honest parties to
produce only classical output (by measuring their entire quantum
state).  Technically, this means that for any dishonest committer
$\tilde{\sf C}$, the joint state of the honest verifier and of
$\tilde{\sf C}$ after the commit phase is a ccq-state \smash{$\rho_{V
    Z \tilde{\sf C}} = \sum_{v,z} P_{VZ}(v,z) \proj{v} \otimes
  \proj{z} \otimes \rho_{\tilde{\sf C}}^{v,z}$}, where the first
register contains the verifier's (classical) output and the remaining
two registers contain $\tilde{\sf C}$'s (partially classical) output.
We propose the following definition.

\begin{definition}\label{def:strongbinding}
  A commitment scheme in the bounded-quantum-storage model is called
  {\em $\eps$-binding}, if for every (dishonest) committer $\tilde{\sf
    C}$, inducing a joint state $\rho_{V Z \tilde{\sf C}}$ after the
  commit phase, there exists a classical binary random variable $D$,
  given by its conditional distribution $P_{D|VZ}$, such that for
  $b=0$ and $b=1$ the state \smash{$\rho_{V Z \tilde{\sf C}}^{b} =
    \sum_v P_{VZ|D}(v,z|b) \proj{v} \otimes \proj{z} \otimes
    \rho_{\tilde{\sf C}}^{v,z}$} satisfies the following condition.
  When executing the opening phase on the state $\rho_{V \tilde{\sf
      C}}^{b}$, for any strategy of \smash{$\tilde{\sf C}$}, the
  honest verifier accepts an opening to $1-b$ with probability at most
  $\eps$.
\end{definition}
% Note that this definition still allows a committer to commit to a
% superposition and otherwise honestly follow the protocol. $D$ is
% then simply defined to be the outcome when the register that carries
% the superposition is measured.  On the other hand, t
It is easy to see that the binding property as defined here implies
the above discussed weak version, namely $p_b \leq P_{D}(b) +
P_{D}(1-b)\eps$ and thus $p_0 + p_1 \leq 1 + \eps$.  Furthermore, it
is straightforward to see that this stronger notion allows for a
formal proof of the obvious reduction of a string to a bit commitment
by committing bit-wise: the $i$-th execution of the bit commitment
scheme guarantees a random variable $D_i$, defined by $P_{D_i|V_i
  Z}$, such that the committer cannot open the $i$-th bit commitment
to $1-D_i$, and thus there exists a random variable $S$, namely $S
= (D_1,\ldots,D_m)$ defined by $P_{D_1\cdots D_m|V_1\cdots V_m Z}
= \prod_i P_{D_i|V_i Z}$, such that for any opening strategy, the
committer cannot open the list of commitments to any other string than
$S$.

% We show in the following that the quantum bit-commitment scheme \comm\
% from~\cite{DFSS05} fulfills the stronger notion of binding from
% Definition~\ref{def:binding} above. For convenience, the protocol
% \comm\ is reproduced in Fig.~\ref{fig:comm} below.  Let
% $\mathfrak{C}_{q}$ denote the set of all possible quantum dishonest
% committers \smash{$\dC$} in \comm\ which have quantum memory of size
% at most $q$ at the start of the opening phase (Step~\ref{it:bound}).
% Then the following holds.

% \begin{theorem}
%   The quantum bit-commitment scheme \comm\ is $\eps$-binding
%   according to Definition~\ref{def:binding} against $\mathfrak{C}_{q}$
%   for a negligible (in $n$) $\eps$ if $n/4 - q \in \Omega(n)$.
% \end{theorem}

% %%%%%%%%%%%%%%%%%%%%%%%%%%%%%%%%%%%%%%%%%%%%%%%%%%

% First, Definition~\ref{def:weakbinding} does not prohibit
% commitment schemes with the following property. There exists a
% strategy for the dishonest committer to produce a commitment in such a
% way that during the course of the opening phase, the committer learns
% some information such that, with probability $\frac12$, it allows him
% to open the commitment to $0$ as well as to $1$ at free will, and,
% with probability $\frac12$, it tells him that he cannot open the
% commitment at all. Even though this does not break the commitment
% scheme completely, it does not capture the expected behavior of a
% commitment scheme.

% Another shortcoming of this notion is that committing bit by bit does
% not yield a secure \index{string commitment}string commitment
% -- the argument that one is
% tempted to use requires independence of the $p_{b}$'s between the
% different executions, which in general does not hold.

% %% one expects that if one commits to a string
% %% bit by bit using a secure bit commitment scheme, then this gives a
% %% secure string commitment scheme. However, this weaker notion does not
% %% seem to allow a proof for this reduction - the argument that
% %%   one is tempted to use requires independency of the $p_{b}$'s between
% %%   the different executions, which in general does not hold. 

% Because of these shortcomings, we propose the following
% \emph{stronger} definition.
% \begin{definition}\label{def:strongbinding}
%   An unconditionally secure commitment scheme is called \emph{strongly
%     binding} against $\mathfrak{C}$ if for all $\{\dC_n\}_{n>0}\in
%   \mathfrak{C}$, there exists a classical binary random variable $D$
%   whose distribution \emph{cannot} by influenced by the (dishonest)
%   committer after the commit phase and with the property that the
%   committer's probability to successfully open the commitment to $1-D$
%   is negligible.
% \end{definition} \index{bit commitment!strong binding}

% Note that this definition still allows a committer to commit to a
% superposition and otherwise honestly follow the protocol. $D$ is
% then simply defined to be the outcome when the register, that carries
% the superposition, is measured.  On the other hand, the definition
% captures exactly what one expects from a commitment
% scheme, except that the bit, to which the committer can open the
% commitment, is not fixed right after the commit phase.
% However, once committed, the dishonest committer \emph{cannot influence} its
% distribution anymore, and thus this is not of any help to him,
% because he can always pretend not to know that bit.

% It is also clear that with this stronger notion of the binding
% property, the obvious reduction from a string- to a bit-commitment
% scheme by committing bit-wise can be proven secure: the $i$th
% execution of the bit commitment scheme guarantees a random variable
% $D_i$ such that the committer cannot open the $i$th bit commitment
% to $1-D_i$, and thus there exists a random variable $S$, namely $S
% = (D_1,D_2,\ldots)$ such that the committer cannot open the list of
% commitments to any other string than $S$.

In Section~\ref{sec:strongbinding}, we show that the bit commitment
\comm\ from Figure~\ref{fig:comm} as a matter of fact satisfies this
stronger and more useful notion of security. This turns out to be a
rather straightforward consequence of the security of the \OT\ scheme
from Chapter~\ref{chap:12OT}.


%%%%%%%%%%%%%%%%%%%%%%%%%%%%%%%%%%%%%%%%%%%%%%%%%%%%%%%%%%%%%%%%%%%%%%%%%%%%%
\section{Weak Binding of the Commitment Scheme} \label{sec:weakbinding}
%%%%%%%%%%%%%%%%%%%%%%%%%%%%%%%%%%%%%%%%%%%%%%%%%%%%%%%%%%%%%%%%%%%%%%%%%%%%%
\index{bit commitment!weak binding|(} In this section, we use the
techniques from the analysis of the \RabinOT protocol from
Chapter~\ref{chap:RabinOT} to prove our commitment scheme \comm\ (or
rather its purified version \eprcomm) weakly binding against
quantum-memory-bounded adversarial committers.

Note that the first two steps of \eprqot\ (from
Figure~\ref{fig:eprot}) and \eprcomm\ (i.e.~before the memory bound applies) are exactly the
same!  This allows us to reuse Corollary~\ref{cor:hadamard} and the
analysis of Section~\ref{sec:otsecurity} to prove the weakly binding
property of \eprcomm.
\begin{theorem}\label{thm:weakbinding}
For any $\gamma<\frac{1}{2}$,
\comm\ is perfectly hiding and weakly binding against $\mathfrak{C}_{\gamma}$.  
\end{theorem}
The proof is given below. It boils down to showing that essentially
$p_0(n) \leq 1 - \qp$ and $p_1(n) \leq 1 - \qt$. The weak binding property
then follows immediately from Corollary~\ref{cor:hadamard}. The
intuition behind $p_0(n) \leq 1 - \qp = 1 - \Qp(S^+)$ is that a committer has only a
fair chance in opening to $0$ if $x$ measured in the $+$-basis has large
probability, i.e., $x \not\in S^+$. The following proof makes this
intuition precise by choosing the $\varepsilon$ and $\delta$'s
correctly.
\begin{proof}
  It remains to show that \eprcomm\ is binding against
  $\mathfrak{C}_{\gamma}$. Let $\eps, \delta > 0$ be such that $\gamma
  + 2h(\delta) + 2\eps < 1/2$, where $h$ is the binary entropy
  function. Recall that the number $\ball{\delta n}$ of $n$-bit strings of Hamming-distance at
  most $\delta n$ from a fixed string is at most $2^{h(\delta) n}$. Let
  $R$ be the basis, determined by the bit that $\dC$ claims in
  Step~\ref{step:open}, and in which $\V$ measures the quantum state
  in Step~\ref{step:last}, and let $X$ be the outcome.
  Corollary~\ref{cor:hadamard} implies the existence of an event $\cal
  E$ such that $\P[\ev|R\!=\!+] + \P[\ev|R\!=\!\times] \geq 1 -
  \negl{n}$ and $\H_{\infty}(X|R\!=\!r,\ev) \geq (\gamma+2h(\delta) +
  2\eps) n$. Applying Corollary~\ref{cor:guess} (with constant $U$ and
  $\eps=0$), it follows that any guess $\hat{X}$ for $X$ satisfies
\begin{align*}
\P\big[ \hat{X} \in \ball{\delta n}(X) \,|\, R\!=\!r,\ev \big]
 &\leq 2^{-\frac{1}{2} (\H_{\infty}(X|X \in S^+)-\gamma n-1) + \log(\ball{\delta
 n}) } \leq 2^{-\eps n + \frac{1}{2}}.
\end{align*}
However, if $\hat{X} \not\in \ball{\delta n}(X)$ then sampling a
random subset of the positions will detect an error except with
probability at most $2^{-\delta n}$. Hence, writing $\qp \assign
\P[\ev|R\!=\!+]$ and $\qt \assign \P[\ev|R\!=\!\times]$,
$$
p_0(n) \leq (1-\qp) + \qp\cdot (2^{-\eps n + \frac12} + 2^{-\delta
  n}) \leq 1-\qp + \negl{n}
$$
and analogously $p_1(n) \leq 1-\qt + \negl{n} $. We conclude that
$$
p_0(n) + p_1(n) \leq  2 - \qp - \qt + \negl{n} \leq  1+ \negl{n} \, .
$$
\end{proof}
\index{bit commitment!weak binding|)}


%%%%%%%%%%%%%%%%%%%%%%%%%%%%%%%%%%%%%%%%%%%%%%%%%%%%%%%%%
\section{Strong Binding of the Commitment Scheme} \label{sec:strongbinding}
%%%%%%%%%%%%%%%%%%%%%%%%%%%%%%%%%%%%%%%%%%%%%%%%%%%%%%%%%
\index{bit commitment!strong binding|(} In this section, we reuse the
analysis of the \OT-protocol from Chapter~\ref{chap:12OT} to prove the
strong binding condition.

\begin{theorem}
  The quantum bit-commitment scheme \comm\ is $\eps$-binding
  according to Definition~\ref{def:strongbinding} against $\mathfrak{C}_{\gamma}$
  for a negligible (in $n$) $\eps$ if $\gamma < \frac14$.
\end{theorem}
% \begin{theorem} 
%   For any $\gamma < \frac14$, \comm\ is perfectly hiding and strongly
%   binding against $\mathfrak{C}_{\gamma}$.
% \end{theorem}
Intuitively, one can argue that $X$ has (smooth) min-entropy about $n/2$ given
$\Theta$.  The \index{min-entropy splitting lemma}
Min-Entropy Splitting Lemma implies that there exists
$D$ such that $X_{1-D}$ has smooth min-entropy about $n/4$ given
$\Theta$ and $D$. Privacy amplification implies that $F(X_{1-D})$ is
close to random given $\Theta, D, F$ and $\dC$'s quantum register of
size $\gamma n$, where $F$ is a \univ one-bit-output hash function,
which in particular implies that $\dC$ cannot guess $X_{1-D}$. The
formal proof is given below.
\begin{proof}
  It remains to show that \eprcomm\ is strongly binding against
  $\mathfrak{C}_{\gamma}$.  Let $\Theta \in \set{+,\times}^n$ be the
  random basis that would correspond to the choice of basis in the
  first step of \comm, i.e. $\theta_i = \set{+,\times}_{[b]}$ for $i
  \in I$ and $\theta_i = \set{+,\times}_{[1-b]}$ for $i \not\in I$. Let
  $X$ be the measurement outcome when $\V$ measures his halves of the
  EPR-pairs in basis $\Theta$.
  
Recall that $h(\cdot)$ denotes the binary Shannon entropy. Choose $\lambda, \lambda', \kappa$ and $\delta$ all positive, but
  small enough such that $\gamma \leq 1/4 - \lambda - \lambda' -
  2h(\delta) - 2 \kappa$, $h(\delta) \leq \lambda'-\kappa$, and
  $h(\delta) \leq \frac{\lambda^4}{32} -
  \kappa$. Before Step~\ref{step:open}, the overall state is given by
  the ccq-state $\rho_{X \Theta \dC}$ after $\dC$ has measured all but
  $\gamma n$ of his qubits, where $X$ describes the outcome of the
  verifier $\V$ measuring his part of the state in random basis
  $\Theta$.  From the uncertainty relation
  (Corollary~\ref{cor:uncertainty}), we know that $\hie{\eps}{X
    \mid \Theta} \geq (1/2 - 2\lambda )n$ for
  $\eps=2^{-\frac{\lambda^4}{32}n}$ % and $\eps'=2^{-\lambda' n}$ 
exponentially small in $n$.  Therefore, by
  Corollary~\ref{cor:ESL}, there exists a binary random variable $D
  \in \set{0,1}$ such that for $\eps'=2^{-\lambda' n}$, it holds that
\begin{align*}
\hie{\eps+\eps'}{X_{1-D} \mid \Theta D} &\geq (1/4 - \lambda - \lambda')n -1\\ 
&\geq (1/4 -\lambda - \lambda')n -1\\
&\geq \gamma n + 2h(\delta)n + 2 \kappa n -1 \, .
\end{align*}

Recall that $\ball{\delta n} \leq 2^{h(\delta) n}$. Applying
Corollary~\ref{cor:guess}, it follows that any guess $\hat{X}$ for $X_{1-D}$
satisfies
\begin{align*}
  \P\big[ \hat{X} \in \ball{\delta n}(X_{1-D}) \big] &\leq 2^{-\frac{1}{2}
    (\hie{\eps+\eps'}{X_{1-D}|\Theta D}-\gamma n-1) + \log(\ball{\delta
      n}) } + (2 \eps + 2 \eps') \ball{\delta n}\\
  &\leq 2^{-\frac{1}{2} (2 \kappa n - 2)} + 2 \cdot 2^{
    -\frac{\lambda^4}{32}n + h(\delta) n } + 2 \cdot 2^{ -
    \lambda' n + h(\delta) n} \\
  &\leq \frac12 2^{-\kappa n } + 2 \cdot 2^{ -\kappa n } + 2 \cdot 2^{
    -\kappa n} \, ,
\end{align*}
which is negligible by the choice of the parameters.
\end{proof}

\index{bit commitment!strong binding|)}
\index{bit commitment!binding|)}



%%%%%%%%%%%%%%%%%%%%%%%%%%%%%%%%%%%%%%%%%%%%%%%%%%%%%%%%%%%%%%%%%%%%%%%%%%%%%
\section{Weakening the Assumptions} \label{sec:weakassumptioncomm}
%%%%%%%%%%%%%%%%%%%%%%%%%%%%%%%%%%%%%%%%%%%%%%%%%%%%%%%%%%%%%%%%%%%%%%%%%%%%%
\index{weak quantum model|(}As argued earlier, assuming that a party can
produce single qubits (with probability~1) is not reasonable given
current technology. Also the assumption that there is no noise on the
quantum channel is impractical.  It can be shown that a
straightforward modification of \comm\ remains secure in the
$(\phi,\eta)$-weak quantum model as introduced in
Section~\ref{sec:weakass} (see also Section~\ref{sec:moreimperfect}),
with $\phi < \frac{1}{2}$ and $\eta < 1- \phi$.

\begin{myfigure}{h}
\begin{myprotocol}{\comm'$(b,\phi)$}
\item $\V$ picks $x \in_R \nbit$ and $\theta \in_R \{+,\times \}^n$ and
  sends $x_i$ in the corresponding bases $\ket{x_1}_{\theta_1},
  \ket{x_2}_{\theta_2}, \ldots, \ket{x_n}_{\theta_n}$ to~$\C$.
 \item $\C$ commits to the bit $b$ by measuring all qubits in basis
   $\{+,\times \}_{[b]}$. Let $x' \in \nbit$ be the result.
 \item To open the commitment, $\C$ sends $b$ and $x'$ to $\V$.
 \item $\V$ verifies that $x_i= x_i'$ for $i$ where $\theta_i = \{+,\times
   \}_{[b]}$. $\V$ accepts if and only if this is the case \emph{for all but
   a $\phi$-fraction of these positions}.
\end{myprotocol}
\caption{Protocol for noise-tolerant quantum bit commitment}\label{fig:commnoisy}
\end{myfigure}


The protocol \comm'\ in Figure~\ref{fig:commnoisy} is the same as
\comm\ from Figure~\ref{fig:comm} except that in the last
Step~\ref{step:last}, $\V$ accepts if and only if $x_i = x'_i$ for all
{\em but about a $\phi$-fraction} of the $i$ where $r_i = \{+,\times
\}_{[b]}$. More precisely, for all but a
$(\phi+\varepsilon)$-fraction, where $\varepsilon > 0$ is sufficiently
small.
\begin{theorem}\label{thm:weakcommsec}
  In the $(\phi,\eta)$-weak quantum model, \comm'\ is perfectly hiding
  and it is weakly binding against $\mathfrak{C}_{\gamma}$ for any $\gamma$
  satisfying $\gamma < \frac{1}{2}(1-\eta) - 2 h(\phi)$.
\end{theorem}
\begin{sketch} %[of Theorem~\ref{thm:weakcommsec}]
  Using \index{Chernoff's inequality} Chernoff's inequality (Lemma~\ref{lem:chernoff}), one can
  argue that for {\em honest} $\C$ and $\V$, the opening of a
  commitment is accepted except with negligible probability.  The
  hiding property holds using the same reasoning as in
  Lemma~\ref{lem:sec:hiding}. And the binding property can be argued
  essentially along the lines of Theorem~\ref{thm:weakbinding}, with
  the following modifications. Let $J$ denote the set of indices $i$
  where $\V$ succeeds in sending a single qubit. We restrict the
  analysis to those $i$'s which are in $J$. By 
  \index{Chernoff's inequality} Chernoff's inequality
  (Lemma~\ref{lem:chernoff}), the cardinality of $J$ is about
  $(1-\eta)n$ (meaning within $(1-\eta\pm \varepsilon)n$), except with
  negligible probability.  Thus, restricting to these $i$'s has the
  same effect as replacing $\gamma$ by $\gamma/(1-\eta)$ (neglecting
  the $\pm \varepsilon$ to simplify notation). Assuming that $\dC$
  knows every $x_i$ for $i \not\in J$, for all $x_i$'s with $i \in J$,
  he has to be able to guess all but about a $\phi/(1-\eta)$-fraction
  correctly, in order to be successful in the opening. Using
  Corollary~\ref{cor:guess}, we can show that for a correctly chosen
  $\delta >0$, the probability of guessing $\hat{X}$ within Hamming
  distance $\delta n$ to the real $X$ is negligible. Therefore, $\dC$
  succeeds with only negligible probability if the fraction of allowed
  errors $\phi/(1-\eta)$ is smaller than $\delta$, i.e.
$$
\phi/(1-\eta) < \delta \, ,$$ 
Additionally, in order for the machinery from Theorem~\ref{thm:weakbinding} to work, $\delta$ must be such that 
$$ \frac{\gamma}{1-\eta} + 2h(\delta) < \frac{1}{2} \, .
$$ 
$\delta$ can be chosen that way if 
$$
\frac{\gamma}{1-\eta} + 2 \, h\!\left( \frac{\phi}{1-\eta} \right) < \frac{1}{2} \, .
$$
Using the fact that $h(\nu p) \leq \nu h(p)$ for any $\nu \geq 1$
and $0 \leq p \leq \frac12$ such that $\nu p \leq 1$, this is clearly
satisfied if
$\gamma + 2 h(\phi) < \frac{1}{2}(1-\eta)$. 
\end{sketch}


\begin{theorem}\label{thm:weakcommsecstrong}
  In the $(\phi,\eta)$-weak quantum model, \comm'\ is perfectly hiding
  and it is strongly binding against $\mathfrak{C}_{\gamma}$ for any $\gamma$
  satisfying $\gamma < \frac{1}{4}(1-\eta) - 3 h(\phi) - \sqrt[4]{32
  \, h(\phi)}$.
\end{theorem}
\begin{sketch}
  The proof goes like the proof of Theorem~\ref{thm:weakcommsec}, but
  uses the techniques from Section~\ref{sec:strongbinding}. In order
  for those to work, we need to choose $\lambda, \lambda'$,
  and $\delta$ all positive and such that
\begin{align} \label{eq:conditions} \begin{split}
\frac{\phi}{1 - \eta} &< \delta,  \\
\frac{\gamma}{1 - \eta} + 2 h(\delta) + \lambda' + \lambda &< 1/4 \,
, \\
h(\delta) &< \lambda' \, , \\
h(\delta) &< \frac{\lambda^4}{32} \, .
\end{split} \end{align}
We verify that the assumption $\gamma < \frac{1}{4}(1-\eta) - 3 h(\phi) - \sqrt[4]{32
  \, h(\phi)}$ on $\gamma$ allows for that. Rearranging
the terms and using that $x < \sqrt[4]{x}$ for $0<x<1$ yields
\[
 \frac{\gamma}{1 - \eta} + 3 \frac{h(\phi)}{1 - \eta} +
 \sqrt[4]{32\frac{h(\phi)}{1- \eta}} < 1/4 \, .
\]
Using as in the previous proof the fact that $h(\nu p) \leq \nu h(p)$ for any
$\nu \geq 1$ and $0 \leq p \leq \frac12$ such that $\nu p \leq 1$, we
get that
\[
 \frac{\gamma}{1 - \eta} + 3 h\left(\frac{\phi}{1-\eta}\right) +
 \sqrt[4]{32 \, h\left(\frac{\phi}{1-\eta}\right)} < 1/4.
\]
That allows to choose $\delta > \frac{\phi}{1-\eta}$ such that
\[
 \frac{\gamma}{1 - \eta} + 2 h(\delta) +  h(\delta) + 
 \sqrt[4]{32 \, h(\delta)} < 1/4,
\]
and therefore, also $\lambda$ and $\lambda'$ can be chosen such that
the conditions \eqref{eq:conditions} are fulfilled.
\end{sketch}

\index{weak quantum model|)}
\index{bit commitment|)}


%%% Local Variables: 
%%% mode: latex
%%% TeX-master: "diss"
%%% End: 
