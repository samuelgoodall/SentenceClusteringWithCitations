\section{Related Work}
% adv
% blackbox
% query efficient score-based blackbox
% a. more info; b. not guarantee success c. #query 

\paragraph{Boundary-based Attack}
% A line of work that is most related to ours is using decision-based attacks.
Boundary Attack~\cite{brendel2017decision} is one of the first work that uses final decisions of a classifier to perform blackbox attacks. The attack process starts from the \sourceimage, which is classified as the adversarial \maliciousclass. Then it employs a reject sampling mechanism to find a \boundaryimage that still belongs to the \maliciousclass by performing random walk along the boundary. The goal is to minimize the distance between the \boundaryimage and the \targetimage. However, as the steps taken are randomly sampled, the convergence of this method is slow and the query number is large.

Several techniques have been proposed to improve the performance of Boundary Attack. \cite{brunner2019guessing,srinivasan2019black,guo2018low} propose to choose the random perturbation in each step more wisely instead of Gaussian perturbation, using Perlin noise, alpha distribution and DCT respectively. \cite{ilyas2018black,khalid2019red,liu2019geometry,chen2019hopskipjumpattack} propose a similar idea - approximating the gradient around the boundary using Monte Carlo algorithm. 
% In our work, we mainly follow the pipeline in \cite{chen2019hopskipjumpattack} because it establishes a good theoretic analysis for the model.

There are two other blackbox attacks which are not based on the boundary. \cite{cheng2018query} proposes to transform the boundary-based output into a continuous metric, so that the score-based attack techniques can be adopted. \cite{dong2019efficient} adopts evolution algorithm to achieve the decision-based attack against face recognition system.
% \Xiaojun{remove redundancy in the background part; mention other gradient estimation on BA} HopSkipJumpAttack~\cite{chen2019hopskipjumpattack} improves upon the basic random walk idea in the original Boundary Attack paper. It proposes an unbiased estimate of the gradient of the model on the decision boundary via a Monte-Carlo algorithm. The \queryimages are generated by adding the \boundaryimage with a set of noise vectors that are randomly sampled from the whole image space. Using the estimated gradient, the paper is able to perform more efficient update to get to a \boundaryimage that is closer to the \targetimage by taking a step towards the gradient direction and then performing binary search back to the decision boundary iteratively. This method reduces the query number compared with Boundary Attack, but since it is sampling from an extremely high-dimensional space(for example, ImageNet data samples lie in $224\times 224\times 3$ dimensional space), the number of queries required to get a fair estimation for updating is still large.
\vspace{-5mm}
\paragraph{Dimension Reduction in Score-based Attack}
Another line of work involves the dimension reduction techniques only for the score-based attacks, which requires access to the prediction of confidence for each class.
In~\cite{guo2018low}, the authors draw intuition from JPEG codec~\cite{wallace1992jpeg} image compression techniques and propose to use discrete cosine transform (DCT) for generating low frequency adversarial perturbations to assist score-based adversarial attack.
AutoZoom~\cite{tu2019autozoom} trains an auto-encoder offline with natural images and uses the decoder network as a dimension reduction tool. Constrained perturbations in the latent space of the auto-encoder are generated and passed through the decoder. The resulting perturbation in the image space is added to the benign one to obtain a query sample. 