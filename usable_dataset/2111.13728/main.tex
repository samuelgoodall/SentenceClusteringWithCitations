\documentclass{article}

\usepackage{arxiv2c}

\usepackage{amsmath,amssymb,amsfonts}

\usepackage{booktabs}   %
\usepackage{subcaption} %
\usepackage[utf8]{inputenc}

\usepackage{amsthm}
\usepackage{color}
\usepackage{proof}
\usepackage{amsmath}
\usepackage{mathtools}
\usepackage{multirow}
\usepackage{caption}
\usepackage{subcaption}
\usepackage{thmtools}
\usepackage{tikz}
\usepackage{thm-restate}
\newcommand{\qstate}[1]{\ensuremath{\vert #1 \rangle}}
\newcommand{\vsep}{\ensuremath{\ \vert\ }}

\newcommand{\myFrameworkName}{\textit{QECV}}
\newcommand{\langname}{\textit{QECV-Lang}}
\newcommand{\assnname}{\textit{QECV-Assn}}

\newcommand{\svar}{\ensuremath{\mathcal{s}}}
\newtheorem{theorem}{Theorem}[section]
\newtheorem{proposition}[theorem]{Proposition}
\newtheorem{corollary}[theorem]{Corollary}
\newtheorem{definition}[theorem]{Definition}
\newtheorem{lemma}[theorem]{Lemma}
\newtheorem{example}[theorem]{Example}
\newtheorem{program}{Program}[section]
\newtheorem{assume}{Assumption}[]
\newtheorem*{remark}{Remark}

\newcounter{cnt}
\newcommand\showcnt{\addtocounter{cnt}{1}\thecnt}
\newcommand{\conf}[1]{\ensuremath{\langle #1 \rangle}}
\newcommand{\denot}[1]{\ensuremath{[\![ #1 ]\!]}}
\newcommand{\ket}[1]{\ensuremath{\vert #1 \rangle}}
\newcommand{\bra}[1]{\ensuremath{\langle #1 \vert}}
\newcommand{\braket}[2]{\ensuremath{\langle #1 \vert #2 \rangle}}
\newcommand{\qif}[3]{\ensuremath{\textbf{if}\ M[#1]\ \textbf{then}\ #2\ \textbf{else}\ #3\ \textbf{end}}}
\newcommand{\stabnum}{\ensuremath{w}}
\newcommand{\qwhilelang}{quantum \textbf{while}-language}
\newcommand{\postpone}[1]{{Details are postponed to Appendix~{#1}}}
\newcommand{\myquad}{{\color{white}-}}
\newcommand{\qifn}[3]{\ensuremath{\textbf{if}\ M[#1]\ \textbf{then}\\\myquad #2\ \\\textbf{else}\\\myquad#3\ \\\textbf{end}}}

\newcommand{\prog}{\text{Prog}}
\newcommand{\projector}{P}

\usepackage{syntax}
\setlength{\grammarparsep}{15pt plus 1pt minus 1pt} %
\setlength{\grammarindent}{4em} %
\usepackage[many]{tcolorbox}
\newcommand{\myfontsize}{9pt}
\newcommand{\mylinesize}{9pt}
\newcommand{\myproof}[1]{{\noindent\textit{Proof}. #1 \hfill \qed}}
\newcommand{\nothmskip}{}
\newcommand{\rulesep}{\unskip\ \vrule\ }


\title{\myFrameworkName: A Verification Framework for Quantum Error Correction Codes}         %

\author{%
	Anbang Wu \\
	Department of Computer Science\\
	University of California, Santa Barbara \\
	\texttt{anbang@ucsb.edu} \\
	\And
	Gushu Li \\
	Department of Electrical \& Computer Engineering\\
	University of California, Santa Barbara \\
	\texttt{gushuli@ece.ucsb.edu} \\
	\AND
	Hezi Zhang \\
	Department of Computer Science \\
	University of California, Santa Barbara \\
	\texttt{hezi@ucsb.edu} \\
	\AND
	Gian Giacomo Guerreschi \\
	Intel Labs \\
	Santa Clara, California \\
	\texttt{gian.giacomo.guerreschi@intel.com}
	\AND
	Yuan Xie \\
	Department of Electrical \& Computer Engineering\\
	University of California, Santa Barbara \\
	\texttt{yuanxie@ucsb.edu} \\
	\AND
	Yufei Ding \\
	Department of Computer Science\\
	University of California, Santa Barbara \\
	\texttt{yufeiding@cs.ucsb.edu} \\
} 

\begin{document}
	
	\maketitle

\begin{abstract}

Quantum Error Correction (QEC) is essential for the functioning of large-scale fault-tolerant quantum computers, and its implementation is a very sophisticated process involving both quantum and classical hardware. Formulating and verifying the decomposition of logical operations into physical ones is a challenge in itself.
In this paper, we propose {\myFrameworkName}, a verification framework that can efficiently verify the formal correctness of stabilizer codes, arguably the most important class of QEC codes.
{\myFrameworkName} first comes with a concise language, {\langname},  where stabilizers are treated as a first-class object, to represent QEC programs.
Stabilizers are also used as predicates in our new assertion language, {\assnname}, as logical and arithmetic operations of stabilizers can be naturally defined.
We derive a sound quantum Hoare logic proof system with a set of inference rules for {\myFrameworkName} to efficiently reason about the correctness of QEC programs.
We demonstrate the effectiveness of {\myFrameworkName} with both theoretical complexity analysis and in-depth case studies of two well-known stabilizer QEC codes, the repetition code and the surface code.

\end{abstract}

\twocolumn
\begin{figure*}[htbp]
    \centering
    % \includegraphics[width=\textwidth]{figs/pipeline.pdf}
    \includegraphics[width=\textwidth]{figs/arxiv_ver/pipeline.jpeg}
    \caption{Pipeline of \name. In this example, the attack goal is to obtain an \advimage that looks like a cat (\targetimage) but be misclassified as a fish ($y_{mal}$). We start from a source-image together with an optimized subspace. We then iterativelly perform gradient estimation with queries, move along the estimated direction, and project the new instance to the decision boundary by binary search towards the \targetimage till converge. The grey solid arrows indicate steps within each iteration. 
    In particular, we show a toy example of how the source-image (purple rectangles) is moved towards the \targetimage (green rectangles), while the intermediate projected \boundaryimage is shown as red rectangles.}
    \label{fig:pipeline}
    % \vspace{-0.5cm}
\end{figure*}

\section{Introduction}
% 1. machine learning is vulnerable, many attack
% 2. blackbox attack is more realistic, transfearbility -- it cannot that it works, 
% gradient estimation - finite diffxxx, but they require logit output
% only based on output -- example of APIs that allow ***
% 3. challenges, 1. #queries, 2. how to reduce dimension, litterateurs -- three perspectives are important, so we plan to make a comprehensive studies about these. 
% 4. in the meantime, what's the key factor for such reduction -- mention the intuition about the theorem and rho
% 5.  exp: datasets, and models, APIs, beat all the stoa
% 6. contribution list: 1. propose blackbox from three xxxxx, and we did comprehensive studies for three types of query reduction; 2. from theoretic perspective, xxxx show xxx is important-- much smaller #queries, and lower magnitude of perturbation, 100% attack success; 3. extensive experiments for models/datasets. 4. real-world APIs

Recent developments of machine learning (ML), especially deep neural networks (DNNs), have advanced a number of real-world applications, including object detection~\cite{ren2015faster}, drug discovery~\cite{chen2018rise}, and robotics~\cite{lenz2015deep}. In the meantime, several safety-critical applications have also adopted ML, such as autonomous driving vehicles~\cite{chen2015deepdriving} and surgical robots~\cite{richter2019open,mlsurgicalrobotics}.
However, recent research have shown that machine learning systems are vulnerable to \emph{adversarial examples}, which are inputs with small magnitude of adversarial perturbations added and therefore cause arbitrarily incorrect predictions during test time~\cite{eykholt2017robust,xiao2018generating,carlini2017towards,goodfellow2014explaining,chaowei2018characterizing,chaowei2018spatially}.
Such adversarial attacks have led to great concerns when applying ML to real-world applications. Thus in-depth analysis of the intrinsic properties of these adversarial attacks as well as potential defense strategies are required.

First, such attacks can be categorized into whitebox and blackbox attacks based on the attacker's knowledge about the victim ML model. In general, the whitebox attacks are possible by leveraging the gradient of the model --- methods like fast gradient sign method (FGSM) ~\cite{goodfellow2014explaining}, optimization based attack~\cite{carlini2017towards}, projected gradient descent based method (PGD)~\cite{madry2017towards} have been proposed. However, whitebox attack is less practical, given the fact that most real-world applications will not release the actual model they are using. In addition, these whitebox attacks are shown to be defendable~\cite{madry2017towards}.
As a result, blackbox adversarial attack have caught a lot of attention in these days.
In blackbox attack, based on whether an attacker needs to query the victim ML model, there are query-free (e.g. transferability based attack) and query-based attacks. Though \emph{transferability} based attack does not require query access to the model, it assumes the attacker has access to the large training data to train a substitute model, and there is no guarantee for the attack success rate. The query based attack includes score-based and boundary-based attacks. Score-based attack assumes the attacker has access to the class probabilities of the model, which is less practical compared with boundary-based attack which only requires the final model prediction, while both require large number of queries. 


% In this paper, we propose a Query-Efficient Boundary-based blackbox Attack (\name) to estimate model decision boundary given only the classification output by leveraging the intrinsic dimensionalities of inputs. 
In this paper,  we propose Query-Efficient Boundary-based blackbox Attack (\name) based only on model's final prediction labels as a general framework to minimize the query number.
Since the gradient estimation consumes the majority of all the queries, the main challenge of reducing the number of queries for boundary-based blackbox attack is that a high-dimensional data (e.g. an image) would require large number of queries to probe the decision boundary. As a result, we propose to search for a small representative subspace for query generation.
% that can serve as the supports for the original gradient space. \Huichen{supports?}
In particular, queries are generated by adding perturbations to an image. We explore the subspace optimization methods from three novel perspectives for perturbation sampling: 1) spatial, 2) frequency, and 3) intrinsic component.
The first one leverages spatial transformation (e.g. linear interpolation) so that the sampling procedure can take place in a low-dimensional space and then project back to the original space.
The second one uses intuition from image compression literature and samples from low frequency subspace and use discrete consine transformation (DCT)~\cite{guo2018low} to project back.
The final one performs scalable gradient matrix decomposition to select the major principle components via principle component analysis (PCA)~\cite{wold1987principal} as subspace to sample from.
% The resulting boundary-based attack from the three methods are called \name-S, \name-F, and \name-I respectively.
% The first one is the spatial transformed subspace. We first sample a low-dimensional vector and then apply spatial transformation (e.g. linear interpolation) to project it back to the original space to get perturbations. 
% Secondly, we propose the low-frequency subspace enabled blackbox attack (\name-F), which leverage discrete consine transformation (DCT)~\cite{guo2018low} to select the low frequency subspace to perform efficient queries against the victim model.
% Finally we also propose the intrinsic component subspace based blackbox attack (\name-I) by performing scalable gradient matrix decomposition to select the major principle components via principle component analysis (PCA)~\cite{wold1987principal} as subspace to conduct the queries.
In addition
% to the three proposed query efficient \name methods
, we theoretically prove the optimality of them on estimating the gradient compared with estimating the gradient directly over the original space.


To demonstrate the effectiveness of the proposed blackbox attack \name methods, 
% , and analyzing different dimension reduction methods, 
we conduct extensive experiments on high dimensional image data including ImageNet~\cite{deng2009imagenet} and CelebA~\cite{liu2018large}. We perform attacks on the ResNet model~\cite{he2016deep}, and show that compared with the-state-of-the-art blackbox attack methods, the different variations of \name can achieve lower magnitude of perturbation with smaller number of queries (attack success rate 100\%). 
\Huichen{I deleted one sentence here because its expression is not accurate.}
% In particular, the \name-S performs slightly better in terms of achieving lower magnitude of perturbation.
In order to show the real-world impact of the proposed attacks, we also perform \name against online commercial APIs including MEGVII Face++\cite{facepp-compare-api} and Microsoft Azure\cite{azure-detect-api}. Our methods can successfully attack the APIs with perturbations of reasonable magnitude.
Towards these different subspaces, our conjecture is that the over-all performance on different subspaces depends on multiple factors including dataset size, model smoothness, adversarial attack goals etc. Therefore, our goal here is to make the first attempt towards providing sufficient empirical observations for these three subspaces, while further extensive studies are required to compare different factors of these subspaces, as well as identifying new types of subspaces.
% We show that we can generate adversarial examples by transferring \name against these real-world applications.
% \bo{add other proud results if any?}

% \Huichen{Do we need to put this paragraph (as in the rebuttal) here?}

% Our intuition for solving this challenge is through dimension reduction. If we can find a small subspace that is representative of the whole gradient space, i.e., most of the gradient vectors lie in this subspace, we can sample from the subspace and reduce the query number. In this paper, we prove a theorem and theoretically show the key to reducing queries is finding the bases that span a good subspace. We further propose three kinds of dimension reduction methods based on hyper pixels, frequency domain, and intrinsic component. Experiments are done on three public datasets: CIFAR10, ImageNet, and CelebA. The results show that compared with state-of-the-art methods, our methods require fewer queries to get to smaller perturbations while sustaining an attack success rate of 100\%. We also successfully attack two real-world APIs: FacePlusPlus and Microsoft Azure.


% We theoretically show why estimating model gradient is not efficient in terms of blackbox attack, and provide the optimality analysis for our dimension reduced decision boundary estimation.

% \Huichen{General information about machine learning/image recognition/face recognition}
% The recent development of machine learning systems has enabled the use of these algorithms in many security-sensitive applications. For example, face recognition is used to authorize log-ins of digital devices and even payment authentications\Huichen{cite}.

% % \Huichen{Illustrate what is a decision-based attack as opposed to other attacks}
% Research and findings bring adversarial examples against machine learning models to the focus of a broad audience\Huichen{cite. maybe also including news reports}. The fact that tiny human-imperceptible perturbations added onto images would alter the predictions of the machine learning systems reveals their vulnerabilities.
% There have been extensive amount of papers discussing the attacks and defenses under various settings, including different attacker goals and threat models. 
% Threat models are generally divided into two categories: white-box and black-box. 
% Under a white-box setting, the attacker has full knowledge about the model's architecture and parameter weights to perform an attack. Thus they can define a loss function with respect to the attack goal and leverage the gradient information of the model to optimize the adversarial example in an end-to-end fashion.
% The more practical assumption 
% on the attacker side however, is the black-box setting where the attacker's capability is limited. T
% As opposed to the white-box setting where the attacker knows everything, the black-box setting is more practical to assume the model architecture is hidden from the attacker and only the output can be accessed. 

% In a score-based attack setting, the victim model's output logits are known to the attackers. The gradients can be estimated by finite difference methods \Huichen{cite} and used to update the attack instance.
% \Huichen{Existing challenges}

% However, systems in the real world tempt to only give a hard decision. For example, in face recognition system for authentication on phones\Huichen{todo: cite}, the system either recognizes the legitimate user and unlocks, or rejects the attempt. It does not present a `score' for the likelihood it believes the person has access to log in.
% This black-box setting is called decision-based attack since it depends only on the final decision of the model, and it is the most realistic among all the threat models. It is also the most challenging one since the hard labels reveal limited information about the model's gradients. Existing works approach this problem by two major directions: transfer-based and boundary-based. 

% Transfer-based approaches train their own substitute models with no direct connection with the victim model. The adversarial examples generated on the substitute models is likely to attack successfully on victim models as well. The substitute model can be an ensemble of various models in order to boost the performance. This method relies heavily on the transferability between models and does not have guarantee on the attack success rate. Also, this attack cannot use the access to the victim model's decisions for improvement. 
% Boundary-based approaches, on the other hand, can make use of the model predictions. This kind of method starts from a point that is predicted as the attacker's desired class, and employs a rejection sampling process to find better points, so the attack success rate is always 100\%. What remains to be optimized is the magnitude of perturbations, which is the distance between the attack sample and a sample with desired appearance. 
% The main challenge of existing boundary-based methods is that they require large number of queries for this optimization. 

% This is due to the large number of dimensions of the image pixel gradient space. For example, in ImageNet, the gradient of a 3-channel image of size $224\times 224$ has a dimension of over 150 k. It is not easy to find a good direction for updates (gradient directions) by randomly sampling from the whole high-dimensional space.
% This makes them unrealistic in practice, since real-world systems would likely reject tens of thousands of incoming queries. 
% Our intuition for solving this challenge is through dimension reduction. If we can find a small subspace that is representative of the whole gradient space, i.e., most of the gradient vectors lie in this subspace, we can sample from the subspace and reduce the query number. In this paper, we prove a theorem and theoretically show the key to reducing queries is finding the bases that span a good subspace. We further propose three kinds of dimension reduction methods based on hyper pixels, frequency domain, and intrinsic component. Experiments are done on three public datasets: CIFAR10, ImageNet, and CelebA. The results show that compared with state-of-the-art methods, our methods require fewer queries to get to smaller perturbations while sustaining an attack success rate of 100\%. We also successfully attack two real-world APIs: FacePlusPlus and Microsoft Azure.

% \Huichen{The contributions of our work}
The \textbf{contributions} of this work are summarized as follows: 
% 1) We propose a general query-efficient blackbox attack \name to reduce the number of queries based on boundary-based attack;
% 2) We propose three different subspace optimization-based blackbox attack approaches, including spatial transformed attack (\name-S), low frequency attack (\name-F), and intrinsic component based attack (\name-I);
1) We propose a general Query-Efficient Boundary-based blackbox Attack \name to reduce the number of queries based on boundary-based attack. The \name contains three variations based on three different representative subspaces including spatial transformed subspace, low frequency subspace, and intrinsic component subspace;
2) We theoretically demonstrate that gradient estimation in the whole gradient space is inefficient in terms of query numbers, and we prove the optimality analysis for our proposed query-efficient gradient estimation methods;
3) We conduct comprehensive experiments on two high resolution image datasets: ImageNet and CelebA. All the different variations of \name outperform the state-of-the-art baseline method by a large margin;
4) We successfully attack two real-world APIs including Face++\cite{facepp-compare-api} and Azure\cite{azure-detect-api} and showcase the effectiveness of \name.

% \begin{enumerate}
    % \item We propose a general query efficient blackbox attack \name to reduce the number of queries based on boundary-based attack;
    % \item We theoretically demonstrate that only gradient estimation is insufficient for performing blackbox attacks, and we provide the optimality analysis for our proposed query efficient decision boundary estimation;
    % \item We propose three different subspace optimization based blackbox attack approaches, including spatial transformed attack (\name-S), low frequency attack (\name-F), and intrinsic component based attack (\name-I);
    % \item We conduct comprehensive experiments on two high resolution image datasets: ImageNet and CelebA. DIfferent variations of \name outperform the state-of-the-art baselines by a large margin;
    % \item We also successfully attack two real-world APIs including xxxx\bo{fill} and showcase the effectiveness of \name.
% \end{enumerate}




\section{Background}
In this section, we introduce the background for our work.
We summarize our key notations in Table~\ref{tab:notations}.
We do not cover the basics of quantum computing (e.g.,  density operator, unitary transformation) and recommend~\cite{nielsen2002quantum} for reference.

\begin{table}[b]
\caption{Notation used in this paper.}\label{tab:notations}
\resizebox{0.45\textwidth}{!}{
\begin{tabular}{p{0.2\linewidth} | p{0.75\linewidth}}
$q$, $\bar{q}$ & qubits, a set of qubits, respectively; \\\hline
$\ket{\psi}$, $\ket{\phi}$, $\ket{0}$, $\ket{1}$, $\ket{+}$, $\ket{-}$ & pure quantum states; \\\hline
$\rho$, $\ket{\psi}\bra{\psi}$ & density operators; \\\hline
$U$ & unitary transformations; \\\hline
$O, M$ & Observable; Use $M$ to stress measurement;\\\hline
$\mathcal{H}$ & the Hilbert space of quantum states;  \\\hline
$\mathcal{D(H)}$ & the set of partial density operators on $\mathcal{H}$; \\
\end{tabular} }
\end{table}

\subsection{Quantum Error Correction and Stabilizer}

Most QEC codes consist of three stages: encoding, decoding, and error correction. The encoding protocol projects an unprotected state of the data qubits into the subspace generated by the logical states. The decoding protocol detects potential errors by performing parity measurements on data qubits. Finally, the correction protocol removes the errors by driving the quantum state back to the logical subspace.



\noindent\textbf{Stabilizer}. 
The stabilizer formalism proposed by Gottesman~\cite{Gottesman1997StabilizerCA} provides a unifying description of many QEC codes. 
Given a $n$-qubit state $\ket{\psi}$ and a Pauli string $s \in \otimes^n \{I,X,Y,Z\}$, %
 we say that $s$ is a \textit{stabilizer} of $\ket{\psi}$, or $\ket{\psi}$ is \textit{stabilized} by $s$, if $s\ket{\psi} = \ket{\psi}$.
 When the stabilized state $\ket{\psi}$ is clear in the context, we will simply say that $s$ is a stabilizer, without referring to the stabilized state. 
We then can use multiple stabilizers to naturally identify a subspace, which is the intersection of the stabilizers' projection subspaces, to represent the logical states.
Moreover, stabilizers can also be observables and they can be measured to ascertain whether the state of the data qubits is in the correct subspace. 
These measurements are called \textbf{stabilizer measurements}. 
Another advantage of the stabilizer formalism is that it can describe standard quantum error channels on stabilized states.
Once the stabilizers are determined, the QEC encoding, decoding, and error correction can be easily derived.

Due to the centrality of stabilizers in QEC codes, including them as fundamental concepts of the verification framework is a promising direction. 
Benefiting from the expressiveness of the stabilizer formalism, we believe that our verification framework for quantum stabilizer codes will be applicable to many existing QEC codes and will help designing novel implementations of fault-tolerant operations. 

In the writing time of this paper, we also notice that Rand, Sundaram, Singhal and Lackey~\cite{Rand2021StaticAO, Rand2021ExtendingGT, Rand2021GottesmanTF} develop an elegant type-checking system for 
general quantum programs based on the stabilizer formalism. %
Despite that we share same high-level insights in utilizing stabilizer formalism~\cite{Gottesman1997StabilizerCA}, we differ significantly in the overall optimization goal and the entire design framework. 
In addition, their work only consider quantum circuits and cannot deal with branch statements (e.g., if and while). 
These statements are indispensable for QEC programs. 
In contrast, we can handle branch statements by incorporating stabilizer variables in the design of quantum predicate logic. Last but not least, this paper also develops a compact language for QEC programs while their work follows the vanilla quantum circuit language~\cite{nielsen2002quantum}.








\subsection{Quantum Program Language}
The {\qwhilelang} proposed by Ying~\cite{Ying2012FloydhoareLF} provides a universal description of purely-quantum programs without classical variables. It focuses on characterizing basic quantum program structures
and its syntax is defined in Backus-Naur form as follows:


%
\begin{tcolorbox}[colback=yellow!10!white,
                  colframe=white!20!black]
\begin{grammar}
%  
\label{equ:qwhile}
  \let\syntleft\relax
  \let\syntright\relax
%
<$\prog$> $\Coloneqq$ \textbf{skip}
\vsep $q \coloneqq \qstate{0}$
\vsep $\bar{q} \coloneqq U[\bar{q}]$
\vsep $\prog_1;\prog_2$
\alt $\textbf{case}\ M[\bar{q}] = \overline{m \to \prog_m}\ \textbf{end}$
\alt $\textbf{while}\  M[\bar{q}]=1\ \textbf{do}\ \prog_1\ \textbf{done}$
%
\end{grammar}
\end{tcolorbox}
%
Here $\prog$ plays the role of a statement of the {\qwhilelang}, but can also indicates the full program when seen as a sequence of statements. In the expressions above, $q$ denotes a quantum variable and $\bar{q}$ represents a quantum register associated with a finite number of quantum variables. The language constructs above are explained as follows: 
\setcounter{cnt}{0}
(\showcnt)  \textbf{skip} does nothing; 
(\showcnt) $q \coloneqq \qstate{0}$ prepare quantum variable $q$ in state $\qstate{0}$; 
(\showcnt) $\bar{q} \coloneqq U[\bar{q}]$ perform unitary operation $U$ on the quantum registers $\bar{q}$; 
(\showcnt) $\prog_1;\prog_2$ is the sequencing of statements; 
(\showcnt) $\textbf{case}\ M[\bar{q}] = \overline{m \to \prog_m}\ \textbf{end}$ measures the quantum variables in $\bar{q}$ with semi-positive Hermitian operators $M=\{M_0,M_1, \cdots, M_m\}$ and executes program $\prog_m$ if the measurement outcome is $m$; 
(\showcnt) $\textbf{while} \; M[\bar{q}]=1\ \textbf{do}\ \prog_1\ \textbf{done}$ measures qubits $\bar{q}$ and executes $\prog_1$ if the measurement outcome is 1. If the measurement outcome is 0, the while loop terminates. Here $M$ is assumed to have only two possible outcomes, $m=0,1$.

The semantics of {\qwhilelang} is developed by assuming partial density operators as quantum program states. 
The details  can be found in ~\cite{Ying2012FloydhoareLF}.

%
\subsection{Quantum Hoare Logic and Quantum Predicates}
\textit{Quantum Hoare logic}~\cite{Ying2012FloydhoareLF} provides a syntax-directed proof system for reasoning about quantum program correctness. The basic computing unit of quantum Hoare logic is the \textit{Hoare tripe}, which is in the form of $\{A\}\prog\{B\}$. Here $A$ and $B$ are \textit{quantum predicates}, $\prog$ is the quantum program. $A$ is often called \textit{precondition} while $B$ is called \textit{post-condition}. The general meaning of the quantum Hoare tripe is that, if the input state satisfies $A$, then the output state of $\prog$ satisfies $B$. The exact mathematical interpretation of the quantum Hoare tripe depends on the type of the predicates used.

 
A general quantum predicate is a Hermitian operator $O$~\cite{Ying2012FloydhoareLF} where $0 \le Tr(O\rho) \le 1, \forall \rho \in \mathcal{D(H)}$. %
A quantum state $\rho$ satisfies the quantum predicate $O$ depending on the value of $Tr(O\rho)$, which represents the expectation value of $O$ on state $\rho$. 
In practice, measuring $Tr(O\rho)$ can be very time-consuming . 
This can be avoided by restricting the predicate from a general Hermitian operator to a projection operator $\projector$ with the property that $\projector^2 = I$.
We say that a quantum state $\rho$ satisfies predicate $\projector$ (denoted by $\rho \models \projector$) if $\projector\rho = \rho$. %
A projection operator $\projector$ can also be described by its subspace with eigenvalue +1, namely $S_{\projector}= \{ \ket{\psi} s.t. \ket{\psi} = \projector\ket{\psi} \}$ in the qubits' Hilbert space $\mathcal{H}$. Birkhoff and Neumann~\cite{Birkhoff1936TheLO} define a quantum logic on the set of subspaces in $\mathcal{H}$ in which, for example, the logical {\bf and} corresponds to the intersection of subspaces. This construction induces logic operations %
on projection operators by considering the equivalent operation on the associated subspaces.







\section{{\langname}}
In this section, we introduce {\langname}, a concise language for QEC Programs. We define its syntax, operational semantics, and denotational semantics. 



\subsection{Syntax}\label{subsect: syntax}
We restate the notation for quantum variables as follows:
Define $\text{qVar}$ as the set of quantum variables,  $q$ as a metavariable ranging over quantum variables, and $\bar{q}$ to be a quantum register associated with a finite set of distinct quantum variables. We denote the state space of $q$ by $\mathcal{H}_q$ which is a two-dimensional Hilbert space spanned by the computational basis states $\{\qstate{0}, \qstate{1}\}$. The state space of $\bar{q}$ is the tensor product of Hilbert spaces
$\mathcal{H}_{\bar{q}} = \otimes_{q\in \bar{q}}\mathcal{H}_q$.


Logical operations in QEC codes are often associated with changes in the set of stabilizer measurements. For example, the surface code~\cite{Fowler2012SurfaceCT} frequently turns on and turn off specific stabilizer measurement circuits to implement logical gates. Besides, the outcomes of stabilizer measurements act as signals for error correction. By introducing a stabilizer variable, which  represents a stabilizer measurement circuit without the need of specifying its actual implementation, we can greatly simplify the description of QEC programs. %
We define the notations for the stabilizer variable as follows:

Define $S$ as the set of stabilizers on $\text{qVar}$, $s$ as an individual stabilizer in $S$, $\text{sVar}$ as the set of stabilzer variables, and $\svar$ as a metavariable ranging over $\text{sVar}$. To avoid $S$ being uncountable, we assume that every $s \in S$ only involves a finite number of qubits. The range of values for the stabilizer variable $\svar$ is $S\cup -S \cup iS \cup -iS$, where $i$ is the imaginary unit.



We define the syntax of {\langname} as follows:
\begin{tcolorbox}[colback=yellow!10!white,
                  colframe=white!20!black]
%
\begin{grammar}

\label{equ:qeclang}
  \let\syntleft\relax
  \let\syntright\relax
%
<$\prog$> $\Coloneqq$ \textbf{skip}
\vsep $q \coloneqq \qstate{0}$
\vsep $\bar{q} \coloneqq U[\bar{q}]$
\vsep $\svar \coloneqq s_e^u$
\alt $\prog_1;\prog_2$
\alt $\textbf{if}\ M[\svar,\bar{q}]\ \textbf{then}\  \prog_1\ \textbf{else}\ \prog_0\ \textbf{end}$
\alt $\textbf{while}\  M[\svar,\bar{q}]\ \textbf{do}\ \prog_1\ \textbf{done}$ %

<$s_e^u$> $\Coloneqq$ $\pm s$ | $\pm i\,s$ | $\pm \svar$
%
\end{grammar}
\end{tcolorbox}




\setcounter{cnt}{0}
The proposed language constructs consisting of instructions as follows: 
(\showcnt)  \textbf{skip} does nothing; 
(\showcnt) $q \coloneqq \qstate{0}$ resets quantum variable $q$ to ground state $\qstate{0}$; 
(\showcnt) $\bar{q} \coloneqq U[\bar{q}]$ perform unitary operation $U$ on quantum register $\bar{q}$; 
(\showcnt) $\svar \coloneqq s_e^u$ assigns a unary stabilizer expression $s_e^u$ to the stabilizer variable $\svar$;
(\showcnt) $\prog_1;\prog_2$ is the sequencing of programs; 
(\showcnt) $\textbf{if}\ M[\svar,\bar{q}]\ \textbf{then}\  \prog_1\ \textbf{else}\ \prog_0\ \textbf{end}$ perform the stabilizer measurement represented by $\svar$ on qubits $\bar{q}$ (or in short, measures $\svar$ on qubits $\bar{q}$)
measures qubits in $\bar{q}$ with the stabilizer $\svar$ and executes program $\prog_1$ if the measurement outcome is $1$. If the measurement outcome is $-1$, i.e. \textbf{the measured state is in the -1 eigenspace of} $\svar$, we first execute $\svar=-\svar$ which flips the sign of $\svar$, then execute $\prog_0$. 
We can see (6) as a short-term for $\textbf{if}\ M[\svar,\bar{q}]=1\ \textbf{then}\ \prog_1\ \textbf{else}\ \svar \coloneqq -\svar ;\ \prog_0\ \textbf{end}$.
(\showcnt) $\textbf{while}\  M[\svar,\bar{q}]\ \textbf{do}\ \prog_1\ \textbf{done}$ measures $\svar$ on qubits $\bar{q}$, and perform $\prog_1$ if the measurement outcome is 1. If the measurement outcome is -1, the sign of $\svar$ will be flipped automatically, and then the while loop terminates. 


The language constructs above are similar to those of the quantum \textbf{while}-language, except the part associated with stabilizer variables. Stabilizer variables can be used to describe operations on stabilizers.
For example, to turn off one stabilizer measurement circuit in a QEC program, we can simply set $\svar = {\rm I}$. The stabilizer variable can also serve to inform the error correction procedure. Every time we detect one or more stabilizer variables with a negative sign (this may happen after a stabilizer measurement), 
the decoder knows that at least one error affected the physical circuit. It proceeds to identifying the specific error and applies the corresponding correction. We define the error correction protocol as a function over stabilizer variables as follows:
\begin{definition}[Error correction protocol] Define \\
$\textbf{correct}(\svar_0,\svar_1,\cdots)$ as an error decoding and correction protocol by measuring $\svar_0,\svar_1,\cdots$.
\end{definition}

As an example, we present a snippet of {\langname} code that corresponds to  one stabilizer measurement of the surface code, where the parity qubit $s$ is connected to four data qubits $\{q_0,q_1,q_2,q_3\}$ with one Z-type stabilizer. Figure~\ref{fig:one-stabilizer}(a) shows the target stabilizer measurement circuit, while panel (b) shows an error correction program based on the stabilizer measurement in (a).  %
The basic idea of the program in Figure~\ref{fig:one-stabilizer}(b) is that, if the state of $q_0q_1q_2q_3$ is not stabilized by the stabilizer variable $\svar_0$, then some errors have happened and the QEC program should correct them.

Comparing to the \qwhilelang, {\langname} avoids the explicit introduction of parity qubits and the  implementation of stabilizer measurement circuits. Instead, it simply provides the stabilizer to measure as the value of variable $\svar$.
This approach makes {\langname} programs very flexible and independent from the specific implementation of stabilizer measurements. The latter would, for example, depend on architectural properties like the underlying hardware connectivity~\cite{Chamberland2020TopologicalAS}.

While being optimized for QEC codes, {\langname} contains all the program constructs necessary to describe general quantum programs. Actually, all programs in {\qwhilelang} can be translated into {\langname} by setting the stabilizer variable to a Pauli Z operator when qubits need to be measured.





\begin{figure}
    \centering
    \begin{subfigure}[b]{0.25\textwidth}
         \centering
         \includegraphics[width=\textwidth]{figure/syntax-zmeasure.pdf}%
         \subcaption{}
    \end{subfigure}
    \begin{subfigure}[b]{0.20\textwidth}
    \begingroup \fontsize{\myfontsize}{\mylinesize}
    \addtolength{\jot}{-4pt}
    \begin{align*}
        & \svar_0 \coloneqq Z_{0}Z_{1}Z_{2}Z_{3} \\ 
        & \textbf{if}\ M[\svar_0, q_3q_2q_1q_0]\ \textbf{then} \\
        & \quad \textbf{skip} \\
        & \textbf{else} \\
        & \quad \text{// correct error....}\\
        & \quad \cdots \\
        & \quad \text{// Recover signal} \\
        & \quad \svar_0 \coloneqq -\svar_0\ 
    \end{align*}%
    \endgroup
    \subcaption{}
    \end{subfigure}
    %
    \caption{An example for {\langname}. 
    (a) is the stabilizer measurement for $Z_{0}Z_{1}Z_{2}Z_{3}$. (b) is an error correction program associated with (a). The detailed error correction operation depends the error correction protocol and are omitted here.  
        }
    \label{fig:one-stabilizer}%
\end{figure}


\input{figtex/lang-operational}
\input{figtex/lang-denotational}

\subsection{Operational Semantics}\label{subsect:operational}

The \textit{operational semantics} of the proposed {\langname} are presented in Figure~\ref{fig:qec-lang-op}.
Different from the \qwhilelang, {\langname} denotes the state in QEC programs by the tuple $(\rho, \sigma)$, where $\rho$ is a partial density matrix that describes the current state of $\bar{q}$ , and $\sigma$ represents the current state of stabilizer variables. 
The quantum state $\rho$ can be regarded as a function over quantum variables $q$, and $\rho(q)$ represents the reduced partial density matrix where quantum variables except $q$ are all traced out.
Likewise, $\sigma$ represents a function over stabilizer variables, and $\sigma(\svar)$ is defined to be the current value of $\svar$.
The stabilizer state $\sigma$ is functionally similar to the classical program state~\cite{Winskel1993TheFS}, and 
we can define the substution rule for $\sigma$ in a similar way, which is then used in Figure~\ref{fig:qec-lang-op}.

\begin{definition}[Substitution rule] The substitution rules for stabilizer state $\sigma$ are defined as follows,
\begin{align*}
\sigma[s/\svar_i](\svar_j) &= \begin{cases}
s, \text{ if }\ i = j;\\
\sigma(\svar_j), \textbf{ otherwise} 
\end{cases}.
\end{align*}
\end{definition}



Rules in Figure~\ref{fig:qec-lang-op} are self-explained and represent reformulation of concepts familiar in quantum computing. The notation in these rules follows the convention in programming language research, for example, the expressions over the bar in the inference rules are \textit{premises} while the expression under the bar is \textit{conclusion}.
For pure quantum state operations in Figure~\ref{fig:qec-lang-op}, the operational semantics follows the flow in {\qwhilelang}~\cite{Ying2012FloydhoareLF}.  
For stabilizer related operations, we introduce extra operational semantics for the unary stabilizer expression $s_e^u$, as shown in the top right corner of Figure~\ref{fig:qec-lang-op}. We then process the assignment operation on stabilizer variables with the substitution rule.
When measuring the stabilizer variable $\svar$, its sign will get flipped if the current quantum state is not a +1 eigenstate of  $\svar$. Thus, we include one operation in the ``If -1'' and ``While -1'' rule to take care of the sign flipping on $\svar$.

To illustrate the use of the operational semantics in Figure~\ref{fig:qec-lang-op}, we revisit the program in Figure~\ref{fig:one-stabilizer}(b). We use $(0,\{\})$ to represent the initial state of $(\rho, \sigma)$ before the program.

\begin{example}[Error correction experiment]
Assume the initialization $q_3q_2q_1q_0 \coloneqq \ket{0000}$ on data qubits is distorted by noise and data qubits are assigned to be $\ket{0001}$. This may happen when, after the initialization in the logical subspace, a Pauli X error affects one of the qubits. 
For illustration purposes we consider that such error was on qubit $q_0$ and therefore use $q_0\coloneqq Xq_0;$ to correct the error. Notice that, in general, the error correcting protocol $\textbf{correct}(\svar_0, \cdots)$ depends on the outcome of multiple stabilizer measurements.
The program in Figure~\ref{fig:one-stabilizer}(b) then becomes
\begin{align*}
    \prog \equiv \ &q_3q_2q_1q_0 \coloneqq \ket{0001}; \svar_0 \coloneqq Z_{0123} ; \\
    &\textbf{if}\ M[\svar_0, q_3q_2q_1q_0]\ \textbf{then}\ \textbf{skip}\ \\
    &\textbf{else}\ q_0\coloneqq Xq_0; \svar_0 \coloneqq -\svar_0\ \textbf{end}.
\end{align*}
We write $Z_{0}Z_{1}Z_{2}Z_{3}$ as $Z_{0123}$ for simplicity.
Then the evaluation of $\prog$ with the operational semantics proceeds as follows:

\noindent
$
    \conf{\prog, \rho}  = \conf{q_3q_2q_1q_0 \coloneqq \ket{1110}; \svar_0 \coloneqq Z_{0123} ; \textbf{if}\ M[\svar_0, \\q_3q_2q_1q_0]\textbf{then}\ 
    \textbf{skip}\ \textbf{else}\ q_0\coloneqq Xq_0; \svar_0 \coloneqq -\svar_0\ \textbf{end}, (0, \{\})} \\
     \to \conf{\svar_0 \coloneqq Z_{0123} ; \textbf{if}\ M[\svar_0, q_3q_2q_1q_0]\ \textbf{then}\ \textbf{skip}\ 
    \  \textbf{else} \ q_0\coloneqq Xq_0; \svar_0 \coloneqq -\svar_0\ \textbf{end}, (\ket{0001}\bra{0001}, \{\})} \\
     \to \conf{\textbf{if}\ M[\svar_0, q_3q_2q_1q_0]\ \textbf{then}\
    \textbf{skip}\ \textbf{else}\ q_0\coloneqq Xq_0; \svar_0 \coloneqq -\svar_0\ \textbf{end}, (\ket{0001}\bra{0001}, \{\svar_0=Z_{0123}\})} \\
     \to \conf{q_0\coloneqq Xq_0; \svar_0 \coloneqq -\svar_0, (\ket{0001}\bra{0001}, \{\svar_0=-Z_{0123}\})} \\
     \to \conf{\svar_0 \coloneqq -\svar_0, (\ket{0000}\bra{0000}, \{\svar_0=-Z_{0123}\})} \\
     \to \conf{E, (\ket{0000}\bra{0000}, \{\svar_0=Z_{0123}\})}
$
\end{example}

\subsection{Denotational Semantics}\label{subsect:denotational}
The denotational semantics of the {\langname}  is given in Figure~\ref{fig:lang-denote}. The program $\prog$ is denoted as a super-operator $\denot{\prog}$ that acts on $(\rho, \sigma)$. While most rules in Figure~\ref{fig:lang-denote} are  self-explained, the \textbf{while} rule relies a partial order on $(\rho, \sigma)$ to compute the fixed point, i.e., the lowest upper bound ($\sqcup$) for the complete partial ordering (CPO) of  $\textbf{while}^{(k)}$.

To define the partial order on state $(\rho, \sigma)$, we first need to define a partial order on the state of stabilizers. Consider a trivial lattice on the stabilizer set $S$, where $s_i \sqsubseteq s_j$ if $s_i = I$, and $s_i$ and $s_j$ cannot be compared if they are both not identity. We can then define the partial order on $\sigma$ as follows: \\
\centerline{$\sigma_1 \sqsubseteq \sigma_2 \ \textbf{if}\ \sigma_1(\svar) \sqsubseteq \sigma_2(\svar), \forall \svar \in \textbf{sVar},$} \\
which immediately induces a partial order on $(\rho, \sigma)$: \\
\centerline{
$(\rho_1, \sigma_1) \sqsubseteq (\rho_2, \sigma_2) \ \textbf{if}\ \rho_1 \sqsubseteq \rho_2 \ \textbf{and}\ \sigma_1 \sqsubseteq \sigma_2,$} \\
where $\rho_1 \sqsubseteq \rho_2$ means that $Tr(O\rho_1) \le Tr(O\rho_2)$, for any semi-positive observable $O$. %

The reason to select the partial order above is that it strictly ensures the consistency of fault-tolerant computation. If any error happens in one while loop and does not get corrected in time, the stabilizer state $\sigma$ will get some variables flipped. Let the resulted stabilizer state be $\sigma'$. Obviously, there does not exist any $\sigma_1$ s.t. $\sigma \sqsubseteq \sigma_1$ and $\sigma' \sqsubseteq \sigma_1$. 
In this case, we just set $\sqcup_{k=0}^{\infty} \denot{\textbf{while}^{(k)}}(\rho, \sigma)$ to be $\perp$, the bottom element of the CPO~\cite{Winskel1993TheFS}, which does not provide any information for the program.  On the other hand, if all errors get corrected in each loop, the resulted state of \textbf{while} can be calculated coordinate-wisely: \\
\centerline{$(\sqcup_{k=0}^{\infty} \denot{\textbf{while}^{(k)}}\rho, \sqcup_{k=0}^{\infty} \denot{\textbf{while}^{(k)}}\sigma)$}.

We connect the denotational semantics to the operational semantics through the following proposition:

\begin{proposition}[Equivalence of the denotational semantics and the operational semantics]\label{prop:equiv}
For a strict QEC program $\prog$ that corrects errors when the errors appear, we have
$
    \denot{\prog}(\rho, \sigma) \equiv \sum \{  (\rho', \sigma'):\conf{\prog, (\rho, \sigma)} \to^* \conf{E, (\rho', \sigma')} \},
$
where $\to^*$ denotes the reflective, transitive closure of $\to$,  and $\{\cdot\}$ represents a multi-set.
\end{proposition}
\myproof{
Except the while loop, other statements can be proved trivially by structural induction. For the while loop, the consistency can be proved with the assumption of a just-in-time error correction.}


\section{{\assnname}}
In this section, 
we first introduce an expressive assertion language \textbf{\assnname} and then derive a Hoare logic to verify QEC programs.
\subsection{Syntax of \assnname}

Stabilizer is a kind of Hermitian operator and can be used  as predicate for QEC programs. 
We observe that  the exponential computational overhead on Hermitian-based predicates may be circumvented by using stabilizers as predicates. In fact arithmetic operations (e.g. addition and multiplication) between stabilizers can be completed within a time polynomial in the number of qubits.
This observation is particularly important for QEC programs in which the majority of logical operations can be described with a few stabilizers and the corresponding predicate transformation can be framed as the multiplication of stabilizers.

However, as predicates, stabilizers are not universal.
There are infinitely many quantum states that are not eigenstates of any non-identity stabilizer, e.g., $\ket{\psi} = \frac{\sqrt{3}}{2}\ket{0} + \frac{1}{2}\ket{1}$. 
Such limitation will cause difficulty in the verification of QEC programs. For example, if we are given some state that is not the +1 eigenstate of any stabilizer (this is possible to happen in future universal fault-tolerant computation), we cannot find any predicate except $I$ to accommodate such state.%
One well-studied way in the quantum information community to address this problem is to use the Pauli expansion of quantum Observable (Hermitian matrices)~\cite{nielsen2002quantum,Wilde2013QuantumIT}: \nothmskip
\begin{lemma}[Pauli expansion]\label{lem:pauli-expansion}
The quantum observable $O$ of a $n$-qubit system can be expressed as a linear combination of Pauli strings:
   $ O = \sum_i w^i\sigma^i_n$,
where $\sigma_n^i \in \{I, X, Y, Z\}^{\otimes n}$ is a length-$n$ Pauli string, and $w_i \in \mathbb{R}$ is its coefficient.
\nothmskip
\end{lemma}

The Pauli expansion motivates the following proposition which provides a universal way to deal with arbitrary logical states in QEC programs: 
\nothmskip
\begin{proposition}\label{prop:universal-stabilizer}
$\forall\, \ket{\psi} \in \mathcal{H}_n$, there is a $\projector$ which is a sum of stabilizers (Pauli strings), that satisfies $\projector \ket{\psi} = \ket{\psi}$, and $\projector \ne I$.
\nothmskip
\end{proposition}
\myproof{
Note that $(\ket{\psi}\bra{\psi})\ket{\psi} = \ket{\psi}$, $\ket{\psi}\bra{\psi}$ is a non-identity observable and by Pauli expansion, it can be represented by a linear combination of Pauli strings, i.e., stabilizers.}

Inspired by Lemma~\ref{lem:pauli-expansion} and Proposition~\ref{prop:universal-stabilizer}, we introduce arithmetic expressions of stabilizers, and define the stabilizer expression $s_e$ as follows, 
\begin{align}
    s_e \Coloneqq s \vsep \lambda_0 s_{e0} + \lambda_1 s_{e1}, \ \lambda_0, \lambda_1 \in \mathbb{C}
\end{align}
where $s$ is a stabilizer. $s_e$ is different from the $s_e^u$ used for stabilizer variables which only consists of unary operations on stabilizers. However, by describing both the stabilizer variable and the predicate within the language of stabilizer, we can easily incorporate the information from stabilizer measurement into predicates.

By Proposition~\ref{prop:universal-stabilizer},
$s_e$ is universal as $\forall\ket{\psi}, \exists s_e\ s.t.\ s_e\ket{\psi} = \ket{\psi}$. For example, the state $\ket{\psi} = \frac{\sqrt{3}}{2}\ket{0} + \frac{1}{2}\ket{1}$ is a +1 eigenstate of  $s_e \coloneqq\,\frac{1}{2}Z + \frac{\sqrt{3}}{2}X$. %
We then formulate the assertion language \textbf{{\assnname}} on QEC programs as follows:
\begin{align}
    A \Coloneqq s_e \vsep A_0 \wedge A_1 \vsep A_0 \vee A_1 \vsep A_0 \Rightarrow A_1.
\end{align}

When $A \coloneqq s_e$, we say a QEC program state $(\rho, \sigma)$ satisfies an assertion $A$ if $A\rho = \rho$ (for $\rho=\ket{\psi}\bra{\psi}$, $A\rho = \rho \Leftrightarrow A\ket{\psi} = \ket{\psi}$) , and $s_e$ is commutable with all stabilizer variables in $\sigma$. We denote this relation by $(\rho,\sigma) \models A$. Requiring $s_e$ to be commutable with stabilizer variables in $\sigma$ is essential for developing the quantum Hoare logic in section~\ref{subsect:partial}.

The semantics of $A_0 \wedge A_1$ and other Boolean expressions can then be derived by structural induction:
\begin{itemize}
    \item $(\rho,\sigma) \models A_1 \wedge A_2$ iff $(\rho,\sigma) \models A_1$ and $(\rho,\sigma) \models A_2$;
    \item $(\rho,\sigma) \models A_1 \vee A_2$ iff $(\rho,\sigma) \models A_1$ or $(\rho,\sigma) \models A_2$;
    \item $(\rho,\sigma) \models (A_1 \Rightarrow A_2)$ iff $((\rho,\sigma) \models A_1) \Rightarrow ((\rho,\sigma) \models A_2)$.
\end{itemize}

If an assertion $A$ is satisfied by any program states $(\rho, \sigma)$, we simply denote such property as $\models A$.



The following lemma presents an important result for {\assnname} that will be frequently utilized in later sections.

\begin{restatable}[Implication rule]{lemma}{implictrule}
	\label{lem:implicitrule}
	For stabilizer expressions,
	\begin{enumerate}
		\item If $(\rho, \sigma) \models s_{e0}$ and $(\rho, \sigma) \models s_{e1}$, we have $(\rho, \sigma) \models s_{e0}s_{e1}$ and $(\rho, \sigma) \models \lambda_0 s_{e0}+\lambda_1 s_{e1}$, where  $\lambda_0+ \lambda_1=1$.
		\item Assume $s_{e0}$ is not singular. If $(\rho, \sigma) \models s_{e0}$ and $(\rho, \sigma) \models s_{e1}s_{e0}$, we have $(\rho, \sigma) \models s_{e1}$.
		\item Assume $(a s_{e0} + bs_{e1})\rho = \rho$, every stabilizer in $\sigma$ is commutable with $s_{e0}$ and $s_{e1}$, and $(\rho, \sigma) \models  s_{e2}$, then
		$(\rho, \sigma) \models a s_{e0} + bs_{e1}s_{e2}$.
	\end{enumerate}
\end{restatable}
\myproof{\postpone{\ref{app:assn}}.}

Rules from classical Boolean predicates can also be used for {\assnname}, such as the rules for disjunction and conjunction. We will use these rules directly without extra description. Especially, the identity operator $I$ represents \textbf{True} and the empty operator $0$ represents \textbf{False} in {\assnname}.



\subsection{Partial Correctness}
\label{subsect:partial}

\begin{figure}
\begingroup\fontsize{\myfontsize}{\mylinesize}
    $\{A\}\textbf{skip}\{A\}\ \hfill\text{(Skip)}$%
    
    $\{A[\ket{0}/q]\} q\coloneqq \ket{0} \{A\} \hfill \text{(Initialization)}$ %
    
    $\{A\} \bar{q}\coloneqq U\bar{q} \{UAU^\dagger\} \hfill\text{(Unitary)}$\\
    \hfill  $U$ is a unitary, but written in the sum of stabilizers.
     %
    
    $\{A\} \svar\coloneqq \pm\svar \{A\} \hfill \text{(Assignment)}$%
    
    $\{A\} \svar\coloneqq s \{A\} \hfill \text{(Assignment)}$ \\
    \hfill  where $s$ is commutable with $A$, otherwise $\{A\} \svar\coloneqq s \{I\}$.
    %
    
    $\infer{\{A\}\prog_1;\prog_2\{B\}}{\{A\}\prog_1\{C\}\quad \{C\}\prog_2\{B\}}\hfill \text{(Sequencing)}$%
    
    
    $\infer{\{\sum_{i=0}^1 A_i M_i\}\textbf{if}\,M[\svar, \bar{q}]\,\textbf{then}\, \prog_1\,\textbf{else}\, \prog_0\,\textbf{end}\{B\}}{\{A_1 \wedge \svar\}\prog_1\{B\}\quad \{A_0 \wedge -\svar \}\prog_0\{B\}} \hfill \text{(Condition)}$
    \\
    
    
    $\infer{\{\sum_{i=0}^1 A_i M_i\} \textbf{while}\, M[\svar, \bar{q}]\, \textbf{do}\, \prog_1\, \textbf{end}\,\{ A_0 \wedge -\svar \} }{\{ A_1 \wedge \svar \}\prog_1\{\sum_{i=0}^1 A_i M_i\} }\hfill \text{(While)}$\\
    
    $\infer{\{A\}\prog\{B\} }{\models (A \Rightarrow A')\quad \{A'\}\prog\{B'\}\quad \models (B' \Rightarrow B)} \hfill \text{(Consequence)}$\\
    
\endgroup
%
    \caption{Hoare rules for partial correctness assertions when $A \coloneqq s_e$.
    }
    \label{fig:pca}%
\end{figure}

A partial correctness assertion in \assnname\ has the form:
\hfill{} $\{A\} c \{B\}$, where $A, B \in $ \assnname, and $c\in$ \langname. We first present the Hoare logic for partial correctness assertions in which the precondition $A$ is a stabilizer expression $s_e$, as shown in Figure~\ref{fig:pca}. We will extend the Hoare logic to Boolean expressions like $A_1 \wedge A_2$ in Proposition~\ref{lem:bool-assn}. %
 
 
The proof rules in Figure~\ref{fig:pca} are syntax-directed and reduce proving a partial correctness assertion of a compound statement to proving partial correctness assertions of its sub-statements. We only explain some rules below, since most rules are self-explained.

In the initialization rule, $(\rho, \sigma)\models A[\ket{0}/q]$ means that $A\rho_0^q = \rho_0^q$ and $A$ commutes with all stabilizer variables in $\sigma$.  This can be seen as the quantum version substitution rule. 
A more useful case of the initialization rule is when all qubits are reset to $\ket{0}$, and for the $n$-qubit system, we have 
\begin{align}
    \{I\}q_{n-1}\cdots q_0\coloneqq \ket{0}^{\otimes n}\{Z_0\wedge Z_1\wedge \cdots \wedge Z_n\}.
\end{align}

In the unitary rule, we represent unitary matrices as the sum of stabilizers in order to utilize the cheap computational cost of stabilizer multiplication. 

The rules for condition and while loop resembles their classical counterparts except the state may be changed by the branching condition. A direct derivative of the Condition rule 
is to make $A_1 = A_0 = A$ as follows,
\begin{lemma}
$\dfrac{\{A \wedge \svar\}\prog_1\{B\}\quad \{A \wedge -\svar \}\prog_0\{B\}}{\{A\}\textbf{if}\,M[\svar, \bar{q}]\,\textbf{then}\, \prog_1\,\textbf{else}\, \prog_0\,\textbf{end}\{B\}}
$.\end{lemma}
\myproof{
Note that $M_0 + M_1 = I$.
}

Likewise, by letting $A_1 = A_0 = A$, we have%
\begin{lemma}
$\dfrac{\{ A \wedge \svar \}\prog_1\{A\} }{\{A\} \textbf{while}\, M[\svar, \bar{q}]\, \textbf{do}\, \prog_1\, \textbf{end}\,\{ A \wedge -\svar \} }$.
\end{lemma}


The consequence rule is a powerful tool for the verification of QEC programs since it can encode facts of QEC codes into partial correctness assertions. 
The following example demonstrates the usage of the proposed Hoare rules, including the consequence rule:
\begin{example}
Assume $\prog\Coloneqq \svar\coloneqq Z_1; \textbf{if}\ M[\svar, q_1]\ \textbf{then}\\\textbf{skip} \textbf{else}\,q_1\coloneqq Xq_1;q_0\coloneqq Xq_0 \,\textbf{end}$. \\We prove  $\{Z_0Z_1\}\prog\{Z_0\}$ as follows:\\
$\{Z_0Z_1\}\svar\coloneqq Z_1; \{Z_0Z_1\}$ \hfill{} (Assignment) \\
$(Z_0Z_1) M_1 = (Z_0Z_1)\frac{I+Z_1}{2} = Z_0\frac{I+Z_1}{2}$, 
$(Z_0Z_1) M_0 = -Z_0\frac{I-Z_1}{2}$ \\
$\{Z_0Z_1 \wedge Z_1\}\textbf{skip}\{Z_0Z_1 \wedge Z_1\}$ \hfill{}(Skip) \\
$\{Z_0Z_1 \wedge -Z_1\}q_1\coloneqq Xq_1 \{-Z_0Z_1 \wedge Z_1\}$ \hfill{} (Unitary) \\
$\{-Z_0Z_1 \wedge Z_1\} q_0\coloneqq Xq_0 \{Z_0Z_1 \wedge Z_1\}$ \hfill{} (Unitary) \\
$\{Z_0Z_1 \wedge -Z_1\}q_1\coloneqq Xq_1; q_0\coloneqq Xq_0 \{Z_0Z_1 \wedge Z_1\}$ \hfill{} (Sequencing)\\
$\{Z_0Z_1= (Z_0Z_1)M_0+ (Z_0Z_1)M_1\}\textbf{if}\ M[\svar, \bar{q}]\ \textbf{then}\  \textbf{skip}\,\\\textbf{else}\,q_1\coloneqq Xq_1; q_0\coloneqq Xq_0 \,\textbf{end}\{Z_0Z_1 \wedge Z_1\}$ \hfill{} (Condition)\\
Then, $\{Z_0Z_1\}\prog\{Z_0Z_1 \wedge Z_1\}$ \hfill{} (Sequencing)\\
$Z_0Z_1 \wedge Z_1 \Rightarrow Z_0$ \hfill (Implication) \\
With the consequence rule, we have 
$\{Z_0Z_1\}\prog\{Z_0\}$.
\end{example}

Now we extend the Hoare rules in Figure~\ref{fig:pca} to other Boolean assertions in \assnname. \nothmskip
\begin{restatable}[]{proposition}{boolassn}
\label{lem:bool-assn}
	We restate the Hoare rules for classical Boolean assertions as follows, \\
	if $\{A_0\}\prog\{B_0\} \wedge \{A_1\}\prog\{B_1\}$,  $\{A_0\wedge A_1\}\prog\{B_0\wedge B_1\}$; \\
	if $\{A_0\}\prog\{B_0\} \vee \{A_1\}\prog\{B_1\}$,  $\{A_0\vee A_1\}\prog\{B_0\vee B_1\}$; \\
	$\{I\}\prog\{I\}, \{0\}\prog\{B\}$, where $B$ is any assertion, and $0$ represents an empty set of program states. For example, if $s_{e_1}$ and $s_{e_2}$ anti-commutes, $s_{e_1} \wedge s_{e_2} = 0$.
\end{restatable}\nothmskip
\myproof{
\postpone{\ref{app:assn}}.}


We prove a lemma for error correction which is frequently used in verification sections later.
\nothmskip
\begin{restatable}[Decoding correctness]{proposition}{decodecorrect}
\label{prop:decodecorrect}
Assume an valid error decoding and correction protocol for $\textbf{correct}$ function. Let $S$ be the set of all active stabilizer measurements in error correction, and  define $ A_S = \wedge_{s_i \in S} s_i$, then \\
$\{I \}\textbf{correct}(\svar_0, \svar_1, \cdots) \{ A_S\}$,\\
$\{A \wedge A_S \}\textbf{correct}(\svar_0, \svar_1, \cdots) \{A \wedge A_S\}$,\\ where $\svar_0, \svar_1, \cdots$ enumerate all elements of $S$.
\end{restatable}
\myproof{
\postpone{\ref{app:assn}}.}

Finally, we prove the soundness of Hoare rules in Figure~\ref{fig:pca}. %
\begin{restatable}[Soundness]{theorem}{soundness}
\label{thm:soundness}
The proof system in Figure~\ref{fig:pca} is sound for the partial correctness assertions.
\end{restatable}\nothmskip
\myproof{
\postpone{\ref{app:assn}}.}

\section{Theoretical Analysis}
In this section, we give a theoretical analysis of %
both the program size  and the computational complexity of our framework for implementing and verifying surface codes~\cite{Fowler2012SurfaceCT, Horsman2012SurfaceCQ}, respectively.


\subsection{Program Size}
We first compare the program size (i.e., the number of statements) when implementing the surface code~\cite{Fowler2012SurfaceCT, Horsman2012SurfaceCQ}, in the qWhile-Lang (i.e., the {\qwhilelang}) and the {\langname} (see row 2-3 of Table~\ref{tab:comp}).
In surface code, we consider two approaches to encode a logical qubit, the planar code and the double defect code (detailed implementation of these codes can be found in \cite{Fowler2012SurfaceCT, Horsman2012SurfaceCQ}).
For the distance-$d$ surface code, the planar version requires $O(d^2)$ data qubits, $O(d^2)$ parity qubits as well as $O(d^2)$ stabilizers. The double-defect version introduces an overhead of a factor 10 in all the three quantities.

As a code size estimation of {\langname}, we only need one statement per stabilizer measurement, whereas the qWhile-Lang requires at least eight gate operations
to describe the circuit measuring a stabilizer~\cite{Fowler2012SurfaceCT}. Thus, {\langname} provides $8\times$ program size compression for the surface code implementations.

\subsection{Verification Complexity of Clifford Gates} %

The defining property of Clifford operations is that, given a Clifford gate $G$ and a stabilizer $s$, $GsG^\dagger$ must also be a stabilizer, i.e. Clifford operations do not increase the number of stabilizers in the assertion. 

By framing both assertions and unitary operations in the language of stabilizers, {\myFrameworkName} can efficiently processes the verification of Clifford operations. The efficiency stems from the low cost of multiplying stabilizers, which is $O(d)$ because the length of the stabilizers for logical states is at most $d$ for a distance-$d$ surface code.
In this way we avoid representing stabilizers as exponentially-large matrices.
Therefore, \myFrameworkName~only incurs $O(d^3)$ computational overhead for the planar surface code and $O(10d^3)$ computational overhead for the double-defect surface code.

However, 
the vanilla quantum Hoare logic in qWhile-Lang can not exploit the property of Clifford operations and the low computational complexity of stabilizer multiplication. 
The Clifford operations are treated like any other unitary operations and the predicate in qWhile-Lang is a Hermitian matrix
of size $O(2^{n_d + n_p}\times 2^{n_d + n_p})$, where $n_d$ is the number of data qubits and $n_p$ is the number of parity qubits. Hoare rules with such predicates incurs at least $O(2^{n_d + n_p}\times 2^{n_d + n_p})$ computational overhead. Thus, the verification with qWhile-Lang requires $O(8d^24^{2d^2})$ time for the planar surface code, and $O(80d^24^{20d^2})$ for the double-defect surface code.
In summary, the proposed language design, assertion design, and the logic proof system can significantly simplify the verification of all Clifford operations of stabilizer codes.


\begin{table}[]
\resizebox{\columnwidth}{!}{
\begin{tabular}{|p{1.8cm}|p{2cm}|p{3cm}|p{2.5cm}|}
\hline
      Metric                &       Method                             & Planar surface code & Double-defect surface code \\ \hline
Statements \#           & qWhile-Lang & $O(8d^2)$           & $O(80d^2)$                 \\ \cline{2-4} 
                                         & {\myFrameworkName}               & $O(d^2)$            & $O(10d^2)$                 \\ \hline
Verification  & qWhile-Lang & $O(8d^24^{2d^2})$     & $O(80d^24^{20d^2})$             \\ \cline{2-4} 
                  Complexity                      & {\myFrameworkName}               & $O(d^3)$            & $O(10d^3)$                 \\ \hline             
\end{tabular}
}
\caption{
Comparison of {{\myFrameworkName}} and  qWhile-Lang on implementing and verifying surface codes. 
}
\label{tab:comp}
%
\end{table}





\subsection{Verification Complexity of T Gate}

The logical T gate is usually the most challenging problem in quantum program verification in general.
The T gate, loosely speaking, represents a fundamental boundary between classical and quantum computing. 
A quantum program with T gates cannot be efficiently and precisely simulated or verified on a classical computer. 
In \myFrameworkName, the number of stabilizers in our predicate will rapidly increase when the program to be verified has some T gates.

With this being said, we argue that verifying a QEC implementation of \textbf{one} logical T gate could be easier in many cases. The exact verification efficiency would be determined by the amount of the non-Clifford operations involved in the implementation of a logical T gate.
In the surface code, for example, there is only \textbf{one} non-Clifford single-qubit physical  gate~\cite{Fowler2015MinimumWP} for a logical T gate. 
The verification complexity will remain $O(d^2)$ because the number of stabilizer terms in a predicate is still $O(1)$. As such, \myFrameworkName~can still hold the exponential advantage for surface code. We remark again that such advantages come from our stabilizer-centric design in developing the verification framework. 


\section{Case Study I: Repetition Code}
In this section and the next section, we give step-by-step case study on two well-known QEC codes to guide through the usage of our framework. 
For each QEC code, we first express its implementation in our {\langname}. %
Then we verify the correctness of the logical operation with our proof system and show that the implemented QEC code can correct local errors on the physical qubits.

We start from the quantum repetition code~\cite{nielsen2002quantum}, which is relatively simple with light error correction overhead. 
This code can correct bit-flip error or phase-flip error, but not when they happen simultaneously. 
The repetition code is mainly deployed on quantum architectures whose underlying physical qubits (e.g., the cat qubit built upon bosonic quantum system~\cite{Chamberland2020BuildingAF}) already have extremely low phase-flip (or bit-flip) error rates. %

\subsection{Quantum Repetition Code}
We consider a three-qubit quantum repetition code to simplify the discussion. At high level, the three-qubit code just encodes one logical qubit with three physical qubits. For example, the $\ket{000}$ state of three physical qubits represents the logical $\ket{0_L}$ state of a logical qubit, and the $\ket{111}$ state represents the logical $\ket{1_L}$ state. 
We first give the circuit diagrams for  primitive operations in quantum repetition code and their code implementations in {\langname} (Figure~\ref{fig:repcode}). 
The interested reader can find detailed explanations of the repetition code design in references~\cite{nielsen2002quantum, google50296}. 


Figure~\ref{fig:repcode} (a) (b) (c) gives the implementations of the qubit initialization, the logical X gate $X_L = X_0X_1X_2$, and the logical Z gate $Z_L = Z_0Z_1Z_2$, respectively. 
The logical CNOT gate between two logical qubits can be implemented by imposing CNOT gates on three pairs of physical qubits, as shown in Figure~\ref{fig:repcode} (d).



\subsection{Verification of Logic Operations}


\input{figtex/repcode}

This part prove the correctness of the code segments in Figure~\ref{fig:repcode} with our frameworks. It contains two major steps, defining the predicates for each logical operation and constructing the proof. 

For the initialization operation, the expected behavior is that, for arbitrary input state, the output state should be in the logical state $\ket{0_L}$, which is the simultaneous eigenstate of the logical Z operator $Z_L$ and the stabilizers $Z_0Z_1$ and $Z_1Z_2$. Thus, we set the precondition to $\{I\}$ and the post-condition to $\{Z_L\wedge Z_0Z_1 \wedge Z_1Z_2\}$, as formulated below. \nothmskip
\begin{proposition}[Initialize to $\ket{0}$]
For the program $\prog$ in Figure~\ref{fig:repcode}(a), we have $\{I\}\prog\{Z_L\wedge Z_0Z_1 \wedge Z_1Z_2\}$.
\end{proposition}\nothmskip
\myproof{
For the initialization, $\{I\}q_0q_1q_2 \coloneqq \ket{000}\{Z_0 \wedge Z_1 \wedge Z_2\}$. And  $\{Z_0 \wedge Z_1 \wedge Z_2\}\svar_0\coloneqq Z_0Z_1\{Z_0\wedge Z_1 \wedge Z_2\}$ and $\{Z_0 \wedge Z_1 \wedge Z_2\}\svar_1\coloneqq Z_1Z_2\{Z_0\wedge Z_1 \wedge Z_2\}$. Since $Z_0\wedge Z_1 \wedge Z_2 \Rightarrow Z_0 Z_1 $ and $Z_0\wedge Z_1 \wedge Z_2 \Rightarrow Z_1 Z_2$, we have $Z_0\wedge Z_1 \wedge Z_2 \Rightarrow Z_0\wedge Z_1 \wedge Z_2 \wedge (Z_0 Z_1) \wedge (Z_1 Z_2)$. Then by Proposition~\ref{prop:decodecorrect},
$\{Z_0\wedge Z_1 \wedge Z_2 \wedge (Z_0 Z_1) \wedge (Z_1 Z_2)\}\textbf{correct}(\svar_0,\svar_1)\{Z_0\wedge Z_1 \wedge Z_2 \wedge (Z_0 Z_1) \wedge (Z_1 Z_2)\}$. By the consequence rule, we get $\{I\}\prog\{Z_L\wedge Z_0Z_1 \wedge Z_{1}Z_2\}$ since $Z_0\wedge Z_1 \wedge Z_2 \Rightarrow Z_L$.
}

We then verify the logical X operation.
It is sufficient to verify two cases, the output state $\ket{1_L}$ under the input state $\ket{0}_L$, and vice versa. Arbitrary logical states can be processed as the linear combination of these two cases by taking advantage of the linearity of the logical X operation.
Since $\ket{0}_L$ corresponds to the predicate $Z_L\wedge Z_0Z_1 \wedge Z_1Z_2$, and $\ket{1}_L$ corresponds to the predicate $-Z_L\wedge Z_0Z_1 \wedge Z_1Z_2$, we have the following proposition: \nothmskip
\begin{proposition}[Logical X gate]
For the program $\prog$ in Figure~\ref{fig:repcode}(b), we have $\{Z_L\wedge Z_0Z_1\wedge Z_1Z_2\}\prog\{-Z_L\wedge Z_0Z_1\wedge Z_1Z_2\}$ and $\{-Z_L\wedge Z_0Z_1\wedge Z_1Z_2\}\prog\{Z_L\wedge Z_0Z_1\wedge Z_1Z_2\}$.
\end{proposition} \nothmskip
\myproof{
Note that $X_0X_1X_2 Z_L X_0 X_1X_2 = -Z_1Z_2Z_3= - Z_L$, $X_0X_1\\X_2Z_0Z_1X_0X_1X_2 = Z_0Z_1$, $X_0X_1X_2Z_1Z_2X_0X_1X_2 = Z_1Z_2$.
}

Likewise, for thel logical Z gate, we only need to verify that, the precondition $\{X_L\wedge Z_0Z_1\wedge Z_1Z_2\}$ relates to the post-condition $\{-X_L\wedge Z_0Z_1 \wedge Z_1Z_2\}$, and vice versa. %
\nothmskip
\begin{proposition}[Logical Z gate]
For the program $\prog$ in Figure~\ref{fig:repcode}(c), $\{X_L\wedge Z_0Z_1\wedge Z_1Z_2\}\prog\{-X_L\wedge Z_0Z_1\wedge Z_1Z_2\}$ and $\{-X_L\wedge Z_0Z_1\wedge Z_1Z_2\}\prog\{X_L\wedge Z_0Z_1\wedge Z_1Z_2\}$.
\end{proposition} \nothmskip
\myproof{
Note that $Z_0Z_1Z_2X_LZ_0Z_1Z_2  = -X_0X_1X_2 = -X_L$.
}


The verification of the logical CNOT gate involves four preconditions: $Z_{L0}I_{L1}$, $X_{L0}I_{L1}$, $I_{L0}X_{L1}$, and $I_{L0}Z_{L1}$, where $Z_0Z_1 \wedge Z_1Z_2 \wedge Z_3Z_4 \wedge Z_4Z_5$ are omitted for simplicity. These four Pauli strings are able to represent any input state by multiplication and addition. The post-conditions for these four preconditions are $Z_{L0}I_{L1}$, $X_{L0}X_{L1}$, $I_{L0}X_{L1}$, and $Z_{L0}Z_{L1}$. While the first three post-conditions are straightforward to understand, we elaborate on the fourth post-condition. 
The precondition $I_{L0}Z_{L1}$ specifies pure states of the form $(a\ket{0}+b\ket{1})_{L0}\ket{0}_{L1}$. After the CNOT gate, the state becomes $a\ket{00}+b\ket{11}$ which is the +1 eigenstate of $Z_{L0}Z_{L1}$, for arbitrary $a$ and $b$. \nothmskip
\begin{restatable}[Logical CNOT]{proposition}{repcnot}
\label{prop:rep-cnot}
For the program $\prog$ in Figure~\ref{fig:repcode}(d), assume $A_S = Z_0Z_1\wedge Z_1Z_2\wedge Z_3Z_4\wedge Z_4Z_5$, we have\\
$\{Z_{L0} I_{L1}\wedge A_S\}\prog\{Z_{L0}I_{L1}\wedge A_S\}$, $\{X_{L0} I_{L1} \wedge A_S\}\prog\{X_{L0}X_{L1}\wedge A_S\}$, $\{I_{L0} X_{L1}\wedge A_S\}\prog\{I_{L0}X_{L1}\wedge A_S\}$, and $\{I_{L0} Z_{L1}\wedge A_S\}\prog\{\\Z_{L0}Z_{L1}\wedge A_S\}$.
\end{restatable} \nothmskip
\myproof{
Note that for control qubit $a$ and target qubit $b$, \\$\text{CNOT}_{ab} = \frac{1}{2}(I + X_b + Z_a - Z_aX_b)$. %
\postpone{\ref{app:rep}}.
}

\subsection{Verification on Noise Injection}
\myFrameworkName~can also reason about the correctness with hardware noise. We assume a minimum weight perfect matching error decoding~\cite{Fowler2015MinimumWP} and correction as follows, \nothmskip
\begin{program}[Quantum repetition code error correction]\label{prog:rep-decoder} For the quantum repetition code in Figure~\ref{fig:repcode}, define the error correction protocol as follows, for $\svar_0 = Z_0Z_1, \svar_1 = Z_1Z_2$, \\ $\textbf{correct}(\svar_0, \svar_1) \Coloneqq \\$
$
\qif{Z_0Z_1, q_0q_1}{\qif{Z_1Z_2, q_1q_2}{\textbf{skip}\\}{q_2\coloneqq Xq_2;\ \svar_1 \coloneqq -\svar_1}\\}{\qif{Z_1Z_2, q_1q_2}{q_0 \coloneqq Xq_0;\ \svar_0 \coloneqq -\svar_0 \\}{q_1\coloneqq Xq_1;\ \svar_0 \coloneqq -\svar_0; \ \svar_1 \coloneqq -\svar_1}}
$
\end{program} \nothmskip

In Figure~\ref{fig:noisyrepcode}(a), we present a noisy logical X gate where an X error occurs on $q_1$. We prove that the expected behavior of the noisy logical X gate is the same as that of the error-free logical X gate with the help of error correction.

\input{figtex/noisyrep}
\nothmskip
\begin{proposition}\label{prop:rep-noise-x} For the program $\prog$ in Figure~\ref{fig:noisyrepcode}(a), which implements a noisy logical X gate, \\
$\{Z_L \wedge Z_0Z_1 \wedge Z_1Z_2\}\prog\{-Z_L \wedge Z_0Z_1 \wedge Z_1Z_2\}$, \\
and $\{-Z_L \wedge Z_0Z_1 \wedge Z_1Z_2\}\prog\{Z_L \wedge Z_0Z_1 \wedge Z_1Z_2\}$.
\end{proposition} \nothmskip
\myproof{
We only prove $\{Z_L\wedge Z_0Z_1 \wedge Z_1Z_2\}{\prog}\{-Z_L\wedge Z_0Z_1 \wedge Z_1Z_2\}$ for simplicity.\\
$Z_L \wedge Z_0Z_1 \wedge Z_1Z_2 \Rightarrow Z_0 \wedge Z_1 \wedge Z_2$; \\
$\{Z_0\wedge Z_1 \wedge Z_2\}\svar_0 \coloneqq I \{Z_0\wedge Z_1 \wedge Z_2\}$; \\
$\{Z_0\wedge Z_1 \wedge Z_2\}\svar_1 \coloneqq I \{Z_0\wedge Z_1 \wedge Z_2\}$;
\\
$\{Z_0\wedge Z_1\wedge Z_2\}q_2q_1q_0 \coloneqq X_2X_1X_0q_2q_1q_0\{-Z_0\wedge -Z_1 \wedge -Z_2\}$; \\
$\{-Z_0\wedge -Z_1 \wedge -Z_2\}q_1 \coloneqq X_1q_1\{-Z_0\wedge Z_1 \wedge -Z_2\}$; 
\\
$\{-Z_0\wedge Z_1 \wedge -Z_2\}\svar_0 \coloneqq Z_0Z_1 \{-Z_0\wedge Z_1 \wedge -Z_2\}$;
\\
$\{-Z_0\wedge Z_1 \wedge -Z_2\}\svar_1 \coloneqq Z_1Z_2 \{-Z_0\wedge Z_1 \wedge -Z_2\}$;
\\
For the \textbf{correct} statement, \\$-Z_0\wedge Z_1 \wedge -Z_2\wedge Z_0Z_1 = 0$, $-Z_0\wedge Z_1 \wedge -Z_2 \wedge Z_1Z_2 = 0$, \\and $\{-Z_0\wedge Z_1\wedge -Z_2 \wedge -Z_0Z_1 \wedge -Z_1Z_2\}q_1\coloneqq Xq_1;\ \svar_0\coloneqq -\svar_0;\ \svar_1 \coloneqq -\svar_1\{-Z_0\wedge -Z_1 \wedge -Z_2 \wedge Z_0Z_1 \wedge Z_1Z_2\}$, \\
so $\{-Z_0\wedge Z_1 \wedge -Z_2\}\textbf{correct}(\svar_0, \svar_1)\{-Z_0\wedge -Z_1 \wedge -Z_2 \wedge Z_0Z_1 \wedge Z_1Z_2 \}$. Then by the consequence rule, we get $\{Z_L \wedge Z_0Z_1 \wedge Z_1Z_2\}\prog\{-Z_L \wedge Z_0Z_1 \wedge Z_1Z_2\}$.
}

However, the error correction protocol in Program~\ref{prog:rep-decoder} can not correct Z errors, as shown in the following proposition, \nothmskip
\begin{proposition} For the program $\prog$ in Figure~\ref{fig:noisyrepcode}(b), where a Z error happens on $q_1$, we have \\
$\{X_L \wedge Z_0Z_1 \wedge Z_1Z_2\}\prog\{-X_L \wedge Z_0Z_1 \wedge Z_1Z_2\}$ and $\{-X_L \wedge Z_0Z_1 \wedge Z_1Z_2\}\prog\{X_L \wedge Z_0Z_1 \wedge Z_1Z_2\}$, \\
which is not the desired behaviour of the logical X gate.
\end{proposition} \nothmskip
\myproof{
Similar to the proof in Proposition~\ref{prop:rep-noise-x}.
}

\section{Case Study II: Surface Code}

In this section, we present the verification of the double-defect surface code~\cite{Fowler2012SurfaceCT}. There are two types of stabilizers in surface code, of which one is called `Z-type' stabilizer as it only consists of Pauli Z operators and the other one is called `X-type' stabilizer as it only contains Pauli X operators. These two types of stabilizers together enable high error tolerance of surface code as well as  the ability to correct both bit-flip error and phase-flip error simultaneously. 
Implementing surface code has been pursued by several major quantum computing vendors including IBM~\cite{Chamberland2020TopologicalAS} and Google~\cite{ChenSatzingerAtalayaKorotkovDunsworthSankQui2}.

\subsection{Surface Code}

Double-defect surface code includes a series of logical operations to support fault-tolerant quantum computation, such as 
initialization, measurement, defect enlarging, defect shrinking, logical single-qubit gates (X, Z, H), qubit moving, braiding and the logical CNOT gate. 
In this section, we only verify qubit initialization, qubit moving, logical Pauli gates and logical qubit braiding. The verification of remaining operations is similar to or can be built upon the verified operations.
For example, qubit measurement is the reversing process of qubit initialization, 
and the logical CNOT gate is the concatenation of three braiding operations.


Without loss of generality, we only consider  the distance-3 surface code.
The logical operations shown in Figure~\ref{fig:surfop} is implemented in {\langname} in Figure~\ref{fig:surfcode}. %
We mainly focus on the operations on the X-cut qubit, which is a kind of logical qubit created by disabling X-type stabilizers.
We present the initialization operation in Figure~\ref{fig:surfop}(a) which initializes a X-cut qubit to the logical state $\ket{+_L}$, i.e., the +1 eigenstate of the logical X operator $X_L$ in Figure~\ref{fig:surfop}.
Figure~\ref{fig:surfop}(b) %
shows the logical Z gate $Z_L$. 
We then implements the qubit moving operation and the loigcal H gate shown in Figure~\ref{fig:surfop}(c)(d).  The qubit moving operation 
will not change the logical state. For the logical H gate, we only presents a simplified version which is enough to demonstrate the core idea of the logical H gate in ~\cite{Fowler2012SurfaceCT}. Finally, we include the verification of qubit initialization (to $\ket{0_L}$), the logical X gate  and the braiding operation in  Appendix~\ref{app:surf}. %

\textbf{All proofs in this section will be postponed to Appendix~\ref{app:surf}.}




\input{figtex/surffig}
\input{figtex/surfcode}

\subsection{Verification of Logic Operations}

As specified by the surface code~\cite{Fowler2012SurfaceCT}, any valid logical state should always be in the +1 eigenspace of all active stabilizers on the surface code array, $S = \{s_0,s_1,\cdots\}$. %
For simplification, we omit the stabilizers that does not involve in proof. For example, %
$(\ket{0_L}\bra{0_L},\sigma)\models Z_L \wedge s_0 \wedge s_1 \cdots$ will be denoted by $(\ket{0_L}\bra{0_L},\sigma)\models Z_L$.
In the cases where we need to stress other active stabilizers in the array, we will have $(\ket{0_L}\bra{0_L},\sigma)\models Z_L \wedge A_S$, where $A_S = \wedge_{s\in S}s$.


We first verify the initialization to  $\ket{+_L}$. 
The expected functionality of the initialization operation is to prepare arbitrary state into a desired state, as shown in the following proposition. The precondition of the partial correctness is just $I$ which allows any state. As for the post-condition,
notice that $\ket{+_L}$ is stabilized by the stabilizer $X_0X_1X_2X_4$ (i.e., $X_L$) and other active stabilizers in $S$. 
The verification of initialization to $\ket{0_L}$ is similar and postponed to Appendix~\ref{app:surf}.
\nothmskip
\begin{proposition}[Initialize $\ket{+}_L$]
For the program $\prog$ in Figure~\ref{fig:surfcode}(a) which initializes a X-cut logical qubit to $\ket{+}_L$, as shown in Figure~\ref{fig:surfop}(a), we have
$\{I\}\prog\{X_0X_1X_2X_4 \wedge A_S\}$.
\end{proposition} \nothmskip


The verification of the logical Z gate is similar to that in quantum repetition code and the precondition and post-condition 
can be derived in a similar way.  The verification of logical X gate is similar and postponed to Appendix~\ref{app:surf}.
\nothmskip
\begin{proposition}[Logical Z gate]
	For program $\prog$ in Figure~\ref{fig:surfcode}(b) which implements the logical Z gate in Figure~\ref{fig:surfop}(b), we have $\{X_L\}\prog\{-X_L\}$ and $\{-X_L\}\prog\{X_L\}$, \\
	where $X_L = X_0X_1X_2X_4$.
\end{proposition} \nothmskip




To reason about qubit moving, the key is to prove that the logical state is preserved. 
We first represent the current state of data qubits with the stabilizer language. The following lemma shows that there is a one-to-one mapping between the stabilizer expressions and the logical  quantum states. %
\nothmskip
\begin{restatable}{lemma}{statestabilizer}
\label{lemma:state-stabilizer}
For a X-cut qubit state $\ket{\psi}$, if $\ket{\psi} = \alpha \ket{0}_L + \beta \ket{1}_L$ ($\vert\alpha\vert^2 + \vert\beta\vert^2 = 1$), then there is a unique $(a Z_L + b X_L)$ s.t. $(a Z_L + b X_L)\ket{\psi} = \ket{\psi}$, and in this case $a = \frac{\alpha^2 - \beta^2}{\alpha^2 + \beta^2}$ and $b = \frac{2\alpha\beta}{\alpha^2 + \beta^2}$ .

\noindent\textbf{Conversely}, for a X-cut qubit state $\ket{\psi}$, if $(\frac{\alpha^2 - \beta^2}{\alpha^2 + \beta^2} Z_L + \frac{2\alpha\beta}{\alpha^2 + \beta^2} X_L)\\ *\ket{\psi} = \ket{\psi}$, and $\ket{\psi}$ is in the space spanned by $\{\ket{0_L}, \ket{1_L}\}$, then $\ket{\psi} = \alpha \ket{0}_L + \beta \ket{1}_L$, up to a global phase.
\end{restatable} \nothmskip
We then apply Lemma~\ref{lemma:state-stabilizer} to verify the vertical qubit moving operation in Figure~\ref{fig:surfop}(c). The verification of the horizontal qubit moving is similar. The following proposition confirms that the logical state is not changed since the precondition and the post-condition have equal coefficients w.r.t. the logical X and logical Z operators. 
\nothmskip
\begin{restatable}[Vertical qubit moving]{proposition}{surfvqmov}
For program $\prog$ in Figure~\ref{fig:surfcode}(c) which implements the qubit moving operation in Figure~\ref{fig:surfop}(c) %
, we have 
$\{a Z_L + b X_L\}\prog\{a Z'_L + b X'_L\}$, 
where $Z_L = Z_0Z_1Z_2$, $X_L = X_{2}X_{3}X_{4}X_{6}$, $Z'_L = Z_{0}Z_{1}Z_{2}Z_{6}$, $X'_L = X_{6}X_{8}X_{9}X_{10}$, $a,b \in \mathbb{C}$.
\end{restatable} \nothmskip



While the implementation of the logical H gate requires isolating the defects of a logical qubit, the core of logical H operation is to perform local H gates for the area outlined in Figure~\ref{fig:surfop}(d), which alone is a distance-$3$ planar surface code. We will only verify that part of program for simplicity. Other parts of the logical H operation can be verified on top of verified operations above. Like in the case in quantum repetition code, we only need to verify that the logical X operator is transformed into the logical Z operator and the logical Z operator is transformed into the logical X operator. 
\nothmskip
\begin{proposition}[Logical H gate]
	For the program $\prog$ in Figure~\ref{fig:surfcode}(d) which implements the simplified logical H gate in Figure~\ref{fig:surfop}(d) , we have
	$\{Z_L\}\prog\{X_L' \}$, $\{X_L\}\prog\{Z_L'\}$, 
	where $Z_L = Z_{1}Z_{6}Z_{11}$, $X_L = X_{5}X_{6}X_{7}$, $Z_L' = Z_{5}Z_{6}Z_{7}$, $X_L' = X_{1}X_{6}X_{11}$ for the distance-$3$ planar surface code  outlined in Figure~\ref{fig:surfop}(d).
\end{proposition} \nothmskip

\subsection{Verification on Noise Injection}
\input{figtex/noisysurf}


To reason the correctness when noise exists, we assume a minimal weight perfect matching (MWPM) decoder~\cite{Fowler2015MinimumWP} and error correction for the surface code array.
When the error only happens on one qubit, it can be easily detected by the decoder and be corrected, as indicated below:
\nothmskip
\begin{proposition}
For the program $\prog$ in Figure~\ref{fig:noisysurfcode}(a) which implements a noisy version of the logical Z gate in Figure~\ref{fig:surfop}(b), \\
$\{X_L \wedge A_S\}\prog\{-X_L \wedge A_S\}, \{-X_L \wedge A_S\}\prog\{X_L \wedge A_S\}$, \\
where $X_L = X_{0}X_{1}X_{2}X_{4}$.
\end{proposition} \nothmskip

When we increase the Z error location by one, the error correction protocol may fail.
\nothmskip
\begin{proposition}\label{prop:noisyfail}
For the program $\prog$ in Figure~\ref{fig:noisysurfcode}(b), \\
$\{X_L \wedge A_S\}\prog\{X_L \wedge A_S\}, \{-X_L \wedge A_S\}\prog\{-X_L \wedge A_S\}$,\\
which is not the desired behavior of the logical Z gate.
\end{proposition} \nothmskip

The proposition~\ref{prop:noisyfail} is expected because a distance-$d$ surface code cannot correct errors on more than $\lfloor\frac{d}{2} \rfloor$ qubits.
More complicated cases can be proved in a similar way with our verification framework. 





\section{Conclusion}\label{sec:conclusion}
\cryptflow\ is the first end-to-end system that translates high-level
\tensorflow inference code to MPC protocols. It has 3 components - a) compiler from \tensorflow to \mpc, b) an improved semi-honest 3PC protocol for DNNs, and c) a generic technique to convert semi-honest secure protocols to malicious secure ones.
% that makes a minimal trust assumption in hardware. 
Using \cryptflow, we demonstrate the first instance of secure inference on large benchmarks such as \resnet\ and \densenet\ on the ImageNet dataset
with both semi-honest (in about thirty seconds) and malicious security (in less than two minutes).
 \cryptflow's modular design supports a variety of backends, and we hope that it can serve as a testbed for benchmarking new MPC protocols in the area. 

Going forward, we would like to plugin protocols like SPDZ~\cite{mpspdz} and Delphi~\cite{delphi} in \cryptflow.
%hope that protocol developers like~\cite{aby3,delphi} would integrate their work as backends to \cryptflow.
Our more ambitious goal is to extend \cryptflow\ to support
\tensorflow training.
%% We would like to consider compiling \tensorflow training code as
%% well.
It is a challenging problem since in the absence of the GPU support, the
overheads of MPC protocols for secure training can be prohibitive.
%% However, none of the existing  \mpc protocols~\cite{secureml,securenn,aby3} for training have GPU support and thus the overheads of secure training are huge.


\bibliographystyle{unsrt}
\bibliography{bib}

\newpage
\appendix
\begin{comment}
\subsection{Deep Neural Networks}
A Deep Neural Network (DNN) consists of multiple layers each performing a specific operation on a layer's input (typically a matrix) to prepare the output to be fed into the next layer as input. We now define some common operations performed by different layers in a DNN. For the rest of the discussion, we assume the input to a layer as a matrix $X$ of dimension $a \times b$ and the output as a matrix $Y$ of dimension $c \times d$. In the following discussion, we talk about 2-dimensional matrices as input and output from layers. The description can be easily extended to n-dimensional matrices.

\begin{itemize}
\item \textsf{ReLU($x$)}: A Rectifier Linear Unit (\textsf{ReLU}) is defined by $\mathsf{max(x, 0)}$. In the context of Deep Neural Networks, \textsf{ReLU} is used as an activation function. A \textsf{ReLU} layer does the following: $\forall$ $(i,j)$ $\in ([a], [b])$, where $[n]=\{0, 1, ..., n-1\}$, outputs $Y$ such that $Y[i][j]=$ \textsf{ReLU($X[i][j]$)}. Note that here $c=a$ and $d=b$.
\item \textsf{Batch Normalization}: This technique has recently been used in popular CNNs where the output of select intermediate layers is normalized across a mini-batch of images. It proceeds as follows: $\forall$ $(i,j)$ $\in ([a], [b])$, where $[n]=\{0, 1, ..., n-1\}$, outputs $Y$ such that $Y[i][j] = c_1\left(\frac{X[i][j]-\mu}{\sigma}\right)+c_2$, where constants $c_1$ and $c_2$ are learned during training, and $\mu$ and $\sigma$ denote mean and standard deviation, respectively, of the mini-batch. During inference, the mean and standard deviation are taken to be those of the entire training set.
\item \textsf{Fully Connected}: A fully connected layer is defined by a weight matrix $W$ with dimension $b \times d$ and a bias matrix $E$ with dimension $c \times d$. This layer outputs $Y$ computed as $Y = X \cdot W + E$, where $\cdot$ defines a regular matrix multiplication operation, with $c=a$.
\item \textsf{Convolution}: A convolution layer is defined by a filter matrix $F$ with dimension $m \times n$ and stride parameters $s_x$ (stride along row) and $s_y$ (stride along column). For this discussion, we assume $s_x = s_y = 1$. Let us define a function $f(A, B)=A \odot B$, where $\odot$ is the dot-product operation (element-wise product followed by additive reduction) between 2 matrices. A convolution layer does the following: $\forall$ $(i,j)$ $\in (I, J)$, where $I = \{0, 1, ..., a-m+1\}$ and $J = \{0, 1, ..., b-n+1\}$, outputs $Y$ such that $Y[i][j] = f(\mathsf{submat}_{m,n}(X, i, j), F)$, where $\mathsf{submat}_{m,n}(A, i, j)$ gives the sub-matrix of dimension equal to that of filter $F$, i.e. $m \times n$, with its top left point at position $(i,j)$ in matrix $A$. Here $c = a-m+1$ and $d = b-n+1$.
\item \textsf{MaxPool} and \textsf{AvgPool}: \textsf{MaxPool} and \textsf{AvgPool} layers are defined by a stencil of dimension $m \times n$ and stride parameters $s_x$ (stride along row) and $s_y$ (stride along column). Just like \textsf{Convolution}, we assume $s_x = s_y = 1$. Let us define a function $f_{\alpha,\beta}(A, i, j) = g(\mathsf{submat}_{\alpha,\beta}(A, i, j))$ ($\mathsf{submat}$ function as defined above in \textsf{Convolution}), where $g$ can be instantiated as any function that takes as input a matrix and outputs a single value. These layers do that following: $\forall$ $(i,j)$ $\in (I, J)$, where $I = \{0, 1, ..., a-m+1\}$ and $J = \{0, 1, ..., b-n+1\}$, output $Y$ such that $Y[i][j] = f_{m, n}(X, i, j)$. For \textsf{MaxPool}, $g(A)$ returns the maximum value from the matrix $A$, and for \textsf{AvgPool}, it returns the average of all the values in the matrix $A$. The output dimension $c \times d$ satisfy $c = a-m+1$ and $d = b-n+1$.
\end{itemize}
\end{comment}
\subsection{Algorithms used by Porthos}\label{appendix:porthos}
The additional algorithms that reshape filters, input, and output, used by Porthos are shown in Algorithms \ref{algo:reshapefilter}, \ref{algo:reshapeinput}, and \ref{algo:reshapeoutput}.

\begin{algorithm}[h]
\KwIn{$X \in \bbZ_L^{f \times f}$.}
\KwOut{$Z \in \bbZ_L^{f^2 \times 1}$.}
1. \textbf{for} $i = \{0, ..., f-1\}$ do \\
2. \hspace{10mm} \textbf{for} $j = \{0, ..., f-1\}$ do \\
3. \hspace{10mm} \hspace{10mm} $Z[i \cdot f + j] = X[i][j]$\\
4. \hspace{10mm} \textbf{end for}\\
5. \textbf{end for}

    \caption{{ ReshapeFilter} \label{algo:reshapefilter}}

\end{algorithm}

\begin{algorithm}[h]
\KwIn{$X \in \mathbb{Z}_L^{m \times m}$.}
\KwOut{$Z \in \mathbb{Z}_L^{n^2 \times f^2}$ where $n = m-f+1$.}
\textbf{Global Information}: Filter dimension $f$.\\
1. \textbf{for} $i = \{0, ..., m-f\}$ do \\
2. \hspace{10mm} \textbf{for} $j = \{0, ..., m-f\}$ do \\
3. \hspace{10mm} \hspace{10mm} \textbf{for} $k = \{0, ..., f-1\}$ do \\
4. \hspace{10mm} \hspace{10mm} \hspace{10mm} \textbf{for} $l = \{0, ..., f-1\}$ do \\
5. \hspace{10mm} \hspace{10mm} \hspace{10mm} \hspace{10mm} $Z[i \cdot (m-f+1) + j][k\cdot f+j] = X[k+i][l+j]$\\
6. \hspace{10mm} \hspace{10mm} \hspace{10mm} \textbf{end for}\\
7. \hspace{10mm} \hspace{10mm} \textbf{end for}\\
8. \hspace{10mm} \textbf{end for}\\
9. \textbf{end for}

    \caption{{ ReshapeInput} \label{algo:reshapeinput}}

\end{algorithm}

\begin{algorithm}[h]
\KwIn{$X \in \bbZ_L^{n^2 \times 1}$.}
\KwOut{$Z \in \bbZ_L^{n \times n}$.}
1. \textbf{for} $i = \{0, ..., n-1\}$ do \\
2. \hspace{10mm} \textbf{for} $j = \{0, ..., n-1\}$ do \\
3. \hspace{10mm} \hspace{10mm} $Z[i][j] = X[i\cdot n + j]$\\
4. \hspace{10mm} \textbf{end for}\\
5. \textbf{end for}

    \caption{{ ReshapeOutput} \label{algo:reshapeoutput}}

\end{algorithm}

\subsection{Batch Normalization}\label{appendix:batchnorm}
Batch Normalization \cite{batchNormPaper} is used to normalize the inputs to intermediate layers across a mini-batch of images. 
For a batch $B$ of inputs, let $\mu_B$ and $\sigma_B^2$ be the mean and the variance respectively. 
For an input $x$, the output of the batch normalization layer is defined as
\[ BN(x) = \gamma\frac{\left( x - \mu_B \right)}{\sqrt{\sigma_B^2 + \epsilon}} + \beta \]
where $\gamma$ and $\beta$ are the model parameters learned during training phase. In the inference phase, $\mu_B$ and $\sigma_B^2$ represent the mean and variance of the entire training dataset.

%\pagebreak
\subsection{Accuracy of Athos}\label{appendix:accuracies}

%\pagebreak
%\begin{comment}
\section{Accuracy of Athos}\label{appendix:accuracies}

% This section presents Top 1 and Top 5 accuracies of Athos on \densenet\ running on ImageNet.
In this section, we present the Top 1 and Top 5 accuracies of Athos on the ImageNet dataset. 

\begin{figure}
  \includegraphics[width=0.8\linewidth]{resnetaccuracy1.pdf}
  \caption{\resnet: Top 1 accuracy vs Scale}
  \label{fig:resnetaccuracy1}
\end{figure}

\begin{figure}
  \includegraphics[width=0.8\linewidth]{resnetaccuracy5.pdf}
  \caption{\resnet: Top 5 accuracy vs Scale}
  \label{fig:resnetaccuracy5}
\end{figure}

%\begin{figure}
%  \includegraphics[width=0.8\linewidth]{squeezenetaccuracy1.pdf}
 % \caption{\squeezenet: Top 1 accuracy vs Scale}
  %\label{fig:squeezenetaccuracy1}
%\end{figure}

%\begin{figure}
 % \includegraphics[width=0.8\linewidth]{squeezenetaccuracy5.pdf}
 % \caption{\squeezenet: Top 5 accuracy vs Scale}
  %\label{fig:squeezenetaccuracy5}
%\end{figure}

\begin{figure}
  \includegraphics[width=0.8\linewidth]{densenetaccuracy1.pdf}
  \caption{\densenet: Top 1 accuracy vs Scale}
  \label{fig:densenetaccuracy1}
\end{figure}

\begin{figure}
  \includegraphics[width=0.8\linewidth]{densenetaccuracy5.pdf}
  \caption{\densenet: Top 5 accuracy vs Scale}
  \label{fig:densenetaccuracy5}
\end{figure}
%\end{comment}
% \begin{figure}
%   \includegraphics[width=0.8\linewidth]{squeezenetaccuracy1.pdf}
%   \caption{\squeezenet: Top 1 accuracy vs Scaling Factor}
%   \label{fig:squeezenetaccuracy1}
% \end{figure}

\subsection{Comparison with 2PC/FHE}
\label{unfaircomp}
See Table~\ref{tab:porthosvspriorcifar10} which validates the well-known fact that 3PC protocols like Porthos are much faster than 2PC/FHE-based approaches.
We omit other 2PC/FHE works (\cite{secureml,ezpc,nhe,nhe2,delphi}, etc.) as the performance comparisons are similar and do not provide additional insights.
\begin{table}
  \centering
%  \resizebox{\columnwidth}{!}{

      \begin{tabular}{|c|c|c|c|c|}
    \hline
    Benchmark & CHET & MiniONN  & Gazelle & Porthos\\
    \hline
	$\squeezenet^*$ (CIFAR) & $1342$ & - & - & $0.05$ \\
	\hline
    MiniONN (CIFAR) & - & $544$  & $12.9$ & $0.36$\\
	\hline
    MiniONN (MNIST) & - & 9.4  & 0.81 & 0.03\\
   \hline 
\end{tabular}
%}
 \caption{Comparison with 2PC  -- All times in seconds. CHET replaces ReLUs in a small \squeezenet\ with square activations.}
\label{tab:porthosvspriorcifar10}
%\tableup
	%\vspace{-0.5cm}
\end{table}
%\mayank{Adding the comparison of Aramis with QuantizedNN here. QuantizedNN uses MP-SDPZ only, so this table serves the purpose of a comparison with both of them.}

\subsection{Proof of malicious security}
\label{app:proof}
For simplicity, consider the case of single malicious party $P_i$. 
Informally, we argue that our technique constrains $P_i$ to follow the instructions of the semi-honest protocol $\pi(\cdot)$ faithfully. 
Or, deviating from faithful execution would result in some honest party to abort. 
%Below, we give a security proof using the standard simulation paradigm (we refer the reader to~\cite{gmw,canetti00} for details on the paradigm).
%
The first $\compute$ invocation of $\fattest^{(\vk_i,\sksign_i)}$ fixes the input of $P_i$ used in the protocol. 
Since every other $\fattest^{(\vk_j,\sksign_j)}$ reliably knows
the verification key $\vk_i$ used by $\fattest^{(\vk_i,\sksign_i)}$,
it checks the signatures on the function description (i.e., $\token^{(i)}_{\pi^*}$)
 as well as the messages of the protocol. The unforgeability of the signature scheme guarantees that $P_i$ cannot forge signatures on incorrectly generated protocol messages. Note that we use this property to ensure that both of the following signatures cannot be forged: (a) signatures under $\vk_i$ on messages generated by $\fattest^{(\vk_i,\sksign_i)}$ and sent to honest $P_j$ (b) signatures under $\vk_j$ on messages sent by $P_j$ being fed into $\fattest^{(\vk_i,\sksign_i)}$. 
%
Also, $\fattest^{(\vk_i,\sksign_i)}$ provides correct randomness to generate messages of $P_i$ in the semi-honest secure protocol. 
%
Hence, all messages from $P_i$ to any honest party $P_j$ are generated correctly as directed by $\pi$. 
This argument can be easily extended to multiple colluding corrupt parties.

%Below, we give a security proof using the standard simulation paradigm (we refer the reader to~\cite{gmw,canetti00} for details on the paradigm).

Formally, we give a security proof using the standard simulation paradigm (we refer the reader to~\cite{gmw,canetti00} for details on the paradigm).
%we prove simulation based security, %(Section~\ref{sec:sim-security}),  
That is, the protocol in Figure~\ref{fig:shtomprotocol} securely realizes the ideal \mpc functionality described in Figure~\ref{fig:fideal} against malicious adversaries. 

\begin{theorem}[Restated]
\label{theorem:maliciousmpc}
Let $\pi(\cdot)$ be a semi-honest secure MPC protocol securely realizing $\fmpc^f$. Then, protocol $\protshtom$ described in Figure \ref{fig:shtomprotocol} securely realizes $\fmpc^f$ in the $\fattest^{(\vk_i,\sksign_i)}-$hybrid model (with $i \in [n]$) against malicious adversaries.
\end{theorem}

\newcommand{\simush}{\simu'}

\noindent\begin{proof}[Proof Sketch]  Let $\adv$ be the real world adversary. 
Ideal world adversary $\simu$ that simulates the view of $\adv$ is as follows: 
Let $\simush$ be the ideal world adversary or the semi-honest simulator for $\pi$ (this exists because $\pi$ is semi-honest secure). 
$\simu$ picks $\{(\vk_k,\sksign_k)\}_{k\in[n]}$ and gives $\{\vk_k\}_{k \in [n]}$ to $\adv$. 
We denote a corrupt party by $P_i$ and honest party by $P_j$.
Next, when $\adv$ invokes an instance of $\fattest^{(\vk_i, \sksign_i)}$ on command $\gcommit$ for a corrupted party $P_i$, $\simu$ simulates the correct behavior of $\fattest^{(\vk_i,\sksign_i)}$. 
Also, $\simu$ sends correctly generated tokens $\{\token^{(j)}_{\pi^*}\}$ for all honest parties to $\adv$. 
When $\simu$ receives token from $\adv$ corresponding to a corrupted party $P_i$, it checks it against $\pi^*$ and $\vk_i$. It aborts if verification fails.
When $\adv$ invokes $\fattest^{(\vk_i, \sksign_i)}$ with $x_i$, $\simu$ stores it as input of $P_i$. When $\adv$ commits to inputs of all corrupt parties, $\simu$ sends these to $\fmpc^f$ to learn output $y$. It sends inputs of corrupt parties and outputs $y$ to $\simush$ that generates the view of the adversary in the semi-honest protocol, that contains the randomness for all corrupt parties as well as the transcript of the protocol. Using this, it is easy for $\simu$ to simulate the view of $\adv$ in the rest of the protocol. 
The indistinguishability of the adversary's view in real and ideal executions follows from the semi-honest security of $\pi$. 
\end{proof}






\end{document}
